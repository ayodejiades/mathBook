\documentclass[a4paper]{book}

% --- FONT AND ENCODING (Core) ---
\usepackage[T1]{fontenc} % Output font encoding
\usepackage{microtype} % Improves typography (spacing, justification)

% --- PAGE LAYOUT AND GEOMETRY ---
\usepackage[left=.7in, right=.7in, top=1.2in, bottom=1in]{geometry}

% --- HEADER AND FOOTER ---
\usepackage{fancyhdr}
\pagestyle{fancy}
\fancyhf{} % Clear all header and footer fields
\fancyhead[LE]{\nouppercase{\leftmark}} % Chapter name on even pages (left)
\fancyhead[RO]{\nouppercase{\rightmark}} % Section name on odd pages (right)
\fancyfoot[C]{\thepage} % Page number in the center of the footer
\renewcommand{\headrulewidth}{0.4pt} % Rule under the header
\renewcommand{\footrulewidth}{0pt} % No rule over the footer
\fancypagestyle{plain}{%
  \fancyhf{}% Clear header/footer for plain pages (e.g., chapter starts)
  \fancyfoot[C]{\thepage}%
  \renewcommand{\headrulewidth}{0pt}%
  \renewcommand{\footrulewidth}{0pt}%
}

% --- MATHEMATICS PACKAGES (Base and Standard Symbols) ---
\usepackage{amsmath}
\usepackage{amsfonts} % Loads amssymb implicitly or defines symbols amssymb relies on
\usepackage{amssymb}  % For additional math symbols
\usepackage{amsthm}   % For theorem-like environments if needed

% --- FONT PACKAGES (Text and Math - load after ams packages) ---
\usepackage{newtxtext, newtxmath} % Times-like font for text and math


% --- CUSTOM DERIVATIVE COMMANDS ---
% For first derivative: \dv{y}{x}
\newcommand{\dv}[2]{\frac{\mathrm{d}#1}{\mathrm{d}#2}}

% For second derivative: \ddv{y}{x} (will typeset d^2y / dx^2)
% Note: This assumes the second argument 'x' is a single variable.
% If you need d^2y / d(t^2)^2 (which is unusual), you'd need a different command or pass {{t^2}}
\newcommand{\ddv}[2]{\frac{\mathrm{d}^2 #1}{\mathrm{d} #2^2}}

% For partial derivatives (optional, but good to have if you plan to use them)
% \pdv{f}{x}
\newcommand{\pdv}[2]{\frac{\partial #1}{\partial #2}}
% \pddv{f}{x} for d^2f/dx^2 (partial)
\newcommand{\pddv}[2]{\frac{\partial^2 #1}{\partial #2^2}}
% \pddv{f}{x}{y} for d^2f/(dx dy) (partial)
\newcommand{\pddvxdy}[3]{\frac{\partial^2 #1}{\partial #2 \partial #3}}

% --- MATHEMATICS UTILITIES (Load after font and core math setup) ---
\usepackage{siunitx}  % For SI units
\AtBeginDocument{%
  \RenewCommandCopy\qty\SI % Makes \qty from physics behave like \SI from siunitx for {value}{unit}
}

% --- LISTS AND TABLES ---
\usepackage{multicol} % For multi-column layout
\usepackage{enumitem}
\setlist[enumerate]{itemsep=3pt, topsep=5pt, partopsep=2pt}
\usepackage{booktabs} % For professional-looking tables
\usepackage{cancel}   % For striking out terms in math

% --- GRAPHICS AND DIAGRAMS ---
\usepackage{graphicx}
\usepackage{tikz}
\usetikzlibrary{positioning}

% --- TABLE OF CONTENTS AND SECTIONING ---
\usepackage{titlesec}
\usepackage{tocloft}
\renewcommand{\cftchapleader}{\cftdotfill{\cftdotsep}}
\renewcommand{\cftbeforechapskip}{10pt}

% --- HYPERLINKS AND REFERENCING ---
\usepackage{hyperref}
\hypersetup{
    colorlinks=true,
    linkcolor=black,
    urlcolor=blue,
    citecolor=purple,
    pdftitle={Acing UTME Maths},
    pdfauthor={Ayodeji Adesegun and Chimobi Nwafor},
    bookmarksopen=true,
    bookmarksnumbered=true
}
\usepackage{cleveref} % Load after hyperref

% --- BIBLIOGRAPHY ---
\usepackage{biblatex}
% \addbibresource{your_bibliography_file.bib}

% --- TITLE PAGE ---
\usepackage{titling}

% --- DOCUMENT START ---
\begin{document}

% --- TITLE PAGE ---
\begin{titlingpage}
\centering
\vspace*{\fill}
\Huge\textbf{Acing UTME Maths}\\
\vspace{1cm}
\Large{A Comprehensive Guide with Past Questions and Solutions}\\
\vspace{2cm}
\Large\textbf{By}\\
\vspace{0.5cm}
\Large\textbf{Ayodeji Adesegun and Chimobi Nwafor}\\
\vspace*{\fill}
\Large{\today}\\
\vspace*{\fill}
\end{titlingpage}

% --- FRONTMATTER ---
\begin{frontmatter}
\pagestyle{plain}

\chapter*{Dedication and Acknowledgements}
This work is dedicated to our families, whose unwavering support has been our greatest strength throughout this journey. Their constant encouragement and belief in our abilities have fueled our passion and perseverance in creating this resource.

We would also like to express our sincere gratitude to the following individuals and institutions for their invaluable contributions:
\begin{itemize}
\item Our mentors and teachers, who instilled in us a love for mathematics and equipped us with the knowledge and skills needed to succeed.
\item The examiners and administrators of the UTME, whose dedication to educational standards ensures a fair and effective assessment process.
\item  Our colleagues and friends, who provided feedback and support throughout the development of this book.
\item The wider academic community, whose research and publications have laid the foundation for our understanding of mathematics.
\end{itemize}
We are truly grateful for the collective effort that has made this book possible. We hope that it will be a valuable resource for students preparing for the UTME and beyond.

\section*{A Note on Preparation}
This book was meticulously prepared using the \LaTeX\ document processing system, a powerful tool for typesetting high-quality scientific and mathematical texts. The diagrams herein were crafted using the TikZ package. We extend our thanks to the developers of the \LaTeX\ system and the numerous packages that made this work possible.

\section*{About the Authors}
\begin{description}
\item[\href{https://linkedin.com/in/ayodejiades}{\textbf{Ayodeji Adesegun}}:] He is a teenager. In his spare time, he enjoys solving mathematical problems from various contests not excluding the Olympiads. You can find more about him \href{https://ayodejiades.vercel.app}{here.}
\item[\href{https://linkedin.com/in/jeremaih-nwafor}{\textbf{Chimobi Nwafor}}:] He is a software engineer trainee at INITS.
\end{description}

\clearpage
\tableofcontents
\clearpage

\chapter*{Preface}
Welcome to ``Acing UTME Maths'', a comprehensive guide designed to help you conquer the upcoming UTME Maths exam. This book provides you with a wealth of past questions, detailed solutions, and insightful strategies to enhance your understanding and preparation.

This book is organized into chapters that follow the official UTME Maths syllabus, covering all key topics and subtopics. Each chapter includes a variety of past questions carefully selected to reflect the types and difficulty levels encountered in the actual exam.

In addition to past questions, we have provided detailed solutions that explain the reasoning behind each step and highlight common mistakes to avoid. We encourage you to work through these solutions carefully and utilize them as learning tools to improve your problem-solving skills.

Furthermore, we have incorporated valuable strategies throughout the book, offering tips and techniques to maximize your efficiency and performance on the exam. These strategies will help you manage your time effectively, approach different question types confidently, and overcome any challenges you may encounter.

We are confident that ``Acing UTME Maths'' will be your ultimate companion on your journey to success. By diligently working through the material and utilizing the resources provided, you will gain the knowledge, skills, and confidence needed to achieve your desired score on the UTME Mathematics exam.

Best of luck!

Ayodeji Adesegun and Chimobi Nwafor
\clearpage
\end{frontmatter}

% --- MAINMATTER ---
\mainmatter
\pagestyle{fancy}

% Chapter organization
\chapter{Number and Numeration}
\section{Number Bases}
\subsection{Questions}

\begin{enumerate}[label={\arabic*.}]
\item The number \(25\) when converted from the tens and units base to the binary base (base \(2\)) is one of the following
	\begin{enumerate}[label={\Alph*.}]
	\item \(10011\)
	\item \(1111011\)
	\item \(111000\)
	\item \(11001\)
	\item \(110011\)
	\end{enumerate}
 
\item The currency used in a country bought \(4\) bags of rice at \(N56\) per bag and \(3\) tins of milk at \(N4\) per tin. What is the total cost of the items she bought?
  \begin{enumerate}[label={\Alph*.}]
    \item \(N245\)
    \item \(N242\)
    \item \(N236\)
    \item \(N341\)
    \item \(N338\)
  \end{enumerate}

\item Evaluate \((212)_3 - (121)_3 + (222)_3\).
  \begin{enumerate}[label={\Alph*.}]
    \item \((313)_3\)
    \item \((1000)_3\)
    \item \((1020)_3\)
    \item \((1222)_3\)
    \item \((1213)_3\)
  \end{enumerate}

\item A trader in a country where their currency 'MONT' (M) is in base five bought \(103_5\) oranges at \(M14_5\) each. If he sold the oranges at \(M24_5\) each, what would be his gain?
  \begin{enumerate}[label={\Alph*.}]
    \item \(M103_5\)
    \item \(M1030_5\)
    \item \(M102_5\)
    \item \(M2002_5\)
    \item \(M3024_5\)
  \end{enumerate}
  
\item Find \(x\) if \((x_4)^2 = (100100)_2\)
	\begin{enumerate}[label={\Alph*.}]
	\item \(6\)
	\item \(12\)
	\item \(100\)
	\item \(210\)
 	\item \(10042\)
	\end{enumerate}
 
\item Convert \(241_5\) to base \(8\).
	\begin{enumerate}[label={\Alph*.}]
	\item \(71_8\)
	\item \(107_8\)
	\item \(176_8\)
	\item \(241_8\)
	\end{enumerate}
 
\item In the equation below, solve for \(x\) \[\frac{(11)_2}{x_2}  = \frac{(1000)_2}{(x)_2 + (101)_2}\].  
	\begin{enumerate}[label={\Alph*.}]
	\item \(101\)
	\item \(11\)
	\item \(110\)
	\item \(111\)
 	\item \(10\)
	\end{enumerate}
 
\item
	\begin{enumerate}[label={\Alph*.}]
	\item \(\)
	\item \(\)
	\item \(\)
	\item \(\)
	\end{enumerate}
\item
	\begin{enumerate}[label={\Alph*.}]
	\item \(\)
	\item \(\)
	\item \(\)
	\item \(\)
	\end{enumerate}
\item
	\begin{enumerate}[label={\Alph*.}]
	\item \(\)
	\item \(\)
	\item \(\)
	\item \(\)
	\end{enumerate}
\item
	\begin{enumerate}[label={\Alph*.}]
	\item \(\)
	\item \(\)
	\item \(\)
	\item \(\)
	\end{enumerate}


\item
	\begin{enumerate}[label={\Alph*.}]
	\item \(\)
	\item \(\)
	\item \(\)
	\item \(\)
	\end{enumerate}
\item
	\begin{enumerate}[label={\Alph*.}]
	\item \(\)
	\item \(\)
	\item \(\)
	\item \(\)
	\end{enumerate}
\item
	\begin{enumerate}[label={\Alph*.}]
	\item \(\)
	\item \(\)
	\item \(\)
	\item \(\)
	\end{enumerate}
\item
	\begin{enumerate}[label={\Alph*.}]
	\item \(\)
	\item \(\)
	\item \(\)
	\item \(\)
	\end{enumerate}
\item
	\begin{enumerate}[label={\Alph*.}]
	\item \(\)
	\item \(\)
	\item \(\)
	\item \(\)
	\end{enumerate}
\item
	\begin{enumerate}[label={\Alph*.}]
	\item \(\)
	\item \(\)
	\item \(\)
	\item \(\)
	\end{enumerate}
\item
	\begin{enumerate}[label={\Alph*.}]
	\item \(\)
	\item \(\)
	\item \(\)
	\item \(\)
	\end{enumerate}
\item
	\begin{enumerate}[label={\Alph*.}]
	\item \(\)
	\item \(\)
	\item \(\)
	\item \(\)
	\end{enumerate}
\item
	\begin{enumerate}[label={\Alph*.}]
	\item \(\)
	\item \(\)
	\item \(\)
	\item \(\)
	\end{enumerate}
\item
	\begin{enumerate}[label={\Alph*.}]
	\item \(\)
	\item \(\)
	\item \(\)
	\item \(\)
	\end{enumerate}
\item
	\begin{enumerate}[label={\Alph*.}]
	\item \(\)
	\item \(\)
	\item \(\)
	\item \(\)
	\end{enumerate}
\item
	\begin{enumerate}[label={\Alph*.}]
	\item \(\)
	\item \(\)
	\item \(\)
	\item \(\)
	\end{enumerate}
\item
	\begin{enumerate}[label={\Alph*.}]
	\item \(\)
	\item \(\)
	\item \(\)
	\item \(\)
	\end{enumerate}
\item
	\begin{enumerate}[label={\Alph*.}]
	\item \(\)
	\item \(\)
	\item \(\)
	\item \(\)
	\end{enumerate}
\item
	\begin{enumerate}[label={\Alph*.}]
	\item \(\)
	\item \(\)
	\item \(\)
	\item \(\)
	\end{enumerate}
\item
	\begin{enumerate}[label={\Alph*.}]
	\item \(\)
	\item \(\)
	\item \(\)
	\item \(\)
	\end{enumerate}
\item
	\begin{enumerate}[label={\Alph*.}]
	\item \(\)
	\item \(\)
	\item \(\)
	\item \(\)
	\end{enumerate}
\item
	\begin{enumerate}[label={\Alph*.}]
	\item \(\)
	\item \(\)
	\item \(\)
	\item \(\)
	\end{enumerate}
\item
	\begin{enumerate}[label={\Alph*.}]
	\item \(\)
	\item \(\)
	\item \(\)
	\item \(\)
	\end{enumerate}
\item
	\begin{enumerate}[label={\Alph*.}]
	\item \(\)
	\item \(\)
	\item \(\)
	\item \(\)
	\end{enumerate}
\item
	\begin{enumerate}[label={\Alph*.}]
	\item \(\)
	\item \(\)
	\item \(\)
	\item \(\)
	\end{enumerate}
\item
	\begin{enumerate}[label={\Alph*.}]
	\item \(\)
	\item \(\)
	\item \(\)
	\item \(\)
	\end{enumerate}
\item
	\begin{enumerate}[label={\Alph*.}]
	\item \(\)
	\item \(\)
	\item \(\)
	\item \(\)
	\end{enumerate}
\item
	\begin{enumerate}[label={\Alph*.}]
	\item \(\)
	\item \(\)
	\item \(\)
	\item \(\)
	\end{enumerate}
\item
	\begin{enumerate}[label={\Alph*.}]
	\item \(\)
	\item \(\)
	\item \(\)
	\item \(\)
	\end{enumerate}
\item
	\begin{enumerate}[label={\Alph*.}]
	\item \(\)
	\item \(\)
	\item \(\)
	\item \(\)
	\end{enumerate}
\item
	\begin{enumerate}[label={\Alph*.}]
	\item \(\)
	\item \(\)
	\item \(\)
	\item \(\)
	\end{enumerate}
\item
	\begin{enumerate}[label={\Alph*.}]
	\item \(\)
	\item \(\)
	\item \(\)
	\item \(\)
	\end{enumerate}
\item
	\begin{enumerate}[label={\Alph*.}]
	\item \(\)
	\item \(\)
	\item \(\)
	\item \(\)
	\end{enumerate}
\item
	\begin{enumerate}[label={\Alph*.}]
	\item \(\)
	\item \(\)
	\item \(\)
	\item \(\)
	\end{enumerate}
\item
	\begin{enumerate}[label={\Alph*.}]
	\item \(\)
	\item \(\)
	\item \(\)
	\item \(\)
	\end{enumerate}
\item
	\begin{enumerate}[label={\Alph*.}]
	\item \(\)
	\item \(\)
	\item \(\)
	\item \(\)
	\end{enumerate}
\item
	\begin{enumerate}[label={\Alph*.}]
	\item \(\)
	\item \(\)
	\item \(\)
	\item \(\)
	\end{enumerate}
\item
	\begin{enumerate}[label={\Alph*.}]
	\item \(\)
	\item \(\)
	\item \(\)
	\item \(\)
	\end{enumerate}
\item
	\begin{enumerate}[label={\Alph*.}]
	\item \(\)
	\item \(\)
	\item \(\)
	\item \(\)
	\end{enumerate}
\item
	\begin{enumerate}[label={\Alph*.}]
	\item \(\)
	\item \(\)
	\item \(\)
	\item \(\)
	\end{enumerate}
\item
	\begin{enumerate}[label={\Alph*.}]
	\item \(\)
	\item \(\)
	\item \(\)
	\item \(\)
	\end{enumerate}
\item
	\begin{enumerate}[label={\Alph*.}]
	\item \(\)
	\item \(\)
	\item \(\)
	\item \(\)
	\end{enumerate}
\item
	\begin{enumerate}[label={\Alph*.}]
	\item \(\)
	\item \(\)
	\item \(\)
	\item \(\)
	\end{enumerate}
\item
	\begin{enumerate}[label={\Alph*.}]
	\item \(\)
	\item \(\)
	\item \(\)
	\item \(\)
	\end{enumerate}
\item
	\begin{enumerate}[label={\Alph*.}]
	\item \(\)
	\item \(\)
	\item \(\)
	\item \(\)
	\end{enumerate}
\item
	\begin{enumerate}[label={\Alph*.}]
	\item \(\)
	\item \(\)
	\item \(\)
	\item \(\)
	\end{enumerate}
\item
	\begin{enumerate}[label={\Alph*.}]
	\item \(\)
	\item \(\)
	\item \(\)
	\item \(\)
	\end{enumerate}
\item
	\begin{enumerate}[label={\Alph*.}]
	\item \(\)
	\item \(\)
	\item \(\)
	\item \(\)
	\end{enumerate}
\item
	\begin{enumerate}[label={\Alph*.}]
	\item \(\)
	\item \(\)
	\item \(\)
	\item \(\)
	\end{enumerate}
\item
	\begin{enumerate}[label={\Alph*.}]
	\item \(\)
	\item \(\)
	\item \(\)
	\item \(\)
	\end{enumerate}
\item
	\begin{enumerate}[label={\Alph*.}]
	\item \(\)
	\item \(\)
	\item \(\)
	\item \(\)
	\end{enumerate}
\item
	\begin{enumerate}[label={\Alph*.}]
	\item \(\)
	\item \(\)
	\item \(\)
	\item \(\)
	\end{enumerate}
\item
	\begin{enumerate}[label={\Alph*.}]
	\item \(\)
	\item \(\)
	\item \(\)
	\item \(\)
	\end{enumerate}
\item
	\begin{enumerate}[label={\Alph*.}]
	\item \(\)
	\item \(\)
	\item \(\)
	\item \(\)
	\end{enumerate}
\item
	\begin{enumerate}[label={\Alph*.}]
	\item \(\)
	\item \(\)
	\item \(\)
	\item \(\)
	\end{enumerate}
\item
	\begin{enumerate}[label={\Alph*.}]
	\item \(\)
	\item \(\)
	\item \(\)
	\item \(\)
	\end{enumerate}
\item
	\begin{enumerate}[label={\Alph*.}]
	\item \(\)
	\item \(\)
	\item \(\)
	\item \(\)
	\end{enumerate}
\item
	\begin{enumerate}[label={\Alph*.}]
	\item \(\)
	\item \(\)
	\item \(\)
	\item \(\)
	\end{enumerate}
\item
	\begin{enumerate}[label={\Alph*.}]
	\item \(\)
	\item \(\)
	\item \(\)
	\item \(\)
	\end{enumerate}
\item
	\begin{enumerate}[label={\Alph*.}]
	\item \(\)
	\item \(\)
	\item \(\)
	\item \(\)
	\end{enumerate}
\item
	\begin{enumerate}[label={\Alph*.}]
	\item \(\)
	\item \(\)
	\item \(\)
	\item \(\)
	\end{enumerate}
\item
	\begin{enumerate}[label={\Alph*.}]
	\item \(\)
	\item \(\)
	\item \(\)
	\item \(\)
	\end{enumerate}
\item
	\begin{enumerate}[label={\Alph*.}]
	\item \(\)
	\item \(\)
	\item \(\)
	\item \(\)
	\end{enumerate}
\item
	\begin{enumerate}[label={\Alph*.}]
	\item \(\)
	\item \(\)
	\item \(\)
	\item \(\)
	\end{enumerate}
\item
	\begin{enumerate}[label={\Alph*.}]
	\item \(\)
	\item \(\)
	\item \(\)
	\item \(\)
	\end{enumerate}
\item
	\begin{enumerate}[label={\Alph*.}]
	\item \(\)
	\item \(\)
	\item \(\)
	\item \(\)
	\end{enumerate}
\item
	\begin{enumerate}[label={\Alph*.}]
	\item \(\)
	\item \(\)
	\item \(\)
	\item \(\)
	\end{enumerate}
\item
	\begin{enumerate}[label={\Alph*.}]
	\item \(\)
	\item \(\)
	\item \(\)
	\item \(\)
	\end{enumerate}


\item
	\begin{enumerate}[label={\Alph*.}]
	\item \(\)
	\item \(\)
	\item \(\)
	\item \(\)
	\end{enumerate}
\item
	\begin{enumerate}[label={\Alph*.}]
	\item \(\)
	\item \(\)
	\item \(\)
	\item \(\)
	\end{enumerate}
\item
	\begin{enumerate}[label={\Alph*.}]
	\item \(\)
	\item \(\)
	\item \(\)
	\item \(\)
	\end{enumerate}
\item
	\begin{enumerate}[label={\Alph*.}]
	\item \(\)
	\item \(\)
	\item \(\)
	\item \(\)
	\end{enumerate}
\item
	\begin{enumerate}[label={\Alph*.}]
	\item \(\)
	\item \(\)
	\item \(\)
	\item \(\)
	\end{enumerate}
\item
	\begin{enumerate}[label={\Alph*.}]
	\item \(\)
	\item \(\)
	\item \(\)
	\item \(\)
	\end{enumerate}
\item
	\begin{enumerate}[label={\Alph*.}]
	\item \(\)
	\item \(\)
	\item \(\)
	\item \(\)
	\end{enumerate}
\item
	\begin{enumerate}[label={\Alph*.}]
	\item \(\)
	\item \(\)
	\item \(\)
	\item \(\)
	\end{enumerate}
\item
	\begin{enumerate}[label={\Alph*.}]
	\item \(\)
	\item \(\)
	\item \(\)
	\item \(\)
	\end{enumerate}
\item
	\begin{enumerate}[label={\Alph*.}]
	\item \(\)
	\item \(\)
	\item \(\)
	\item \(\)
	\end{enumerate}
\item
	\begin{enumerate}[label={\Alph*.}]
	\item \(\)
	\item \(\)
	\item \(\)
	\item \(\)
	\end{enumerate}
\item
	\begin{enumerate}[label={\Alph*.}]
	\item \(\)
	\item \(\)
	\item \(\)
	\item \(\)
	\end{enumerate}
\item
	\begin{enumerate}[label={\Alph*.}]
	\item \(\)
	\item \(\)
	\item \(\)
	\item \(\)
	\end{enumerate}
\item
	\begin{enumerate}[label={\Alph*.}]
	\item \(\)
	\item \(\)
	\item \(\)
	\item \(\)
	\end{enumerate}
\item
	\begin{enumerate}[label={\Alph*.}]
	\item \(\)
	\item \(\)
	\item \(\)
	\item \(\)
	\end{enumerate}
\item
	\begin{enumerate}[label={\Alph*.}]
	\item \(\)
	\item \(\)
	\item \(\)
	\item \(\)
	\end{enumerate}
\item
	\begin{enumerate}[label={\Alph*.}]
	\item \(\)
	\item \(\)
	\item \(\)
	\item \(\)
	\end{enumerate}
\item
	\begin{enumerate}[label={\Alph*.}]
	\item \(\)
	\item \(\)
	\item \(\)
	\item \(\)
	\end{enumerate}
\item
	\begin{enumerate}[label={\Alph*.}]
	\item \(\)
	\item \(\)
	\item \(\)
	\item \(\)
	\end{enumerate}
\item
	\begin{enumerate}[label={\Alph*.}]
	\item \(\)
	\item \(\)
	\item \(\)
	\item \(\)
	\end{enumerate}
\item
	\begin{enumerate}[label={\Alph*.}]
	\item \(\)
	\item \(\)
	\item \(\)
	\item \(\)
	\end{enumerate}
\item
	\begin{enumerate}[label={\Alph*.}]
	\item \(\)
	\item \(\)
	\item \(\)
	\item \(\)
	\end{enumerate}
\item
	\begin{enumerate}[label={\Alph*.}]
	\item \(\)
	\item \(\)
	\item \(\)
	\item \(\)
	\end{enumerate}
\item
	\begin{enumerate}[label={\Alph*.}]
	\item \(\)
	\item \(\)
	\item \(\)
	\item \(\)
	\end{enumerate}
\item
	\begin{enumerate}[label={\Alph*.}]
	\item \(\)
	\item \(\)
	\item \(\)
	\item \(\)
	\end{enumerate}
\end{enumerate}

32. 4243_5 - 12x4_5 = y344. what is the difference between x and y?
 a. 4	b. 2	c. 1	d. 3	e. 5

\chapter{Fraction and Decimals}
\section{Percentage Error and Approximation}
\subsection{Question}
\begin{multicols}{2}
\begin{enumerate}[label={\arabic*.}]
\item The sum of 3\(\frac {7}{8}\) and 1\(\frac{1}{3}\) is less than the difference between \(\frac {3}{8}\) and 1\(\frac{2}{3}\) by.
    \begin{enumerate}[label={\Alph*.}]
    \item \(3\frac{2}{3}\)
    \item \(1\frac{1}{2}\)
    \item \(8\frac{1}{8}\)
    \item \(0\)
    \item \(5\frac{1}{4}\)

    \end{enumerate}
\item  After getting a rise of 15\%, a man's new monthly salary is N345. How much per month did he earn befor the increase? 
    \begin{enumerate}[label={\Alph*.}]
    \item N\(360\)
    \item N\(300\)
    \item N\(293.25\)
    \item N\(330\)
    \item N\(396.75\)

    \end{enumerate}
\item Find correct to 3 significant figures, the value of \(\sqrt[]{41830}\)
    \begin{enumerate}[label={\Alph*.}]
    \item \(647\)
    \item \(2050\)
    \item \(205\)
    \item \(647\)
    \item \(6470\)

    \end{enumerate}
\item 12 men complete a job in 9days. How many men working at the sam rate, would be required to complete the job in 6 days?. 
    \begin{enumerate}[label={\Alph*.}]
    \item \(24\)
    \item \(9\)
    \item \(8\)
    \item \(12\)
    \item \(18\)

    \end{enumerate}
\item Simplify 2\(\frac{5}{12}\) - 1\(\frac{7}{8}\) \(\times\) \(\frac{6}{5}\).
    \begin{enumerate}[label={\Alph*.}]
    \item \(\frac{11}{30}\)
    \item \(\frac{9}{4}\)
    \item \(\frac{1}{6}\)
    \item \(\frac{5}{3}\)
    \item \(\frac{13}{20}\)

    \end{enumerate}
\item By selling an article for N45.00 a man makes a profit of 8\%. For how much should he have sold it in order to make a profit of 32\%? 
    \begin{enumerate}[label={\Alph*.}]
    \item N\(59.00\)
    \item N\(55.00\)
    \item N\(180.00\)
    \item N\(63.00\)
    \item N\(42.00\)
    

    \end{enumerate}
\item Which of the following fractions is less than one-third?
    \begin{enumerate}[label={\Alph*.}]
    \item \(\frac{4}{11}\)
    \item \(\frac{122}{383}\)
    \item \(\frac{15}{46}\)
    \item \(\frac{22}{63}\)
    \item \(\frac{6}{25}\)

    \end{enumerate}
\item The ratio of the price of load of bread to the price of a packet of sugar in 1975 was \(a:x\). In 1980 teh price of a loaf of bread went up by 25\% and that of a packet of sugar by 10\%. Their new ratio is now
    \begin{enumerate}[label={\Alph*.}]
    \item \(50a:44x\)
    \item \(44a:50x\)
    \item \(40a:44x\)
    \item \(55a:44x\)
    \item \(44a:55x\)

    \end{enumerate}
\item Simplify 1 + \(\frac{2}{3 + \frac{4}{5+ \frac{6}{7}}}\)
    \begin{enumerate}[label={\Alph*.}]
    \item \(\frac{7}{95}\)
    \item \(\frac{177}{95}\)
    \item \(\frac{233}{151}\)
    \item \(\frac{17}{10}\)
    \item \(\frac{3}{10}\)

    \end{enumerate}
\item Evaluate and correct of 4 decimal place 827.51 \(\times\) 0.015.
    \begin{enumerate}[label={\Alph*.}]
    \item \(124.1265\)
    \item \(8.8415\)
    \item \(12.4127\)
    \item \(12.4120\)
    \item \(124.1265\)

    \end{enumerate}
\item A micrometer is defined as one millionth of a millimeter. A length of of \(12,000\) micrometer may be represented as
    \begin{enumerate}[label={\Alph*.}]
    \item \(0.000012m\)
    \item \(0.12m\)
    \item \(0.00000012m\)
    \item \(0.0000000012m\)
    \item \(0.0000012m\)

    \end{enumerate}
\item The difference between 4\(\frac{5}{7}\) and 2\(\frac{1}{4}\) is greater than the sum of \(\frac{1}{14}\) and 1\(\frac{1}{2}\) by.
    \begin{enumerate}[label={\Alph*.}]
    \item \(\frac{27}{28}\)
    \item \(\frac{23}{28}\)
    \item \(\frac{50}{56}\)
    \item \(\frac{48}{56}\)
    \item \(\frac{24}{48}\)

    \end{enumerate}
\item When a dealer sells a bicycle for N81, he makes a profit of 8\%. What did he pay for the bicycle. 
    \begin{enumerate}[label={\Alph*.}]
    \item N\(75\)
    \item N\(75.52\)
    \item N\(74.52\)
    \item N\(87.48\)
    \item N\(73\)

    \end{enumerate}
\item A man and wife went to by an article costing N\(400\). The woman had 10\% of the cost and the man 40\% of the remainder. How much did they have altogether?
    \begin{enumerate}[label={\Alph*.}]
    \item N\(186\)
    \item N\(184\)
    \item N\(200\)
    \item N\(144\)
    \item N\(100\)

    \end{enumerate}
\item A sum of money invested at 5\% per annum simple interest amount to N285.20 after 3 years. How long will it take the same sum to amount to N434.00 at 7\(\frac{1}{2}\)\% per annum simple interest?
    \begin{enumerate}[label={\Alph*.}]
    \item \(10\) years
    \item \(12\) years
    \item \(7\frac{1}{2}\) years
    \item \(14\) years
    \item \(5\) years

    \end{enumerate}
\item A construction company is owned by two partners \(A\) and \(B\) and it is agreed that their profit will be divided in ratio 4:5, at the end of the year, B recieved N5,000 more than A. What is the total profit of the company for the year? 
    \begin{enumerate}[label={\Alph*.}]
    \item N\(45,000\)
    \item N\(30,000\)
    \item N\(150,000\)
    \item N\(25,000\)
    \item N\(30,000\)

    \end{enumerate}
\item The diameter of metal rod is meased as \(23.40\)cm to 4 significant figures. What isthe maximum error in the measurement?
    \begin{enumerate}[label={\Alph*.}]
    \item \(0.0004\)cm
    \item \(0.05\)cm
    \item \(0.005\)cm
    \item \(0.5\)cm
    \item \(0.45\)cm

    \end{enumerate}
\item Simplify: 3 - \(\frac{2}{\frac{4}{5} + \frac{1}{2}}\)
    \begin{enumerate}[label={\Alph*.}]
    \item \(1\frac{9}{10}\)
    \item \(1\frac{3}{10}\)
    \item \(1\frac{3}{4}\)
    \item \(-1\)
    \item \(1\)

    \end{enumerate}
\item Given that \(x:y\) = \(\frac{1}{3}:\frac{1}{2}\) and \(\psi:\theta\) = \(\frac{2}{5}:\frac{4}{7}\), find \(x:\theta\).
    \begin{enumerate}[label={\Alph*.}]
    \item \(20:21\)
    \item \(7:15\)
    \item \(3:20\)
    \item \(2:35\)
    \item \(4:105\)

    \end{enumerate}
\item If N560 is shared in the ratio \(7:2:1\), what is the smallest share?
    \begin{enumerate}[label={\Alph*.}]
    \item N\(392\)
    \item N\(113.40\)
    \item N\(56.70\)
    \item N\(87.48\)
    \item N\(126.41\)

    \end{enumerate}
\item Simplify: \(\frac{1}{2} + \frac{1}{2 + \frac{1}{2 - \frac{1}{4 + \frac{1}{5}}}}\)
    \begin{enumerate}[label={\Alph*.}]
    \item \(\frac{169}{190}\)
    \item \(-\frac{1}{3}\)
    \item \(\frac{13}{15}\)
    \item \(-\frac{3}{4}\)
    \item \(-\frac{14}{27}\)

    \end{enumerate}
\item 22\(\frac{1}{2}\)\% of the Nigerian Naira equals 17\(\frac{1}{10}\)\% of a foreign currency \(M\). What is the conversion rate of \(M\) to Naira?
    \begin{enumerate}[label={\Alph*.}]
    \item \(2\frac{11}{57}\) Naria
    \item \(1\frac{18}{57}\) Naria
    \item \(\frac{15}{59}\) Naria
    \item \(\frac{15}{57}\) Naria
    \item \(38\frac{1}{4}\) Naria

    \end{enumerate}
\item Divide the LCM of 48,64 and 80 by their HCF.
    \begin{enumerate}[label={\Alph*.}]
    \item \(30\)
    \item \(48\)
    \item \(52\)
    \item \(20\)
    \item \(60\)

    \end{enumerate}
\item A sum of money was invested at 8\% per annum simple interest. If after 4years the money amount s to N330.00, find the amount originally invested.
    \begin{enumerate}[label={\Alph*.}]
    \item N\(150\)
    \item N\(200\)
    \item N\(165\)
    \item N\(180\)
    \item N\(250\)

    \end{enumerate}
\item P sold his bicycle to Q at a profit of 10\%. Q sold to R for N209 at a loss of 5\%. How much did the bicycle cost P? 
    \begin{enumerate}[label={\Alph*.}]
    \item N\(150\)
    \item N\(205\)
    \item N\(180\)
    \item N\(196\)
    \item N\(200\)

    \end{enumerate}
\item Find the smallest number by which 252 can be multiplied to obtain a perfect square. 
    \begin{enumerate}[label={\Alph*.}]
    \item \(2\)
    \item \(3\)
    \item \(5\)
    \item \(7\)
    \item \(9\)

    \end{enumerate}
\item Find the reciprocal of: \(\frac{\frac{2}{3}}{\frac{1}{2} + \frac{1}{3}}\)
    \begin{enumerate}[label={\Alph*.}]
    \item \(\frac{4}{5}\)
    \item \(\frac{2}{5}\)
    \item \(\frac{6}{9}\)
    \item \(\frac{5}{4}\)
    \item \(\frac{3}{4}\)

    \end{enumerate}
\item Three boys shared some oranges, the first recieved \(\frac{1}{3}\) of the of the oranges, the second received \(\frac{2}{3}\) of the remainder, if the third boy
    \begin{enumerate}[label={\Alph*.}]
    \item \(48\)
    \item \(72\)
    \item \(54\)
    \item \(42\)
    \item \(60\)

    \end{enumerate}
\item Udoh deposited N150.00 in the bank. At the end of 5years the simple interest on the principal was N55.00. At what rate per annum was the interest paid 
    \begin{enumerate}[label={\Alph*.}]
    \item \(7\frac{1}{3}\%\)
    \item \(5\%\)
    \item \(11\%\)
    \item \(3\frac{1}{2}\%\)s
    \item \(4\frac{2}{5}\%\)

    \end{enumerate}
\item A number of pencil were shared among Desmond, Florence and Kevin in ratio \(2:3:5\) respectively. If Desmond got 5, how many were shared out?
    \begin{enumerate}[label={\Alph*.}]
    \item \(30\)
    \item \(15\)
    \item \(25\)
    \item \(20\)
    \item \(35\)

    \end{enumerate}
\item Find the least length of a rod which can be cut into exacly equal strips, each of 40cm or 48cm in length.
    \begin{enumerate}[label={\Alph*.}]
    \item \(240\) cm
    \item \(480\) cm
    \item \(360\) cm
    \item \(120\) cm
    \item \(480\) cm
    

    \end{enumerate}
\item A rectangular lawn has an area of 1815 square yards. If its length is 50metres, find it width in meters. Given that 1metres equals 1.1 yards.
    \begin{enumerate}[label={\Alph*.}]
    \item \(30.00\) m
    \item \(33.00\) m
    \item \(32.00\) m
    \item \(39.93\) m
    \item \(36.45\) m

    \end{enumerate}
\item Reduce each number to two significant figure and then evaluate \(\frac{0.021741 \times 1.2047}{0.023789}\)
    \begin{enumerate}[label={\Alph*.}]
    \item \(0.8\)
    \item \(1.2\)
    \item \(1.1\)
    \item \(0.9\)
    \item \(0.6\)
e
    \end{enumerate}
\item A cinema hall contains a certain number of people. if 27\(\frac{1}{2}\)\% are children, 47\(\frac{1}{2}\)\% are men and 84 are 
women, find the number of men in the hall
    \begin{enumerate}[label={\Alph*.}]
    \item \(133\)
    \item \(84\)
    \item \(63\)
    \item \(113\)

    \end{enumerate}
\item A woman buys 270 oranges for N\(1,800\) and sells at 5 for N\(40\). What is her profit?
    \begin{enumerate}[label={\Alph*.}]
    \item N \(1,620\)
    \item N \(630\)
    \item N \(360\)
    \item N \(2,160\)
    

    \end{enumerate}
\item If a car travels 120km on 45 litres of petrol, how much petrol is needed for a journey of 600km?
    \begin{enumerate}[label={\Alph*.}]
    \item \(720\) litres
    \item \(225\) litres
    \item \(960\) litres
    \item \(160\) litres


    \end{enumerate}
\item Simplify 1-(\(\frac{1}{7} \times 3\frac{1}{2}\)) / \(\frac{3}{4}\)
    \begin{enumerate}[label={\Alph*.}]
    \item \(2\)
    \item \(1\)
    \item \(\frac{1}{3}\)
    \item \(\frac{2}{3}\)

    \end{enumerate}
\item Evaluate: \(\frac{12.02 \times 20.06}{26.04 \times 60.06}\), correct to 3 significant figures
    \begin{enumerate}[label={\Alph*.}]
    \item \(0.154\)
    \item \(0.155\)
    \item \(0.158\)
    \item \(0.157\)

    \end{enumerate}
\item Evaluate: \(\frac{0.8 \times 0.43 \times 0.031}{0.05 \times 0.72 \times 0.021}\), correct to 3 significant figures
    \begin{enumerate}[label={\Alph*.}]
    \item \(14.1\)
    \item \(14.09\)
    \item \(14.12\)
    \item \(14.11\)

    \end{enumerate}
\item A man bought a car for N\(500, 000\) and was able to sell it for N\(350, 000\), what was his percentage loss?
    \begin{enumerate}[label={\Alph*.}]
    \item \(50\%\)
    \item \(30\%\)
    \item \(70\%\)
    \item \(60\%\)
    

    \end{enumerate}
\item Simplify: 1\(\frac{2}{3}\) + 4\(\frac{1}{4}\) + 1\(\frac{5}{12}\)
    \begin{enumerate}[label={\Alph*.}]
    \item \(4\frac{1}{3}\)
    \item \(4\frac{2}{3}\)
    \item \(4\frac{12}{17}\)
    \item \(4\frac{3}{17}\)

    \end{enumerate}
\item A man donates 16\% of his monthly net earning to the church, if it amounts to N\(4,500\), 
what is his monthly income? 
    \begin{enumerate}[label={\Alph*.}]
    \item N\(40,500\)
    \item N\(52,000\)
    \item N\(52,500\)
    \item N\(45,000\)
    

    \end{enumerate}
\item If a student measured the lengt of a table to be 2.30m insted of 2.50m. 
What was his percentage error in measuring the length?
    \begin{enumerate}[label={\Alph*.}]
    \item \(7\)\%
    \item \(10\)\%
    \item \(9\)\%
    \item \(8\)\%
    

    \end{enumerate}
\item A man bought a second-hand photocopy machine for \(34,000\). He serviced 
it at a cost of N\(2,000\) and then sold it at a profit of 15\%. 
what was the selling price? 
    \begin{enumerate}[label={\Alph*.}]
    \item \(37,550\)
    \item \(40,000\)
    \item \(41,400\)
    \item \(42,400\)

    \end{enumerate}
\item A student spent \(\frac{1}{5}\) of his allowance on books, 
\(\frac{1}{3}\) of the remainder on food and kept the rest for contingencies. 
What fraction was kept?
    \begin{enumerate}[label={\Alph*.}]
    \item \(\frac{8}{15}\)
    \item \(\frac{4}{5}\)
    \item \(\frac{2}{3}\)
    \item \(\frac{7}{15}\)

    \end{enumerate}
\item if \(p:q\) = \(\frac{2}{3}:\frac{5}{6}\) and \(\frac{3}{4}:\frac{1}{2}\),
find \(p:q:r\)
    \begin{enumerate}[label={\Alph*.}]
    \item \(12:15:10\)
    \item \(10:15:24\)
    \item \(9:10:15\)
    \item \(12:15:16\)

    \end{enumerate}
\item Simplify: \(\frac{3\frac{2}{3}\times\frac{5}{6}\times\frac{2}{3}}{\frac{11}{25}\times\frac{3}{4}\times\frac{2}{27}}\)
    \begin{enumerate}[label={\Alph*.}]
    \item \(4\frac{1}{3}\)
    \item \(30\)
    \item \(5\frac{2}{3}\)
    \item \(50\)

    \end{enumerate}
\item A man earns N\(3,500\) per month out of which he spend 15\% on his children's education. If he spends additional N\(1,950\) on food, how much does he have left? 
    \begin{enumerate}[label={\Alph*.}]
    \item N\(2,975\)
    \item N\(1,950\)
    \item N\(525\)
    \item N\(1025\)
    

    \end{enumerate}
\item Evaluate \(\frac{21}{9}\) to 3 significant figures
    \begin{enumerate}[label={\Alph*.}]
    \item \(2.30\)
    \item \(2.31\)
    \item \(2.32\)
    \item \(2.33\)

    \end{enumerate}
\item A girl share a number of apples in the ratio 5:3:2. if the highest share is 40. find the smallest share.
    \begin{enumerate}[label={\Alph*.}]
    \item \(74\)
    \item \(38\)
    \item \(36\)
    \item \(16\)

    \end{enumerate}
\item Calculate the time taken for N\(3,000\) to earn N600 at 8\% simple interest.
    \begin{enumerate}[label={\Alph*.}]
    \item \(3\) years
    \item \(2\frac{1}{2}\) years
    \item \(1\frac{1}{2}\) years
    \item \(3\frac{1}{2}\) years
    

    \end{enumerate}
\item Find the tax on an income of N\(20,000\) if no tax is paid on the first N\(10,000\) and tax is paid at N50 and in N\(1,000\)
on the next N\(5,000\) and at N\(55\) and N\(1000\) on the remainder. 
    \begin{enumerate}[label={\Alph*.}]
    \item N\(225\)
    \item N\(525\)
    \item N\(552\)
    \item N\(500\)
    

    \end{enumerate}
\item The time taken to do a piece of work is inversely proportional to the number of men employed. 
if it takes 30 men to do a piece of work in 6days, how many men are required to do the work in 4 days?
    \begin{enumerate}[label={\Alph*.}]
    \item \(35\)
    \item \(45\)
    \item \(25\)
    \item \(60\)

    \end{enumerate}
\item Three boys shared oranges. The first received \(\frac{1}{3}\) of the oranges and the second received \(\frac{2}{3}\)
of the remainder. If the third boy received the remaining 12 oranges, how much oranges did they share?
    \begin{enumerate}[label={\Alph*.}]
    \item \(42\)
    \item \(60\)
    \item \(54\)
    \item \(48\)

    \end{enumerate}
\item A farmer planted \(5,000\) grains of maize and harves 5,000 cobs, each bearing 500 grains. What is the ratio of hte number of grains sowed 
to the number harvested? 
    \begin{enumerate}[label={\Alph*.}]
    \item \(1:5,000\)
    \item \(1:25,000\)
    \item \(1:500\)
    \item \(1:250,000\)

    \end{enumerate}
\item Evaluate: \(\frac{0.21 \times 0.072 \times 0.00054}{0.006 \times 1.68 \times 0.063}\)
    \begin{enumerate}[label={\Alph*.}]
    \item \(0.1286\)
    \item \(0.01285\)
    \item \(0.01286\)
    \item \(0.1285\)

    \end{enumerate}
\item A man's initial salary is N\(540\) a month and increases after a period of six months by N36 a
month. Find his salary in the eight month of the third year. 
    \begin{enumerate}[label={\Alph*.}]
    \item \(828\)
    \item \(756\)
    \item \(720\)
    \item \(684\)

    \end{enumerate}
\item Find correct to 3 decimal places: (\(\frac{1}{0.05}\))/(\(\frac{1}{5.005}\)) -(0.05 \(\times\) 2.05)
    \begin{enumerate}[label={\Alph*.}]
    \item \(\)
    \item \(\)
    \item \(\)
    \item \(\)

    \end{enumerate}
\item 
    \begin{enumerate}[label={\Alph*.}]
    \item \(\)
    \item \(\)
    \item \(\)
    \item \(\)

    \end{enumerate}
\item 
    \begin{enumerate}[label={\Alph*.}]
    \item \(\)
    \item \(\)
    \item \(\)
    \item \(\)

    \end{enumerate}
\item 
    \begin{enumerate}[label={\Alph*.}]
    \item \(\)
    \item \(\)
    \item \(\)
    \item \(\)

    \end{enumerate}
\item 
    \begin{enumerate}[label={\Alph*.}]
    \item \(\)
    \item \(\)
    \item \(\)
    \item \(\)

    \end{enumerate}
\item 
    \begin{enumerate}[label={\Alph*.}]
    \item \(\)
    \item \(\)
    \item \(\)
    \item \(\)

    \end{enumerate}
\item 
    \begin{enumerate}[label={\Alph*.}]
    \item \(\)
    \item \(\)
    \item \(\)
    \item \(\)

    \end{enumerate}
\item 
    \begin{enumerate}[label={\Alph*.}]
    \item \(\)
    \item \(\)
    \item \(\)
    \item \(\)

    \end{enumerate}
\item 
    \begin{enumerate}[label={\Alph*.}]
    \item \(\)
    \item \(\)
    \item \(\)
    \item \(\)

    \end{enumerate}
\item 
    \begin{enumerate}[label={\Alph*.}]
    \item \(\)
    \item \(\)
    \item \(\)
    \item \(\)

    \end{enumerate}
\item 
    \begin{enumerate}[label={\Alph*.}]
    \item \(\)
    \item \(\)
    \item \(\)
    \item \(\)

    \end{enumerate}
\item 
    \begin{enumerate}[label={\Alph*.}]
    \item \(\)
    \item \(\)
    \item \(\)
    \item \(\)

    \end{enumerate}
\item 
    \begin{enumerate}[label={\Alph*.}]
    \item \(\)
    \item \(\)
    \item \(\)
    \item \(\)

    \end{enumerate}
\item 
    \begin{enumerate}[label={\Alph*.}]
    \item \(\)
    \item \(\)
    \item \(\)
    \item \(\)

    \end{enumerate}
\item 
    \begin{enumerate}[label={\Alph*.}]
    \item \(\)
    \item \(\)
    \item \(\)
    \item \(\)

    \end{enumerate}
\item 
    \begin{enumerate}[label={\Alph*.}]
    \item \(\)
    \item \(\)
    \item \(\)
    \item \(\)

    \end{enumerate}
\item 
    \begin{enumerate}[label={\Alph*.}]
    \item \(\)
    \item \(\)
    \item \(\)
    \item \(\)

    \end{enumerate}
\item 
    \begin{enumerate}[label={\Alph*.}]
    \item \(\)
    \item \(\)
    \item \(\)
    \item \(\)

    \end{enumerate}
\item 
    \begin{enumerate}[label={\Alph*.}]
    \item \(\)
    \item \(\)
    \item \(\)
    \item \(\)

    \end{enumerate}
\item 
    \begin{enumerate}[label={\Alph*.}]
    \item \(\)
    \item \(\)
    \item \(\)
    \item \(\)

    \end{enumerate}
\item 
    \begin{enumerate}[label={\Alph*.}]
    \item \(\)
    \item \(\)
    \item \(\)
    \item \(\)

    \end{enumerate}
\item 
    \begin{enumerate}[label={\Alph*.}]
    \item \(\)
    \item \(\)
    \item \(\)
    \item \(\)

    \end{enumerate}
\item 
    \begin{enumerate}[label={\Alph*.}]
    \item \(\)
    \item \(\)
    \item \(\)
    \item \(\)

    \end{enumerate}
\item 
    \begin{enumerate}[label={\Alph*.}]
    \item \(\)
    \item \(\)
    \item \(\)
    \item \(\)

    \end{enumerate}
\item 
    \begin{enumerate}[label={\Alph*.}]
    \item \(\)
    \item \(\)
    \item \(\)
    \item \(\)

    \end{enumerate}
\item 
    \begin{enumerate}[label={\Alph*.}]
    \item \(\)
    \item \(\)
    \item \(\)
    \item \(\)

    \end{enumerate}
\item 
    \begin{enumerate}[label={\Alph*.}]
    \item \(\)
    \item \(\)
    \item \(\)
    \item \(\)

    \end{enumerate}
\item 
    \begin{enumerate}[label={\Alph*.}]
    \item \(\)
    \item \(\)
    \item \(\)
    \item \(\)

    \end{enumerate}
\item 
    \begin{enumerate}[label={\Alph*.}]
    \item \(\)
    \item \(\)
    \item \(\)
    \item \(\)

    \end{enumerate}
\item 
    \begin{enumerate}[label={\Alph*.}]
    \item \(\)
    \item \(\)
    \item \(\)
    \item \(\)

    \end{enumerate}
\item 
    \begin{enumerate}[label={\Alph*.}]
    \item \(\)
    \item \(\)
    \item \(\)
    \item \(\)

    \end{enumerate}
\item 
    \begin{enumerate}[label={\Alph*.}]
    \item \(\)
    \item \(\)
    \item \(\)
    \item \(\)

    \end{enumerate}
\item 
    \begin{enumerate}[label={\Alph*.}]
    \item \(\)
    \item \(\)
    \item \(\)
    \item \(\)

    \end{enumerate}
\item 
    \begin{enumerate}[label={\Alph*.}]
    \item \(\)
    \item \(\)
    \item \(\)
    \item \(\)

    \end{enumerate}
\item 
    \begin{enumerate}[label={\Alph*.}]
    \item \(\)
    \item \(\)
    \item \(\)
    \item \(\)

    \end{enumerate}
\item 
    \begin{enumerate}[label={\Alph*.}]
    \item \(\)
    \item \(\)
    \item \(\)
    \item \(\)

    \end{enumerate}
\item 
    \begin{enumerate}[label={\Alph*.}]
    \item \(\)
    \item \(\)
    \item \(\)
    \item \(\)

    \end{enumerate}
\item 
    \begin{enumerate}[label={\Alph*.}]
    \item \(\)
    \item \(\)
    \item \(\)
    \item \(\)

    \end{enumerate}
\item 
    \begin{enumerate}[label={\Alph*.}]
    \item \(\)
    \item \(\)
    \item \(\)
    \item \(\)

    \end{enumerate}


\end{enumerate}
\end{multicols}
\chapter{Algbera}
\section{}
\subsection{Questions}
\begin{enumerate}[label={\arabic*.}]
\item
	\begin{enumerate}[label={\Alph*.}]
	\item \(\)
	\item \(\)
	\item \(\)
	\item \(\)
	\end{enumerate}
\item
	\begin{enumerate}[label={\Alph*.}]
	\item \(\)
	\item \(\)
	\item \(\)
	\item \(\)
	\end{enumerate}
\item
	\begin{enumerate}[label={\Alph*.}]
	\item \(\)
	\item \(\)
	\item \(\)
	\item \(\)
	\end{enumerate}
\item
	\begin{enumerate}[label={\Alph*.}]
	\item \(\)
	\item \(\)
	\item \(\)
	\item \(\)
	\end{enumerate}
\item
	\begin{enumerate}[label={\Alph*.}]
	\item \(\)
	\item \(\)
	\item \(\)
	\item \(\)
	\end{enumerate}
\item
	\begin{enumerate}[label={\Alph*.}]
	\item \(\)
	\item \(\)
	\item \(\)
	\item \(\)
	\end{enumerate}
\item
	\begin{enumerate}[label={\Alph*.}]
	\item \(\)
	\item \(\)
	\item \(\)
	\item \(\)
	\end{enumerate}
\item
	\begin{enumerate}[label={\Alph*.}]
	\item \(\)
	\item \(\)
	\item \(\)
	\item \(\)
	\end{enumerate}
\item
	\begin{enumerate}[label={\Alph*.}]
	\item \(\)
	\item \(\)
	\item \(\)
	\item \(\)
	\end{enumerate}
\item
	\begin{enumerate}[label={\Alph*.}]
	\item \(\)
	\item \(\)
	\item \(\)
	\item \(\)
	\end{enumerate}
\item
	\begin{enumerate}[label={\Alph*.}]
	\item \(\)
	\item \(\)
	\item \(\)
	\item \(\)
	\end{enumerate}


\item
	\begin{enumerate}[label={\Alph*.}]
	\item \(\)
	\item \(\)
	\item \(\)
	\item \(\)
	\end{enumerate}
\item
	\begin{enumerate}[label={\Alph*.}]
	\item \(\)
	\item \(\)
	\item \(\)
	\item \(\)
	\end{enumerate}
\item
	\begin{enumerate}[label={\Alph*.}]
	\item \(\)
	\item \(\)
	\item \(\)
	\item \(\)
	\end{enumerate}
\item
	\begin{enumerate}[label={\Alph*.}]
	\item \(\)
	\item \(\)
	\item \(\)
	\item \(\)
	\end{enumerate}
\item
	\begin{enumerate}[label={\Alph*.}]
	\item \(\)
	\item \(\)
	\item \(\)
	\item \(\)
	\end{enumerate}
\item
	\begin{enumerate}[label={\Alph*.}]
	\item \(\)
	\item \(\)
	\item \(\)
	\item \(\)
	\end{enumerate}
\item
	\begin{enumerate}[label={\Alph*.}]
	\item \(\)
	\item \(\)
	\item \(\)
	\item \(\)
	\end{enumerate}
\item
	\begin{enumerate}[label={\Alph*.}]
	\item \(\)
	\item \(\)
	\item \(\)
	\item \(\)
	\end{enumerate}
\item
	\begin{enumerate}[label={\Alph*.}]
	\item \(\)
	\item \(\)
	\item \(\)
	\item \(\)
	\end{enumerate}
\item
	\begin{enumerate}[label={\Alph*.}]
	\item \(\)
	\item \(\)
	\item \(\)
	\item \(\)
	\end{enumerate}
\item
	\begin{enumerate}[label={\Alph*.}]
	\item \(\)
	\item \(\)
	\item \(\)
	\item \(\)
	\end{enumerate}
\item
	\begin{enumerate}[label={\Alph*.}]
	\item \(\)
	\item \(\)
	\item \(\)
	\item \(\)
	\end{enumerate}
\item
	\begin{enumerate}[label={\Alph*.}]
	\item \(\)
	\item \(\)
	\item \(\)
	\item \(\)
	\end{enumerate}
\item
	\begin{enumerate}[label={\Alph*.}]
	\item \(\)
	\item \(\)
	\item \(\)
	\item \(\)
	\end{enumerate}
\item
	\begin{enumerate}[label={\Alph*.}]
	\item \(\)
	\item \(\)
	\item \(\)
	\item \(\)
	\end{enumerate}
\item
	\begin{enumerate}[label={\Alph*.}]
	\item \(\)
	\item \(\)
	\item \(\)
	\item \(\)
	\end{enumerate}
\item
	\begin{enumerate}[label={\Alph*.}]
	\item \(\)
	\item \(\)
	\item \(\)
	\item \(\)
	\end{enumerate}
\item
	\begin{enumerate}[label={\Alph*.}]
	\item \(\)
	\item \(\)
	\item \(\)
	\item \(\)
	\end{enumerate}
\item
	\begin{enumerate}[label={\Alph*.}]
	\item \(\)
	\item \(\)
	\item \(\)
	\item \(\)
	\end{enumerate}
\item
	\begin{enumerate}[label={\Alph*.}]
	\item \(\)
	\item \(\)
	\item \(\)
	\item \(\)
	\end{enumerate}
\item
	\begin{enumerate}[label={\Alph*.}]
	\item \(\)
	\item \(\)
	\item \(\)
	\item \(\)
	\end{enumerate}
\item
	\begin{enumerate}[label={\Alph*.}]
	\item \(\)
	\item \(\)
	\item \(\)
	\item \(\)
	\end{enumerate}
\item
	\begin{enumerate}[label={\Alph*.}]
	\item \(\)
	\item \(\)
	\item \(\)
	\item \(\)
	\end{enumerate}
\item
	\begin{enumerate}[label={\Alph*.}]
	\item \(\)
	\item \(\)
	\item \(\)
	\item \(\)
	\end{enumerate}
\item
	\begin{enumerate}[label={\Alph*.}]
	\item \(\)
	\item \(\)
	\item \(\)
	\item \(\)
	\end{enumerate}
\item
	\begin{enumerate}[label={\Alph*.}]
	\item \(\)
	\item \(\)
	\item \(\)
	\item \(\)
	\end{enumerate}
\item
	\begin{enumerate}[label={\Alph*.}]
	\item \(\)
	\item \(\)
	\item \(\)
	\item \(\)
	\end{enumerate}
\item
	\begin{enumerate}[label={\Alph*.}]
	\item \(\)
	\item \(\)
	\item \(\)
	\item \(\)
	\end{enumerate}
\item
	\begin{enumerate}[label={\Alph*.}]
	\item \(\)
	\item \(\)
	\item \(\)
	\item \(\)
	\end{enumerate}
\item
	\begin{enumerate}[label={\Alph*.}]
	\item \(\)
	\item \(\)
	\item \(\)
	\item \(\)
	\end{enumerate}
\item
	\begin{enumerate}[label={\Alph*.}]
	\item \(\)
	\item \(\)
	\item \(\)
	\item \(\)
	\end{enumerate}
\item
	\begin{enumerate}[label={\Alph*.}]
	\item \(\)
	\item \(\)
	\item \(\)
	\item \(\)
	\end{enumerate}
\item
	\begin{enumerate}[label={\Alph*.}]
	\item \(\)
	\item \(\)
	\item \(\)
	\item \(\)
	\end{enumerate}
\item
	\begin{enumerate}[label={\Alph*.}]
	\item \(\)
	\item \(\)
	\item \(\)
	\item \(\)
	\end{enumerate}
\item
	\begin{enumerate}[label={\Alph*.}]
	\item \(\)
	\item \(\)
	\item \(\)
	\item \(\)
	\end{enumerate}
\item
	\begin{enumerate}[label={\Alph*.}]
	\item \(\)
	\item \(\)
	\item \(\)
	\item \(\)
	\end{enumerate}
\item
	\begin{enumerate}[label={\Alph*.}]
	\item \(\)
	\item \(\)
	\item \(\)
	\item \(\)
	\end{enumerate}
\item
	\begin{enumerate}[label={\Alph*.}]
	\item \(\)
	\item \(\)
	\item \(\)
	\item \(\)
	\end{enumerate}
\item
	\begin{enumerate}[label={\Alph*.}]
	\item \(\)
	\item \(\)
	\item \(\)
	\item \(\)
	\end{enumerate}
\item
	\begin{enumerate}[label={\Alph*.}]
	\item \(\)
	\item \(\)
	\item \(\)
	\item \(\)
	\end{enumerate}
\item
	\begin{enumerate}[label={\Alph*.}]
	\item \(\)
	\item \(\)
	\item \(\)
	\item \(\)
	\end{enumerate}
\item
	\begin{enumerate}[label={\Alph*.}]
	\item \(\)
	\item \(\)
	\item \(\)
	\item \(\)
	\end{enumerate}
\item
	\begin{enumerate}[label={\Alph*.}]
	\item \(\)
	\item \(\)
	\item \(\)
	\item \(\)
	\end{enumerate}
\item
	\begin{enumerate}[label={\Alph*.}]
	\item \(\)
	\item \(\)
	\item \(\)
	\item \(\)
	\end{enumerate}
\item
	\begin{enumerate}[label={\Alph*.}]
	\item \(\)
	\item \(\)
	\item \(\)
	\item \(\)
	\end{enumerate}
\item
	\begin{enumerate}[label={\Alph*.}]
	\item \(\)
	\item \(\)
	\item \(\)
	\item \(\)
	\end{enumerate}
\item
	\begin{enumerate}[label={\Alph*.}]
	\item \(\)
	\item \(\)
	\item \(\)
	\item \(\)
	\end{enumerate}
\item
	\begin{enumerate}[label={\Alph*.}]
	\item \(\)
	\item \(\)
	\item \(\)
	\item \(\)
	\end{enumerate}
\item
	\begin{enumerate}[label={\Alph*.}]
	\item \(\)
	\item \(\)
	\item \(\)
	\item \(\)
	\end{enumerate}
\item
	\begin{enumerate}[label={\Alph*.}]
	\item \(\)
	\item \(\)
	\item \(\)
	\item \(\)
	\end{enumerate}
\item
	\begin{enumerate}[label={\Alph*.}]
	\item \(\)
	\item \(\)
	\item \(\)
	\item \(\)
	\end{enumerate}
\item
	\begin{enumerate}[label={\Alph*.}]
	\item \(\)
	\item \(\)
	\item \(\)
	\item \(\)
	\end{enumerate}
\item
	\begin{enumerate}[label={\Alph*.}]
	\item \(\)
	\item \(\)
	\item \(\)
	\item \(\)
	\end{enumerate}
\item
	\begin{enumerate}[label={\Alph*.}]
	\item \(\)
	\item \(\)
	\item \(\)
	\item \(\)
	\end{enumerate}
\item
	\begin{enumerate}[label={\Alph*.}]
	\item \(\)
	\item \(\)
	\item \(\)
	\item \(\)
	\end{enumerate}
\item
	\begin{enumerate}[label={\Alph*.}]
	\item \(\)
	\item \(\)
	\item \(\)
	\item \(\)
	\end{enumerate}
\item
	\begin{enumerate}[label={\Alph*.}]
	\item \(\)
	\item \(\)
	\item \(\)
	\item \(\)
	\end{enumerate}
\item
	\begin{enumerate}[label={\Alph*.}]
	\item \(\)
	\item \(\)
	\item \(\)
	\item \(\)
	\end{enumerate}
\item
	\begin{enumerate}[label={\Alph*.}]
	\item \(\)
	\item \(\)
	\item \(\)
	\item \(\)
	\end{enumerate}
\item
	\begin{enumerate}[label={\Alph*.}]
	\item \(\)
	\item \(\)
	\item \(\)
	\item \(\)
	\end{enumerate}
\item
	\begin{enumerate}[label={\Alph*.}]
	\item \(\)
	\item \(\)
	\item \(\)
	\item \(\)
	\end{enumerate}
\item
	\begin{enumerate}[label={\Alph*.}]
	\item \(\)
	\item \(\)
	\item \(\)
	\item \(\)
	\end{enumerate}
\item
	\begin{enumerate}[label={\Alph*.}]
	\item \(\)
	\item \(\)
	\item \(\)
	\item \(\)
	\end{enumerate}
\item
	\begin{enumerate}[label={\Alph*.}]
	\item \(\)
	\item \(\)
	\item \(\)
	\item \(\)
	\end{enumerate}


\item
	\begin{enumerate}[label={\Alph*.}]
	\item \(\)
	\item \(\)
	\item \(\)
	\item \(\)
	\end{enumerate}
\item
	\begin{enumerate}[label={\Alph*.}]
	\item \(\)
	\item \(\)
	\item \(\)
	\item \(\)
	\end{enumerate}
\item
	\begin{enumerate}[label={\Alph*.}]
	\item \(\)
	\item \(\)
	\item \(\)
	\item \(\)
	\end{enumerate}
\item
	\begin{enumerate}[label={\Alph*.}]
	\item \(\)
	\item \(\)
	\item \(\)
	\item \(\)
	\end{enumerate}
\item
	\begin{enumerate}[label={\Alph*.}]
	\item \(\)
	\item \(\)
	\item \(\)
	\item \(\)
	\end{enumerate}
\item
	\begin{enumerate}[label={\Alph*.}]
	\item \(\)
	\item \(\)
	\item \(\)
	\item \(\)
	\end{enumerate}
\item
	\begin{enumerate}[label={\Alph*.}]
	\item \(\)
	\item \(\)
	\item \(\)
	\item \(\)
	\end{enumerate}
\item
	\begin{enumerate}[label={\Alph*.}]
	\item \(\)
	\item \(\)
	\item \(\)
	\item \(\)
	\end{enumerate}
\item
	\begin{enumerate}[label={\Alph*.}]
	\item \(\)
	\item \(\)
	\item \(\)
	\item \(\)
	\end{enumerate}
\item
	\begin{enumerate}[label={\Alph*.}]
	\item \(\)
	\item \(\)
	\item \(\)
	\item \(\)
	\end{enumerate}
\item
	\begin{enumerate}[label={\Alph*.}]
	\item \(\)
	\item \(\)
	\item \(\)
	\item \(\)
	\end{enumerate}
\item
	\begin{enumerate}[label={\Alph*.}]
	\item \(\)
	\item \(\)
	\item \(\)
	\item \(\)
	\end{enumerate}
\item
	\begin{enumerate}[label={\Alph*.}]
	\item \(\)
	\item \(\)
	\item \(\)
	\item \(\)
	\end{enumerate}
\item
	\begin{enumerate}[label={\Alph*.}]
	\item \(\)
	\item \(\)
	\item \(\)
	\item \(\)
	\end{enumerate}
\item
	\begin{enumerate}[label={\Alph*.}]
	\item \(\)
	\item \(\)
	\item \(\)
	\item \(\)
	\end{enumerate}
\item
	\begin{enumerate}[label={\Alph*.}]
	\item \(\)
	\item \(\)
	\item \(\)
	\item \(\)
	\end{enumerate}
\item
	\begin{enumerate}[label={\Alph*.}]
	\item \(\)
	\item \(\)
	\item \(\)
	\item \(\)
	\end{enumerate}
\item
	\begin{enumerate}[label={\Alph*.}]
	\item \(\)
	\item \(\)
	\item \(\)
	\item \(\)
	\end{enumerate}
\item
	\begin{enumerate}[label={\Alph*.}]
	\item \(\)
	\item \(\)
	\item \(\)
	\item \(\)
	\end{enumerate}
\item
	\begin{enumerate}[label={\Alph*.}]
	\item \(\)
	\item \(\)
	\item \(\)
	\item \(\)
	\end{enumerate}
\item
	\begin{enumerate}[label={\Alph*.}]
	\item \(\)
	\item \(\)
	\item \(\)
	\item \(\)
	\end{enumerate}
\item
	\begin{enumerate}[label={\Alph*.}]
	\item \(\)
	\item \(\)
	\item \(\)
	\item \(\)
	\end{enumerate}
\item
	\begin{enumerate}[label={\Alph*.}]
	\item \(\)
	\item \(\)
	\item \(\)
	\item \(\)
	\end{enumerate}
\item
	\begin{enumerate}[label={\Alph*.}]
	\item \(\)
	\item \(\)
	\item \(\)
	\item \(\)
	\end{enumerate}
\item
	\begin{enumerate}[label={\Alph*.}]
	\item \(\)
	\item \(\)
	\item \(\)
	\item \(\)
	\end{enumerate}
\end{enumerate}

\section{Indices and Standard Form}
\subsection{Questions}
\begin{multicols}{2}
\begin{enumerate}[label={\arabic*.}] 
\item If $(25)^{x-1} = 64\left(\cfrac{5}{2}\right)^6$, then $x$ has the value
	\begin{enumerate}[label={\Alph*.}]
	\item \(7\)
	\item \(4\)
	\item \(32\)
	\item \(5\)
	\item \(64\)
	\end{enumerate}
\item Simplify \(\cfrac{5^x \times 25^{x -1}}{125^{x+1}}\)
	\begin{enumerate}[label={\Alph*.}]
	\item \(5^{x+2}\)
	\item \(5^{2x -1}\)
	\item \(5^{x+1}\)
	\item \(5^3\)
	\item \(5^{-5}\)
	\end{enumerate}
\item Express $37.05 \times 0.0042$ in standard form 
	\begin{enumerate}[label={\Alph*.}]
	\item \(15.561 \times 10^2\)
	\item \(1.556 \times 10^1\)
	\item \(1.5561 \times 10^{-4}\)
	\item \(1.5561 \times 10^{-1}\)
	\item \(1.5561 \times 10^2\)
	\end{enumerate}
\item Simplify: $\sqrt[3]{(64r^{-6})^\frac{1}{2}}$
	\begin{enumerate}[label={\Alph*.}]
	\item \(\cfrac{1}{2r}\)
	\item \(\cfrac{2}{r}\)
	\item \(2\)
	\item \(\cfrac{1}{2}\)
	\end{enumerate}
\item What are the values of $y$ that satisfy this equation:
\[9^y -4(3^y) + 3 = 0\]
	\begin{enumerate}[label={\Alph*.}]
	\item \(-1 \text{ and } 0\)
	\item \(1 \text{ and } 3\)
	\item \(0 \text{ and } 1\)
	\item \(-1 \text{ and } 1\)
	\end{enumerate}
\item Simplify:  $\cfrac{9^{\frac{1}{3}} \times 27^{-\frac{1}{2}}}{3^{-\frac{1}{6}} \times 3^{-\frac{2}{3}}}$
	\begin{enumerate}[label={\Alph*.}]
	\item \(\cfrac{1}{3}\)
	\item \(\cfrac{1}{9}\)
	\item \(3\)
	\item \(1\)
	\item \(9\)
	\end{enumerate}
\item If \(\sqrt{3^x} = \sqrt[3]{9}\)
	\begin{enumerate}[label={\Alph*.}]
	\item \(\cfrac{3}{4}\)
	\item \(\cfrac{4}{3}\)
	\item \(\cfrac{1}{3}\)
	\item \(\cfrac{2}{3}\)
	\item \(\cfrac{1}{2}\)
	\end{enumerate}
\item Find the value of $\left(4^{\frac{1}{2}}\right)^6$
	\begin{enumerate}[label={\Alph*.}]
	\item \(6\)
	\item \(2\)
	\item \(1\)
	\item \(4\)
	\item \(8\)
	\end{enumerate}
\item Simplify: $\cfrac{3(2^{n-1}) - 4(2^{n-1})}{2^{n+1} - 2^n}$
	\begin{enumerate}[label={\Alph*.}]
	\item \(-2^{n-1}\)
	\item \(2^{n+1}\)
	\item \(-2^1\)
	\item \(-2^{-1}\)
	\end{enumerate}
\item Evaluate: $\cfrac{27^{\frac{1}{3}} - 8^{\frac{2}{3}}}{16^{\frac{2}{4}} \times 2}$ 
	\begin{enumerate}[label={\Alph*.}]
	\item \(-\cfrac{1}{8}\)
	\item \(\cfrac{21}{7}\)
	\item \(\cfrac{23}{5}\)
	\item \(-\cfrac{23}{5}\)
	\item \(-\cfrac{23}{6}\)
	\end{enumerate}
\item If \(\cfrac{4^{x+3}}{16^{2x -3}} = 1\), find $x$
	\begin{enumerate}[label={\Alph*.}]
	\item \(1\)
	\item \(-1\)
	\item \(-3\)
	\item \(3\)
	\item \(-3\)
	\end{enumerate}
\item Evaluate without using tables: $(0.008)^{-\frac{1}{3}} \times (0.16)^{-\frac{3}{2}}$
	\begin{enumerate}[label={\Alph*.}]
	\item \(\cfrac{8}{625}\)
	\item \(8\)
	\item \(\cfrac{625}{8}\)
	\item \(\cfrac{1}{8}\)
	\end{enumerate}
\item Simplify: $\cfrac{3^n - 3^{n-1}}{3^3 \times 3^n - 27 \times 3^{n-1}}$
	\begin{enumerate}[label={\Alph*.}]
	\item \(0\)
	\item \(\cfrac{1}{27}\)
	\item \(3^n - 3^{n-1}\)
	\item \(1\)
	\item \(\cfrac{2}{27}\)
	\end{enumerate}
\item Evaluate and leave your answer in standard form: 
\[\sqrt{\cfrac{0.0048 \times 0.81 \times 10^{-7}}{0.027 \times 0.04 \times 10^6}}\]
	\begin{enumerate}[label={\Alph*.}]
	\item \(6 \times 10^{-14}\)
	\item \(6 \times 10^{-7}\)
	\item \(6 \times 10^7\)
	\item \(6 \times 10^{14}\)
	\end{enumerate}
\item If $3^{2y} - 6(3^y) = 27$, find $y$
	\begin{enumerate}[label={\Alph*.}]
	\item \(3\)
	\item \(-1\)
	\item \(2\)
	\item \(-3\)
	\item \(1\)
	\end{enumerate}
\item If it is given that $5^{x+1} + 5^x = 150$, then the value of $x$ is equal to
	\begin{enumerate}[label={\Alph*.}]
	\item \(2\)
	\item \(3\)
	\item \(\cfrac{1}{2}\)
	\item \(1\)
	\item \(4\)
	\end{enumerate}
\item Given that $10^{2n + 1} = 0.0000001$, find $n$
	\begin{enumerate}[label={\Alph*.}]
	\item \(-7\)
	\item \(-6\)
	\item \(-\cfrac{3}{4}\)
	\item \(4\)
	\item \(-4\)
	\end{enumerate}
\item The result of dividing $\left(\cfrac{x^a}{x^b}\right)^{a-b} \text{ by } \left(\cfrac{x^{a+b}}{x^{a-b}}\right)^{\frac{a^2}{b}}$
	\begin{enumerate}[label={\Alph*.}]
	\item \(\)
	\item \(\)
	\item \(\)
	\item \(\)
	\end{enumerate}
\item If $\sqrt[3]{81} = 3^x$, find the value of $x$
	\begin{enumerate}[label={\Alph*.}]
	\item \(\cfrac{4}{3}\)
	\item \(-\cfrac{4}{3}\)
	\item \(\cfrac{3}{4}\)
	\item \(-\cfrac{3}{4}\)
	\end{enumerate}
\item Simplify: $\cfrac{x(x+1)^{-\frac{1}{2}} -(x+1)^{\frac{1}{2}}}{(x+1)^{\frac{1}{2}}} $
	\begin{enumerate}[label={\Alph*.}]
	\item \(\cfrac{1}{x+1}\)
	\item \(-\cfrac{1}{x+1}\)
	\item \(\cfrac{1}{x}\)
	\item \(-\cfrac{1}{\sqrt{x+1}}\)
	\end{enumerate}
\item Express in standard form \, $$\cfrac{0.8 \times 0.8 \times 0.8 - 0.5 \times 0.5 \times 0.5}{0.8 \times 0.8 + 0.8 \times 0.5 + 0.5 \times 0.5}$$
	\begin{enumerate}[label={\Alph*.}]
	\item \(8 \times 10^{-1}\)
	\item \(4 \times 10^{-1}\)
	\item \(3 \times 10^{-1}\)
	\item \(1.3 \times 10^{-1}\)
	\end{enumerate}
\item Express in standard form $$\cfrac{69842 \times 69842 - 30158 \times 30158}{69842 - 30158}$$
	\begin{enumerate}[label={\Alph*.}]
	\item \(3.0158 \times 10^{-4}\)
	\item \(10^{-4}\)
	\item \(6.9842 \times 10^{-5}\)
	\item \(10^{-5}\)
	\item \(10^5\)
	\end{enumerate}
\item The value of $\cfrac{9^2 \times 18^4}{3^{16}}$ is:
	\begin{enumerate}[label={\Alph*.}]
	\item \(\cfrac{2}{3}\)
	\item \(\cfrac{4}{9}\)
	\item \(\cfrac{32}{243}\)
	\item \(\cfrac{16}{81}\)
	\end{enumerate}
\item If $m$ and $n$ are whole numbers such that $m^n = 121$ then $(m-1)^{n+1} = ?$
	\begin{enumerate}[label={\Alph*.}]
	\item \(10\)
	\item \(10^2\)
	\item \(10^3\)
	\item \(10^4\)
	\end{enumerate}
\item Simplify: $\cfrac{a^\frac{1}{2} + a^{-\frac{1}{2}}}{1-a} + \cfrac{1- a^{\frac{1}{2}}}{1+ \sqrt{a}}$
	\begin{enumerate}[label={\Alph*.}]
	\item \(\cfrac{a}{a-1}\)
	\item \(\cfrac{a-1}{2}\)
	\item \(\cfrac{2}{a-1}\)
	\item \(\cfrac{2}{1-a}\)
	\end{enumerate}
\item Simplify: $\left(\cfrac{1}{64}\right)^0 + (64)^{-\frac{1}{2}} + (-32)^{\frac{4}{5}}$
	\begin{enumerate}[label={\Alph*.}]
	\item \(17\frac{1}{8}\)
	\item \(11\frac{7}{8}\)
	\item \(17\frac{3}{8}\)
	\item \(17\frac{7}{8}\)
	\end{enumerate}
\item If $\left(\cfrac{x}{y}\right)^{5a-3} = \left(\cfrac{y}{x}\right)^{17-3a}$, what is the value of $a$
	\begin{enumerate}[label={\Alph*.}]
	\item \(-7\)
	\item \(-5\)
	\item \(0\)
	\item \(3\)
	\end{enumerate}
\item Evaluate: $\cfrac{(0.064 - 0.008)(0.16 -0.04)}{(0.16+0.08+0.04)(0.4+0.2)^3}$
	\begin{enumerate}[label={\Alph*.}]
	\item \(\cfrac{1}{3}\)
	\item \(3\)
	\item \(\cfrac{3}{2}\)
	\item \(\cfrac{2}{3}\)
	\end{enumerate}
\item The value of $\left[\left(\sqrt[n]{x^2}\right)^{n/2}\right]^2$
	\begin{enumerate}[label={\Alph*.}]
	\item \(\cfrac{1}{x^2}\)
	\item \(x\)
	\item \(x^2\)
	\item \(x^{\frac{n}{2}}\)
	\end{enumerate}
\item Solve for $x$ if $3^x - 3^{x-1} = 486$
	\begin{enumerate}[label={\Alph*.}]
	\item \(5\)
	\item \(6\)
	\item \(7\)
	\item \(9\)
	\end{enumerate}
\item If $5\sqrt{5} \times 5^3 \div 5^{-\frac{3}{2}} = 5^{a+2}$, then the value of $a$ is 
	\begin{enumerate}[label={\Alph*.}]
	\item \(4\)
	\item \(5\)
	\item \(6\)
	\item \(8\)
	\end{enumerate}
\item If $\left(\sqrt{3}\right)^5 \times 9^2 = 3^n \times 3\sqrt{3}$, then find $n$
	\begin{enumerate}[label={\Alph*.}]
	\item \(2\)
	\item \(3\)
	\item \(4\)
	\item \(5\)
	\end{enumerate}
\item The value of \, $\cfrac{243^{\frac{n}{5}} \times 3^{2n + 1}}{9^n \times 3^{n-1}}$
	\begin{enumerate}[label={\Alph*.}]
	\item \(3\)
	\item \(6\)
	\item \(9\)
	\item \(12\)
	\end{enumerate}
\item If $k^ak^bk^c = 1$, then the value of $a^3 + b^3 + c^3$ is:
	\begin{enumerate}[label={\Alph*.}]
	\item \(9\)
	\item \(a + b + c\)
	\item \(abc\)
	\item \(3abc\)
	\end{enumerate}
\item The value of $\cfrac{81^{3.6} \times 9^{2.7}}{81^{4.2} \times 3}$ is
	\begin{enumerate}[label={\Alph*.}]
	\item \(3\)
	\item \(6\)
	\item \(9\)
	\item \(8.2\)
	\end{enumerate}
\item Simplify $\cfrac{6^{2n +1} \times 9^n \times 4^{2n}}{18^n \times 2^n \times 12^{2n}}$
	\begin{enumerate}[label={\Alph*.}]
	\item \(3^{2n}\)
	\item \(3 \times 2^{n+1}\)
	\item \(2n\)
	\item \(6\)
	\item \(1\)
	\end{enumerate}
\item Solve the systems of equations: $2^{x+y} = 32$  and $3^{3y -x} = 27$, find $(x,y)$ respectively
	\begin{enumerate}[label={\Alph*.}]
	\item \((-3,2)\)
	\item \((-3,-2)\)
	\item \((3,2)\)
	\item \((2,2)\)
	\item \((3,-2)\)
	\end{enumerate}
\item
	\begin{enumerate}[label={\Alph*.}]
	\item \(\)
	\item \(\)
	\item \(\)
	\item \(\)
	\end{enumerate}
\item
	\begin{enumerate}[label={\Alph*.}]
	\item \(\)
	\item \(\)
	\item \(\)
	\item \(\)
	\end{enumerate}
\item
	\begin{enumerate}[label={\Alph*.}]
	\item \(\)
	\item \(\)
	\item \(\)
	\item \(\)
	\end{enumerate}
\item
	\begin{enumerate}[label={\Alph*.}]
	\item \(\)
	\item \(\)
	\item \(\)
	\item \(\)
	\end{enumerate}
\item
	\begin{enumerate}[label={\Alph*.}]
	\item \(\)
	\item \(\)
	\item \(\)
	\item \(\)
	\end{enumerate}
\item
	\begin{enumerate}[label={\Alph*.}]
	\item \(\)
	\item \(\)
	\item \(\)
	\item \(\)
	\end{enumerate}
\item
	\begin{enumerate}[label={\Alph*.}]
	\item \(\)
	\item \(\)
	\item \(\)
	\item \(\)
	\end{enumerate}
\item
	\begin{enumerate}[label={\Alph*.}]
	\item \(\)
	\item \(\)
	\item \(\)
	\item \(\)
	\end{enumerate}
\item
	\begin{enumerate}[label={\Alph*.}]
	\item \(\)
	\item \(\)
	\item \(\)
	\item \(\)
	\end{enumerate}
\item
	\begin{enumerate}[label={\Alph*.}]
	\item \(\)
	\item \(\)
	\item \(\)
	\item \(\)
	\end{enumerate}
\item
	\begin{enumerate}[label={\Alph*.}]
	\item \(\)
	\item \(\)
	\item \(\)
	\item \(\)
	\end{enumerate}
\item
	\begin{enumerate}[label={\Alph*.}]
	\item \(\)
	\item \(\)
	\item \(\)
	\item \(\)
	\end{enumerate}
\item
	\begin{enumerate}[label={\Alph*.}]
	\item \(\)
	\item \(\)
	\item \(\)
	\item \(\)
	\end{enumerate}
\item
	\begin{enumerate}[label={\Alph*.}]
	\item \(\)
	\item \(\)
	\item \(\)
	\item \(\)
	\end{enumerate}
\item
	\begin{enumerate}[label={\Alph*.}]
	\item \(\)
	\item \(\)
	\item \(\)
	\item \(\)
	\end{enumerate}
\item
	\begin{enumerate}[label={\Alph*.}]
	\item \(\)
	\item \(\)
	\item \(\)
	\item \(\)
	\end{enumerate}
\item
	\begin{enumerate}[label={\Alph*.}]
	\item \(\)
	\item \(\)
	\item \(\)
	\item \(\)
	\end{enumerate}
\item
	\begin{enumerate}[label={\Alph*.}]
	\item \(\)
	\item \(\)
	\item \(\)
	\item \(\)
	\end{enumerate}
\item
	\begin{enumerate}[label={\Alph*.}]
	\item \(\)
	\item \(\)
	\item \(\)
	\item \(\)
	\end{enumerate}
\item
	\begin{enumerate}[label={\Alph*.}]
	\item \(\)
	\item \(\)
	\item \(\)
	\item \(\)
	\end{enumerate}
\item
	\begin{enumerate}[label={\Alph*.}]
	\item \(\)
	\item \(\)
	\item \(\)
	\item \(\)
	\end{enumerate}
\item
	\begin{enumerate}[label={\Alph*.}]
	\item \(\)
	\item \(\)
	\item \(\)
	\item \(\)
	\end{enumerate}
\item
	\begin{enumerate}[label={\Alph*.}]
	\item \(\)
	\item \(\)
	\item \(\)
	\item \(\)
	\end{enumerate}
\item
	\begin{enumerate}[label={\Alph*.}]
	\item \(\)
	\item \(\)
	\item \(\)
	\item \(\)
	\end{enumerate}
\item
	\begin{enumerate}[label={\Alph*.}]
	\item \(\)
	\item \(\)
	\item \(\)
	\item \(\)
	\end{enumerate}
\item
	\begin{enumerate}[label={\Alph*.}]
	\item \(\)
	\item \(\)
	\item \(\)
	\item \(\)
	\end{enumerate}
\item
	\begin{enumerate}[label={\Alph*.}]
	\item \(\)
	\item \(\)
	\item \(\)
	\item \(\)
	\end{enumerate}
\item
	\begin{enumerate}[label={\Alph*.}]
	\item \(\)
	\item \(\)
	\item \(\)
	\item \(\)
	\end{enumerate}
\item
	\begin{enumerate}[label={\Alph*.}]
	\item \(\)
	\item \(\)
	\item \(\)
	\item \(\)
	\end{enumerate}
\item
	\begin{enumerate}[label={\Alph*.}]
	\item \(\)
	\item \(\)
	\item \(\)
	\item \(\)
	\end{enumerate}
\item
	\begin{enumerate}[label={\Alph*.}]
	\item \(\)
	\item \(\)
	\item \(\)
	\item \(\)
	\end{enumerate}
\item
	\begin{enumerate}[label={\Alph*.}]
	\item \(\)
	\item \(\)
	\item \(\)
	\item \(\)
	\end{enumerate}
\item
	\begin{enumerate}[label={\Alph*.}]
	\item \(\)
	\item \(\)
	\item \(\)
	\item \(\)
	\end{enumerate}
\item
	\begin{enumerate}[label={\Alph*.}]
	\item \(\)
	\item \(\)
	\item \(\)
	\item \(\)
	\end{enumerate}
\item
	\begin{enumerate}[label={\Alph*.}]
	\item \(\)
	\item \(\)
	\item \(\)
	\item \(\)
	\end{enumerate}
\item
	\begin{enumerate}[label={\Alph*.}]
	\item \(\)
	\item \(\)
	\item \(\)
	\item \(\)
	\end{enumerate}
\item
	\begin{enumerate}[label={\Alph*.}]
	\item \(\)
	\item \(\)
	\item \(\)
	\item \(\)
	\end{enumerate}
\item
	\begin{enumerate}[label={\Alph*.}]
	\item \(\)
	\item \(\)
	\item \(\)
	\item \(\)
	\end{enumerate}


\item
	\begin{enumerate}[label={\Alph*.}]
	\item \(\)
	\item \(\)
	\item \(\)
	\item \(\)
	\end{enumerate}
\item
	\begin{enumerate}[label={\Alph*.}]
	\item \(\)
	\item \(\)
	\item \(\)
	\item \(\)
	\end{enumerate}
\item
	\begin{enumerate}[label={\Alph*.}]
	\item \(\)
	\item \(\)
	\item \(\)
	\item \(\)
	\end{enumerate}
\item
	\begin{enumerate}[label={\Alph*.}]
	\item \(\)
	\item \(\)
	\item \(\)
	\item \(\)
	\end{enumerate}
\item
	\begin{enumerate}[label={\Alph*.}]
	\item \(\)
	\item \(\)
	\item \(\)
	\item \(\)
	\end{enumerate}
\item
	\begin{enumerate}[label={\Alph*.}]
	\item \(\)
	\item \(\)
	\item \(\)
	\item \(\)
	\end{enumerate}
\item
	\begin{enumerate}[label={\Alph*.}]
	\item \(\)
	\item \(\)
	\item \(\)
	\item \(\)
	\end{enumerate}
\item
	\begin{enumerate}[label={\Alph*.}]
	\item \(\)
	\item \(\)
	\item \(\)
	\item \(\)
	\end{enumerate}
\item
	\begin{enumerate}[label={\Alph*.}]
	\item \(\)
	\item \(\)
	\item \(\)
	\item \(\)
	\end{enumerate}
\item
	\begin{enumerate}[label={\Alph*.}]
	\item \(\)
	\item \(\)
	\item \(\)
	\item \(\)
	\end{enumerate}
\item
	\begin{enumerate}[label={\Alph*.}]
	\item \(\)
	\item \(\)
	\item \(\)
	\item \(\)
	\end{enumerate}
\item
	\begin{enumerate}[label={\Alph*.}]
	\item \(\)
	\item \(\)
	\item \(\)
	\item \(\)
	\end{enumerate}
\item
	\begin{enumerate}[label={\Alph*.}]
	\item \(\)
	\item \(\)
	\item \(\)
	\item \(\)
	\end{enumerate}
\item
	\begin{enumerate}[label={\Alph*.}]
	\item \(\)
	\item \(\)
	\item \(\)
	\item \(\)
	\end{enumerate}
\item
	\begin{enumerate}[label={\Alph*.}]
	\item \(\)
	\item \(\)
	\item \(\)
	\item \(\)
	\end{enumerate}
\item
	\begin{enumerate}[label={\Alph*.}]
	\item \(\)
	\item \(\)
	\item \(\)
	\item \(\)
	\end{enumerate}
\item
	\begin{enumerate}[label={\Alph*.}]
	\item \(\)
	\item \(\)
	\item \(\)
	\item \(\)
	\end{enumerate}
\item
	\begin{enumerate}[label={\Alph*.}]
	\item \(\)
	\item \(\)
	\item \(\)
	\item \(\)
	\end{enumerate}
\item
	\begin{enumerate}[label={\Alph*.}]
	\item \(\)
	\item \(\)
	\item \(\)
	\item \(\)
	\end{enumerate}
\item
	\begin{enumerate}[label={\Alph*.}]
	\item \(\)
	\item \(\)
	\item \(\)
	\item \(\)
	\end{enumerate}
\item
	\begin{enumerate}[label={\Alph*.}]
	\item \(\)
	\item \(\)
	\item \(\)
	\item \(\)
	\end{enumerate}
\item
	\begin{enumerate}[label={\Alph*.}]
	\item \(\)
	\item \(\)
	\item \(\)
	\item \(\)
	\end{enumerate}
\item
	\begin{enumerate}[label={\Alph*.}]
	\item \(\)
	\item \(\)
	\item \(\)
	\item \(\)
	\end{enumerate}
\item
	\begin{enumerate}[label={\Alph*.}]
	\item \(\)
	\item \(\)
	\item \(\)
	\item \(\)
	\end{enumerate}
\item
	\begin{enumerate}[label={\Alph*.}]
	\item \(\)
	\item \(\)
	\item \(\)
	\item \(\)
	\end{enumerate}
\end{enumerate}
\end{multicols}
\section{Surds}
\subsection{Questions}
\begin{multicols}{2}
\begin{enumerate}[label={\arabic*.}] 
\item Simplify: $\sqrt{50}$
	\begin{enumerate}[label={\Alph*.}]
	\item $5\sqrt{2}$
	\item $2\sqrt{5}$
	\item $10\sqrt{2}$
	\item $25\sqrt{2}$
	\end{enumerate}

\item Simplify: $\sqrt{75} + \sqrt{12}$
	\begin{enumerate}[label={\Alph*.}]
	\item $7\sqrt{3}$
	\item $5\sqrt{3} + 2\sqrt{3}$
	\item $\sqrt{87}$
	\item $8\sqrt{3}$
	\end{enumerate}

\item Rationalize the denominator: $\cfrac{6}{\sqrt{3}}$
	\begin{enumerate}[label={\Alph*.}]
	\item $2\sqrt{3}$
	\item $3\sqrt{2}$
	\item $6\sqrt{3}$
	\item $\sqrt{18}$
	\end{enumerate}

\item Simplify: $\sqrt{128}$
	\begin{enumerate}[label={\Alph*.}]
	\item $8\sqrt{2}$
	\item $4\sqrt{8}$
	\item $16\sqrt{2}$
	\item $2\sqrt{32}$
	\end{enumerate}

\item Evaluate: $(\sqrt{5})^2$
	\begin{enumerate}[label={\Alph*.}]
	\item $25$
	\item $5$
	\item $10$
	\item $\sqrt{25}$
	\end{enumerate}

\item Simplify: $\sqrt{18} - \sqrt{8}$
	\begin{enumerate}[label={\Alph*.}]
	\item $\sqrt{10}$
	\item $\sqrt{2}$
	\item $2\sqrt{2}$
	\item $3\sqrt{2} - 2\sqrt{2}$
	\end{enumerate}

\item Multiply: $\sqrt{3} \times \sqrt{12}$
	\begin{enumerate}[label={\Alph*.}]
	\item $6$
	\item $3\sqrt{4}$
	\item $\sqrt{36}$
	\item $2\sqrt{6}$
	\end{enumerate}

\item Rationalize: $\cfrac{10}{\sqrt{5}}$
	\begin{enumerate}[label={\Alph*.}]
	\item $2\sqrt{5}$
	\item $5\sqrt{2}$
	\item $\sqrt{20}$
	\item $10\sqrt{5}$
	\end{enumerate}

\item Simplify: $3\sqrt{8} + 2\sqrt{2}$
	\begin{enumerate}[label={\Alph*.}]
	\item $5\sqrt{10}$
	\item $8\sqrt{2}$
	\item $5\sqrt{2}$
	\item $6\sqrt{2} + 2\sqrt{2}$
	\end{enumerate}

\item Express in simplest form: $\sqrt{200}$
	\begin{enumerate}[label={\Alph*.}]
	\item $10\sqrt{2}$
	\item $20\sqrt{2}$
	\item $100\sqrt{2}$
	\item $2\sqrt{100}$
	\end{enumerate}

\item Simplify: $\cfrac{\sqrt{48}}{\sqrt{3}}$
	\begin{enumerate}[label={\Alph*.}]
	\item $4$
	\item $16$
	\item $\sqrt{16}$
	\item $2\sqrt{4}$
	\end{enumerate}

\item Evaluate: $\sqrt{2} \times \sqrt{8}$
	\begin{enumerate}[label={\Alph*.}]
	\item $4$
	\item $\sqrt{16}$
	\item $2\sqrt{4}$
	\item $\sqrt{10}$
	\end{enumerate}

\item Simplify: $\sqrt{45} - \sqrt{20}$
	\begin{enumerate}[label={\Alph*.}]
	\item $\sqrt{25}$
	\item $\sqrt{5}$
	\item $3\sqrt{5} - 2\sqrt{5}$
	\item $5\sqrt{5}$
	\end{enumerate}

\item Rationalize: $\cfrac{8}{\sqrt{2}}$
	\begin{enumerate}[label={\Alph*.}]
	\item $4\sqrt{2}$
	\item $8\sqrt{2}$
	\item $2\sqrt{8}$
	\item $16\sqrt{2}$
	\end{enumerate}

\item Simplify: $2\sqrt{27} + \sqrt{12}$
	\begin{enumerate}[label={\Alph*.}]
	\item $8\sqrt{3}$
	\item $6\sqrt{3} + 2\sqrt{3}$
	\item $7\sqrt{3}$
	\item $9\sqrt{3}$
	\end{enumerate}

\item Express $\sqrt{72}$ in simplest surd form
	\begin{enumerate}[label={\Alph*.}]
	\item $6\sqrt{2}$
	\item $8\sqrt{3}$
	\item $9\sqrt{2}$
	\item $12\sqrt{6}$
	\end{enumerate}

\item Multiply and simplify: $\sqrt{6} \times \sqrt{24}$
	\begin{enumerate}[label={\Alph*.}]
	\item $12$
	\item $\sqrt{144}$
	\item $6\sqrt{4}$
	\item $2\sqrt{36}$
	\end{enumerate}

\item Rationalize the denominator: $\cfrac{15}{\sqrt{3}}$
	\begin{enumerate}[label={\Alph*.}]
	\item $5\sqrt{3}$
	\item $3\sqrt{5}$
	\item $15\sqrt{3}$
	\item $\sqrt{75}$
	\end{enumerate}

\item Simplify: $\sqrt{98} + \sqrt{32}$
	\begin{enumerate}[label={\Alph*.}]
	\item $11\sqrt{2}$
	\item $7\sqrt{2} + 4\sqrt{2}$
	\item $\sqrt{130}$
	\item $9\sqrt{2}$
	\end{enumerate}

\item Evaluate: $\cfrac{\sqrt{50}}{\sqrt{2}}$
	\begin{enumerate}[label={\Alph*.}]
	\item $5$
	\item $25$
	\item $\sqrt{25}$
	\item $10$
	\end{enumerate}

\item Simplify: $4\sqrt{5} - 2\sqrt{5}$
	\begin{enumerate}[label={\Alph*.}]
	\item $2\sqrt{5}$
	\item $6\sqrt{5}$
	\item $2\sqrt{0}$
	\item $8\sqrt{5}$
	\end{enumerate}

\item Expand and simplify: $(\sqrt{3} + 2)(\sqrt{3} - 2)$
	\begin{enumerate}[label={\Alph*.}]
	\item $-1$
	\item $1$
	\item $3 - 4$
	\item $7$
	\end{enumerate}

\item Rationalize: $\cfrac{12}{\sqrt{6}}$
	\begin{enumerate}[label={\Alph*.}]
	\item $2\sqrt{6}$
	\item $6\sqrt{2}$
	\item $\sqrt{24}$
	\item $12\sqrt{6}$
	\end{enumerate}

\item Simplify: $\sqrt{180}$
	\begin{enumerate}[label={\Alph*.}]
	\item $6\sqrt{5}$
	\item $9\sqrt{2}$
	\item $3\sqrt{20}$
	\item $18\sqrt{10}$
	\end{enumerate}

\item Evaluate: $\sqrt{7} \times \sqrt{7}$
	\begin{enumerate}[label={\Alph*.}]
	\item $7$
	\item $14$
	\item $49$
	\item $\sqrt{14}$
	\end{enumerate}

\item Simplify: $\sqrt{125} - \sqrt{45}$
	\begin{enumerate}[label={\Alph*.}]
	\item $2\sqrt{5}$
	\item $5\sqrt{5} - 3\sqrt{5}$
	\item $8\sqrt{5}$
	\item $\sqrt{80}$
	\end{enumerate}

\item Rationalize: $\cfrac{20}{\sqrt{10}}$
	\begin{enumerate}[label={\Alph*.}]
	\item $2\sqrt{10}$
	\item $10\sqrt{2}$
	\item $\sqrt{200}$
	\item $20\sqrt{10}$
	\end{enumerate}

\item Simplify: $3\sqrt{12} + 4\sqrt{3}$
	\begin{enumerate}[label={\Alph*.}]
	\item $10\sqrt{3}$
	\item $6\sqrt{3} + 4\sqrt{3}$
	\item $7\sqrt{15}$
	\item $12\sqrt{3}$
	\end{enumerate}

\item Express in simplest form: $\sqrt{162}$
	\begin{enumerate}[label={\Alph*.}]
	\item $9\sqrt{2}$
	\item $6\sqrt{3}$
	\item $81\sqrt{2}$
	\item $3\sqrt{18}$
	\end{enumerate}

\item Multiply: $\sqrt{10} \times \sqrt{40}$
	\begin{enumerate}[label={\Alph*.}]
	\item $20$
	\item $\sqrt{400}$
	\item $10\sqrt{4}$
	\item $4\sqrt{100}$
	\end{enumerate}

\item Rationalize the denominator: $\cfrac{18}{\sqrt{2}}$
	\begin{enumerate}[label={\Alph*.}]
	\item $9\sqrt{2}$
	\item $2\sqrt{9}$
	\item $18\sqrt{2}$
	\item $\sqrt{162}$
	\end{enumerate}

\item Simplify: $\sqrt{300}$
	\begin{enumerate}[label={\Alph*.}]
	\item $10\sqrt{3}$
	\item $3\sqrt{100}$
	\item $30\sqrt{10}$
	\item $100\sqrt{3}$
	\end{enumerate}

\item Evaluate: $\cfrac{\sqrt{72}}{\sqrt{8}}$
	\begin{enumerate}[label={\Alph*.}]
	\item $3$
	\item $9$
	\item $\sqrt{9}$
	\item $6$
	\end{enumerate}

\item Simplify: $5\sqrt{2} + 3\sqrt{2}$
	\begin{enumerate}[label={\Alph*.}]
	\item $8\sqrt{2}$
	\item $15\sqrt{4}$
	\item $8\sqrt{4}$
	\item $\sqrt{64}$
	\end{enumerate}

\item Expand: $(\sqrt{5} + 1)^2$
	\begin{enumerate}[label={\Alph*.}]
	\item $6 + 2\sqrt{5}$
	\item $5 + 1$
	\item $26$
	\item $6$
	\end{enumerate}

\item Rationalize: $\cfrac{24}{\sqrt{8}}$
	\begin{enumerate}[label={\Alph*.}]
	\item $3\sqrt{8}$
	\item $6\sqrt{2}$
	\item $8\sqrt{3}$
	\item $12\sqrt{2}$
	\end{enumerate}

\item Simplify: $\sqrt{112}$
	\begin{enumerate}[label={\Alph*.}]
	\item $4\sqrt{7}$
	\item $7\sqrt{4}$
	\item $8\sqrt{14}$
	\item $56\sqrt{2}$
	\end{enumerate}

\item Multiply and simplify: $\sqrt{5} \times \sqrt{20}$
	\begin{enumerate}[label={\Alph*.}]
	\item $10$
	\item $5\sqrt{4}$
	\item $\sqrt{100}$
	\item $20$
	\end{enumerate}

\item Simplify: $\sqrt{147} - \sqrt{27}$
	\begin{enumerate}[label={\Alph*.}]
	\item $4\sqrt{3}$
	\item $7\sqrt{3} - 3\sqrt{3}$
	\item $\sqrt{120}$
	\item $10\sqrt{3}$
	\end{enumerate}

\item Rationalize: $\cfrac{14}{\sqrt{7}}$
	\begin{enumerate}[label={\Alph*.}]
	\item $2\sqrt{7}$
	\item $7\sqrt{2}$
	\item $\sqrt{98}$
	\item $14\sqrt{7}$
	\end{enumerate}

\item Simplify: $2\sqrt{18} + 3\sqrt{2}$
	\begin{enumerate}[label={\Alph*.}]
	\item $9\sqrt{2}$
	\item $6\sqrt{2} + 3\sqrt{2}$
	\item $5\sqrt{20}$
	\item $11\sqrt{2}$
	\end{enumerate}

\item Express in simplest form: $\sqrt{242}$
	\begin{enumerate}[label={\Alph*.}]
	\item $11\sqrt{2}$
	\item $121\sqrt{2}$
	\item $2\sqrt{121}$
	\item $22\sqrt{11}$
	\end{enumerate}

\item Evaluate: $\cfrac{\sqrt{108}}{\sqrt{3}}$
	\begin{enumerate}[label={\Alph*.}]
	\item $6$
	\item $36$
	\item $\sqrt{36}$
	\item $3\sqrt{4}$
	\end{enumerate}

\item Simplify: $7\sqrt{3} - 2\sqrt{3}$
	\begin{enumerate}[label={\Alph*.}]
	\item $5\sqrt{3}$
	\item $9\sqrt{3}$
	\item $5\sqrt{0}$
	\item $14\sqrt{6}$
	\end{enumerate}

\item Expand and simplify: $(\sqrt{2} + 3)(\sqrt{2} - 3)$
	\begin{enumerate}[label={\Alph*.}]
	\item $-7$
	\item $2 - 9$
	\item $11$
	\item $-5$
	\end{enumerate}

\item Rationalize: $\cfrac{30}{\sqrt{5}}$
	\begin{enumerate}[label={\Alph*.}]
	\item $6\sqrt{5}$
	\item $5\sqrt{6}$
	\item $\sqrt{180}$
	\item $30\sqrt{5}$
	\end{enumerate}

\item Simplify: $\sqrt{245}$
	\begin{enumerate}[label={\Alph*.}]
	\item $7\sqrt{5}$
	\item $5\sqrt{7}$
	\item $49\sqrt{5}$
	\item $\sqrt{49 \times 5}$
	\end{enumerate}

\item Multiply: $\sqrt{15} \times \sqrt{15}$
	\begin{enumerate}[label={\Alph*.}]
	\item $15$
	\item $30$
	\item $225$
	\item $\sqrt{30}$
	\end{enumerate}

\item Simplify: $\sqrt{192} - \sqrt{48}$
	\begin{enumerate}[label={\Alph*.}]
	\item $4\sqrt{3}$
	\item $8\sqrt{3} - 4\sqrt{3}$
	\item $\sqrt{144}$
	\item $12\sqrt{3}$
	\end{enumerate}

\item Rationalize: $\cfrac{16}{\sqrt{4}}$
	\begin{enumerate}[label={\Alph*.}]
	\item $8$
	\item $4\sqrt{4}$
	\item $16\sqrt{4}$
	\item $32$
	\end{enumerate}

\item Simplify: $4\sqrt{20} + 2\sqrt{5}$
	\begin{enumerate}[label={\Alph*.}]
	\item $10\sqrt{5}$
	\item $8\sqrt{5} + 2\sqrt{5}$
	\item $6\sqrt{25}$
	\item $20\sqrt{5}$
	\end{enumerate}

\item Express in simplest form: $\sqrt{288}$
	\begin{enumerate}[label={\Alph*.}]
	\item $12\sqrt{2}$
	\item $144\sqrt{2}$
	\item $2\sqrt{144}$
	\item $24\sqrt{3}$
	\end{enumerate}

\item Evaluate: $\cfrac{\sqrt{125}}{\sqrt{5}}$
	\begin{enumerate}[label={\Alph*.}]
	\item $5$
	\item $25$
	\item $\sqrt{25}$
	\item $15$
	\end{enumerate}

\item Simplify: $9\sqrt{7} + 4\sqrt{7}$
	\begin{enumerate}[label={\Alph*.}]
	\item $13\sqrt{7}$
	\item $36\sqrt{14}$
	\item $13\sqrt{14}$
	\item $\sqrt{169 \times 7}$
	\end{enumerate}

\item Expand: $(\sqrt{6} + 2)^2$
	\begin{enumerate}[label={\Alph*.}]
	\item $10 + 4\sqrt{6}$
	\item $6 + 4$
	\item $40$
	\item $8$
	\end{enumerate}

\item Rationalize: $\cfrac{21}{\sqrt{3}}$
	\begin{enumerate}[label={\Alph*.}]
	\item $7\sqrt{3}$
	\item $3\sqrt{7}$
	\item $21\sqrt{3}$
	\item $\sqrt{147}$
	\end{enumerate}

\item Simplify: $\sqrt{343}$
	\begin{enumerate}[label={\Alph*.}]
	\item $7\sqrt{7}$
	\item $49\sqrt{7}$
	\item $\sqrt{49 \times 7}$
	\item $343\sqrt{1}$
	\end{enumerate}

\item Multiply and simplify: $\sqrt{8} \times \sqrt{18}$
	\begin{enumerate}[label={\Alph*.}]
	\item $12$
	\item $\sqrt{144}$
	\item $2\sqrt{36}$
	\item $4\sqrt{9}$
	\end{enumerate}

\item Simplify: $\sqrt{175} + \sqrt{63}$
	\begin{enumerate}[label={\Alph*.}]
	\item $8\sqrt{7}$
	\item $5\sqrt{7} + 3\sqrt{7}$
	\item $\sqrt{238}$
	\item $15\sqrt{7}$
	\end{enumerate}

\item Rationalize: $\cfrac{25}{\sqrt{5}}$
	\begin{enumerate}[label={\Alph*.}]
	\item $5\sqrt{5}$
	\item $25\sqrt{5}$
	\item $\sqrt{125}$
	\item $125$
	\end{enumerate}

\item Simplify: $5\sqrt{32} - 2\sqrt{2}$
	\begin{enumerate}[label={\Alph*.}]
	\item $18\sqrt{2}$
	\item $20\sqrt{2} - 2\sqrt{2}$
	\item $3\sqrt{30}$
	\item $8\sqrt{2}$
	\end{enumerate}

\item Express in simplest form: $\sqrt{392}$
	\begin{enumerate}[label={\Alph*.}]
	\item $14\sqrt{2}$
	\item $196\sqrt{2}$
	\item $2\sqrt{196}$
	\item $28\sqrt{7}$
	\end{enumerate}

\item Evaluate: $\cfrac{\sqrt{200}}{\sqrt{8}}$
	\begin{enumerate}[label={\Alph*.}]
	\item $5$
	\item $25$
	\item $\sqrt{25}$
	\item $10$
	\end{enumerate}

\item Simplify: $6\sqrt{11} - 2\sqrt{11}$
	\begin{enumerate}[label={\Alph*.}]
	\item $4\sqrt{11}$
	\item $8\sqrt{11}$
	\item $4\sqrt{0}$
	\item $12\sqrt{22}$
	\end{enumerate}

\item Expand and simplify: $(\sqrt{10} + 1)(\sqrt{10} - 1)$
	\begin{enumerate}[label={\Alph*.}]
	\item $9$
	\item $10 - 1$
	\item $11$
	\item $-9$
	\end{enumerate}

\item Rationalize: $\cfrac{35}{\sqrt{7}}$
	\begin{enumerate}[label={\Alph*.}]
	\item $5\sqrt{7}$
	\item $7\sqrt{5}$
	\item $\sqrt{245}$
	\item $35\sqrt{7}$
	\end{enumerate}

\item Simplify: $\sqrt{450}$
	\begin{enumerate}[label={\Alph*.}]
	\item $15\sqrt{2}$
	\item $225\sqrt{2}$
	\item $2\sqrt{225}$
	\item $9\sqrt{50}$
	\end{enumerate}

\item Multiply: $\sqrt{6} \times \sqrt{54}$
	\begin{enumerate}[label={\Alph*.}]
	\item $18$
	\item $\sqrt{324}$
	\item $6\sqrt{9}$
	\item $3\sqrt{36}$
	\end{enumerate}

\item Simplify: $\sqrt{243} - \sqrt{48}$
	\begin{enumerate}[label={\Alph*.}]
	\item $5\sqrt{3}$
	\item $9\sqrt{3} - 4\sqrt{3}$
	\item $\sqrt{195}$
	\item $13\sqrt{3}$
	\end{enumerate}

\item Rationalize: $\cfrac{40}{\sqrt{10}}$
	\begin{enumerate}[label={\Alph*.}]
	\item $4\sqrt{10}$
	\item $10\sqrt{4}$
	\item $\sqrt{160}$
	\item $40\sqrt{10}$
	\end{enumerate}

\item Simplify: $3\sqrt{50} + 4\sqrt{2}$
	\begin{enumerate}[label={\Alph*.}]
	\item $19\sqrt{2}$
	\item $15\sqrt{2} + 4\sqrt{2}$
	\item $7\sqrt{52}$
	\item $12\sqrt{100}$
	\end{enumerate}

\item Express in simplest form: $\sqrt{500}$
	\begin{enumerate}[label={\Alph*.}]
	\item $10\sqrt{5}$
	\item $250\sqrt{2}$
	\item $5\sqrt{100}$
	\item $100\sqrt{5}$
	\end{enumerate}

\item Evaluate: $\cfrac{\sqrt{180}}{\sqrt{5}}$
	\begin{enumerate}[label={\Alph*.}]
	\item $6$
	\item $36$
	\item $\sqrt{36}$
	\item $3\sqrt{4}$
	\end{enumerate}

\item Simplify: $10\sqrt{3} + 5\sqrt{3}$
	\begin{enumerate}[label={\Alph*.}]
	\item $15\sqrt{3}$
	\item $50\sqrt{9}$
	\item $15\sqrt{6}$
	\item $\sqrt{675}$
	\end{enumerate}

\item Expand: $(\sqrt{11} + 3)^2$
	\begin{enumerate}[label={\Alph*.}]
	\item $20 + 6\sqrt{11}$
	\item $11 + 9$
	\item $121$
	\item $14$
	\end{enumerate}

\item Rationalize: $\cfrac{42}{\sqrt{6}}$
	\begin{enumerate}[label={\Alph*.}]
	\item $7\sqrt{6}$
	\item $6\sqrt{7}$
	\item $42\sqrt{6}$
	\item $\sqrt{252}$
	\end{enumerate}

\item Simplify: $\sqrt{512}$
	\begin{enumerate}[label={\Alph*.}]
	\item $16\sqrt{2}$
	\item $256\sqrt{2}$
	\item $2\sqrt{256}$
	\item $8\sqrt{64}$
	\end{enumerate}

\item Multiply and simplify: $\sqrt{12} \times \sqrt{27}$
	\begin{enumerate}[label={\Alph*.}]
	\item $18$
	\item $\sqrt{324}$
	\item $6\sqrt{9}$
	\item $3\sqrt{36}$
	\end{enumerate}

\item Simplify: $\sqrt{320} + \sqrt{80}$
	\begin{enumerate}[label={\Alph*.}]
	\item $12\sqrt{5}$
	\item $8\sqrt{5} + 4\sqrt{5}$
	\item $\sqrt{400}$
	\item $20\sqrt{5}$
	\end{enumerate}

\item Rationalize: $\cfrac{48}{\sqrt{12}}$
	\begin{enumerate}[label={\Alph*.}]
	\item $4\sqrt{12}$
	\item $12\sqrt{4}$
	\item $4\sqrt{3}$
	\item $8\sqrt{3}$
	\end{enumerate}

\item Simplify: $6\sqrt{45} - 3\sqrt{5}$
	\begin{enumerate}[label={\Alph*.}]
	\item $15\sqrt{5}$
	\item $18\sqrt{5} - 3\sqrt{5}$
	\item $3\sqrt{40}$
	\item $21\sqrt{5}$
	\end{enumerate}

\item Express in simplest form: $\sqrt{578}$
	\begin{enumerate}[label={\Alph*.}]
	\item $17\sqrt{2}$
	\item $289\sqrt{2}$
	\item $2\sqrt{289}$
	\item $34\sqrt{17}$
	\end{enumerate}

\item Evaluate: $\cfrac{\sqrt{243}}{\sqrt{3}}$
	\begin{enumerate}[label={\Alph*.}]
	\item $9$
	\item $81$
	\item $\sqrt{81}$
	\item $27$
	\end{enumerate}

\item Simplify: $8\sqrt{13} + 7\sqrt{13}$
	\begin{enumerate}[label={\Alph*.}]
	\item $15\sqrt{13}$
	\item $56\sqrt{26}$
	\item $15\sqrt{26}$
	\item $\sqrt{3575}$
	\end{enumerate}

\item Expand and simplify: $(\sqrt{7} + 4)(\sqrt{7} - 4)$
	\begin{enumerate}[label={\Alph*.}]
	\item $-9$
	\item $7 - 16$
	\item $23$
	\item $-7$
	\end{enumerate}

\item Rationalize: $\cfrac{56}{\sqrt{8}}$
	\begin{enumerate}[label={\Alph*.}]
	\item $7\sqrt{8}$
	\item $14\sqrt{2}$
	\item $8\sqrt{7}$
	\item $28\sqrt{2}$
	\end{enumerate}

\item Simplify: $\sqrt{605}$
	\begin{enumerate}[label={\Alph*.}]
	\item $11\sqrt{5}$
	\item $121\sqrt{5}$
	\item $5\sqrt{121}$
	\item $\sqrt{121 \times 5}$
	\end{enumerate}

\item Multiply: $\sqrt{11} \times \sqrt{44}$
	\begin{enumerate}[label={\Alph*.}]
	\item $22$
	\item $\sqrt{484}$
	\item $11\sqrt{4}$
	\item $2\sqrt{121}$
	\end{enumerate}

\item Simplify: $\sqrt{363} - \sqrt{27}$
	\begin{enumerate}[label={\Alph*.}]
	\item $8\sqrt{3}$
	\item $11\sqrt{3} - 3\sqrt{3}$
	\item $\sqrt{336}$
	\item $14\sqrt{3}$
	\end{enumerate}

\item Rationalize: $\cfrac{4}{\sqrt{3} + 1}$
	\begin{enumerate}[label={\Alph*.}]
	\item $2\sqrt{3} - 2$
	\item $2(\sqrt{3} - 1)$
	\item $4\sqrt{3} - 4$
	\item $\sqrt{3} - 1$
	\end{enumerate}

\item Simplify: $\cfrac{1}{\sqrt{5} - 2}$
	\begin{enumerate}[label={\Alph*.}]
	\item $\sqrt{5} + 2$
	\item $\cfrac{\sqrt{5} + 2}{5}$
	\item $5 + 2$
	\item $\sqrt{5} - 2$
	\end{enumerate}

\item Evaluate: $(\sqrt{8} + \sqrt{2})(\sqrt{8} - \sqrt{2})$
	\begin{enumerate}[label={\Alph*.}]
	\item $6$
	\item $8 - 2$
	\item $10$
	\item $4$
	\end{enumerate}

\item Rationalize: $\cfrac{6}{\sqrt{7} - \sqrt{3}}$
	\begin{enumerate}[label={\Alph*.}]
	\item $\cfrac{3(\sqrt{7} + \sqrt{3})}{2}$
	\item $\sqrt{7} + \sqrt{3}$
	\item $3\sqrt{7} + 3\sqrt{3}$
	\item $6(\sqrt{7} + \sqrt{3})$
	\end{enumerate}

\item Simplify: $\sqrt{12} + \sqrt{27} - \sqrt{75}$
	\begin{enumerate}[label={\Alph*.}]
	\item $0$
	\item $-2\sqrt{3}$
	\item $2\sqrt{3}$
	\item $\sqrt{3}$
	\end{enumerate}

\item Evaluate: $\cfrac{\sqrt{98} - \sqrt{32}}{\sqrt{2}}$
	\begin{enumerate}[label={\Alph*.}]
	\item $3$
	\item $7 - 4$
	\item $11$
	\item $\sqrt{33}$
	\end{enumerate}

\item Rationalize: $\cfrac{10}{\sqrt{6} + \sqrt{2}}$
	\begin{enumerate}[label={\Alph*.}]
	\item $\cfrac{5(\sqrt{6} - \sqrt{2})}{2}$
	\item $5\sqrt{6} - 5\sqrt{2}$
	\item $\sqrt{6} - \sqrt{2}$
	\item $10(\sqrt{6} - \sqrt{2})$
	\end{enumerate}

\item Simplify: $(\sqrt{5} + \sqrt{3})^2$
	\begin{enumerate}[label={\Alph*.}]
	\item $8 + 2\sqrt{15}$
	\item $5 + 3$
	\item $64$
	\item $8 + \sqrt{15}$
	\end{enumerate}

\item Evaluate: $\cfrac{\sqrt{72} + \sqrt{50}}{\sqrt{2}}$
	\begin{enumerate}[label={\Alph*.}]
	\item $11$
	\item $6 + 5$
	\item $\sqrt{61}$
	\item $8$
	\end{enumerate}

\item Rationalize: $\cfrac{8}{\sqrt{5} + \sqrt{3}}$
	\begin{enumerate}[label={\Alph*.}]
	\item $4(\sqrt{5} - \sqrt{3})$
	\item $8\sqrt{5} - 8\sqrt{3}$
	\item $\sqrt{5} - \sqrt{3}$
	\item $2(\sqrt{5} - \sqrt{3})$
	\end{enumerate}
	
\item Simplify and rationalize the expression \(\displaystyle \frac{2}{\sqrt{3} - \sqrt{2}}\).
	\begin{enumerate}[label={\Alph*.}]
	\item \(\sqrt{3} - \sqrt{2}\)
	\item \(2\sqrt{3} - 2\sqrt{2}\)
	\item \(2\sqrt{3} + 2\sqrt{2}\)
	\item \(\sqrt{3} + \sqrt{2}\)
	\end{enumerate}
\end{enumerate}
\end{multicols}
\section{Logarithms}
\subsection{Questions}
\begin{multicols}{2}
\begin{enumerate}[label={\arabic*.}] 
\item Evaluate: \(\log_2 8\).
	\begin{enumerate}[label={\Alph*.}]
	\item $2$
	\item $3$
	\item $4$
	\item $8$
	\end{enumerate}

\item If \(\log_{10} 2 = 0.3010\), find \(\log_{10} 20\).
	\begin{enumerate}[label={\Alph*.}]
	\item $0.6020$
	\item $1.3010$
	\item $2.3010$
	\item $6.020$
	\end{enumerate}

\item Simplify: \(\log_5 125\).
	\begin{enumerate}[label={\Alph*.}]
	\item $1$
	\item $2$
	\item $3$
	\item $5$
	\end{enumerate}

\item If \(\log_x 81 = 4\), find \(x\).
	\begin{enumerate}[label={\Alph*.}]
	\item $3$
	\item $4$
	\item $9$
	\item $27$
	\end{enumerate}

\item Evaluate: \(\log_3 1\).
	\begin{enumerate}[label={\Alph*.}]
	\item $0$
	\item $1$
	\item $3$
	\item $\text{undefined}$
	\end{enumerate}

\item Simplify: \(\log_7 49 + \log_7 7\).
	\begin{enumerate}[label={\Alph*.}]
	\item $1$
	\item $2$
	\item $3$
	\item $4$
	\end{enumerate}

\item If \(\log_{10} 5 = 0.6990\), find \(\log_{10} 50\).
	\begin{enumerate}[label={\Alph*.}]
	\item $0.3010$
	\item $1.3010$
	\item $1.6990$
	\item $3.4950$
	\end{enumerate}

\item Evaluate: \(\log_4 64\).
	\begin{enumerate}[label={\Alph*.}]
	\item $2$
	\item $3$
	\item $4$
	\item $16$
	\end{enumerate}

\item Simplify: \(\log_2 32 - \log_2 4\).
	\begin{enumerate}[label={\Alph*.}]
	\item $2$
	\item $3$
	\item $4$
	\item $8$
	\end{enumerate}

\item If \(\log_5 x = 3\), find \(x\).
	\begin{enumerate}[label={\Alph*.}]
	\item $8$
	\item $15$
	\item $25$
	\item $125$
	\end{enumerate}

\item Evaluate: \(2\log_3 9\).
	\begin{enumerate}[label={\Alph*.}]
	\item $2$
	\item $3$
	\item $4$
	\item $6$
	\end{enumerate}

\item Simplify: \(\log_6 36 + \log_6 1\).
	\begin{enumerate}[label={\Alph*.}]
	\item $0$
	\item $1$
	\item $2$
	\item $3$
	\end{enumerate}

\item If \(\log_{10} 3 = 0.4771\), find \(\log_{10} 30\).
	\begin{enumerate}[label={\Alph*.}]
	\item $0.9542$
	\item $1.4771$
	\item $4.771$
	\item $14.313$
	\end{enumerate}

\item Evaluate: \(\log_8 512\).
	\begin{enumerate}[label={\Alph*.}]
	\item $2$
	\item $3$
	\item $4$
	\item $64$
	\end{enumerate}

\item Simplify: \(\log_5 25 \times \log_5 5\).
	\begin{enumerate}[label={\Alph*.}]
	\item $1$
	\item $2$
	\item $3$
	\item $5$
	\end{enumerate}

\item If \(\log_2 x = 5\), find \(x\).
	\begin{enumerate}[label={\Alph*.}]
	\item $10$
	\item $25$
	\item $32$
	\item $64$
	\end{enumerate}

\item Evaluate: \(\log_{10} 100\).
	\begin{enumerate}[label={\Alph*.}]
	\item $1$
	\item $2$
	\item $10$
	\item $100$
	\end{enumerate}

\item Simplify: \(\log_3 81 - \log_3 9\).
	\begin{enumerate}[label={\Alph*.}]
	\item $1$
	\item $2$
	\item $3$
	\item $4$
	\end{enumerate}

\item If \(\log_{10} 2 = 0.3010\), find \(\log_{10} 8\).
	\begin{enumerate}[label={\Alph*.}]
	\item $0.6020$
	\item $0.9030$
	\item $2.408$
	\item $8.030$
	\end{enumerate}

\item Evaluate: \(\log_9 729\).
	\begin{enumerate}[label={\Alph*.}]
	\item $2$
	\item $3$
	\item $4$
	\item $81$
	\end{enumerate}

\item Simplify: \(3\log_2 4\).
	\begin{enumerate}[label={\Alph*.}]
	\item $2$
	\item $4$
	\item $6$
	\item $8$
	\end{enumerate}

\item If \(\log_x 16 = 2\), find \(x\).
	\begin{enumerate}[label={\Alph*.}]
	\item $2$
	\item $4$
	\item $8$
	\item $32$
	\end{enumerate}

\item Evaluate: \(\log_5 1\).
	\begin{enumerate}[label={\Alph*.}]
	\item $0$
	\item $1$
	\item $5$
	\item $\text{undefined}$
	\end{enumerate}

\item Simplify: \(\log_4 16 + \log_4 4\).
	\begin{enumerate}[label={\Alph*.}]
	\item $2$
	\item $3$
	\item $4$
	\item $5$
	\end{enumerate}

\item If \(\log_{10} 7 = 0.8451\), find \(\log_{10} 70\).
	\begin{enumerate}[label={\Alph*.}]
	\item $1.6902$
	\item $1.8451$
	\item $8.451$
	\item $59.157$
	\end{enumerate}

\item Evaluate: \(\log_6 216\).
	\begin{enumerate}[label={\Alph*.}]
	\item $2$
	\item $3$
	\item $4$
	\item $36$
	\end{enumerate}

\item Simplify: \(\log_7 343 - \log_7 49\).
	\begin{enumerate}[label={\Alph*.}]
	\item $0$
	\item $1$
	\item $2$
	\item $7$
	\end{enumerate}

\item If \(\log_3 x = 4\), find \(x\).
	\begin{enumerate}[label={\Alph*.}]
	\item $12$
	\item $27$
	\item $64$
	\item $81$
	\end{enumerate}

\item Evaluate: \(\log_{10} 1000\).
	\begin{enumerate}[label={\Alph*.}]
	\item $2$
	\item $3$
	\item $10$
	\item $100$
	\end{enumerate}

\item Simplify: \(2\log_5 5 + \log_5 25\).
	\begin{enumerate}[label={\Alph*.}]
	\item $2$
	\item $3$
	\item $4$
	\item $5$
	\end{enumerate}

\item If \(\log_x 27 = 3\), find \(x\).
	\begin{enumerate}[label={\Alph*.}]
	\item $3$
	\item $9$
	\item $27$
	\item $81$
	\end{enumerate}

\item Evaluate: \(\log_2 \cfrac{1}{8}\).
	\begin{enumerate}[label={\Alph*.}]
	\item $-3$
	\item $-2$
	\item $\cfrac{1}{3}$
	\item $3$
	\end{enumerate}

\item Simplify: \(\log_3 27 + \log_3 3\).
	\begin{enumerate}[label={\Alph*.}]
	\item $2$
	\item $3$
	\item $4$
	\item $9$
	\end{enumerate}

\item If \(\log_{10} 4 = 0.6021\), find \(\log_{10} 400\).
	\begin{enumerate}[label={\Alph*.}]
	\item $1.2042$
	\item $1.6021$
	\item $2.6021$
	\item $240.84$
	\end{enumerate}

\item Evaluate: \(\log_5 \cfrac{1}{25}\).
	\begin{enumerate}[label={\Alph*.}]
	\item $-2$
	\item $-1$
	\item $\cfrac{1}{2}$
	\item $2$
	\end{enumerate}

\item Simplify: \(\log_8 64 - \log_8 8\).
	\begin{enumerate}[label={\Alph*.}]
	\item $0$
	\item $1$
	\item $2$
	\item $8$
	\end{enumerate}

\item If \(\log_4 x = 3\), find \(x\).
	\begin{enumerate}[label={\Alph*.}]
	\item $12$
	\item $16$
	\item $64$
	\item $81$
	\end{enumerate}

\item Evaluate: \(3\log_3 3\).
	\begin{enumerate}[label={\Alph*.}]
	\item $1$
	\item $3$
	\item $9$
	\item $27$
	\end{enumerate}

\item Simplify: \(\log_2 16 + \log_2 2\).
	\begin{enumerate}[label={\Alph*.}]
	\item $3$
	\item $4$
	\item $5$
	\item $32$
	\end{enumerate}

\item If \(\log_{10} 6 = 0.7782\), find \(\log_{10} 60\).
	\begin{enumerate}[label={\Alph*.}]
	\item $1.5564$
	\item $1.7782$
	\item $7.782$
	\item $46.692$
	\end{enumerate}

\item Evaluate: \(\log_{10} \cfrac{1}{100}\).
	\begin{enumerate}[label={\Alph*.}]
	\item $-2$
	\item $-1$
	\item $\cfrac{1}{2}$
	\item $2$
	\end{enumerate}

\item Simplify: \(\log_9 81 + \log_9 9\).
	\begin{enumerate}[label={\Alph*.}]
	\item $2$
	\item $3$
	\item $4$
	\item $9$
	\end{enumerate}

\item If \(\log_7 x = 2\), find \(x\).
	\begin{enumerate}[label={\Alph*.}]
	\item $14$
	\item $49$
	\item $98$
	\item $343$
	\end{enumerate}

\item Evaluate: \(\log_3 \cfrac{1}{27}\).
	\begin{enumerate}[label={\Alph*.}]
	\item $-3$
	\item $-2$
	\item $\cfrac{1}{3}$
	\item $3$
	\end{enumerate}

\item Simplify: \(2\log_4 4 + \log_4 16\).
	\begin{enumerate}[label={\Alph*.}]
	\item $2$
	\item $3$
	\item $4$
	\item $6$
	\end{enumerate}

\item If \(\log_{10} 8 = 0.9031\), find \(\log_{10} 800\).
	\begin{enumerate}[label={\Alph*.}]
	\item $1.8062$
	\item $2.9031$
	\item $9.031$
	\item $722.48$
	\end{enumerate}

\item Evaluate: \(\log_6 \cfrac{1}{36}\).
	\begin{enumerate}[label={\Alph*.}]
	\item $-2$
	\item $-1$
	\item $\cfrac{1}{2}$
	\item $2$
	\end{enumerate}

\item Simplify: \(\log_{10} 1000 - \log_{10} 10\).
	\begin{enumerate}[label={\Alph*.}]
	\item $1$
	\item $2$
	\item $3$
	\item $100$
	\end{enumerate}

\item If \(\log_2 x = 6\), find \(x\).
	\begin{enumerate}[label={\Alph*.}]
	\item $8$
	\item $12$
	\item $32$
	\item $64$
	\end{enumerate}

\item Evaluate: \(4\log_2 2\).
	\begin{enumerate}[label={\Alph*.}]
	\item $2$
	\item $4$
	\item $8$
	\item $16$
	\end{enumerate}

\item Simplify: \(\log_5 125 + \log_5 5\).
	\begin{enumerate}[label={\Alph*.}]
	\item $2$
	\item $3$
	\item $4$
	\item $5$
	\end{enumerate}

\item If \(\log_x 125 = 3\), find \(x\).
	\begin{enumerate}[label={\Alph*.}]
	\item $5$
	\item $25$
	\item $125$
	\item $625$
	\end{enumerate}

\item Evaluate: \(\log_4 \cfrac{1}{16}\).
	\begin{enumerate}[label={\Alph*.}]
	\item $-2$
	\item $-1$
	\item $\cfrac{1}{4}$
	\item $4$
	\end{enumerate}

\item Simplify: \(\log_7 49 + \log_7 7\).
	\begin{enumerate}[label={\Alph*.}]
	\item $2$
	\item $3$
	\item $4$
	\item $7$
	\end{enumerate}

\item If \(\log_{10} 9 = 0.9542\), find \(\log_{10} 900\).
	\begin{enumerate}[label={\Alph*.}]
	\item $1.9084$
	\item $2.9542$
	\item $9.542$
	\item $858.78$
	\end{enumerate}

\item Evaluate: \(\log_8 \cfrac{1}{64}\).
	\begin{enumerate}[label={\Alph*.}]
	\item $-2$
	\item $-1$
	\item $\cfrac{1}{2}$
	\item $2$
	\end{enumerate}

\item Simplify: \(\log_3 243 - \log_3 27\).
	\begin{enumerate}[label={\Alph*.}]
	\item $1$
	\item $2$
	\item $3$
	\item $9$
	\end{enumerate}

\item If \(\log_6 x = 3\), find \(x\).
	\begin{enumerate}[label={\Alph*.}]
	\item $18$
	\item $36$
	\item $216$
	\item $1296$
	\end{enumerate}

\item Evaluate: \(\log_{10} \cfrac{1}{1000}\).
	\begin{enumerate}[label={\Alph*.}]
	\item $-3$
	\item $-2$
	\item $\cfrac{1}{3}$
	\item $3$
	\end{enumerate}

\item Simplify: \(3\log_5 5 + \log_5 1\).
	\begin{enumerate}[label={\Alph*.}]
	\item $1$
	\item $2$
	\item $3$
	\item $5$
	\end{enumerate}

\item If \(\log_x 64 = 3\), find \(x\).
	\begin{enumerate}[label={\Alph*.}]
	\item $2$
	\item $4$
	\item $8$
	\item $16$
	\end{enumerate}

\item Evaluate: \(\log_9 \cfrac{1}{81}\).
	\begin{enumerate}[label={\Alph*.}]
	\item $-2$
	\item $-1$
	\item $\cfrac{1}{2}$
	\item $2$
	\end{enumerate}

\item Simplify: \(\log_2 64 + \log_2 4\).
	\begin{enumerate}[label={\Alph*.}]
	\item $6$
	\item $8$
	\item $10$
	\item $256$
	\end{enumerate}

\item If \(\log_8 x = 2\), find \(x\).
	\begin{enumerate}[label={\Alph*.}]
	\item $16$
	\item $64$
	\item $128$
	\item $512$
	\end{enumerate}

\item Evaluate: \(5\log_3 3\).
	\begin{enumerate}[label={\Alph*.}]
	\item $3$
	\item $5$
	\item $15$
	\item $243$
	\end{enumerate}

\item Simplify: \(\log_6 216 - \log_6 36\).
	\begin{enumerate}[label={\Alph*.}]
	\item $0$
	\item $1$
	\item $2$
	\item $6$
	\end{enumerate}

\item If \(\log_{10} x = 2\), find \(x\).
	\begin{enumerate}[label={\Alph*.}]
	\item $2$
	\item $10$
	\item $20$
	\item $100$
	\end{enumerate}

\item Evaluate: \(\log_7 \cfrac{1}{343}\).
	\begin{enumerate}[label={\Alph*.}]
	\item $-3$
	\item $-2$
	\item $\cfrac{1}{3}$
	\item $3$
	\end{enumerate}

\item Simplify: \(2\log_6 6 + \log_6 36\).
	\begin{enumerate}[label={\Alph*.}]
	\item $2$
	\item $3$
	\item $4$
	\item $6$
	\end{enumerate}

\item If \(\log_5 x = 4\), find \(x\).
	\begin{enumerate}[label={\Alph*.}]
	\item $20$
	\item $125$
	\item $625$
	\item $3125$
	\end{enumerate}

\item Evaluate: \(\log_{10} \sqrt{10}\).
	\begin{enumerate}[label={\Alph*.}]
	\item $\cfrac{1}{2}$
	\item $1$
	\item $2$
	\item $5$
	\end{enumerate}

\item Simplify: \(\log_4 256 - \log_4 16\).
	\begin{enumerate}[label={\Alph*.}]
	\item $1$
	\item $2$
	\item $3$
	\item $4$
	\end{enumerate}

\item If \(\log_x 32 = 5\), find \(x\).
	\begin{enumerate}[label={\Alph*.}]
	\item $2$
	\item $4$
	\item $8$
	\item $16$
	\end{enumerate}

\item Evaluate: \(\log_5 \cfrac{1}{125}\).
	\begin{enumerate}[label={\Alph*.}]
	\item $-3$
	\item $-2$
	\item $\cfrac{1}{3}$
	\item $3$
	\end{enumerate}

\item Simplify: \(\log_8 512 + \log_8 8\).
	\begin{enumerate}[label={\Alph*.}]
	\item $3$
	\item $4$
	\item $5$
	\item $8$
	\end{enumerate}

\item If \(\log_9 x = 2\), find \(x\).
	\begin{enumerate}[label={\Alph*.}]
	\item $18$
	\item $81$
	\item $162$
	\item $729$
	\end{enumerate}

\item Evaluate: \(\log_{10} \sqrt[3]{10}\).
	\begin{enumerate}[label={\Alph*.}]
	\item $\cfrac{1}{3}$
	\item $1$
	\item $3$
	\item $10$
	\end{enumerate}

\item Simplify: \(3\log_2 2 + 2\log_2 4\).
	\begin{enumerate}[label={\Alph*.}]
	\item $5$
	\item $6$
	\item $7$
	\item $8$
	\end{enumerate}

\item If \(\log_3 x = 5\), find \(x\).
	\begin{enumerate}[label={\Alph*.}]
	\item $15$
	\item $81$
	\item $243$
	\item $729$
	\end{enumerate}

\item Evaluate: \(\log_{10} 0.01\).
	\begin{enumerate}[label={\Alph*.}]
	\item $-2$
	\item $-1$
	\item $0.01$
	\item $2$
	\end{enumerate}

\item Simplify: \(\log_5 625 - \log_5 25\).
	\begin{enumerate}[label={\Alph*.}]
	\item $1$
	\item $2$
	\item $3$
	\item $5$
	\end{enumerate}

\item If \(\log_x 1000 = 3\), find \(x\).
	\begin{enumerate}[label={\Alph*.}]
	\item $10$
	\item $100$
	\item $333$
	\item $1000$
	\end{enumerate}

\item Evaluate: \(\log_{10} \sqrt[4]{10}\).
	\begin{enumerate}[label={\Alph*.}]
	\item $\cfrac{1}{4}$
	\item $1$
	\item $2.5$
	\item $4$
	\end{enumerate}

\item Simplify: \(\log_7 2401 - \log_7 343\).
	\begin{enumerate}[label={\Alph*.}]
	\item $0$
	\item $1$
	\item $2$
	\item $7$
	\end{enumerate}

\item If \(\log_4 x = 4\), find \(x\).
	\begin{enumerate}[label={\Alph*.}]
	\item $16$
	\item $64$
	\item $128$
	\item $256$
	\end{enumerate}

\item Evaluate: \(\log_{10} 0.001\).
	\begin{enumerate}[label={\Alph*.}]
	\item $-3$
	\item $-2$
	\item $0.001$
	\item $3$
	\end{enumerate}

\item Simplify: \(2\log_3 9 + \log_3 3\).
	\begin{enumerate}[label={\Alph*.}]
	\item $3$
	\item $4$
	\item $5$
	\item $9$
	\end{enumerate}

\item If \(\log_{10} x = -1\), find \(x\).
	\begin{enumerate}[label={\Alph*.}]
	\item $0.01$
	\item $0.1$
	\item $1$
	\item $10$
	\end{enumerate}

\item Evaluate: \(\log_2 \sqrt{8}\).
	\begin{enumerate}[label={\Alph*.}]
	\item $\cfrac{3}{2}$
	\item $2$
	\item $3$
	\item $4$
	\end{enumerate}

\item Simplify: \(\log_9 729 + \log_9 9\).
	\begin{enumerate}[label={\Alph*.}]
	\item $3$
	\item $4$
	\item $5$
	\item $9$
	\end{enumerate}

\item If \(\log_7 x = 3\), find \(x\).
	\begin{enumerate}[label={\Alph*.}]
	\item $21$
	\item $49$
	\item $147$
	\item $343$
	\end{enumerate}

\item Evaluate: \(\log_{10} 0.0001\).
	\begin{enumerate}[label={\Alph*.}]
	\item $-4$
	\item $-3$
	\item $0.0001$
	\item $4$
	\end{enumerate}

\item Simplify: \(3\log_4 4 + \log_4 64\).
	\begin{enumerate}[label={\Alph*.}]
	\item $4$
	\item $5$
	\item $6$
	\item $8$
	\end{enumerate}

\item If \(\log_{10} x = -2\), find \(x\).
	\begin{enumerate}[label={\Alph*.}]
	\item $0.001$
	\item $0.01$
	\item $0.1$
	\item $1$
	\end{enumerate}

\item Evaluate: \(\log_3 \sqrt{27}\).
	\begin{enumerate}[label={\Alph*.}]
	\item $\cfrac{3}{2}$
	\item $2$
	\item $3$
	\item $9$
	\end{enumerate}

\item Simplify: \(\log_{10} 10000 - \log_{10} 100\).
	\begin{enumerate}[label={\Alph*.}]
	\item $1$
	\item $2$
	\item $3$
	\item $100$
	\end{enumerate}

\item If \(\log_2 x = 7\), find \(x\).
	\begin{enumerate}[label={\Alph*.}]
	\item $14$
	\item $64$
	\item $128$
	\item $256$
	\end{enumerate}

\item Evaluate: \(\log_{10} \sqrt{100}\).
	\begin{enumerate}[label={\Alph*.}]
	\item $\cfrac{1}{2}$
	\item $1$
	\item $2$
	\item $10$
	\end{enumerate}

\item Simplify: \(\log_6 1296 - \log_6 216\).
	\begin{enumerate}[label={\Alph*.}]
	\item $0$
	\item $1$
	\item $2$
	\item $6$
	\end{enumerate}
	
\item If \(\log_{10} 2 = 0.3010\) and \(\log_{10} 3 = 0.4771\), evaluate \(\log_{10} 18\).
	\begin{enumerate}[label={\Alph*.}]
	\item \(1.2552\)  
	\item \(1.2551\)  
	\item \(1.7781\)  
	\item\(1.7782\)
	\end{enumerate}
\end{enumerate}
\end{multicols}
\include{geometry}
\section{Co-ordinate Geometry}
\subsection{Questions}
\begin{multicols}{2}
\begin{enumerate}[label={\arabic*.}]
\item Find the distance between points $A(3, 4)$ and $B(6, 8)$
	\begin{enumerate}[label={\Alph*.}]
	\item $3$
	\item $4$
	\item $5$
	\item $7$
	\end{enumerate}

\item What is the midpoint of the line joining $(-2, 5)$ and $(4, -3)$?
	\begin{enumerate}[label={\Alph*.}]
	\item $(1, 1)$
	\item $(2, 2)$
	\item $(3, 4)$
	\item $(-1, -1)$
	\end{enumerate}

\item Find the gradient of the line passing through $(2, 3)$ and $(5, 9)$
	\begin{enumerate}[label={\Alph*.}]
	\item $1$
	\item $2$
	\item $3$
	\item $4$
	\end{enumerate}

\item The gradient of the line joining $(-2, 0)$ and $(0, -4)$ is
	\begin{enumerate}[label={\Alph*.}]
	\item $-2$
	\item $-1$
	\item $2$
	\item $4$
	\end{enumerate}

\item Find the distance between points $(0, 0)$ and $(3, 4)$
	\begin{enumerate}[label={\Alph*.}]
	\item $3$
	\item $4$
	\item $5$
	\item $7$
	\end{enumerate}

\item What is the midpoint between $(6, 8)$ and $(2, 4)$?
	\begin{enumerate}[label={\Alph*.}]
	\item $(4, 6)$
	\item $(8, 12)$
	\item $(2, 2)$
	\item $(3, 5)$
	\end{enumerate}

\item Find the gradient of the line $2y = 4x + 6$
	\begin{enumerate}[label={\Alph*.}]
	\item $1$
	\item $2$
	\item $3$
	\item $4$
	\end{enumerate}

\item The distance between $(1, 2)$ and $(4, 6)$ is
	\begin{enumerate}[label={\Alph*.}]
	\item $3$
	\item $4$
	\item $5$
	\item $7$
	\end{enumerate}

\item Find the equation of a line with gradient $3$ passing through $(0, 2)$
	\begin{enumerate}[label={\Alph*.}]
	\item $y = 3x + 2$
	\item $y = 2x + 3$
	\item $y = 3x - 2$
	\item $y = x + 5$
	\end{enumerate}

\item What is the midpoint of $(10, 14)$ and $(6, 2)$?
	\begin{enumerate}[label={\Alph*.}]
	\item $(8, 8)$
	\item $(16, 16)$
	\item $(4, 6)$
	\item $(2, 3)$
	\end{enumerate}

\item Find the gradient of the line passing through $(0, 5)$ and $(3, 11)$
	\begin{enumerate}[label={\Alph*.}]
	\item $1$
	\item $2$
	\item $3$
	\item $6$
	\end{enumerate}

\item The distance between $(-3, 4)$ and $(3, 4)$ is
	\begin{enumerate}[label={\Alph*.}]
	\item $0$
	\item $3$
	\item $6$
	\item $8$
	\end{enumerate}

\item Find the gradient of the line $3y + 6x = 9$
	\begin{enumerate}[label={\Alph*.}]
	\item $-2$
	\item $-1$
	\item $2$
	\item $3$
	\end{enumerate}

\item What is the midpoint between $(5, 7)$ and $(9, 11)$?
	\begin{enumerate}[label={\Alph*.}]
	\item $(7, 9)$
	\item $(14, 18)$
	\item $(4, 4)$
	\item $(2, 2)$
	\end{enumerate}

\item Find the distance between $(2, 1)$ and $(5, 5)$
	\begin{enumerate}[label={\Alph*.}]
	\item $3$
	\item $4$
	\item $5$
	\item $6$
	\end{enumerate}

\item The gradient of a line parallel to $y = 4x + 1$ is
	\begin{enumerate}[label={\Alph*.}]
	\item $1$
	\item $2$
	\item $4$
	\item $-4$
	\end{enumerate}

\item Find the midpoint of $(0, 0)$ and $(8, 6)$
	\begin{enumerate}[label={\Alph*.}]
	\item $(4, 3)$
	\item $(8, 6)$
	\item $(2, 1.5)$
	\item $(16, 12)$
	\end{enumerate}

\item What is the gradient of the line passing through $(1, 1)$ and $(4, 7)$?
	\begin{enumerate}[label={\Alph*.}]
	\item $1$
	\item $2$
	\item $3$
	\item $6$
	\end{enumerate}

\item Find the distance between $(-1, -1)$ and $(2, 3)$
	\begin{enumerate}[label={\Alph*.}]
	\item $3$
	\item $4$
	\item $5$
	\item $6$
	\end{enumerate}

\item The equation of a line with gradient $2$ and y-intercept $-3$ is
	\begin{enumerate}[label={\Alph*.}]
	\item $y = 2x - 3$
	\item $y = -3x + 2$
	\item $y = 2x + 3$
	\item $y = 3x - 2$
	\end{enumerate}

\item Find the midpoint between $(-4, 6)$ and $(10, -2)$
	\begin{enumerate}[label={\Alph*.}]
	\item $(3, 2)$
	\item $(6, 4)$
	\item $(7, 4)$
	\item $(14, 8)$
	\end{enumerate}

\item What is the gradient of the line $y = -3x + 7$?
	\begin{enumerate}[label={\Alph*.}]
	\item $-3$
	\item $3$
	\item $7$
	\item $-7$
	\end{enumerate}

\item Find the distance between $(5, 0)$ and $(0, 12)$
	\begin{enumerate}[label={\Alph*.}]
	\item $11$
	\item $12$
	\item $13$
	\item $17$
	\end{enumerate}

\item The gradient of a line perpendicular to $y = 2x + 5$ is
	\begin{enumerate}[label={\Alph*.}]
	\item $2$
	\item $-2$
	\item $\cfrac{1}{2}$
	\item $-\cfrac{1}{2}$
	\end{enumerate}

\item Find the midpoint of $(7, 9)$ and $(3, 5)$
	\begin{enumerate}[label={\Alph*.}]
	\item $(5, 7)$
	\item $(10, 14)$
	\item $(4, 4)$
	\item $(2, 2)$
	\end{enumerate}

\item What is the gradient of the line passing through $(2, 5)$ and $(2, 9)$?
	\begin{enumerate}[label={\Alph*.}]
	\item $0$
	\item $1$
	\item $\text{undefined}$
	\item $4$
	\end{enumerate}

\item Find the distance between $(6, 8)$ and $(6, 2)$
	\begin{enumerate}[label={\Alph*.}]
	\item $0$
	\item $4$
	\item $6$
	\item $10$
	\end{enumerate}

\item The equation of a line passing through $(3, 4)$ with gradient $-1$ is
	\begin{enumerate}[label={\Alph*.}]
	\item $y = -x + 7$
	\item $y = x + 1$
	\item $y = -x - 1$
	\item $y = x + 7$
	\end{enumerate}

\item Find the midpoint between $(12, 8)$ and $(4, 16)$
	\begin{enumerate}[label={\Alph*.}]
	\item $(8, 12)$
	\item $(16, 24)$
	\item $(6, 6)$
	\item $(4, 4)$
	\end{enumerate}

\item What is the gradient of a horizontal line?
	\begin{enumerate}[label={\Alph*.}]
	\item $0$
	\item $1$
	\item $-1$
	\item $\text{undefined}$
	\end{enumerate}

\item Find the distance between $(8, 15)$ and $(8, 3)$
	\begin{enumerate}[label={\Alph*.}]
	\item $0$
	\item $8$
	\item $12$
	\item $18$
	\end{enumerate}

\item The gradient of the line $4y = 8x - 12$ is
	\begin{enumerate}[label={\Alph*.}]
	\item $1$
	\item $2$
	\item $4$
	\item $8$
	\end{enumerate}

\item Find the midpoint of $(-5, -3)$ and $(7, 9)$
	\begin{enumerate}[label={\Alph*.}]
	\item $(1, 3)$
	\item $(2, 6)$
	\item $(6, 6)$
	\item $(12, 12)$
	\end{enumerate}

\item What is the gradient of the line passing through $(3, 2)$ and $(7, 2)$?
	\begin{enumerate}[label={\Alph*.}]
	\item $0$
	\item $1$
	\item $4$
	\item $\text{undefined}$
	\end{enumerate}

\item Find the distance between $(0, 5)$ and $(12, 0)$
	\begin{enumerate}[label={\Alph*.}]
	\item $11$
	\item $12$
	\item $13$
	\item $17$
	\end{enumerate}

\item The equation of a line with gradient $\cfrac{1}{2}$ passing through $(4, 3)$ is
	\begin{enumerate}[label={\Alph*.}]
	\item $y = \cfrac{1}{2}x + 1$
	\item $y = 2x - 5$
	\item $y = \cfrac{1}{2}x - 1$
	\item $y = x + 1$
	\end{enumerate}

\item Find the midpoint between $(9, 12)$ and $(15, 4)$
	\begin{enumerate}[label={\Alph*.}]
	\item $(12, 8)$
	\item $(24, 16)$
	\item $(6, 8)$
	\item $(3, 4)$
	\end{enumerate}

\item What is the gradient of a vertical line?
	\begin{enumerate}[label={\Alph*.}]
	\item $0$
	\item $1$
	\item $-1$
	\item $\text{undefined}$
	\end{enumerate}

\item Find the distance between $(-2, -3)$ and $(4, 5)$
	\begin{enumerate}[label={\Alph*.}]
	\item $8$
	\item $9$
	\item $10$
	\item $11$
	\end{enumerate}

\item The gradient of the line $5y - 10x = 15$ is
	\begin{enumerate}[label={\Alph*.}]
	\item $-2$
	\item $2$
	\item $5$
	\item $10$
	\end{enumerate}

\item Find the midpoint of $(20, 30)$ and $(10, 10)$
	\begin{enumerate}[label={\Alph*.}]
	\item $(15, 20)$
	\item $(30, 40)$
	\item $(10, 20)$
	\item $(5, 10)$
	\end{enumerate}

\item What is the gradient of the line passing through $(-1, 4)$ and $(2, 10)$?
	\begin{enumerate}[label={\Alph*.}]
	\item $1$
	\item $2$
	\item $3$
	\item $6$
	\end{enumerate}

\item Find the distance between $(7, 24)$ and $(0, 0)$
	\begin{enumerate}[label={\Alph*.}]
	\item $24$
	\item $25$
	\item $31$
	\item $7$
	\end{enumerate}

\item The gradient of a line perpendicular to $y = -\cfrac{1}{3}x + 4$ is
	\begin{enumerate}[label={\Alph*.}]
	\item $-3$
	\item $-\cfrac{1}{3}$
	\item $\cfrac{1}{3}$
	\item $3$
	\end{enumerate}

\item Find the midpoint between $(8, 14)$ and $(12, 6)$
	\begin{enumerate}[label={\Alph*.}]
	\item $(10, 10)$
	\item $(20, 20)$
	\item $(4, 8)$
	\item $(2, 4)$
	\end{enumerate}

\item What is the gradient of the line $y - 3x = 6$?
	\begin{enumerate}[label={\Alph*.}]
	\item $-3$
	\item $3$
	\item $6$
	\item $-6$
	\end{enumerate}

\item Find the distance between $(9, 40)$ and $(0, 0)$
	\begin{enumerate}[label={\Alph*.}]
	\item $40$
	\item $41$
	\item $49$
	\item $9$
	\end{enumerate}

\item The equation of a line passing through the origin with gradient $5$ is
	\begin{enumerate}[label={\Alph*.}]
	\item $y = 5x$
	\item $y = x + 5$
	\item $y = 5x + 1$
	\item $y = -5x$
	\end{enumerate}

\item Find the midpoint of $(-6, 8)$ and $(14, -4)$
	\begin{enumerate}[label={\Alph*.}]
	\item $(4, 2)$
	\item $(8, 4)$
	\item $(10, 6)$
	\item $(20, 12)$
	\end{enumerate}

\item What is the gradient of the line passing through $(5, 1)$ and $(1, 5)$?
	\begin{enumerate}[label={\Alph*.}]
	\item $1$
	\item $-1$
	\item $4$
	\item $-4$
	\end{enumerate}

\item Find the distance between $(3, 7)$ and $(6, 11)$
	\begin{enumerate}[label={\Alph*.}]
	\item $3$
	\item $4$
	\item $5$
	\item $7$
	\end{enumerate}

\item The gradient of a line parallel to $y = -5x + 2$ is
	\begin{enumerate}[label={\Alph*.}]
	\item $-5$
	\item $5$
	\item $2$
	\item $-2$
	\end{enumerate}

\item Find the midpoint between $(18, 24)$ and $(6, 8)$
	\begin{enumerate}[label={\Alph*.}]
	\item $(12, 16)$
	\item $(24, 32)$
	\item $(6, 8)$
	\item $(3, 4)$
	\end{enumerate}

\item What is the gradient of the line $6y + 12x = 18$?
	\begin{enumerate}[label={\Alph*.}]
	\item $-2$
	\item $2$
	\item $6$
	\item $-6$
	\end{enumerate}

\item Find the distance between $(10, 24)$ and $(10, 4)$
	\begin{enumerate}[label={\Alph*.}]
	\item $0$
	\item $10$
	\item $20$
	\item $28$
	\end{enumerate}

\item The equation of a line with gradient $-3$ and y-intercept $5$ is
	\begin{enumerate}[label={\Alph*.}]
	\item $y = -3x + 5$
	\item $y = 3x + 5$
	\item $y = -3x - 5$
	\item $y = 5x - 3$
	\end{enumerate}

\item Find the midpoint of $(25, 35)$ and $(15, 15)$
	\begin{enumerate}[label={\Alph*.}]
	\item $(20, 25)$
	\item $(40, 50)$
	\item $(10, 20)$
	\item $(5, 10)$
	\end{enumerate}

\item What is the gradient of the line passing through $(0, 3)$ and $(4, 11)$?
	\begin{enumerate}[label={\Alph*.}]
	\item $1$
	\item $2$
	\item $3$
	\item $4$
	\end{enumerate}

\item Find the distance between $(8, 6)$ and $(2, -2)$
	\begin{enumerate}[label={\Alph*.}]
	\item $6$
	\item $8$
	\item $10$
	\item $14$
	\end{enumerate}

\item The gradient of a line perpendicular to $y = \cfrac{2}{3}x - 1$ is
	\begin{enumerate}[label={\Alph*.}]
	\item $\cfrac{2}{3}$
	\item $-\cfrac{2}{3}$
	\item $\cfrac{3}{2}$
	\item $-\cfrac{3}{2}$
	\end{enumerate}

\item Find the midpoint between $(5, 13)$ and $(11, 7)$
	\begin{enumerate}[label={\Alph*.}]
	\item $(8, 10)$
	\item $(16, 20)$
	\item $(6, 6)$
	\item $(3, 3)$
	\end{enumerate}

\item What is the gradient of the line $2y = -6x + 10$?
	\begin{enumerate}[label={\Alph*.}]
	\item $-3$
	\item $3$
	\item $-6$
	\item $2$
	\end{enumerate}

\item Find the distance between $(15, 20)$ and $(15, 8)$
	\begin{enumerate}[label={\Alph*.}]
	\item $0$
	\item $12$
	\item $15$
	\item $28$
	\end{enumerate}

\item The equation of a line passing through $(2, 1)$ with gradient $4$ is
	\begin{enumerate}[label={\Alph*.}]
	\item $y = 4x - 7$
	\item $y = 4x + 7$
	\item $y = 4x - 1$
	\item $y = 2x + 4$
	\end{enumerate}

\item Find the midpoint of $(30, 40)$ and $(10, 20)$
	\begin{enumerate}[label={\Alph*.}]
	\item $(20, 30)$
	\item $(40, 60)$
	\item $(15, 15)$
	\item $(10, 10)$
	\end{enumerate}

\item What is the gradient of the line passing through $(6, 2)$ and $(10, 10)$?
	\begin{enumerate}[label={\Alph*.}]
	\item $1$
	\item $2$
	\item $4$
	\item $8$
	\end{enumerate}

\item Find the distance between $(5, 12)$ and $(0, 0)$
	\begin{enumerate}[label={\Alph*.}]
	\item $12$
	\item $13$
	\item $17$
	\item $5$
	\end{enumerate}

\item The gradient of a line parallel to $y = \cfrac{1}{4}x - 3$ is
	\begin{enumerate}[label={\Alph*.}]
	\item $\cfrac{1}{4}$
	\item $-\cfrac{1}{4}$
	\item $4$
	\item $-4$
	\end{enumerate}

\item Find the midpoint between $(22, 18)$ and $(14, 10)$
	\begin{enumerate}[label={\Alph*.}]
	\item $(18, 14)$
	\item $(36, 28)$
	\item $(8, 8)$
	\item $(4, 4)$
	\end{enumerate}

\item What is the gradient of the line $3y + 9x = 12$?
	\begin{enumerate}[label={\Alph*.}]
	\item $-3$
	\item $3$
	\item $-9$
	\item $9$
	\end{enumerate}

\item Find the distance between $(20, 21)$ and $(20, 5)$
	\begin{enumerate}[label={\Alph*.}]
	\item $0$
	\item $16$
	\item $20$
	\item $26$
	\end{enumerate}

\item The equation of a line with gradient $\cfrac{3}{4}$ passing through $(8, 5)$ is
	\begin{enumerate}[label={\Alph*.}]
	\item $y = \cfrac{3}{4}x - 1$
	\item $y = \cfrac{3}{4}x + 1$
	\item $y = \cfrac{4}{3}x - 1$
	\item $y = 3x + 4$
	\end{enumerate}

\item Find the midpoint of $(16, 22)$ and $(8, 14)$
	\begin{enumerate}[label={\Alph*.}]
	\item $(12, 18)$
	\item $(24, 36)$
	\item $(8, 8)$
	\item $(4, 4)$
	\end{enumerate}

\item What is the gradient of the line passing through $(7, 5)$ and $(7, 15)$?
	\begin{enumerate}[label={\Alph*.}]
	\item $0$
	\item $1$
	\item $10$
	\item $\text{undefined}$
	\end{enumerate}

\item Find the distance between $(9, 12)$ and $(0, 0)$
	\begin{enumerate}[label={\Alph*.}]
	\item $12$
	\item $15$
	\item $21$
	\item $9$
	\end{enumerate}

\item The gradient of a line perpendicular to $y = 4x + 7$ is
	\begin{enumerate}[label={\Alph*.}]
	\item $4$
	\item $-4$
	\item $\cfrac{1}{4}$
	\item $-\cfrac{1}{4}$
	\end{enumerate}

\item Find the midpoint between $(28, 36)$ and $(12, 20)$
	\begin{enumerate}[label={\Alph*.}]
	\item $(20, 28)$
	\item $(40, 56)$
	\item $(16, 16)$
	\item $(8, 8)$
	\end{enumerate}

\item What is the gradient of the line $y + 5x = 15$?
	\begin{enumerate}[label={\Alph*.}]
	\item $-5$
	\item $5$
	\item $15$
	\item $-15$
	\end{enumerate}

\item Find the distance between $(24, 7)$ and $(0, 0)$
	\begin{enumerate}[label={\Alph*.}]
	\item $24$
	\item $25$
	\item $31$
	\item $7$
	\end{enumerate}

\item The equation of a line passing through $(5, 6)$ with gradient $-2$ is
	\begin{enumerate}[label={\Alph*.}]
	\item $y = -2x + 16$
	\item $y = 2x - 4$
	\item $y = -2x - 4$
	\item $y = -2x + 6$
	\end{enumerate}

\item Find the midpoint of $(35, 45)$ and $(25, 25)$
	\begin{enumerate}[label={\Alph*.}]
	\item $(30, 35)$
	\item $(60, 70)$
	\item $(10, 20)$
	\item $(5, 10)$
	\end{enumerate}

\item What is the gradient of the line passing through $(1, 8)$ and $(5, 20)$?
	\begin{enumerate}[label={\Alph*.}]
	\item $2$
	\item $3$
	\item $4$
	\item $12$
	\end{enumerate}

\item Find the distance between $(6, 8)$ and $(1, -4)$
	\begin{enumerate}[label={\Alph*.}]
	\item $11$
	\item $12$
	\item $13$
	\item $17$
	\end{enumerate}

\item The gradient of a line parallel to $y = -7x + 3$ is
	\begin{enumerate}[label={\Alph*.}]
	\item $-7$
	\item $7$
	\item $3$
	\item $-3$
	\end{enumerate}

\item Find the midpoint between $(40, 50)$ and $(20, 30)$
	\begin{enumerate}[label={\Alph*.}]
	\item $(30, 40)$
	\item $(60, 80)$
	\item $(20, 20)$
	\item $(10, 10)$
	\end{enumerate}

\item What is the gradient of the line $4y - 8x = 20$?
	\begin{enumerate}[label={\Alph*.}]
	\item $-2$
	\item $2$
	\item $4$
	\item $-4$
	\end{enumerate}

\item Find the distance between $(13, 84)$ and $(13, 0)$
	\begin{enumerate}[label={\Alph*.}]
	\item $0$
	\item $13$
	\item $84$
	\item $97$
	\end{enumerate}

\item The equation of a line with gradient $6$ and y-intercept $-2$ is
	\begin{enumerate}[label={\Alph*.}]
	\item $y = 6x - 2$
	\item $y = -6x + 2$
	\item $y = 6x + 2$
	\item $y = 2x - 6$
	\end{enumerate}

\item Find the midpoint of $(50, 60)$ and $(30, 40)$
	\begin{enumerate}[label={\Alph*.}]
	\item $(40, 50)$
	\item $(80, 100)$
	\item $(20, 20)$
	\item $(10, 10)$
	\end{enumerate}

\item What is the gradient of the line passing through $(3, 10)$ and $(9, 22)$?
	\begin{enumerate}[label={\Alph*.}]
	\item $1$
	\item $2$
	\item $3$
	\item $4$
	\end{enumerate}

\item Find the distance between $(11, 60)$ and $(0, 0)$
	\begin{enumerate}[label={\Alph*.}]
	\item $60$
	\item $61$
	\item $71$
	\item $11$
	\end{enumerate}

\item The gradient of a line perpendicular to $y = -\cfrac{5}{2}x + 1$ is
	\begin{enumerate}[label={\Alph*.}]
	\item $-\cfrac{5}{2}$
	\item $\cfrac{5}{2}$
	\item $-\cfrac{2}{5}$
	\item $\cfrac{2}{5}$
	\end{enumerate}

\item Find the midpoint between $(45, 55)$ and $(35, 35)$
	\begin{enumerate}[label={\Alph*.}]
	\item $(40, 45)$
	\item $(80, 90)$
	\item $(10, 20)$
	\item $(5, 10)$
	\end{enumerate}

\item What is the gradient of the line $7y + 14x = 21$?
	\begin{enumerate}[label={\Alph*.}]
	\item $-2$
	\item $2$
	\item $7$
	\item $-7$
	\end{enumerate}

\item Find the coordinate of the midpoint of the line joining \(P(-3, 5)\) and \(Q(5, -3)\).
	\begin{enumerate}[label={\Alph*.}]
	\item \((1, 2)\)
	\item \((2, 1)\)
	\item \((1, 1)\)
	\item \((2, 2)\)
	\end{enumerate}

\item Find the gradient of the line joining \((2, 7)\) and \((5, 1)\).
	\begin{enumerate}[label={\Alph*.}]
	\item \(2\)
	\item \(-2\)
	\item \(3\)
	\item \(-3\)
	\end{enumerate}

\item \(P(-6, 1)\) and \(Q(6, 6)\) are the two ends of the diameter of a given circle. Calculate the radius.
	\begin{enumerate}[label={\Alph*.}]
	\item \(6.5\) units
	\item \(13.0\) units
	\item \(3.5\) units
	\item \(7.0\) units
	\end{enumerate}

\item Find the distance between the points \((4, 3)\) and \((1, -1)\).
	\begin{enumerate}[label={\Alph*.}]
	\item \(3\)
	\item \(4\)
	\item \(5\)
	\item \(6\)
	\end{enumerate}

\item Find the equation of the line passing through $(2, 3)$ with gradient $-2$.
	\begin{enumerate}[label={\Alph*.}]
	\item $y = -2x + 7$
	\item $y = -2x - 7$
	\item $y = 2x + 7$
	\item $y = 2x - 7$
	\end{enumerate}

\item Calculate the area of a triangle with vertices at 0 0$, 5 0$ and 0 8.
	\begin{enumerate}[label={\Alph*.}]
	\item $ square units
	\item $ square units
	\item $ square units
	\item $ square units
	\end{enumerate}

\end{enumerate}
\end{multicols}

\chapter{Calculus}
\section{Differentiation}
\subsection{Questions}
\begin{multicols}{2}
\begin{enumerate}[label={\arabic*.}]
  \item The minimum point on the curve \(y = x^2 - 6x + 5\) is at?
        \begin{enumerate}[label={\Alph*.}]
            \item (1,5)
            \item (2,3)
            \item (3,4)
            \item (-3,4)
            \item (3,-4)
        \end{enumerate}
  \item At what value of \(x\) is the function \(y = x^2 + x + 1\) minimum? 
        \begin{enumerate}[label={\Alph*.}]
            \item \(-1\)
            \item \(-\frac{1}{2}\)
            \item \(\frac{1}{2}\)
            \item \(1\)
        \end{enumerate}
  \item At what value of \(x\) is the function \(y = x^2 - 2x - 3\) minimum?
    \begin{enumerate}[label={\Alph*.}]
            \item \(1\) 
            \item  \(-1\) 
            \item \(-4\)
            \item \(4\)
        \end{enumerate}
  \item Find the maximum value of \(y = x^2 - 2x - 3\)
        \begin{enumerate}[label={\Alph*.}]
			\item \(-4\)
			\item \(-1\)
			\item \(1\)
			\item \(4\)
        \end{enumerate}
  \item Find the maximum value of \(y = 3x^2 - x^3\)
        \begin{enumerate}[label={\Alph*.}]
            \item \(2\)
            \item \(4\)
            \item \(6\)
            \item \(0\)
        \end{enumerate}
  \item Find the minimum value of \(y = x^3 + x^2 - x + 1\)
        \begin{enumerate}[label={\Alph*.}]
            \item \(\)
			\item \(\)
			\item \(\)
			\item \(\)
			\item \(\)
        \end{enumerate}
  \item Find the value of \(x\) for which the function \(f(x) = 2x^3 - x^2 -4x + 4\) has a maximum value.
        \begin{enumerate}[label={\Alph*.}]
            \item \(\frac{2}{3}\)
            \item \(1\)
            \item \(-1\)
            \item \(-\frac{2}{3}\)
        \end{enumerate}
  \item Find the value of \(x\) for which the function \(f(x) = 3x^3-9x^2\) is minimum
        \begin{enumerate}[label={\Alph*.}]
            \item \(2\)
            \item \(0\)
            \item \(5\)
            \item \(3\)
        \end{enumerate}
  \item Find the maximum value of the function \(f(x) = 2 + x - x^2\)
        \begin{enumerate}[label={\Alph*.}]
            \item \(\frac{9}{4}\)
            \item \(\frac{7}{4}\)
            \item \(\frac{3}{2}\)
            \item \(\frac{1}{2}\)
        \end{enumerate}
  \item Find the maximum value of \(y\) in the equation: \(y = 1 - 2 - 3x^2\)
        \begin{enumerate}[label={\Alph*.}]
            \item \(\frac{4}{3}\)
            \item \(\frac{5}{4}\)
            \item \(\frac{3}{4}\)
            \item \(\frac{5}{3}\)
        \end{enumerate}
  \item The minimum value of \(y\) in the equation: \(y = x^2 - 6x + 8\) is
        \begin{enumerate}[label={\Alph*.}]
            \item \(8\)
            \item \(3\)
            \item \(0\)
            \item \(-1\)
        \end{enumerate}
  \item Obtain a maximum value of the function: \(f(x) = x^3 - 12x + 11\)
        \begin{enumerate}[label={\Alph*.}]
            \item \(-5\)
            \item \(-2\)
            \item \(2\)
            \item \(27\)
        \end{enumerate}
  \item Find the value of \(h\) if the maximum value of \(y = 1 + hx - 3x^2\) is 13.
        \begin{enumerate}[label={\Alph*.}]
            \item \(10\)
            \item \(11\)
            \item \(12\)
            \item \(13\)
        \end{enumerate}
  \item A trader realizes \(10x - x^2\) naira profit from the sale of \(x\) bags of corn. How many bags will give him the maximum profit?
        \begin{enumerate}[label={\Alph*.}]
            \item \(4\)
            \item \(5\)
            \item \(6\)
            \item \(7\)
        \end{enumerate}
  \item Find the value of \(x\) for which the function \(y = x^3 - x\) has a minimum value.
        \begin{enumerate}[label={\Alph*.}]
            \item \(\frac{\sqrt{3}}{3}\)
            \item -\(\frac{\sqrt{3}}{3}\)
            \item \(\sqrt{3}\)
            \item -\(\sqrt{3}\)
        \end{enumerate}
  \item If \(f(x) = x^2 - 2x - 3\), find the least value of \(f(x)\) and the corresponding value of \(x\).
        \begin{enumerate}[label={\Alph*.}]
            \item \(f(x) = -3, x = 1\)
            \item \(f(x) = -3, x = 3\)
            \item \(f(x) = 1, x = -4\)
            \item \(f(x) = 1, x = -4\)
        \end{enumerate}
  \item If \(y = 3 \cos\left(\frac{x}{3}\right)\), find \(\dv{y}{x}\) when \(x = \frac{3\pi}{2}\).
        \begin{enumerate}[label={\Alph*.}]
            \item \(-1\)
            \item \(1\)
            \item \(2\)
            \item  \(3\)
        \end{enumerate}
  \item What is the rate of change of the volume \(v\) of a hemisphere with respect to its radius \(r\) when \(r = 2\)?
        \begin{enumerate}[label={\Alph*.}]
            \item  \(2\pi\)
            \item  \(4\pi\)
            \item  \(8\pi\)
            \item  \(16\pi\)
        \end{enumerate}
  \item If \(y = (1-2x)^3\), find the value of \(\dv{y}{x}\) at \(x = -1\).
         \begin{enumerate}[label={\Alph*.}]
            \item  \(22\)
            \item \(57\)
            \item \(-6\)
            \item \(-54\)
        \end{enumerate}
  \item Find the derivative of \(y = \sin(2x^3+3x-4)\).
      \begin{enumerate}[label={\Alph*.}]
            \item \(\cos(2x^2+3x-4)\)
            \item \(-\cos(2x^2+3x-4)\)
            \item  \(-(6x^2+3)\cos(2x^2+3x-4)\)
            \item \((6x^2 + 3) \cos(2x^2+3x-4)\)
        \end{enumerate}
  \item The radius \(r\) of a circular disc is increasing at the rate of \(0.5\ \text{cm/sec}\). At what rate is the area of the disc increasing when its radius is \(6\ \text{cm}\)?
       \begin{enumerate}[label={\Alph*.}]
            \item  \(3\pi\ \text{cm}^2/\text{sec}\)
            \item  \(18\pi\ \text{cm}^2/\text{sec}\)
            \item  \(6\pi\ \text{cm}^2/\text{sec}\)
            \item \(36\pi\ \text{cm}^2/\text{sec}\)
        \end{enumerate}
  \item Find \(\dv{y}{x}\), if \(y = \cos x\).
       \begin{enumerate}[label={\Alph*.}]
            \item  \(\sin x\)
            \item  \(-\sin x\)
            \item  \(\tan x\)
            \item  \(-\tan x\)
        \end{enumerate}
  \item Differentiate: \((\cos \theta - \sin \theta)^2\) with respect to \(\theta\).
     \begin{enumerate}[label={\Alph*.}]
            \item \(1-2\cos 2\theta\)
            \item  \(-2\sin 2\theta\)
            \item  \(-2\cos 2\theta\)
            \item  \(1-2\sin 2\theta\)
        \end{enumerate}
  \item Differentiate: \(\left(x^2 + \frac{1}{x}\right)^2\) with respect to \(x\).
       \begin{enumerate}[label={\Alph*.}]
            \item \(4x^3 - 2 + \frac{2}{x^3}\)
            \item  \(4x^3 - 2 - \frac{2}{x^3}\)
            \item   \(4x^3 - 4x - \frac{2}{x}\)
            \item    \(4x^3 - 3x + \frac{2}{x}\)
        \end{enumerate}
\item Find the point \(x,y\) on the Euclidean plane where the curve \(y = 2x^{2} - 2x +3\) has \(2\) as the gradient.
    \begin{enumerate}[label={\Alph*.}]
        \item \((1, 4)\)
        \item \((2, 2)\)
        \item \((3, 4)\)
        \item \((3, 2)\)
    \end{enumerate}  
\item For what value of \(x\) is the tangent to the curve \(y = x^{2} - 4x + 3\) parallel to the \(x\)-axis?
    \begin{enumerate}[label={\Alph*.}]
        \item \(0\)
        \item \(1\)
        \item \(2\)
        \item \(3\)
    \end{enumerate}   
\item If \(y = x \sin x\), find \( \frac{d^2y}{dx^2} \).
	\begin{enumerate}[label={\Alph*.}]
	\item \(2\cos x - \sin x\)
	\item \(\sin x + \cos x\)
	\item \(\sin x  - \cos  x\)
	\item \( \cos x  - 2\sin x \)
	\end{enumerate}
\item Differentiate: \(\frac{6x^{3} - 5x^{2} + 1}{3x^{2}}\) with respect to \(x\)
	\begin{enumerate}[label={\Alph*.}]
	\item \(2 + \frac{2}{3{x}^{3}}\)
	\item \(2 + \frac{1}{6x}\)
	\item \(\sin x - \cos x\)
	\item \(\cos x - 2\sin x\)
	\end{enumerate}
\item If \(y = (1+x)^{2}\), find \( \dv{y}{x} \).
	\begin{enumerate}[label={\Alph*.}]
	\item \(x+1\)
	\item \(2x-1\)
	\item \(2 + 2x\)
	\item \(1+2x\)
	\end{enumerate}
\item Differentiate: \(3x^3+2x^2+3x+1\) with respect to \(x\)
	\begin{enumerate}[label={\Alph*.}]
	\item \(9x^2+4x+3\)
	\item \(9x^2+4x-3\)
	\item \(9x^2-4x-3\)
	\item \(9x^2-4x+3\)
	\end{enumerate}
\item Differentiate: \(\frac{2}{3}x^{3}-\frac{4}{x}\)
	\begin{enumerate}[label={\Alph*.}]
	\item \(2x^2 + \frac{4}{x^2}\)
	\item \(2x^{2}-\frac{4}{x}\)
	\item \(3x^{2}-\frac{4}{x}\)
	\item \(3x^{2}+\frac{4}{x^2}\)
	\end{enumerate}
\item Find the derivative of \(\frac{\sin {x}}{\cos {x}}\)
	\begin{enumerate}[label={\Alph*.}]
	\item \(\tan{x}\cos{x}\)
	\item \(\csc{x}\sec{x}\)
	\item \(\sec^{2}{x}\)
	\item \(\cot^{2}{x}\)
	\end{enumerate}
\item If \(y=x^2-3x+4\), find \(\dv{y}{x}\) at \(x=5\).
	\begin{enumerate}[label={\Alph*.}]
	\item \(9\)
	\item \(7\)
	\item \(5\)
	\item \(3\)
	\end{enumerate}
\item If \(y= 2x\cos{2x} -\sin{2x}\), find \(\dv{y}{x}\) when \(x=\frac{\pi}{4}\).
	\begin{enumerate}[label={\Alph*.}]
	\item \(9\)
	\item \(7\)
	\item \(5\)
	\item \(3\)
	\end{enumerate}
\item If \(y=3\cos{4x}\), find \(\dv{y}{x}\)
	\begin{enumerate}[label={\Alph*.}]
	\item \(-24\sin{4x}\)
	\item \(12\sin{4x}\)
	\item \(-12 \sin{4x}\)
	\item \(6\sin{8x}\)
	\end{enumerate}
\item Find the derivative of \((2+3x)(1-x)\) with respect to \(x\).
	\begin{enumerate}[label={\Alph*.}]
	\item \(6x-1\)
	\item \(1-6x\)
	\item \(-3\)
	\item \(6\)
	\end{enumerate}
\item Find \(\dv{y}{x}\), if \(y = 3{x}^{3}+2{x}^{2}+3x+1\).
	\begin{enumerate}[label={\Alph*.}]
	\item \(9{x}^{2}+4{x}+3\)
	\item \(9{x}^{2}+4{x}-3\)
	\item \(9{x}^{2}-4{x}+3\)
	\item \(9{x}^{2}-4{x}-3\)
	\end{enumerate}
\item If \(y = 2{x}^{3}+6{x}^{2}+6x+1\), find \(\dv{y}{x}\).
	\begin{enumerate}[label={\Alph*.}]
	\item \(6{x}^{2}+12{x}+1\)
	\item \(6{x}^{2}-12{x}+1\)
	\item \(6{x}^{2}+12{x}+6\)
	\item \(6{x}^{2}+6{x}+6\)
	\end{enumerate}
\item Find the derivative of \(y ={\left(\dfrac{1}{3}x + 6\right)}^{2}\).
	\begin{enumerate}[label={\Alph*.}]
	\item \(2{\left(\dfrac{1}{3}x + 6\right)}\)
	\item \(\frac{2}{3}{\left(\dfrac{1}{3}x + 6\right)}\)
	\item \(\frac{1}{3}{\left(\dfrac{1}{3}x + 6\right)}^{2}\)
	\item \(\frac{2}{3}{\left(\dfrac{1}{3}x + 6\right)}^{2}\)
	\end{enumerate}
\item If \(y = {x}^{2}-3{x}+4\), find \(\dv{y}{x}\) at \(x = 5\).
	\begin{enumerate}[label={\Alph*.}]
	\item \(9\)
	\item \(7\)
	\item \(5\)
	\item \(3\)
	\end{enumerate}
\item If  \(y = {x}^{2}+\sqrt{x}\), find \(\dv{y}{x}\).
	\begin{enumerate}[label={\Alph*.}]
	\item \(2x-\frac{1}{2}x^{\frac{1}{2}}\)
	\item \(2x-\frac{1}{2}x^{-{\frac{1}{2}}}\)
	\item \(2x+x^{-{\frac{1}{2}}}\)
	\item \(2x+\frac{1}{2}x^{-{\frac{1}{2}}}\)
	\end{enumerate}
\item Find \(\dv{y}{x}\), if \(y = \frac{2}{3}{x}^{3}-\frac{4}{x}\)
	\begin{enumerate}[label={\Alph*.}]
	\item \({3}{x}^{2}-\frac{4}{x}\)
	\item \({3}{x}^{2}+\frac{4}{x^{2}}\)
	\item \({2}{x}^{2}-\frac{4}{x}\)
	\item \({2}{x}^{2}+\frac{4}{x^{2}}\)
	\end{enumerate}
\item If \(y = \cos{(3x)}\), find \(\dv{y}{x}\).
	\begin{enumerate}[label={\Alph*.}]
	\item \(\frac{1}{3}\sin{(3x)}\)
	\item \({3}\sin{(3x)}\)
	\item \(-{\frac{1}{3}}\sin{(3x)}\)
	\item \(-3\sin{(3x)}\)
	\end{enumerate}
\item Find \(\dv{y}{x}\), if \(y = \cos{x}\)
	\begin{enumerate}[label={\Alph*.}]
	\item \(\sin{x}\)
	\item \(-\sin{x}\)
	\item \(\tan{x}\)
	\item \(-\tan{x}\)
	\end{enumerate}
\item Find the slope of the curve: \(y = 2x^{3}+ 5{x}-3\) at \((1,4)\).
	\begin{enumerate}[label={\Alph*.}]
	\item \(4\)
	\item \(6\)
	\item \(7\)
	\item \(9\)
	\end{enumerate}
\item Find the derivative of \(y = \sin^{2}{(5x)}\) with respect to \(x\).
	\begin{enumerate}[label={\Alph*.}]
	\item \(5 \sin{(5x)\cos{(5x)}}\)
	\item \(2 \sin{(5x)\cos{(5x)}}\)
	\item \(15 \sin{(5x)\cos{(5x)}}\)
	\item \(10 \sin{(5x)\cos{(5x)}}\)
	\end{enumerate}
\item The slope of the tangent to the curve: \(y=3{x}^{2}-2x+5\) at the point \((1,6)\) is
	\begin{enumerate}[label={\Alph*.}]
	\item \(1\)
	\item \(4\)
	\item \(5\)
	\item \(6\)
	\end{enumerate}
\item If the gradient of the curve \(y=2k{x}^{2}+x+1\) at \(x = 1\) is 9, find the value of \(k\)
	\begin{enumerate}[label={\Alph*.}]
	\item \(\)
	\item \(\)
	\item \(\)
	\item \(\)
	\end{enumerate}
\item Find the 
	\begin{enumerate}[label={\Alph*.}]
	\item \(\)
	\item \(\)
	\item \(\)
	\item \(\)
	\end{enumerate}
\item
	\begin{enumerate}[label={\Alph*.}]
	\item \(\)
	\item \(\)
	\item \(\)
	\item \(\)
	\end{enumerate}
\item
	\begin{enumerate}[label={\Alph*.}]
	\item \(\)
	\item \(\)
	\item \(\)
	\item \(\)
	\end{enumerate}
\item
	\begin{enumerate}[label={\Alph*.}]
	\item \(\)
	\item \(\)
	\item \(\)
	\item \(\)
	\end{enumerate}
\item
	\begin{enumerate}[label={\Alph*.}]
	\item \(\)
	\item \(\)
	\item \(\)
	\item \(\)
	\end{enumerate}
\item
	\begin{enumerate}[label={\Alph*.}]
	\item \(\)
	\item \(\)
	\item \(\)
	\item \(\)
	\end{enumerate}
\item
	\begin{enumerate}[label={\Alph*.}]
	\item \(\)
	\item \(\)
	\item \(\)
	\item \(\)
	\end{enumerate}
\item
	\begin{enumerate}[label={\Alph*.}]
	\item \(\)
	\item \(\)
	\item \(\)
	\item \(\)
	\end{enumerate}
\item
	\begin{enumerate}[label={\Alph*.}]
	\item \(\)
	\item \(\)
	\item \(\)
	\item \(\)
	\end{enumerate}
\item
	\begin{enumerate}[label={\Alph*.}]
	\item \(\)
	\item \(\)
	\item \(\)
	\item \(\)
	\end{enumerate}
\item
	\begin{enumerate}[label={\Alph*.}]
	\item \(\)
	\item \(\)
	\item \(\)
	\item \(\)
	\end{enumerate}
\item
	\begin{enumerate}[label={\Alph*.}]
	\item \(\)
	\item \(\)
	\item \(\)
	\item \(\)
	\end{enumerate}
\item
	\begin{enumerate}[label={\Alph*.}]
	\item \(\)
	\item \(\)
	\item \(\)
	\item \(\)
	\end{enumerate}
\item
	\begin{enumerate}[label={\Alph*.}]
	\item \(\)
	\item \(\)
	\item \(\)
	\item \(\)
	\end{enumerate}
\item
	\begin{enumerate}[label={\Alph*.}]
	\item \(\)
	\item \(\)
	\item \(\)
	\item \(\)
	\end{enumerate}
\item
	\begin{enumerate}[label={\Alph*.}]
	\item \(\)
	\item \(\)
	\item \(\)
	\item \(\)
	\end{enumerate}
\item
	\begin{enumerate}[label={\Alph*.}]
	\item \(\)
	\item \(\)
	\item \(\)
	\item \(\)
	\end{enumerate}
\item
	\begin{enumerate}[label={\Alph*.}]
	\item \(\)
	\item \(\)
	\item \(\)
	\item \(\)
	\end{enumerate}
\item
	\begin{enumerate}[label={\Alph*.}]
	\item \(\)
	\item \(\)
	\item \(\)
	\item \(\)
	\end{enumerate}
\item
	\begin{enumerate}[label={\Alph*.}]
	\item \(\)
	\item \(\)
	\item \(\)
	\item \(\)
	\end{enumerate}
\item
	\begin{enumerate}[label={\Alph*.}]
	\item \(\)
	\item \(\)
	\item \(\)
	\item \(\)
	\end{enumerate}
\item
	\begin{enumerate}[label={\Alph*.}]
	\item \(\)
	\item \(\)
	\item \(\)
	\item \(\)
	\end{enumerate}
\item
	\begin{enumerate}[label={\Alph*.}]
	\item \(\)
	\item \(\)
	\item \(\)
	\item \(\)
	\end{enumerate}
\item
	\begin{enumerate}[label={\Alph*.}]
	\item \(\)
	\item \(\)
	\item \(\)
	\item \(\)
	\end{enumerate}
\item
	\begin{enumerate}[label={\Alph*.}]
	\item \(\)
	\item \(\)
	\item \(\)
	\item \(\)
	\end{enumerate}
\item
	\begin{enumerate}[label={\Alph*.}]
	\item \(\)
	\item \(\)
	\item \(\)
	\item \(\)
	\end{enumerate}
\item
	\begin{enumerate}[label={\Alph*.}]
	\item \(\)
	\item \(\)
	\item \(\)
	\item \(\)
	\end{enumerate}
\item
	\begin{enumerate}[label={\Alph*.}]
	\item \(\)
	\item \(\)
	\item \(\)
	\item \(\)
	\end{enumerate}
\item
	\begin{enumerate}[label={\Alph*.}]
	\item \(\)
	\item \(\)
	\item \(\)
	\item \(\)
	\end{enumerate}
\item
	\begin{enumerate}[label={\Alph*.}]
	\item \(\)
	\item \(\)
	\item \(\)
	\item \(\)
	\end{enumerate}
\item
	\begin{enumerate}[label={\Alph*.}]
	\item \(\)
	\item \(\)
	\item \(\)
	\item \(\)
	\end{enumerate}
\item
	\begin{enumerate}[label={\Alph*.}]
	\item \(\)
	\item \(\)
	\item \(\)
	\item \(\)
	\end{enumerate}
\item
	\begin{enumerate}[label={\Alph*.}]
	\item \(\)
	\item \(\)
	\item \(\)
	\item \(\)
	\end{enumerate}
\item
	\begin{enumerate}[label={\Alph*.}]
	\item \(\)
	\item \(\)
	\item \(\)
	\item \(\)
	\end{enumerate}
\item
	\begin{enumerate}[label={\Alph*.}]
	\item \(\)
	\item \(\)
	\item \(\)
	\item \(\)
	\end{enumerate}
\item
	\begin{enumerate}[label={\Alph*.}]
	\item \(\)
	\item \(\)
	\item \(\)
	\item \(\)
	\end{enumerate}
\item
	\begin{enumerate}[label={\Alph*.}]
	\item \(\)
	\item \(\)
	\item \(\)
	\item \(\)
	\end{enumerate}
\item
	\begin{enumerate}[label={\Alph*.}]
	\item \(\)
	\item \(\)
	\item \(\)
	\item \(\)
	\end{enumerate}
\item
	\begin{enumerate}[label={\Alph*.}]
	\item \(\)
	\item \(\)
	\item \(\)
	\item \(\)
	\end{enumerate}
\item
	\begin{enumerate}[label={\Alph*.}]
	\item \(\)
	\item \(\)
	\item \(\)
	\item \(\)
	\end{enumerate}
\item
	\begin{enumerate}[label={\Alph*.}]
	\item \(\)
	\item \(\)
	\item \(\)
	\item \(\)
	\end{enumerate}
\item
	\begin{enumerate}[label={\Alph*.}]
	\item \(\)
	\item \(\)
	\item \(\)
	\item \(\)
	\end{enumerate}
\item
	\begin{enumerate}[label={\Alph*.}]
	\item \(\)
	\item \(\)
	\item \(\)
	\item \(\)
	\end{enumerate}
\item
	\begin{enumerate}[label={\Alph*.}]
	\item \(\)
	\item \(\)
	\item \(\)
	\item \(\)
	\end{enumerate}
\item
	\begin{enumerate}[label={\Alph*.}]
	\item \(\)
	\item \(\)
	\item \(\)
	\item \(\)
	\end{enumerate}
\item
	\begin{enumerate}[label={\Alph*.}]
	\item \(\)
	\item \(\)
	\item \(\)
	\item \(\)
	\end{enumerate}
\item
	\begin{enumerate}[label={\Alph*.}]
	\item \(\)
	\item \(\)
	\item \(\)
	\item \(\)
	\end{enumerate}
\item
	\begin{enumerate}[label={\Alph*.}]
	\item \(\)
	\item \(\)
	\item \(\)
	\item \(\)
	\end{enumerate}
\item
	\begin{enumerate}[label={\Alph*.}]
	\item \(\)
	\item \(\)
	\item \(\)
	\item \(\)
	\end{enumerate}
\item
	\begin{enumerate}[label={\Alph*.}]
	\item \(\)
	\item \(\)
	\item \(\)
	\item \(\)
	\end{enumerate}
\item
	\begin{enumerate}[label={\Alph*.}]
	\item \(\)
	\item \(\)
	\item \(\)
	\item \(\)
	\end{enumerate}
\item
	\begin{enumerate}[label={\Alph*.}]
	\item \(\)
	\item \(\)
	\item \(\)
	\item \(\)
	\end{enumerate}
\end{enumerate}
\end{multicols}
\section{Integration}
\subsection{Questions}
\begin{multicols}{2}
  \begin{enumerate}[label={\arabic*.}]
    \item Find the integral of \(y = 3{x}^{2}-2x-1\) with respect to \(x\).
      \begin{enumerate}[label={\Alph*.}]
        \item \({x}^{3} - {x}^{2} - x + C\)
        \item \({x}^{3} + {x}^{2} - x + C\)
        \item \({x}^{3} + {x}^{2} + x + C\)
        \item \({x}^{3} - {x}^{2} + x + C\)
      \end{enumerate}
    \item Integrate the expression \(6{x}^{2} - 2x + 1\) with respect to \(x\).
      \begin{enumerate}[label={\Alph*.}]
        \item \(3{x}^{3} - 2{x}^{2} + x + c\)
        \item \(2{x}^{3} - x^{2} + x + c\)
        \item \(2{x}^{3} - 3{x}^{2} + c\)
        \item \({x}^{3} + {x}^{2} - x + c\)
      \end{enumerate}
    \item Integrate \(x^{-2} + \cos {x}\) with respect to \(x\).
      \begin{enumerate}[label={\Alph*.}]
        \item \(\dfrac{1}{x} + \sin {x} + k\)
        \item \(-\dfrac{1}{x} + \sin {x} + k\)
        \item \(-\dfrac{1}{x} - \sin {x} + k\)
        \item \(\ln|x| + \sin {x} + k\)
      \end{enumerate}
    \item If the expression \(a{x}^{2} + bx + c\) equals \(5\) at \(x = 1\). If its derivative is \(2x + 1\), what are the values of \(a\),\(b\), \(c\) respectively?
      \begin{enumerate}[label={\Alph*.}]
        \item \(1, 1, 3\)
        \item \(1, 3, 1\)
        \item \(1, 2, 1\)
        \item \(2, 1, 1\)
      \end{enumerate}
    \item Integrate the expression \({(2x+1)}^{3}\) with respect to \(x\).
      \begin{enumerate}[label={\Alph*.}]
        \item \(\dfrac{{(2x+1)}^{3}}{8} + k\)
        \item \(\dfrac{{(2x+1)}^{4}}{8} + k\)
        \item \(\dfrac{{(2x+1)}^{4}}{6} + k\)
        \item \(\dfrac{{(2x+1)}^{2}}{8} + k\)
      \end{enumerate}
    \item Evaluate \(\displaystyle \int (4{x}^{-3} - 7{x}^{2} + 5x - 6)\ \mathrm{d}x\)
      \begin{enumerate}[label={\Alph*.}]
        \item \(-2x^{-2}-\dfrac{7}{3}x^{3}+\dfrac{5}{2}x^2-6x + C\)
        \item \(2x^{2}+\dfrac{7}{3}x^{3}+5x^{2}-6 + C\)
        \item \(12x^{2}+14x-5 + C\)
        \item \(-12x^{-4}-14x+5 + C\)
      \end{enumerate}
    \item Evaluate \(\displaystyle \int_{-1}^{2}\left(2x^{2} + x\right)\ \mathrm{d}x\)
      \begin{enumerate}[label={\Alph*.}]
        \item \(4\frac{1}{2}\)
        \item \(3\frac{1}{2}\)
        \item \(7\frac{1}{2}\)
        \item \(5\frac{1}{4}\)
      \end{enumerate}
    \item Integrate \(\dfrac{x^{2}-\sqrt{x}}{x}\) with respect to \(x\).
      \begin{enumerate}[label={\Alph*.}]
        \item \(\dfrac{x^{2}}{2} - 2\sqrt{x} + k\)
        \item \(\dfrac{2(x^{2}-x)}{3x} + k\)
        \item \(\dfrac{x^{2}}{2} - \sqrt{x} + k\)
        \item \(\dfrac{x^{2}-x}{3x} + k\)
      \end{enumerate}
    \item Evaluate \(\displaystyle \int_{-1}^{1} {\left(2x + 1\right)}^{2}\ \mathrm{d}x\)
      \begin{enumerate}[label={\Alph*.}]
        \item \(3\frac{2}{3}\)
        \item \(4\)
        \item \(4\frac{1}{3}\)
        \item \(4\frac{2}{3}\)
      \end{enumerate}
    \item Evaluate \(\displaystyle \int \left(\cos{4x} + \sin{3x}\right)\ \mathrm{d}x\)
      \begin{enumerate}[label={\Alph*.}]
        \item \(\sin{4x} - \cos{3x} + k\)
        \item \(\sin{4x} + \cos{3x} + k\)
        \item \(\cfrac{1}{4}\sin{4x} - \cfrac{1}{3}\cos{3x} + k\)
        \item \(\cfrac{1}{4}\sin{4x} + \cfrac{1}{3}\cos{3x} + k\)
      \end{enumerate}
    \item Evaluate \(\displaystyle \int_{0}^{\frac{\pi}{2}} \sin{x}\ \mathrm{d}x\)
      \begin{enumerate}[label={\Alph*.}]
        \item \(-2\)
        \item \(-1\)
        \item \(1\)
        \item \(2\)
      \end{enumerate}
    \item Evaluate \(\displaystyle \int_{1}^{2}\frac{5}{x}\ \mathrm{d}x\)
      \begin{enumerate}[label={\Alph*.}]
        \item \(1.47\)
        \item \(2.67\)
        \item \(3.23\)
        \item \(3.47\)
      \end{enumerate}
    \item Evaluate the integral \(\displaystyle \int_{\frac{\pi}{12}}^{\frac{\pi}{4}} 2\cos{2x}\ \mathrm{d}x\)
      \begin{enumerate}[label={\Alph*.}]
        \item \(-{\cfrac{1}{2}}\)
        \item \(-1\)
        \item \(\cfrac{1}{2}\)
        \item \(1\)
      \end{enumerate}
    \item Evaluate \(\displaystyle \int {\left(2x+3\right)}^{\frac{1}{2}}\ \mathrm{d}x\)
      \begin{enumerate}[label={\Alph*.}]
        \item \(\cfrac{1}{12}{(2x+3)}^6 + k\)
        \item \(\cfrac{1}{3}{(2x+3)}^{\frac{1}{2}} + k\)
        \item \(\cfrac{1}{3}{(2x+3)}^{\frac{3}{2}} + k\)
        \item \(\cfrac{1}{12}{(2x+3)}^{\frac{3}{4}} + k\)
      \end{enumerate}
    \item Evaluate \(\displaystyle \int \left(\sin{x} - 5{x}^{2}\right)\ \mathrm{d}x\)
      \begin{enumerate}[label={\Alph*.}]
        \item \(-\cos{x} - 10x + k\)
        \item \(\cos{x} - \cfrac{5x^3}{3} + k\)
        \item \(-\cos{x} - \cfrac{5x^3}{3} + k\)
        \item \(\cos{x} - 10x + k\)
      \end{enumerate}
    \item Evaluate \(\displaystyle \int \sin{2x}\ \mathrm{d}x\)
      \begin{enumerate}[label={\Alph*.}]
        \item \(\cos{2x} + k\)
        \item \(\cfrac{1}{2}\cos{2x} + k\)
        \item \(-\cfrac{1}{2}\cos{2x} + k\)
        \item \(-\cos{2x} + k\)
      \end{enumerate}
    \item If \(y = x(x^4 + x + 1)\), evaluate \(\displaystyle \int_{0}^{1} y\ \mathrm{d}x\)
      \begin{enumerate}[label={\Alph*.}]
        \item \(\cfrac{11}{12}\)
        \item \(1\)
        \item \(\cfrac{5}{6}\)
        \item \(0\)
      \end{enumerate}
    \item Evaluate \(\displaystyle \int_{2}^{\pi} (\sec^{2}{x} - \tan^{2}{x})\ \mathrm{d}x\)
      \begin{enumerate}[label={\Alph*.}]
        \item \(\cfrac{\pi}{2}\)
        \item \(\cfrac{\pi}{3}\)
        \item \(\pi - 2\)
        \item \(\pi + 2\)
      \end{enumerate}
    \item Evaluate \(\displaystyle \int_{0}^{\frac{\pi}{4}} (\sin{x} - \cos{x})\ \mathrm{d}x\)
      \begin{enumerate}[label={\Alph*.}]
        \item \(\sqrt{2} + 1\)
        \item \(\sqrt{2} - 1\)
        \item \(1 - \sqrt{2}\)
        \item \(-\sqrt{2} - 1\)
      \end{enumerate}
    \item Evaluate \(\displaystyle \int_{-2}^{1} {\left(x - 1\right)}^{2}\ \mathrm{d}x\)
      \begin{enumerate}[label={\Alph*.}]
        \item \(-{\cfrac{10}{3}}\)
        \item \(7\)
        \item \(9\)
        \item \(11\)
      \end{enumerate}
    \item A function \(f(x)\) passes through the origin and its first derivative is \(3x + 2\). What is \(f(x)\)?
      \begin{enumerate}[label={\Alph*.}]
        \item \(f(x) = \cfrac{3{x}^{2}}{2} + 2x \)
        \item \(f(x) = \cfrac{3{x}^{2}}{2} + x\)
        \item \(f(x) = 3{x}^{2} + \cfrac{x}{2}\)
        \item \(f(x) = 3{x}^{2} + 2x\)
      \end{enumerate}
    \item Evaluate \(\displaystyle \int_{2}^{3} \left(x^2-2x\right)\ \mathrm{d}x\)
      \begin{enumerate}[label={\Alph*.}]
        \item \(4\)
        \item \(2\)
        \item \(\cfrac{4}{3}\)
        \item \(\cfrac{1}{3}\)
      \end{enumerate}
    \item Evaluate \(\displaystyle \int_{-4}^{0} \left(1-2x\right)\ \mathrm{d}x\)
      \begin{enumerate}[label={\Alph*.}]
        \item \(-20\)
        \item \(-16\)
        \item \(10\)
        \item \(20\)
      \end{enumerate}
    \item Evaluate \(\displaystyle \int_{1}^{2} \left(6x^2-2x\right)\ \mathrm{d}x\)
      \begin{enumerate}[label={\Alph*.}]
        \item \(11\)
        \item \(12\)
        \item \(13\)
        \item \(16\)
      \end{enumerate}
    \item Evaluate \(\displaystyle \int_{-{\frac{\pi}{2}}}^{\frac{\pi}{2}} \cos{x}\ \mathrm{d}x\)
      \begin{enumerate}[label={\Alph*.}]
        \item \(0\)
        \item \(1\)
        \item \(2\)
        \item \(3\)
      \end{enumerate}
    \item Evaluate \(\displaystyle \int_{0}^{2} \left({x}^{3}+{x}^{2}\right)\ \mathrm{d}x\)
      \begin{enumerate}[label={\Alph*.}]
        \item \(4\frac{5}{6}\)
        \item \(6\frac{2}{3}\)
        \item \(1\frac{5}{6}\)
        \item \(2\frac{5}{6}\)
      \end{enumerate}
    \item Evaluate \(\displaystyle \int_{1}^{2} 2\ \mathrm{d}x\)
      \begin{enumerate}[label={\Alph*.}]
        \item \(3\)
        \item \(5\)
        \item \(2\)
        \item \(6\)
      \end{enumerate}
    \item Evaluate \(\displaystyle \int_{1}^{2} \left({x}^{2}-4x\right)\ \mathrm{d}x\)
      \begin{enumerate}[label={\Alph*.}]
        \item \(\cfrac{11}{3}\)
        \item \(\cfrac{3}{11}\)
        \item \(-{\cfrac{3}{11}}\)
        \item \(-{\cfrac{11}{3}}\)
      \end{enumerate}
    \item Evaluate \(\displaystyle \int \left(\sin{x}+2\right)\ \mathrm{d}x\)
      \begin{enumerate}[label={\Alph*.}]
        \item \(\cos{x} + {x}^{2} + k\)
        \item \(\cos{x} + 2x + k\)
        \item \(-{\cos{x}} + {x}^{2} + k\)
        \item \(-{\cos{x}} + 2x + k\)
      \end{enumerate}
    \item Evaluate \(\displaystyle \int \cos{4x}\ \mathrm{d}x\)
      \begin{enumerate}[label={\Alph*.}]
        \item \(\cfrac{3}{4}\sin{4x} + k\)
        \item \(-{\cfrac{1}{4}\sin{4x}} + k\)
        \item \(-{\cfrac{3}{4}\sin{4x}} + k\)
        \item \({\cfrac{1}{4}\sin{4x}} + k\)
      \end{enumerate}
    \item Integrate \(\dfrac{1+x}{{x}^{3}}\) with respect to \(x\).
      \begin{enumerate}[label={\Alph*.}]
        \item \(2{x}^{2} -\cfrac{1}{x} + k \)
        \item \({x}^{2} -\cfrac{1}{x} + k \)
        \item \(-{\cfrac{{x}^{2}}{2}} -\cfrac{1}{x} + k \)
        \item \(-{\cfrac{1}{{2x}^{2}}} -\cfrac{1}{x} + k \)
      \end{enumerate}
    \item Evaluate \(\displaystyle \int \left({x}^{2}+3x-5\right)\ \mathrm{d}x\)
      \begin{enumerate}[label={\Alph*.}]
        \item \({\cfrac{{x}^{3}}{3}} - {\cfrac{3{x}^{2}}{2}} - 5x + k \)
        \item \({\cfrac{{x}^{3}}{3}} - {\cfrac{3{x}^{2}}{2}} + 5x + k \)
        \item \({\cfrac{{x}^{3}}{3}} + {\cfrac{3{x}^{2}}{2}} - 5x + k \)
        \item \({\cfrac{{x}^{3}}{3}} + {\cfrac{3{x}^{2}}{2}} + 5x + k \)
      \end{enumerate}
    \item Integrate \(\dfrac{2x^3+2x}{x}\) with respect to \(x\).
      \begin{enumerate}[label={\Alph*.}]
        \item \({\cfrac{2{x}^{3}}{3}} - 2x + k \)
        \item \({\cfrac{2{x}^{3}}{3}} + 2x + k \)
        \item \({x}^{3} - 2x + k \)
        \item \({x}^{3} + 2x + k \)
      \end{enumerate}
    \item Evaluate \(\displaystyle \int \left(5{x}^{3} + 7{x}^{2} -2x + 5\right)\ \mathrm{d}x\)
      \begin{enumerate}[label={\Alph*.}]
        \item \({\cfrac{5{x}^{4}}{4}} + {\cfrac{7{x}^{3}}{3}} + 2x + C \)
        \item \({\cfrac{5{x}^{4}}{4}} + {\cfrac{7{x}^{3}}{3}} - {x}^{2} + 5x + C \)
        \item \({\cfrac{5{x}^{3}}{3}} + {\cfrac{7{x}^{2}}{2}} - {x} + C \)
        \item \({\cfrac{2{x}^{2}}{3}} + {\cfrac{x}{5}} - C \)
      \end{enumerate}
    \item Find the value of \(\displaystyle \int^{\pi}_{0}\cfrac{\cos^{2}\theta-1}{\sin^{2}\theta}\ \mathrm{d}\theta\)
      \begin{enumerate}[label={\Alph*.}]
        \item \(\pi\)
        \item \(-\pi\)
        \item \(\cfrac{\pi}{2}\)
        \item \(-{\cfrac{\pi}{2}}\)
      \end{enumerate}
    \item The area enclosed by \(y=x^2-1\), \(y=3\), and \(x \ge 0\) is revolved around the y-axis. If the volume is \(K\pi\), find \(K\).
      \begin{enumerate}[label={\Alph*.}]
        \item \(7\)
        \item \(\cfrac{15}{2}\)
        \item \(8\)
        \item \(\cfrac{17}{2}\)
      \end{enumerate}
    \item Evaluate \(\displaystyle \int (2x-5)^4 \mathrm{d}x\)
      \begin{enumerate}[label={\Alph*.}]
        \item \(\dfrac{(2x-5)^5}{5} + C\)
        \item \(\dfrac{(2x-5)^5}{10} + C\)
        \item \(8(2x-5)^3 + C\)
        \item \(\dfrac{(2x-5)^3}{6} + C\)
      \end{enumerate}
    \item Find \(\displaystyle \int e^{3x} \mathrm{d}x\)
      \begin{enumerate}[label={\Alph*.}]
        \item \(3e^{3x} + C\)
        \item \(e^{3x} + C\)
        \item \(\dfrac{1}{3}e^{3x} + C\)
        \item \(\dfrac{1}{3}e^{x} + C\)
      \end{enumerate}
    \item Evaluate \(\displaystyle \int_{0}^{1} (x^2 - x + 1) \mathrm{d}x\)
      \begin{enumerate}[label={\Alph*.}]
        \item \(\dfrac{1}{6}\)
        \item \(\dfrac{5}{6}\)
        \item \(1\)
        \item \(\dfrac{7}{6}\)
      \end{enumerate}
    \item Find \(\displaystyle \int \sec^2(3x) \mathrm{d}x\)
      \begin{enumerate}[label={\Alph*.}]
        \item \(\tan(3x) + C\)
        \item \(\dfrac{1}{3}\tan(3x) + C\)
        \item \(3\tan(3x) + C\)
        \item \(\sec(3x)\tan(3x) + C\)
      \end{enumerate}
    \item Evaluate \(\displaystyle \int_{1}^{e} \frac{1}{x} \mathrm{d}x\)
      \begin{enumerate}[label={\Alph*.}]
        \item \(0\)
        \item \(1\)
        \item \(e\)
        \item \(e-1\)
      \end{enumerate}
    \item Find the area under the curve \(y=x^2\) from \(x=0\) to \(x=3\).
      \begin{enumerate}[label={\Alph*.}]
        \item \(3\)
        \item \(6\)
        \item \(9\)
        \item \(27\)
      \end{enumerate}
    \item Evaluate \(\displaystyle \int \frac{2}{x+1} \mathrm{d}x\)
      \begin{enumerate}[label={\Alph*.}]
        \item \(2\ln|x+1| + C\)
        \item \(\ln|x+1| + C\)
        \item \(\dfrac{-2}{(x+1)^2} + C\)
        \item \(2\arctan(x) + C\)
      \end{enumerate}
    \item If \(\dv{y}{x} = 2x - 3\) and \(y=2\) when \(x=1\), find \(y\) in terms of \(x\).
      \begin{enumerate}[label={\Alph*.}]
        \item \(y = x^2 - 3x + 2\)
        \item \(y = x^2 - 3x + 4\)
        \item \(y = 2x^2 - 3x + 3\)
        \item \(y = x^2 - 3x\)
      \end{enumerate}
    \item Evaluate \(\displaystyle \int_{0}^{\pi} \cos x \mathrm{d}x\)
      \begin{enumerate}[label={\Alph*.}]
        \item \(0\)
        \item \(1\)
        \item \(-1\)
        \item \(2\)
      \end{enumerate}
    \item Find \(\displaystyle \int (3-4x)^{-2} \mathrm{d}x\)
      \begin{enumerate}[label={\Alph*.}]
        \item \(\dfrac{(3-4x)^{-1}}{-4} + C\) % Error in original, should be -1/-4 = 1/4
        \item \(\dfrac{(3-4x)^{-1}}{4} + C\)
        \item \(\dfrac{(3-4x)^{-3}}{-12} + C\)
        \item \(-4(3-4x)^{-3} + C\)
      \end{enumerate}
    \item What is the area bounded by the curve \(y=4-x^2\) and the x-axis?
      \begin{enumerate}[label={\Alph*.}]
        \item \(\dfrac{8}{3}\)
        \item \(\dfrac{16}{3}\)
        \item \(\dfrac{32}{3}\)
        \item \(16\)
      \end{enumerate}
    \item Evaluate \(\displaystyle \int_{0}^{2} (3x^2+4x-5) \mathrm{d}x\)
      \begin{enumerate}[label={\Alph*.}]
        \item \(2\)
        \item \(4\)
        \item \(6\)
        \item \(8\)
      \end{enumerate}
    \item Find \(\displaystyle \int (x+1)(x-2) \mathrm{d}x\)
      \begin{enumerate}[label={\Alph*.}]
        \item \(\dfrac{x^3}{3} - \dfrac{x^2}{2} - 2x + C\)
        \item \(\dfrac{x^3}{3} + \dfrac{x^2}{2} - 2x + C\)
        \item \(x^2 - x - 2 + C\)
        \item \(\dfrac{(x+1)^2(x-2)^2}{4} + C\)
      \end{enumerate}
    \item Evaluate \(\displaystyle \int_{1}^{4} \sqrt{x} \mathrm{d}x\)
      \begin{enumerate}[label={\Alph*.}]
        \item \(\dfrac{7}{3}\)
        \item \(\dfrac{14}{3}\)
        \item \(2\)
        \item \(3\)
      \end{enumerate}
    \item The gradient of a curve is given by \(4x+3\). If the curve passes through the point \((1,5)\), find its equation.
      \begin{enumerate}[label={\Alph*.}]
        \item \(y = 2x^2 + 3x\)
        \item \(y = 2x^2 + 3x + 5\)
        \item \(y = 2x^2 + 3x - 0\) % Note: To make A the correct answer, C must be 0.
        \item \(y = 4x^2 + 3x - 2\)
      \end{enumerate}
    \item Evaluate \(\displaystyle \int e^{-x/2} \mathrm{d}x\)
      \begin{enumerate}[label={\Alph*.}]
        \item \(-2e^{-x/2} + C\)
        \item \(-\dfrac{1}{2}e^{-x/2} + C\)
        \item \(2e^{-x/2} + C\)
        \item \(e^{-x/2} + C\)
      \end{enumerate}
    \item Find the area enclosed by the curve \(y=x^3\), the x-axis, and the lines \(x=1\) and \(x=2\).
      \begin{enumerate}[label={\Alph*.}]
        \item \(\dfrac{7}{4}\)
        \item \(\dfrac{15}{4}\)
        \item \(4\)
        \item \(7\)
      \end{enumerate}
    \item Evaluate \(\displaystyle \int_{0}^{\pi/6} \sec x \tan x \mathrm{d}x\)
      \begin{enumerate}[label={\Alph*.}]
        \item \(\dfrac{2\sqrt{3}}{3} - 1\)
        \item \(1 - \dfrac{2\sqrt{3}}{3}\)
        \item \(\dfrac{\sqrt{3}}{3}\)
        \item \(2 - \sqrt{3}\)
      \end{enumerate}
    \item If \(\displaystyle \int_{0}^{a} (2x+1) \mathrm{d}x = 4\), find the positive value of \(a\).
      \begin{enumerate}[label={\Alph*.}]
        \item \(1\)
        \item \(\dfrac{3}{2}\)
        \item \(2\)
        \item \(3\)
      \end{enumerate}
    \item Find \(\displaystyle \int \frac{x^2+1}{x^2} \mathrm{d}x\)
      \begin{enumerate}[label={\Alph*.}]
        \item \(x - \dfrac{1}{x} + C\)
        \item \(x + \dfrac{1}{x} + C\)
        \item \(1 - \dfrac{2}{x^3} + C\)
        \item \(2x + C\)
      \end{enumerate}
    \item Evaluate \(\displaystyle \int (1 - x)^3 \mathrm{d}x\)
      \begin{enumerate}[label={\Alph*.}]
        \item \(\dfrac{(1-x)^4}{4} + C\)
        \item \(-\dfrac{(1-x)^4}{4} + C\)
        \item \(-3(1-x)^2 + C\)
        \item \(3(1-x)^2 + C\)
      \end{enumerate}
    \item The area bounded by the curve \(y=x\), the x-axis, \(x=0\) and \(x=2\) is revolved around the x-axis. Find the volume of the solid generated.
      \begin{enumerate}[label={\Alph*.}]
        \item \(\dfrac{2\pi}{3}\)
        \item \(\dfrac{4\pi}{3}\)
        \item \(\dfrac{8\pi}{3}\)
        \item \(4\pi\)
      \end{enumerate}
    \item Evaluate \(\displaystyle \int_{0}^{\pi/3} \sin(3x) \mathrm{d}x\)
      \begin{enumerate}[label={\Alph*.}]
        \item \(0\)
        \item \(\dfrac{1}{3}\)
        \item \(\dfrac{2}{3}\)
        \item \(1\)
      \end{enumerate}
    \item Evaluate \(\displaystyle \int (x^2 + 1)^2 dx \)
      \begin{enumerate}[label={\Alph*.}]
        \item \( \dfrac{x^5}{5} + \dfrac{2x^3}{3} + x + C \)
        \item \( \dfrac{(x^2+1)^3}{3} + C \)
        \item \( \dfrac{x^5}{5} + x^3 + x + C \)
        \item \( \dfrac{(x^2+1)^3}{6x} + C \)
      \end{enumerate}
    \item If \(\displaystyle \int_{1}^{k} \frac{1}{x^2} dx = \frac{1}{2} \), find the value of \(k\).
      \begin{enumerate}[label={\Alph*.}]
        \item \(1\)
        \item \(2\)
        \item \(\dfrac{1}{2}\)
        \item \(4\)
      \end{enumerate}
    \item Find the indefinite integral of \( \sec^2 x e^{\tan x} \).
      \begin{enumerate}[label={\Alph*.}]
        \item \( e^{\tan x} + C \)
        \item \( \tan x e^{\tan x} + C \)
        \item \( \sec x e^{\tan x} + C \)
        \item \( e^{\sec^2 x} + C \)
      \end{enumerate}
    \item Evaluate \(\displaystyle \int \frac{\ln x}{x} dx \).
      \begin{enumerate}[label={\Alph*.}]
        \item \( \ln| \ln x | + C \)
        \item \( (\ln x)^2 + C \)
        \item \( \dfrac{1}{2}(\ln x)^2 + C \)
        \item \( \dfrac{1}{x^2} + C \)
      \end{enumerate}
    \item The area of the region bounded by \(y = e^x\), the x-axis, \(x=0\) and \(x=1\) is
      \begin{enumerate}[label={\Alph*.}]
        \item \(e\)
        \item \(e-1\)
        \item \(1-e\)
        \item \(1\)
      \end{enumerate}
    \item Evaluate \(\displaystyle \int 2^x \mathrm{d}x\)
      \begin{enumerate}[label={\Alph*.}]
        \item \(2^x + C\)
        \item \(\dfrac{2^x}{\ln 2} + C\)
        \item \(2^x \ln 2 + C\)
        \item \(x 2^{x-1} + C\)
      \end{enumerate}
    \item Evaluate \(\displaystyle \int_0^1 (e^x + e^{-x}) \mathrm{d}x\)
      \begin{enumerate}[label={\Alph*.}]
        \item \(e - \dfrac{1}{e}\)
        \item \(e + \dfrac{1}{e}\)
        \item \(e - \dfrac{1}{e} - 2\)
        \item \(0\)
      \end{enumerate}
    \item Find \(\displaystyle \int \frac{1}{2x+3} \mathrm{d}x\)
      \begin{enumerate}[label={\Alph*.}]
        \item \(\ln|2x+3| + C\)
        \item \(2\ln|2x+3| + C\)
        \item \(\dfrac{1}{2}\ln|2x+3| + C\)
        \item \(\dfrac{-1}{(2x+3)^2} + C\)
      \end{enumerate}
    \item Find the area bounded by \(y=\sin x\), the x-axis, from \(x=0\) to \(x=\pi\).
      \begin{enumerate}[label={\Alph*.}]
        \item \(0\)
        \item \(1\)
        \item \(2\)
        \item \(\pi\)
      \end{enumerate}
    \item Evaluate \(\displaystyle \int x e^{x^2} \mathrm{d}x\).
      \begin{enumerate}[label={\Alph*.}]
        \item \(e^{x^2} + C\)
        \item \(x^2 e^{x^2} + C\)
        \item \(\dfrac{1}{2}e^{x^2} + C\)
        \item \(2e^{x^2} + C\)
      \end{enumerate}
    \item Evaluate \(\displaystyle \int_{0}^{\pi/4} \tan x \sec^2 x \mathrm{d}x\)
      \begin{enumerate}[label={\Alph*.}]
        \item \(\dfrac{1}{4}\)
        \item \(\dfrac{1}{2}\)
        \item \(1\)
        \item \(2\)
      \end{enumerate}
    \item Find \(\displaystyle \int \cos^2 x \mathrm{d}x\) (Hint: \(\cos 2x = 2\cos^2 x - 1\))
      \begin{enumerate}[label={\Alph*.}]
        \item \(\dfrac{x}{2} + \dfrac{\sin 2x}{4} + C\)
        \item \(\dfrac{x}{2} - \dfrac{\sin 2x}{4} + C\)
        \item \(x + \sin 2x + C\)
        \item \(\dfrac{\cos^3 x}{3} + C\)
      \end{enumerate}
    \item If \(\displaystyle \int_{0}^{b} x \mathrm{d}x = 8\), find \(b > 0\).
      \begin{enumerate}[label={\Alph*.}]
        \item \(2\)
        \item \(4\)
        \item \(8\)
        \item \(16\)
      \end{enumerate}
    \item Evaluate \(\displaystyle \int \frac{1}{\sqrt{1-x^2}} \mathrm{d}x\).
      \begin{enumerate}[label={\Alph*.}]
        \item \(\arcsin x + C\)
        \item \(\arccos x + C\)
        \item \(\ln|\sqrt{1-x^2}| + C\)
        \item \(2\sqrt{1-x^2} + C\)
      \end{enumerate}
    \item The volume generated by revolving the area under \(y=\sqrt{x}\) from \(x=0\) to \(x=4\) about the x-axis is:
      \begin{enumerate}[label={\Alph*.}]
        \item \(4\pi\)
        \item \(8\pi\)
        \item \(16\pi\)
        \item \(\dfrac{8\pi}{3}\)
      \end{enumerate}
    \item Evaluate \(\displaystyle \int_{-1}^{1} x^3 \mathrm{d}x\).
      \begin{enumerate}[label={\Alph*.}]
        \item \(0\)
        \item \(\dfrac{1}{4}\)
        \item \(\dfrac{1}{2}\)
        \item \(1\)
      \end{enumerate}
    \item Find \(\displaystyle \int \frac{e^x}{1+e^x} \mathrm{d}x\).
      \begin{enumerate}[label={\Alph*.}]
        \item \(e^x \ln|1+e^x| + C\)
        \item \(\ln(1+e^x) + C\)
        \item \(\dfrac{e^{2x}}{2+e^x} + C\)
        \item \(\arctan(e^x) + C\)
      \end{enumerate}
    \item Evaluate \(\displaystyle \int_{1}^{2} (x+\frac{1}{x})^2 \mathrm{d}x\).
      \begin{enumerate}[label={\Alph*.}]
        \item \(\dfrac{29}{6}\)
        \item \(\dfrac{17}{3}\)
        \item \(5\)
        \item \(\dfrac{10}{3}\)
      \end{enumerate}
    \item If \(\dv{y}{x} = \sin x + x\) and \(y(0) = 1\), find \(y\).
      \begin{enumerate}[label={\Alph*.}]
        \item \(y = \cos x + \dfrac{x^2}{2}\)
        \item \(y = -\cos x + \dfrac{x^2}{2} + 1\)
        \item \(y = -\cos x + \dfrac{x^2}{2} + 2\)
        \item \(y = \cos x + \dfrac{x^2}{2} + 1\)
      \end{enumerate}
    \item Evaluate \(\displaystyle \int 5 \mathrm{d}x\).
      \begin{enumerate}[label={\Alph*.}]
        \item \(5+C\)
        \item \(5x+C\)
        \item \(\dfrac{x^2}{2} + 5x + C\)
        \item \(C\)
      \end{enumerate}
    \item Evaluate \(\displaystyle \int_{0}^{2} |x-1| \mathrm{d}x\).
      \begin{enumerate}[label={\Alph*.}]
        \item \(0\)
        \item \(\dfrac{1}{2}\)
        \item \(1\)
        \item \(2\)
      \end{enumerate}
    \item Find \(\displaystyle \int \frac{1}{x \ln x} \mathrm{d}x\).
      \begin{enumerate}[label={\Alph*.}]
        \item \((\ln x)^2 + C\)
        \item \(\ln|\ln x| + C\)
        \item \(\dfrac{1}{(\ln x)^2} + C\)
        \item \(\ln x^2 + C\)
      \end{enumerate}
    \item The area bounded by \(y=2x\), the x-axis, \(x=1\) and \(x=3\) is:
      \begin{enumerate}[label={\Alph*.}]
        \item \(4\)
        \item \(6\)
        \item \(8\)
        \item \(10\)
      \end{enumerate}
    \item Evaluate \(\displaystyle \int \sin^2 x \cos x \mathrm{d}x\).
      \begin{enumerate}[label={\Alph*.}]
        \item \(\dfrac{\sin^3 x}{3} + C\)
        \item \(\dfrac{\cos^3 x}{3} + C\)
        \item \(\sin x \cos x + C\)
        \item \(2\sin x \cos x + C\)
      \end{enumerate}
    \item \(\displaystyle \int_{e}^{e^2} \frac{\mathrm{d}x}{x \ln x}\).
      \begin{enumerate}[label={\Alph*.}]
        \item \(1\)
        \item \(\ln 2\)
        \item \(2\)
        \item \(e\)
      \end{enumerate}
    \item Find \(\displaystyle \int (x+1)^5 \mathrm{d}x\).
      \begin{enumerate}[label={\Alph*.}]
        \item \(5(x+1)^4 + C\)
        \item \(\dfrac{(x+1)^6}{6} + C\)
        \item \((x+1)^6 + C\)
        \item \(\dfrac{x^6}{6} + x^5 + \cdots + C\)
      \end{enumerate}
    \item Evaluate \(\displaystyle \int_0^{\pi/2} \cos^3 x \sin x \mathrm{d}x\).
      \begin{enumerate}[label={\Alph*.}]
        \item \(\dfrac{1}{4}\)
        \item \(\dfrac{1}{3}\)
        \item \(0\)
        \item \(1\)
      \end{enumerate}
    \item The value of \(\displaystyle \int_0^1 \frac{dx}{1+x^2}\) is:
      \begin{enumerate}[label={\Alph*.}]
        \item \(\pi\)
        \item \(\dfrac{\pi}{2}\)
        \item \(\dfrac{\pi}{4}\)
        \item \(1\)
      \end{enumerate}
    \item Integrate \( \sqrt{ax+b} \) with respect to x.
      \begin{enumerate}[label={\Alph*.}]
        \item \( \dfrac{1}{2a\sqrt{ax+b}} + C \)
        \item \( \dfrac{2}{3a}(ax+b)^{3/2} + C \)
        \item \( \dfrac{1}{a}(ax+b)^{3/2} + C \)
        \item \( \dfrac{2}{a}(ax+b)^{1/2} + C \)
      \end{enumerate}
    \item If \(f'(x) = x - \dfrac{1}{x^2}\) and \(f(1) = \dfrac{1}{2}\), find \(f(x)\).
      \begin{enumerate}[label={\Alph*.}]
        \item \(\dfrac{x^2}{2} + \dfrac{1}{x} + 1\)
        \item \(\dfrac{x^2}{2} - \dfrac{1}{x} + 1\)
        \item \(\dfrac{x^2}{2} + \dfrac{1}{x} - 1\)
        \item \(\dfrac{x^2}{2} - \dfrac{1}{x} - \dfrac{1}{2}\)
      \end{enumerate}
    \item Evaluate \(\displaystyle \int_{-2}^{2} (x^3 + \sin x) \mathrm{d}x\).
      \begin{enumerate}[label={\Alph*.}]
        \item \(0\)
        \item \(4\)
        \item \(8\)
        \item \(16/3\)
      \end{enumerate}
    \item Find \(\displaystyle \int x \sqrt{x^2+1} \mathrm{d}x\).
      \begin{enumerate}[label={\Alph*.}]
        \item \(\dfrac{1}{2}(x^2+1)^{3/2} + C\)
        \item \(\dfrac{1}{3}(x^2+1)^{3/2} + C\)
        \item \((x^2+1)^{1/2} + C\)
        \item \(x^2\sqrt{x^2+1} + C\)
      \end{enumerate}
    \item What is \(\displaystyle \int_{a}^{b} f(x) \mathrm{d}x + \int_{b}^{c} f(x) \mathrm{d}x\)?
      \begin{enumerate}[label={\Alph*.}]
        \item \(\displaystyle \int_{a}^{c} f(x) \mathrm{d}x\)
        \item \(\displaystyle \int_{c}^{a} f(x) \mathrm{d}x\)
        \item \(\displaystyle \int_{b}^{a} f(x) \mathrm{d}x + \int_{c}^{b} f(x) \mathrm{d}x\)
        \item \(0\)
      \end{enumerate}
    \item Evaluate \(\displaystyle \int {(e^{2x} + e^{-2x})}^2 \mathrm{d}x\).
      \begin{enumerate}[label={\Alph*.}]
        \item \(\dfrac{1}{4}e^{4x} + 2x - \dfrac{1}{4}e^{-4x} + C\)
        \item \(\dfrac{1}{2}e^{4x} + 2x - \dfrac{1}{2}e^{-4x} + C\)
        \item \(e^{4x} + 2 + e^{-4x} + C\)
        \item \(\dfrac{{(e^{2x} + e^{-2x})}^3}{3} + C\)
      \end{enumerate}
    \item Find the area between the curves \(y=x^2\) and \(y=x\).
      \begin{enumerate}[label={\Alph*.}]
        \item \(\dfrac{1}{3}\)
        \item \(\dfrac{1}{6}\)
        \item \(\dfrac{1}{2}\)
        \item \(1\)
      \end{enumerate}
    \item \(\displaystyle \int \tan^2 x \mathrm{d}x\).
      \begin{enumerate}[label={\Alph*.}]
        \item \(\sec^2 x - x + C\)
        \item \(\tan x - x + C\)
        \item \(\dfrac{\tan^3 x}{3} + C\)
        \item \(\ln|\sec x| + C\)
      \end{enumerate}
    \item Evaluate \(\displaystyle \int_{0}^{\ln 2} e^x \mathrm{d}x\).
      \begin{enumerate}[label={\Alph*.}]
        \item \(1\)
        \item \(2\)
        \item \(2 - \ln 2\)
        \item \(\ln 2\)
      \end{enumerate}
    \item Find \(\displaystyle \int \frac{\cos x}{1+\sin^2 x} \mathrm{d}x\).
      \begin{enumerate}[label={\Alph*.}]
        \item \(\ln(1+\sin^2 x) + C\)
        \item \(\arctan(\sin x) + C\)
        \item \(\arcsin(\cos x) + C\)
        \item \(\dfrac{-\sin x}{(1+\sin^2 x)^2} + C\)
      \end{enumerate}
    \item Find the volume of the solid generated by revolving the region bounded by \(y=\dfrac{1}{x}\), the x-axis, from \(x=1\) to \(x=2\).
      \begin{enumerate}[label={\Alph*.}]
        \item \(\pi\)
        \item \(\dfrac{\pi}{2}\)
        \item \(\dfrac{\pi}{3}\)
        \item \(2\pi\)
      \end{enumerate}
    \item Evaluate \(\displaystyle \int \frac{2x+3}{x^2+3x+5} \mathrm{d}x\).
      \begin{enumerate}[label={\Alph*.}]
        \item \(2\ln|x^2+3x+5| + C\)
        \item \(\ln|x^2+3x+5| + C\)
        \item \(\dfrac{1}{2}\ln|x^2+3x+5| + C\)
        \item \(\arctan(x^2+3x+5) + C\)
      \end{enumerate}
    \item Evaluate \(\displaystyle \int_{0}^{1} {x(x^2+1)}^3 \mathrm{d}x\).
      \begin{enumerate}[label={\Alph*.}]
        \item \(\dfrac{15}{8}\)
        \item \(\dfrac{7}{4}\)
        \item \(22\)
		\item \(\dfrac{17}{8}\)
      \end{enumerate}
  \end{enumerate}
\end{multicols}

\chapter{Combinatorics}
\section{Combination \& Permutation}
\subsection{Questions}
\begin{enumerate}[label={\arabic*.}]
\item Ralia has \(7\) different posters to be hung in her bedroom, living room, and kitchen. Assuming she has plans to plant at least a poster in each of the \(3\) rooms, how many choices does she have?
	\begin{enumerate}[label={\Alph*.}]
	\item \(49\)
	\item \(170\)
	\item \(210\)
	\item \(21\)
	\end{enumerate}
\item In how many ways can a committee of \(2\) women and \(3\) men be chosen from \(6\) men and \(5\) women?
	\begin{enumerate}[label={\Alph*.}]
	\item \(200\)
	\item \(100\)
	\item \(50\)
	\item \(30\)
	\end{enumerate}
\item In how many ways can the word MATHEMATICS be arranged?
	\begin{enumerate}[label={\Alph*.}]
	\item \(\frac{11!}{9!2!}\)
	\item \(\frac{11!}{9!2!2!}\)
	\item \(\frac{11!}{2!2!2!}\)
	\item \(\frac{11!}{2!2!}\)
	\end{enumerate}
\item In how many ways can the word ACCEPTANCE be arranged?
	\begin{enumerate}[label={\Alph*.}]
	\item \(\frac{10!}{2!2!3!}\)
	\item \(\frac{10!}{2!2!}\)
	\item \(10!\)
	\item \(\frac{10!}{2!3!}\)
	\end{enumerate}
\item Five people are to be arranged in a row for a group photograph. How many arrangements are there if a married couple in the group insist on sitting next to each other?
	\begin{enumerate}[label={\Alph*.}]
	\item \(48\)
	\item \(12\)
	\item \(7\)
	\item \(10\)
	\end{enumerate}
\item In how many ways can \(6\) subjects be selected from \(10\) subjects for an examination
	\begin{enumerate}[label={\Alph*.}]
	\item \(215\)
	\item \(218\)
	\item \(216\)
	\item \(210\)
	\end{enumerate}
\item In how many ways can a delegation of \(3\) be chosen from \(5\) men and \(3\) women, if atleast \(1\) man and \(1\) woman must be included?
	\begin{enumerate}[label={\Alph*.}]
	\item \(28\)
	\item \(30\)
	\item \(15\)
	\item \(45\)
	\end{enumerate}
\item
	\begin{enumerate}[label={\Alph*.}]
	\item \(\)
	\item \(\)
	\item \(\)
	\item \(\)
	\end{enumerate}
\item
	\begin{enumerate}[label={\Alph*.}]
	\item \(\)
	\item \(\)
	\item \(\)
	\item \(\)
	\end{enumerate}
\item
	\begin{enumerate}[label={\Alph*.}]
	\item \(\)
	\item \(\)
	\item \(\)
	\item \(\)
	\end{enumerate}
\item
	\begin{enumerate}[label={\Alph*.}]
	\item \(\)
	\item \(\)
	\item \(\)
	\item \(\)
	\end{enumerate}


\item
	\begin{enumerate}[label={\Alph*.}]
	\item \(\)
	\item \(\)
	\item \(\)
	\item \(\)
	\end{enumerate}
\item
	\begin{enumerate}[label={\Alph*.}]
	\item \(\)
	\item \(\)
	\item \(\)
	\item \(\)
	\end{enumerate}
\item
	\begin{enumerate}[label={\Alph*.}]
	\item \(\)
	\item \(\)
	\item \(\)
	\item \(\)
	\end{enumerate}
\item
	\begin{enumerate}[label={\Alph*.}]
	\item \(\)
	\item \(\)
	\item \(\)
	\item \(\)
	\end{enumerate}
\item
	\begin{enumerate}[label={\Alph*.}]
	\item \(\)
	\item \(\)
	\item \(\)
	\item \(\)
	\end{enumerate}
\item
	\begin{enumerate}[label={\Alph*.}]
	\item \(\)
	\item \(\)
	\item \(\)
	\item \(\)
	\end{enumerate}
\item
	\begin{enumerate}[label={\Alph*.}]
	\item \(\)
	\item \(\)
	\item \(\)
	\item \(\)
	\end{enumerate}
\item
	\begin{enumerate}[label={\Alph*.}]
	\item \(\)
	\item \(\)
	\item \(\)
	\item \(\)
	\end{enumerate}
\item
	\begin{enumerate}[label={\Alph*.}]
	\item \(\)
	\item \(\)
	\item \(\)
	\item \(\)
	\end{enumerate}
\item
	\begin{enumerate}[label={\Alph*.}]
	\item \(\)
	\item \(\)
	\item \(\)
	\item \(\)
	\end{enumerate}
\item
	\begin{enumerate}[label={\Alph*.}]
	\item \(\)
	\item \(\)
	\item \(\)
	\item \(\)
	\end{enumerate}
\item
	\begin{enumerate}[label={\Alph*.}]
	\item \(\)
	\item \(\)
	\item \(\)
	\item \(\)
	\end{enumerate}
\item
	\begin{enumerate}[label={\Alph*.}]
	\item \(\)
	\item \(\)
	\item \(\)
	\item \(\)
	\end{enumerate}
\item
	\begin{enumerate}[label={\Alph*.}]
	\item \(\)
	\item \(\)
	\item \(\)
	\item \(\)
	\end{enumerate}
\item
	\begin{enumerate}[label={\Alph*.}]
	\item \(\)
	\item \(\)
	\item \(\)
	\item \(\)
	\end{enumerate}
\item
	\begin{enumerate}[label={\Alph*.}]
	\item \(\)
	\item \(\)
	\item \(\)
	\item \(\)
	\end{enumerate}
\item
	\begin{enumerate}[label={\Alph*.}]
	\item \(\)
	\item \(\)
	\item \(\)
	\item \(\)
	\end{enumerate}
\item
	\begin{enumerate}[label={\Alph*.}]
	\item \(\)
	\item \(\)
	\item \(\)
	\item \(\)
	\end{enumerate}
\item
	\begin{enumerate}[label={\Alph*.}]
	\item \(\)
	\item \(\)
	\item \(\)
	\item \(\)
	\end{enumerate}
\item
	\begin{enumerate}[label={\Alph*.}]
	\item \(\)
	\item \(\)
	\item \(\)
	\item \(\)
	\end{enumerate}
\item
	\begin{enumerate}[label={\Alph*.}]
	\item \(\)
	\item \(\)
	\item \(\)
	\item \(\)
	\end{enumerate}
\item
	\begin{enumerate}[label={\Alph*.}]
	\item \(\)
	\item \(\)
	\item \(\)
	\item \(\)
	\end{enumerate}
\item
	\begin{enumerate}[label={\Alph*.}]
	\item \(\)
	\item \(\)
	\item \(\)
	\item \(\)
	\end{enumerate}
\item
	\begin{enumerate}[label={\Alph*.}]
	\item \(\)
	\item \(\)
	\item \(\)
	\item \(\)
	\end{enumerate}
\item
	\begin{enumerate}[label={\Alph*.}]
	\item \(\)
	\item \(\)
	\item \(\)
	\item \(\)
	\end{enumerate}
\item
	\begin{enumerate}[label={\Alph*.}]
	\item \(\)
	\item \(\)
	\item \(\)
	\item \(\)
	\end{enumerate}
\item
	\begin{enumerate}[label={\Alph*.}]
	\item \(\)
	\item \(\)
	\item \(\)
	\item \(\)
	\end{enumerate}
\item
	\begin{enumerate}[label={\Alph*.}]
	\item \(\)
	\item \(\)
	\item \(\)
	\item \(\)
	\end{enumerate}
\item
	\begin{enumerate}[label={\Alph*.}]
	\item \(\)
	\item \(\)
	\item \(\)
	\item \(\)
	\end{enumerate}
\item
	\begin{enumerate}[label={\Alph*.}]
	\item \(\)
	\item \(\)
	\item \(\)
	\item \(\)
	\end{enumerate}
\item
	\begin{enumerate}[label={\Alph*.}]
	\item \(\)
	\item \(\)
	\item \(\)
	\item \(\)
	\end{enumerate}
\item
	\begin{enumerate}[label={\Alph*.}]
	\item \(\)
	\item \(\)
	\item \(\)
	\item \(\)
	\end{enumerate}
\item
	\begin{enumerate}[label={\Alph*.}]
	\item \(\)
	\item \(\)
	\item \(\)
	\item \(\)
	\end{enumerate}
\item
	\begin{enumerate}[label={\Alph*.}]
	\item \(\)
	\item \(\)
	\item \(\)
	\item \(\)
	\end{enumerate}
\item
	\begin{enumerate}[label={\Alph*.}]
	\item \(\)
	\item \(\)
	\item \(\)
	\item \(\)
	\end{enumerate}
\item
	\begin{enumerate}[label={\Alph*.}]
	\item \(\)
	\item \(\)
	\item \(\)
	\item \(\)
	\end{enumerate}
\item
	\begin{enumerate}[label={\Alph*.}]
	\item \(\)
	\item \(\)
	\item \(\)
	\item \(\)
	\end{enumerate}
\item
	\begin{enumerate}[label={\Alph*.}]
	\item \(\)
	\item \(\)
	\item \(\)
	\item \(\)
	\end{enumerate}
\item
	\begin{enumerate}[label={\Alph*.}]
	\item \(\)
	\item \(\)
	\item \(\)
	\item \(\)
	\end{enumerate}
\item
	\begin{enumerate}[label={\Alph*.}]
	\item \(\)
	\item \(\)
	\item \(\)
	\item \(\)
	\end{enumerate}
\item
	\begin{enumerate}[label={\Alph*.}]
	\item \(\)
	\item \(\)
	\item \(\)
	\item \(\)
	\end{enumerate}
\item
	\begin{enumerate}[label={\Alph*.}]
	\item \(\)
	\item \(\)
	\item \(\)
	\item \(\)
	\end{enumerate}
\item
	\begin{enumerate}[label={\Alph*.}]
	\item \(\)
	\item \(\)
	\item \(\)
	\item \(\)
	\end{enumerate}
\item
	\begin{enumerate}[label={\Alph*.}]
	\item \(\)
	\item \(\)
	\item \(\)
	\item \(\)
	\end{enumerate}
\item
	\begin{enumerate}[label={\Alph*.}]
	\item \(\)
	\item \(\)
	\item \(\)
	\item \(\)
	\end{enumerate}
\item
	\begin{enumerate}[label={\Alph*.}]
	\item \(\)
	\item \(\)
	\item \(\)
	\item \(\)
	\end{enumerate}
\item
	\begin{enumerate}[label={\Alph*.}]
	\item \(\)
	\item \(\)
	\item \(\)
	\item \(\)
	\end{enumerate}
\item
	\begin{enumerate}[label={\Alph*.}]
	\item \(\)
	\item \(\)
	\item \(\)
	\item \(\)
	\end{enumerate}
\item
	\begin{enumerate}[label={\Alph*.}]
	\item \(\)
	\item \(\)
	\item \(\)
	\item \(\)
	\end{enumerate}
\item
	\begin{enumerate}[label={\Alph*.}]
	\item \(\)
	\item \(\)
	\item \(\)
	\item \(\)
	\end{enumerate}
\item
	\begin{enumerate}[label={\Alph*.}]
	\item \(\)
	\item \(\)
	\item \(\)
	\item \(\)
	\end{enumerate}
\item
	\begin{enumerate}[label={\Alph*.}]
	\item \(\)
	\item \(\)
	\item \(\)
	\item \(\)
	\end{enumerate}
\item
	\begin{enumerate}[label={\Alph*.}]
	\item \(\)
	\item \(\)
	\item \(\)
	\item \(\)
	\end{enumerate}
\item
	\begin{enumerate}[label={\Alph*.}]
	\item \(\)
	\item \(\)
	\item \(\)
	\item \(\)
	\end{enumerate}
\item
	\begin{enumerate}[label={\Alph*.}]
	\item \(\)
	\item \(\)
	\item \(\)
	\item \(\)
	\end{enumerate}
\item
	\begin{enumerate}[label={\Alph*.}]
	\item \(\)
	\item \(\)
	\item \(\)
	\item \(\)
	\end{enumerate}
\item
	\begin{enumerate}[label={\Alph*.}]
	\item \(\)
	\item \(\)
	\item \(\)
	\item \(\)
	\end{enumerate}
\item
	\begin{enumerate}[label={\Alph*.}]
	\item \(\)
	\item \(\)
	\item \(\)
	\item \(\)
	\end{enumerate}
\item
	\begin{enumerate}[label={\Alph*.}]
	\item \(\)
	\item \(\)
	\item \(\)
	\item \(\)
	\end{enumerate}
\item
	\begin{enumerate}[label={\Alph*.}]
	\item \(\)
	\item \(\)
	\item \(\)
	\item \(\)
	\end{enumerate}
\item
	\begin{enumerate}[label={\Alph*.}]
	\item \(\)
	\item \(\)
	\item \(\)
	\item \(\)
	\end{enumerate}
\item
	\begin{enumerate}[label={\Alph*.}]
	\item \(\)
	\item \(\)
	\item \(\)
	\item \(\)
	\end{enumerate}
\item
	\begin{enumerate}[label={\Alph*.}]
	\item \(\)
	\item \(\)
	\item \(\)
	\item \(\)
	\end{enumerate}
\item
	\begin{enumerate}[label={\Alph*.}]
	\item \(\)
	\item \(\)
	\item \(\)
	\item \(\)
	\end{enumerate}


\item
	\begin{enumerate}[label={\Alph*.}]
	\item \(\)
	\item \(\)
	\item \(\)
	\item \(\)
	\end{enumerate}
\item
	\begin{enumerate}[label={\Alph*.}]
	\item \(\)
	\item \(\)
	\item \(\)
	\item \(\)
	\end{enumerate}
\item
	\begin{enumerate}[label={\Alph*.}]
	\item \(\)
	\item \(\)
	\item \(\)
	\item \(\)
	\end{enumerate}
\item
	\begin{enumerate}[label={\Alph*.}]
	\item \(\)
	\item \(\)
	\item \(\)
	\item \(\)
	\end{enumerate}
\item
	\begin{enumerate}[label={\Alph*.}]
	\item \(\)
	\item \(\)
	\item \(\)
	\item \(\)
	\end{enumerate}
\item
	\begin{enumerate}[label={\Alph*.}]
	\item \(\)
	\item \(\)
	\item \(\)
	\item \(\)
	\end{enumerate}
\item
	\begin{enumerate}[label={\Alph*.}]
	\item \(\)
	\item \(\)
	\item \(\)
	\item \(\)
	\end{enumerate}
\item
	\begin{enumerate}[label={\Alph*.}]
	\item \(\)
	\item \(\)
	\item \(\)
	\item \(\)
	\end{enumerate}
\item
	\begin{enumerate}[label={\Alph*.}]
	\item \(\)
	\item \(\)
	\item \(\)
	\item \(\)
	\end{enumerate}
\item
	\begin{enumerate}[label={\Alph*.}]
	\item \(\)
	\item \(\)
	\item \(\)
	\item \(\)
	\end{enumerate}
\item
	\begin{enumerate}[label={\Alph*.}]
	\item \(\)
	\item \(\)
	\item \(\)
	\item \(\)
	\end{enumerate}
\item
	\begin{enumerate}[label={\Alph*.}]
	\item \(\)
	\item \(\)
	\item \(\)
	\item \(\)
	\end{enumerate}
\item
	\begin{enumerate}[label={\Alph*.}]
	\item \(\)
	\item \(\)
	\item \(\)
	\item \(\)
	\end{enumerate}
\item
	\begin{enumerate}[label={\Alph*.}]
	\item \(\)
	\item \(\)
	\item \(\)
	\item \(\)
	\end{enumerate}
\item
	\begin{enumerate}[label={\Alph*.}]
	\item \(\)
	\item \(\)
	\item \(\)
	\item \(\)
	\end{enumerate}
\item
	\begin{enumerate}[label={\Alph*.}]
	\item \(\)
	\item \(\)
	\item \(\)
	\item \(\)
	\end{enumerate}
\item
	\begin{enumerate}[label={\Alph*.}]
	\item \(\)
	\item \(\)
	\item \(\)
	\item \(\)
	\end{enumerate}
\item
	\begin{enumerate}[label={\Alph*.}]
	\item \(\)
	\item \(\)
	\item \(\)
	\item \(\)
	\end{enumerate}
\item
	\begin{enumerate}[label={\Alph*.}]
	\item \(\)
	\item \(\)
	\item \(\)
	\item \(\)
	\end{enumerate}
\item
	\begin{enumerate}[label={\Alph*.}]
	\item \(\)
	\item \(\)
	\item \(\)
	\item \(\)
	\end{enumerate}
\item
	\begin{enumerate}[label={\Alph*.}]
	\item \(\)
	\item \(\)
	\item \(\)
	\item \(\)
	\end{enumerate}
\item
	\begin{enumerate}[label={\Alph*.}]
	\item \(\)
	\item \(\)
	\item \(\)
	\item \(\)
	\end{enumerate}
\item
	\begin{enumerate}[label={\Alph*.}]
	\item \(\)
	\item \(\)
	\item \(\)
	\item \(\)
	\end{enumerate}
\item
	\begin{enumerate}[label={\Alph*.}]
	\item \(\)
	\item \(\)
	\item \(\)
	\item \(\)
	\end{enumerate}
\item
	\begin{enumerate}[label={\Alph*.}]
	\item \(\)
	\item \(\)
	\item \(\)
	\item \(\)
	\end{enumerate}
\end{enumerate}
\section{Probability}
\subsection{Questions}
\begin{multicols}{2}
\begin{enumerate}[label={\arabic*.}]
\item A bag contains 5 red balls and 3 blue balls. What is the probability of selecting a red ball?
	\begin{enumerate}[label={\Alph*.}]
	\item \(\dfrac{5}{8}\)
	\item \(\dfrac{3}{8}\)
	\item \(\dfrac{5}{3}\)
	\item \(\dfrac{3}{5}\)
	\end{enumerate}

\item Two dice are thrown together. What is the probability that the sum of the numbers is 7?
	\begin{enumerate}[label={\Alph*.}]
	\item \(\dfrac{1}{6}\)
	\item \(\dfrac{1}{12}\)
	\item \(\dfrac{5}{36}\)
	\item \(\dfrac{7}{36}\)
	\end{enumerate}

\item A box contains 4 red, 3 white, and 5 black balls. If one ball is drawn at random, what is the probability that it is not red?
	\begin{enumerate}[label={\Alph*.}]
	\item \(\dfrac{2}{3}\)
	\item \(\dfrac{1}{3}\)
	\item \(\dfrac{1}{2}\)
	\item \(\dfrac{3}{4}\)
	\end{enumerate}

\item A card is drawn at random from a standard deck of 52 cards. What is the probability of drawing a king?
	\begin{enumerate}[label={\Alph*.}]
	\item \(\dfrac{1}{13}\)
	\item \(\dfrac{1}{52}\)
	\item \(\dfrac{4}{52}\)
	\item \(\dfrac{1}{4}\)
	\end{enumerate}

\item If $P(A)$ = 0.6 and $P(B)$ = 0.4, and A and B are mutually exclusive events, find $P(A \text{or} B)$.
	\begin{enumerate}[label={\Alph*.}]
	\item \(1.0\)
	\item \(0.24\)
	\item \(0.8\)
	\item \(0.5\)
	\end{enumerate}

\item A bag contains 6 white and 4 black balls. Two balls are drawn at random without replacement. What is the probability that both are white?
	\begin{enumerate}[label={\Alph*.}]
	\item \(\dfrac{1}{3}\)
	\item \(\dfrac{2}{5}\)
	\item \(\dfrac{1}{2}\)
	\item \(\dfrac{3}{10}\)
	\end{enumerate}

\item The probability that it will rain tomorrow is 0.7. What is the probability that it will not rain?
	\begin{enumerate}[label={\Alph*.}]
	\item \(0.3\)
	\item \(0.7\)
	\item \(1.0\)
	\item \(0.5\)
	\end{enumerate}

\item Three coins are tossed simultaneously. What is the probability of getting at least two heads?
	\begin{enumerate}[label={\Alph*.}]
	\item \(\dfrac{1}{2}\)
	\item \(\dfrac{1}{4}\)
	\item \(\dfrac{3}{8}\)
	\item \(\dfrac{5}{8}\)
	\end{enumerate}

\item A die is rolled. What is the probability of getting an even number?
	\begin{enumerate}[label={\Alph*.}]
	\item \(\dfrac{1}{2}\)
	\item \(\dfrac{1}{3}\)
	\item \(\dfrac{2}{3}\)
	\item \(\dfrac{1}{6}\)
	\end{enumerate}

\item A bag contains 8 red and 5 blue marbles. If two marbles are drawn at random with replacement, what is the probability that both are red?
	\begin{enumerate}[label={\Alph*.}]
	\item \(\dfrac{64}{169}\)
	\item \(\dfrac{8}{13}\)
	\item \(\dfrac{56}{169}\)
	\item \(\dfrac{40}{169}\)
	\end{enumerate}

\item From a deck of 52 cards, what is the probability of drawing a heart or a king?
	\begin{enumerate}[label={\Alph*.}]
	\item \(\dfrac{4}{13}\)
	\item \(\dfrac{17}{52}\)
	\item \(\dfrac{13}{52}\)
	\item \(\dfrac{16}{52}\)
	\end{enumerate}

\item A bag contains 5 red, 4 blue, and 3 green balls. What is the probability of drawing a blue ball?
	\begin{enumerate}[label={\Alph*.}]
	\item \(\dfrac{1}{3}\)
	\item \(\dfrac{1}{4}\)
	\item \(\dfrac{5}{12}\)
	\item \(\dfrac{1}{2}\)
	\end{enumerate}

\item Two cards are drawn from a deck without replacement. What is the probability that both are aces?
	\begin{enumerate}[label={\Alph*.}]
	\item \(\dfrac{1}{221}\)
	\item \(\dfrac{4}{663}\)
	\item \(\dfrac{1}{169}\)
	\item \(\dfrac{2}{221}\)
	\end{enumerate}

\item A coin is tossed three times. What is the probability of getting exactly two tails?
	\begin{enumerate}[label={\Alph*.}]
	\item \(\dfrac{3}{8}\)
	\item \(\dfrac{1}{4}\)
	\item \(\dfrac{1}{2}\)
	\item \(\dfrac{1}{8}\)
	\end{enumerate}

\item If $P(A)$ = 0.5, $P(B)$ = 0.4, and $P(A \cap B)$ = 0.2, find $P(A \cup B)$.
	\begin{enumerate}[label={\Alph*.}]
	\item \(0.7\)
	\item \(0.9\)
	\item \(0.6\)
	\item \(0.8\)
	\end{enumerate}

\item A box contains 7 red and 5 white balls. Three balls are drawn at random. What is the probability that all are red?
	\begin{enumerate}[label={\Alph*.}]
	\item \(\dfrac{7}{44}\)
	\item \(\dfrac{35}{264}\)
	\item \(\dfrac{1}{12}\)
	\item \(\dfrac{7}{22}\)
	\end{enumerate}

\item A die is rolled twice. What is the probability that the sum is 10?
	\begin{enumerate}[label={\Alph*.}]
	\item \(\dfrac{1}{12}\)
	\item \(\dfrac{1}{9}\)
	\item \(\dfrac{1}{18}\)
	\item \(\dfrac{5}{36}\)
	\end{enumerate}

\item A card is drawn from a deck of 52 cards. What is the probability that it is either a spade or a face card?
	\begin{enumerate}[label={\Alph*.}]
	\item \(\dfrac{11}{26}\)
	\item \(\dfrac{25}{52}\)
	\item \(\dfrac{13}{26}\)
	\item \(\dfrac{22}{52}\)
	\end{enumerate}

\item In a class of 30 students, 18 play football and 12 play basketball. If 8 play both games, what is the probability that a randomly selected student plays at least one game?
	\begin{enumerate}[label={\Alph*.}]
	\item \(\dfrac{11}{15}\)
	\item \(\dfrac{2}{3}\)
	\item \(\dfrac{3}{5}\)
	\item \(\dfrac{4}{5}\)
	\end{enumerate}

\item A bag contains 4 red, 5 blue, and 6 green balls. Two balls are drawn at random. What is the probability that they are of different colors?
	\begin{enumerate}[label={\Alph*.}]
	\item \(\dfrac{37}{70}\)
	\item \(\dfrac{74}{105}\)
	\item \(\dfrac{31}{70}\)
	\item \(\dfrac{33}{70}\)
	\end{enumerate}

\item What is the probability of getting at least one head when a coin is tossed four times?
	\begin{enumerate}[label={\Alph*.}]
	\item \(\dfrac{15}{16}\)
	\item \(\dfrac{1}{2}\)
	\item \(\dfrac{7}{8}\)
	\item \(\dfrac{3}{4}\)
	\end{enumerate}

\item A number is selected at random from the numbers 1 to 20. What is the probability that it is divisible by 3 or 5?
	\begin{enumerate}[label={\Alph*.}]
	\item \(\dfrac{9}{20}\)
	\item \(\dfrac{1}{2}\)
	\item \(\dfrac{11}{20}\)
	\item \(\dfrac{2}{5}\)
	\end{enumerate}

\item Two dice are rolled. What is the probability that at least one shows a 6?
	\begin{enumerate}[label={\Alph*.}]
	\item \(\dfrac{11}{36}\)
	\item \(\dfrac{1}{3}\)
	\item \(\dfrac{5}{18}\)
	\item \(\dfrac{1}{6}\)
	\end{enumerate}

\item A bag contains 3 red, 4 white, and 5 black balls. If three balls are drawn at random, what is the probability that they are of the same color?
	\begin{enumerate}[label={\Alph*.}]
	\item \(\dfrac{3}{44}\)
	\item \(\dfrac{7}{44}\)
	\item \(\dfrac{1}{11}\)
	\item \(\dfrac{5}{44}\)
	\end{enumerate}

\item If the probability of an event is \(\dfrac{3}{7}\), what is the probability of its complement?
	\begin{enumerate}[label={\Alph*.}]
	\item \(\dfrac{4}{7}\)
	\item \(\dfrac{3}{7}\)
	\item \(\dfrac{1}{7}\)
	\item \(\dfrac{10}{7}\)
	\end{enumerate}

\item A letter is chosen at random from the word "MATHEMATICS". What is the probability that it is a vowel?
	\begin{enumerate}[label={\Alph*.}]
	\item \(\dfrac{4}{11}\)
	\item \(\dfrac{5}{11}\)
	\item \(\dfrac{3}{11}\)
	\item \(\dfrac{6}{11}\)
	\end{enumerate}

\item Three cards are drawn from a deck without replacement. What is the probability that all three are diamonds?
	\begin{enumerate}[label={\Alph*.}]
	\item \(\dfrac{11}{850}\)
	\item \(\dfrac{13}{204}\)
	\item \(\dfrac{1}{64}\)
	\item \(\dfrac{1}{52}\)
	\end{enumerate}

\item A box contains 6 defective and 14 non-defective items. If 3 items are drawn at random, what is the probability that at least one is defective?
	\begin{enumerate}[label={\Alph*.}]
	\item \(\dfrac{73}{95}\)
	\item \(\dfrac{22}{95}\)
	\item \(\dfrac{3}{10}\)
	\item \(\dfrac{91}{190}\)
	\end{enumerate}

\item A die is rolled. What is the probability of getting a prime number?
	\begin{enumerate}[label={\Alph*.}]
	\item \(\dfrac{1}{2}\)
	\item \(\dfrac{1}{3}\)
	\item \(\dfrac{2}{3}\)
	\item \(\dfrac{5}{6}\)
	\end{enumerate}

\item If two events A and B are independent with $P(A)$ = 0.6 and $P(B)$ = 0.5, find $P(A \text{and} B)$.
	\begin{enumerate}[label={\Alph*.}]
	\item \(0.3\)
	\item \(0.5\)
	\item \(0.8\)
	\item \(0.25\)
	\end{enumerate}

\item A bag contains 5 red, 6 blue, and 4 green balls. What is the probability of drawing either a red or green ball?
	\begin{enumerate}[label={\Alph*.}]
	\item \(\dfrac{3}{5}\)
	\item \(\dfrac{2}{5}\)
	\item \(\dfrac{1}{3}\)
	\item \(\dfrac{9}{15}\)
	\end{enumerate}

\item Four coins are tossed simultaneously. What is the probability of getting exactly three heads?
	\begin{enumerate}[label={\Alph*.}]
	\item \(\dfrac{1}{4}\)
	\item \(\dfrac{3}{16}\)
	\item \(\dfrac{1}{8}\)
	\item \(\dfrac{1}{2}\)
	\end{enumerate}

\item A number is chosen at random from 1 to 50. What is the probability that it is a multiple of 7?
	\begin{enumerate}[label={\Alph*.}]
	\item \(\dfrac{7}{50}\)
	\item \(\dfrac{1}{7}\)
	\item \(\dfrac{3}{25}\)
	\item \(\dfrac{2}{25}\)
	\end{enumerate}

\item Two events A and B are such that $P(A)$ = 0.4, $P(B)$ = 0.5, and $P(A \cup B)$ = 0.7. Find $P(A \cap B)$.
	\begin{enumerate}[label={\Alph*.}]
	\item \(0.2\)
	\item \(0.3\)
	\item \(0.1\)
	\item \(0.6\)
	\end{enumerate}

\item A bag contains 7 white and 8 black balls. If two balls are drawn at random, what is the probability that one is white and one is black?
	\begin{enumerate}[label={\Alph*.}]
	\item \(\dfrac{56}{105}\)
	\item \(\dfrac{8}{15}\)
	\item \(\dfrac{7}{15}\)
	\item \(\dfrac{1}{2}\)
	\end{enumerate}

\item A die is rolled three times. What is the probability of getting three different numbers?
	\begin{enumerate}[label={\Alph*.}]
	\item \(\dfrac{5}{9}\)
	\item \(\dfrac{20}{27}\)
	\item \(\dfrac{2}{3}\)
	\item \(\dfrac{25}{36}\)
	\end{enumerate}

\item From a group of 5 men and 4 women, a committee of 3 is selected at random. What is the probability that it contains at least one woman?
	\begin{enumerate}[label={\Alph*.}]
	\item \(\dfrac{5}{6}\)
	\item \(\dfrac{37}{42}\)
	\item \(\dfrac{2}{3}\)
	\item \(\dfrac{3}{4}\)
	\end{enumerate}

\item A card is drawn from a deck and replaced. This is done three times. What is the probability of getting exactly two spades?
	\begin{enumerate}[label={\Alph*.}]
	\item \(\dfrac{9}{64}\)
	\item \(\dfrac{3}{16}\)
	\item \(\dfrac{27}{256}\)
	\item \(\dfrac{1}{8}\)
	\end{enumerate}

\item In a single throw of two dice, what is the probability of getting a doublet?
	\begin{enumerate}[label={\Alph*.}]
	\item \(\dfrac{1}{6}\)
	\item \(\dfrac{1}{12}\)
	\item \(\dfrac{1}{36}\)
	\item \(\dfrac{5}{36}\)
	\end{enumerate}

\item A box contains 4 red, 3 blue, and 5 green marbles. Three marbles are drawn at random without replacement. What is the probability that all three are green?
	\begin{enumerate}[label={\Alph*.}]
	\item \(\dfrac{1}{22}\)
	\item \(\dfrac{5}{66}\)
	\item \(\dfrac{10}{132}\)
	\item \(\dfrac{1}{11}\)
	\end{enumerate}

\item If $P(A)$ = 0.7, $P(B)$ = 0.6, and $P(A \cap B)$ = 0.4, find $P(A' \cap B')$.
	\begin{enumerate}[label={\Alph*.}]
	\item \(0.1\)
	\item \(0.2\)
	\item \(0.3\)
	\item \(0.4\)
	\end{enumerate}

\item A letter is chosen at random from the word "PROBABILITY". What is the probability that it is the letter B?
	\begin{enumerate}[label={\Alph*.}]
	\item \(\dfrac{2}{11}\)
	\item \(\dfrac{1}{11}\)
	\item \(\dfrac{3}{11}\)
	\item \(\dfrac{1}{5}\)
	\end{enumerate}

\item Three dice are thrown simultaneously. What is the probability that the sum is 10?
	\begin{enumerate}[label={\Alph*.}]
	\item \(\dfrac{1}{8}\)
	\item \(\dfrac{25}{216}\)
	\item \(\dfrac{27}{216}\)
	\item \(\dfrac{1}{12}\)
	\end{enumerate}

\item A bag contains 5 white, 7 red, and 8 black balls. If three balls are drawn at random, what is the probability that they are of different colors?
	\begin{enumerate}[label={\Alph*.}]
	\item \(\dfrac{7}{19}\)
	\item \(\dfrac{280}{1140}\)
	\item \(\dfrac{14}{57}\)
	\item \(\dfrac{140}{570}\)
	\end{enumerate}

\item A number is selected at random from 1 to 100. What is the probability that it is divisible by both 3 and 5?
	\begin{enumerate}[label={\Alph*.}]
	\item \(\dfrac{1}{15}\)
	\item \(\dfrac{2}{25}\)
	\item \(\dfrac{3}{50}\)
	\item \(\dfrac{7}{100}\)
	\end{enumerate}

\item Two cards are drawn from a deck with replacement. What is the probability that both are hearts?
	\begin{enumerate}[label={\Alph*.}]
	\item \(\dfrac{1}{16}\)
	\item \(\dfrac{1}{4}\)
	\item \(\dfrac{13}{204}\)
	\item \(\dfrac{1}{8}\)
	\end{enumerate}

\item A box contains 10 items, 3 of which are defective. If 2 items are selected at random without replacement, what is the probability that both are defective?
	\begin{enumerate}[label={\Alph*.}]
	\item \(\dfrac{1}{15}\)
	\item \(\dfrac{3}{45}\)
	\item \(\dfrac{2}{15}\)
	\item \(\dfrac{1}{10}\)
	\end{enumerate}

\item A coin is tossed 5 times. What is the probability of getting at least 4 heads?
	\begin{enumerate}[label={\Alph*.}]
	\item \(\dfrac{3}{16}\)
	\item \(\dfrac{1}{8}\)
	\item \(\dfrac{5}{32}\)
	\item \(\dfrac{6}{32}\)
	\end{enumerate}

\item From a pack of 52 cards, two cards are drawn at random. What is the probability that both are red?
	\begin{enumerate}[label={\Alph*.}]
	\item \(\dfrac{25}{102}\)
	\item \(\dfrac{1}{4}\)
	\item \(\dfrac{13}{51}\)
	\item \(\dfrac{26}{51}\)
	\end{enumerate}

\item A die is rolled. Given that an even number appears, what is the probability that it is 4?
	\begin{enumerate}[label={\Alph*.}]
	\item \(\dfrac{1}{3}\)
	\item \(\dfrac{1}{2}\)
	\item \(\dfrac{1}{6}\)
	\item \(\dfrac{2}{3}\)
	\end{enumerate}

\item A bag contains 6 red, 5 blue, and 4 green balls. If two balls are drawn at random with replacement, what is the probability that both are of the same color?
	\begin{enumerate}[label={\Alph*.}]
	\item \(\dfrac{77}{225}\)
	\item \(\dfrac{1}{3}\)
	\item \(\dfrac{37}{105}\)
	\item \(\dfrac{61}{225}\)
	\end{enumerate}

\item Three students A, B, and C have probabilities \(\dfrac{1}{2}\), \(\dfrac{1}{3}\), and \(\dfrac{1}{4}\) of solving a problem independently. What is the probability that the problem will be solved?
	\begin{enumerate}[label={\Alph*.}]
	\item \(\dfrac{3}{4}\)
	\item \(\dfrac{2}{3}\)
	\item \(\dfrac{11}{12}\)
	\item \(\dfrac{5}{6}\)
	\end{enumerate}

\item A box contains 8 balls numbered 1 to 8. Two balls are drawn at random. What is the probability that their sum is odd?
	\begin{enumerate}[label={\Alph*.}]
	\item \(\dfrac{4}{7}\)
	\item \(\dfrac{1}{2}\)
	\item \(\dfrac{3}{7}\)
	\item \(\dfrac{16}{28}\)
	\end{enumerate}

\item A card is drawn from a deck of 52 cards. What is the probability that it is a black card or a queen?
	\begin{enumerate}[label={\Alph*.}]
	\item \(\dfrac{7}{13}\)
	\item \(\dfrac{1}{2}\)
	\item \(\dfrac{28}{52}\)
	\item \(\dfrac{15}{26}\)
	\end{enumerate}

\item Four persons are chosen at random from a group of 3 men, 2 women, and 4 children. What is the probability that exactly two of them are children?
	\begin{enumerate}[label={\Alph*.}]
	\item \(\dfrac{10}{21}\)
	\item \(\dfrac{3}{7}\)
	\item \(\dfrac{1}{2}\)
	\item \(\dfrac{2}{5}\)
	\end{enumerate}

\item A number is chosen at random from the first 30 natural numbers. What is the probability that it is a perfect square?
	\begin{enumerate}[label={\Alph*.}]
	\item \(\dfrac{1}{6}\)
	\item \(\dfrac{1}{5}\)
	\item \(\dfrac{1}{10}\)
	\item \(\dfrac{2}{15}\)
	\end{enumerate}

\item Two dice are thrown. What is the probability that the difference of numbers shown is 2?
	\begin{enumerate}[label={\Alph*.}]
	\item \(\dfrac{2}{9}\)
	\item \(\dfrac{1}{6}\)
	\item \(\dfrac{1}{9}\)
	\item \(\dfrac{7}{36}\)
	\end{enumerate}

\item A bag contains 4 white, 5 black, and 6 red balls. Three balls are drawn at random. What is the probability that at least one is white?
	\begin{enumerate}[label={\Alph*.}]
	\item \(\dfrac{37}{91}\)
	\item \(\dfrac{69}{91}\)
	\item \(\dfrac{54}{91}\)
	\item \(\dfrac{22}{91}\)
	\end{enumerate}

\item If $P(A)$ = 0.3, $P(B)$ = 0.4, and A and B are independent, find $P(A' \cap B')$.
	\begin{enumerate}[label={\Alph*.}]
	\item \(0.42\)
	\item \(0.58\)
	\item \(0.28\)
	\item \(0.12\)
	\end{enumerate}

\item A letter is selected at random from the word "ARRANGEMENT". What is the probability that it is a consonant?
	\begin{enumerate}[label={\Alph*.}]
	\item \(\dfrac{6}{11}\)
	\item \(\dfrac{5}{11}\)
	\item \(\dfrac{7}{11}\)
	\item \(\dfrac{4}{11}\)
	\end{enumerate}

\item Three cards are drawn from a deck with replacement. What is the probability that all three are aces?
	\begin{enumerate}[label={\Alph*.}]
	\item \(\dfrac{1}{2197}\)
	\item \(\dfrac{64}{140608}\)
	\item \(\dfrac{1}{169}\)
	\item \(\dfrac{8}{2197}\)
	\end{enumerate}

\item A box contains 5 red, 4 blue, and 3 yellow marbles. Two marbles are drawn without replacement. What is the probability that the first is red and the second is blue?
	\begin{enumerate}[label={\Alph*.}]
	\item \(\dfrac{5}{33}\)
	\item \(\dfrac{20}{132}\)
	\item \(\dfrac{10}{66}\)
	\item \(\dfrac{1}{6}\)
	\end{enumerate}

\item A die is thrown twice. What is the probability that the product of the numbers is even?
	\begin{enumerate}[label={\Alph*.}]
	\item \(\dfrac{3}{4}\)
	\item \(\dfrac{1}{2}\)
	\item \(\dfrac{5}{12}\)
	\item \(\dfrac{2}{3}\)
	\end{enumerate}

\item From 10 boys and 8 girls, a committee of 5 is selected. What is the probability that it contains exactly 3 boys?
	\begin{enumerate}[label={\Alph*.}]
	\item \(\dfrac{3360}{8568}\)
	\item \(\dfrac{1}{3}\)
	\item \(\dfrac{120}{357}\)
	\item \(\dfrac{280}{714}\)
	\end{enumerate}

\item A number is chosen at random from 1 to 60. What is the probability that it is divisible by 4 or 6?
	\begin{enumerate}[label={\Alph*.}]
	\item \(\dfrac{1}{2}\)
	\item \(\dfrac{13}{30}\)
	\item \(\dfrac{7}{15}\)
	\item \(\dfrac{2}{5}\)
	\end{enumerate}

\item Two events A and B are such that $P(A)$ = 0.5, $P(B)$ = 0.6, and $P(A \cap B)$ = 0.3. Find $P(A|B)$.
	\begin{enumerate}[label={\Alph*.}]
	\item \(0.5\)
	\item \(0.6\)
	\item \(0.3\)
	\item \(0.8\)
	\end{enumerate}

\item A bag contains 7 red, 5 white, and 6 blue balls. If one ball is drawn at random, what is the probability that it is either red or white?
	\begin{enumerate}[label={\Alph*.}]
	\item \(\dfrac{2}{3}\)
	\item \(\dfrac{7}{18}\)
	\item \(\dfrac{5}{9}\)
	\item \(\dfrac{12}{18}\)
	\end{enumerate}

\item Four coins are tossed. What is the probability of getting at least two tails?
	\begin{enumerate}[label={\Alph*.}]
	\item \(\dfrac{11}{16}\)
	\item \(\dfrac{5}{8}\)
	\item \(\dfrac{3}{4}\)
	\item \(\dfrac{1}{2}\)
	\end{enumerate}

\item A number is selected at random from integers 1 to 25. What is the probability that it is prime?
	\begin{enumerate}[label={\Alph*.}]
	\item \(\dfrac{9}{25}\)
	\item \(\dfrac{2}{5}\)
	\item \(\dfrac{8}{25}\)
	\item \(\dfrac{1}{3}\)
	\end{enumerate}

\item Two dice are rolled. What is the probability that the sum is greater than 9?
	\begin{enumerate}[label={\Alph*.}]
	\item \(\dfrac{1}{6}\)
	\item \(\dfrac{5}{36}\)
	\item \(\dfrac{1}{4}\)
	\item \(\dfrac{7}{36}\)
	\end{enumerate}

\item A bag contains 6 identical red balls and 4 identical blue balls. If 3 balls are drawn at random, what is the probability of getting 2 red and 1 blue?
	\begin{enumerate}[label={\Alph*.}]
	\item \(\dfrac{1}{2}\)
	\item \(\dfrac{3}{5}\)
	\item \(\dfrac{2}{5}\)
	\item \(\dfrac{60}{120}\)
	\end{enumerate}

\item If events A and B are independent with $P(A)$ = 0.7 and $P(B)$ = 0.5, find $P(A \cup B)$.
	\begin{enumerate}[label={\Alph*.}]
	\item \(0.85\)
	\item \(0.9\)
	\item \(1.2\)
	\item \(0.35\)
	\end{enumerate}

\item A card is drawn from a deck. What is the probability that it is a red face card?
	\begin{enumerate}[label={\Alph*.}]
	\item \(\dfrac{3}{26}\)
	\item \(\dfrac{1}{13}\)
	\item \(\dfrac{6}{52}\)
	\item \(\dfrac{1}{4}\)
	\end{enumerate}

\item Three students attempt a problem independently. Their probabilities of solving it are \(\dfrac{1}{2}\), \(\dfrac{1}{3}\), and \(\dfrac{1}{5}\). What is the probability that exactly one solves it?
	\begin{enumerate}[label={\Alph*.}]
	\item \(\dfrac{7}{15}\)
	\item \(\dfrac{13}{30}\)
	\item \(\dfrac{1}{2}\)
	\item \(\dfrac{3}{10}\)
	\end{enumerate}

\item A box contains 9 tickets numbered 1 to 9. Two tickets are drawn at random. What is the probability that the sum is even?
	\begin{enumerate}[label={\Alph*.}]
	\item \(\dfrac{4}{9}\)
	\item \(\dfrac{5}{9}\)
	\item \(\dfrac{1}{2}\)
	\item \(\dfrac{20}{36}\)
	\end{enumerate}

\item A die is rolled four times. What is the probability of getting at least one six?
	\begin{enumerate}[label={\Alph*.}]
	\item \(\dfrac{671}{1296}\)
	\item \(\dfrac{625}{1296}\)
	\item \(\dfrac{2}{3}\)
	\item \(\dfrac{1}{2}\)
	\end{enumerate}

\item From a group of 6 men and 5 women, two people are selected at random. What is the probability that both are women?
	\begin{enumerate}[label={\Alph*.}]
	\item \(\dfrac{2}{11}\)
	\item \(\dfrac{10}{55}\)
	\item \(\dfrac{5}{22}\)
	\item \(\dfrac{1}{6}\)
	\end{enumerate}

\item A bag contains 5 white and 7 black balls. If two balls are drawn at random with replacement, what is the probability that at least one is white?
	\begin{enumerate}[label={\Alph*.}]
	\item \(\dfrac{95}{144}\)
	\item \(\dfrac{49}{144}\)
	\item \(\dfrac{5}{12}\)
	\item \(\dfrac{2}{3}\)
	\end{enumerate}

\item Two cards are drawn from a deck without replacement. What is the probability that the first is a king and the second is a queen?
	\begin{enumerate}[label={\Alph*.}]
	\item \(\dfrac{4}{663}\)
	\item \(\dfrac{16}{2652}\)
	\item \(\dfrac{1}{169}\)
	\item \(\dfrac{8}{663}\)
	\end{enumerate}

\item A number is chosen at random from 1 to 40. What is the probability that it is divisible by 2 and 3?
	\begin{enumerate}[label={\Alph*.}]
	\item \(\dfrac{1}{6}\)
	\item \(\dfrac{3}{20}\)
	\item \(\dfrac{7}{40}\)
	\item \(\dfrac{1}{5}\)
	\end{enumerate}

\item Three dice are thrown simultaneously. What is the probability of getting the same number on all three?
	\begin{enumerate}[label={\Alph*.}]
	\item \(\dfrac{1}{36}\)
	\item \(\dfrac{1}{216}\)
	\item \(\dfrac{6}{216}\)
	\item \(\dfrac{1}{6}\)
	\end{enumerate}

\item A bag contains 4 red, 3 blue, and 5 green marbles. Three marbles are drawn at random. What is the probability that no two are of the same color?
	\begin{enumerate}[label={\Alph*.}]
	\item \(\dfrac{3}{11}\)
	\item \(\dfrac{60}{220}\)
	\item \(\dfrac{2}{11}\)
	\item \(\dfrac{30}{110}\)
	\end{enumerate}

\item If $P(A)$ = 0.4, $P(B)$ = 0.5, and $P(A \cap B)$ = 0.2, find $P(B|A)$.
	\begin{enumerate}[label={\Alph*.}]
	\item \(0.5\)
	\item \(0.4\)
	\item \(0.2\)
	\item \(0.25\)
	\end{enumerate}

\item A coin is biased such that heads is twice as likely as tails. What is the probability of getting tails?
	\begin{enumerate}[label={\Alph*.}]
	\item \(\dfrac{1}{3}\)
	\item \(\dfrac{1}{2}\)
	\item \(\dfrac{2}{3}\)
	\item \(\dfrac{1}{4}\)
	\end{enumerate}

\item A letter is chosen at random from the word "SUCCESS". What is the probability that it is the letter S?
	\begin{enumerate}[label={\Alph*.}]
	\item \(\dfrac{3}{7}\)
	\item \(\dfrac{4}{7}\)
	\item \(\dfrac{2}{7}\)
	\item \(\dfrac{1}{2}\)
	\end{enumerate}

\item Two dice are rolled. What is the probability that at least one die shows a number greater than 4?
	\begin{enumerate}[label={\Alph*.}]
	\item \(\dfrac{5}{9}\)
	\item \(\dfrac{2}{3}\)
	\item \(\dfrac{20}{36}\)
	\item \(\dfrac{4}{9}\)
	\end{enumerate}

\item A box contains 8 red, 6 blue, and 4 green balls. If three balls are drawn at random without replacement, what is the probability that all are blue?
	\begin{enumerate}[label={\Alph*.}]
	\item \(\dfrac{5}{204}\)
	\item \(\dfrac{20}{816}\)
	\item \(\dfrac{1}{34}\)
	\item \(\dfrac{6}{204}\)
	\end{enumerate}

\item If events A and B are mutually exclusive with $P(A)$ = 0.3 and $P(B)$ = 0.5, find $P(A' \cap B')$.
	\begin{enumerate}[label={\Alph*.}]
	\item \(0.2\)
	\item \(0.8\)
	\item \(0.35\)
	\item \(0.15\)
	\end{enumerate}

\item A number is selected at random from integers 10 to 30. What is the probability that it is divisible by 3?
	\begin{enumerate}[label={\Alph*.}]
	\item \(\dfrac{1}{3}\)
	\item \(\dfrac{7}{21}\)
	\item \(\dfrac{8}{21}\)
	\item \(\dfrac{2}{7}\)
	\end{enumerate}

\item Three cards are drawn from a deck without replacement. What is the probability that at least one is a spade?
	\begin{enumerate}[label={\Alph*.}]
	\item \(\dfrac{133}{204}\)
	\item \(\dfrac{71}{204}\)
	\item \(\dfrac{2}{3}\)
	\item \(\dfrac{39}{68}\)
	\end{enumerate}

\item A bag contains 10 balls numbered 1 to 10. If two balls are drawn at random, what is the probability that their product is odd?
	\begin{enumerate}[label={\Alph*.}]
	\item \(\dfrac{2}{9}\)
	\item \(\dfrac{10}{45}\)
	\item \(\dfrac{1}{3}\)
	\item \(\dfrac{5}{18}\)
	\end{enumerate}

\item A bag contains 7 red, 5 blue, and 3 yellow balls. If three balls are drawn at random without replacement, what is the probability that exactly two are red?
	\begin{enumerate}[label={\Alph*.}]
	\item \(\dfrac{21}{65}\)
	\item \(\dfrac{168}{455}\)
	\item \(\dfrac{42}{91}\)
	\item \(\dfrac{3}{13}\)
	\end{enumerate}

\item Two friends agree to meet between 2:00 PM and 3:00 PM. Each arrives at a random time and waits for 15 minutes. What is the probability that they meet?
	\begin{enumerate}[label={\Alph*.}]
	\item \(\dfrac{7}{16}\)
	\item \(\dfrac{1}{4}\)
	\item \(\dfrac{1}{2}\)
	\item \(\dfrac{3}{8}\)
	\end{enumerate}
	
\item Two events A and B are independent with $P(A) = 0.4$ and $P(A \cup B) = 0.7$. Find $P(B)$.
	\begin{enumerate}[label={\Alph*.}]
	\item \(0.5\)
	\item \(0.6\)
	\item \(0.3\)
	\item \(0.4\)
	\end{enumerate}

\item A die is rolled. What is the probability of getting a number less than 5?
	\begin{enumerate}[label={\Alph*.}]
	\item \(\dfrac{2}{3}\)
	\item \(\dfrac{1}{2}\)
	\item \(\dfrac{5}{6}\)
	\item \(\dfrac{3}{4}\)
	\end{enumerate}

\item From a deck of 52 cards, three cards are drawn with replacement. What is the probability that all three are kings?
	\begin{enumerate}[label={\Alph*.}]
	\item \(\dfrac{1}{2197}\)
	\item \(\dfrac{64}{140608}\)
	\item \(\dfrac{8}{2197}\)
	\item \(\dfrac{1}{169}\)
	\end{enumerate}

\item A box contains 5 red, 4 blue, and 6 green balls. Two balls are drawn without replacement. What is the probability that both are of the same color?
	\begin{enumerate}[label={\Alph*.}]
	\item \(\dfrac{16}{105}\)
	\item \(\dfrac{31}{105}\)
	\item \(\dfrac{1}{3}\)
	\item \(\dfrac{37}{105}\)
	\end{enumerate}

\item A coin is tossed 6 times. What is the probability of getting exactly 4 heads?
	\begin{enumerate}[label={\Alph*.}]
	\item \(\dfrac{15}{64}\)
	\item \(\dfrac{5}{16}\)
	\item \(\dfrac{3}{8}\)
	\item \(\dfrac{1}{4}\)
	\end{enumerate}

\item A number is chosen at random from 1 to 80. What is the probability that it is a multiple of both 4 and 6?
	\begin{enumerate}[label={\Alph*.}]
	\item \(\dfrac{1}{20}\)
	\item \(\dfrac{3}{40}\)
	\item \(\dfrac{7}{80}\)
	\item \(\dfrac{1}{16}\)
	\end{enumerate}

\item Two dice are thrown. What is the probability that the product of the numbers is a perfect square?
	\begin{enumerate}[label={\Alph*.}]
	\item \(\dfrac{1}{4}\)
	\item \(\dfrac{11}{36}\)
	\item \(\dfrac{7}{36}\)
	\item \(\dfrac{5}{18}\)
	\end{enumerate}
\end{enumerate}
\end{multicols}
\chapter{Statistics}
\section{Measures of Central Tendency}
\subsection{Questions}
\begin{multicols}{2}
\begin{enumerate}[label={\arabic*.}]
\item Find the mean of the numbers 3, 5, 7, 9, and 11.
	\begin{enumerate}[label={\Alph*.}]
	\item \(7\)
	\item \(6\)
	\item \(8\)
	\item \(5\)
	\end{enumerate}

\item The median of the data set 2, 5, 8, 11, 14 is
	\begin{enumerate}[label={\Alph*.}]
	\item \(8\)
	\item \(5\)
	\item \(11\)
	\item \(7\)
	\end{enumerate}

\item Find the mode of the following data: 3, 5, 3, 7, 3, 9, 5.
	\begin{enumerate}[label={\Alph*.}]
	\item \(3\)
	\item \(5\)
	\item \(7\)
	\item \(9\)
	\end{enumerate}

\item The mean of five numbers is 12. If four of the numbers are 10, 11, 13, and 15, find the fifth number.
	\begin{enumerate}[label={\Alph*.}]
	\item \(11\)
	\item \(12\)
	\item \(10\)
	\item \(9\)
	\end{enumerate}

\item Find the median of 15, 20, 18, 12, 25, 22, 16.
	\begin{enumerate}[label={\Alph*.}]
	\item \(18\)
	\item \(20\)
	\item \(16\)
	\item \(19\)
	\end{enumerate}

\item The data set 4, 7, 7, 10, 10, 10, 13 has a mode of
	\begin{enumerate}[label={\Alph*.}]
	\item \(10\)
	\item \(7\)
	\item \(13\)
	\item \(4\)
	\end{enumerate}

\item If the mean of 8, 10, x, 14, and 16 is 12, find the value of x.
	\begin{enumerate}[label={\Alph*.}]
	\item \(12\)
	\item \(11\)
	\item \(10\)
	\item \(13\)
	\end{enumerate}

\item Find the mean of the first five prime numbers.
	\begin{enumerate}[label={\Alph*.}]
	\item \(5.6\)
	\item \(6\)
	\item \(5\)
	\item \(6.5\)
	\end{enumerate}

\item The median of 3, 7, 5, 9, 11, 13 is
	\begin{enumerate}[label={\Alph*.}]
	\item \(8\)
	\item \(7\)
	\item \(9\)
	\item \(7.5\)
	\end{enumerate}

\item Which measure of central tendency is most affected by extreme values?
	\begin{enumerate}[label={\Alph*.}]
	\item Mean
	\item Median
	\item Mode
	\item Range
	\end{enumerate}

\item Find the mode of 12, 15, 12, 18, 20, 12, 15, 18, 12.
	\begin{enumerate}[label={\Alph*.}]
	\item \(12\)
	\item \(15\)
	\item \(18\)
	\item \(20\)
	\end{enumerate}

\item The mean of 6, 8, 10, 12, and x is 10. Find x.
	\begin{enumerate}[label={\Alph*.}]
	\item \(14\)
	\item \(12\)
	\item \(10\)
	\item \(16\)
	\end{enumerate}

\item Find the median of 2, 4, 6, 8, 10, 12, 14, 16.
	\begin{enumerate}[label={\Alph*.}]
	\item \(8\)
	\item \(9\)
	\item \(10\)
	\item \(7\)
	\end{enumerate}

\item The ages of five students are 15, 16, 14, 17, and 18 years. Find their mean age.
	\begin{enumerate}[label={\Alph*.}]
	\item \(16\)
	\item \(15\)
	\item \(17\)
	\item \(14\)
	\end{enumerate}

\item If the mean of 7 numbers is 15 and the mean of 3 of them is 12, find the mean of the remaining 4 numbers.
	\begin{enumerate}[label={\Alph*.}]
	\item \(17.25\)
	\item \(16.5\)
	\item \(18\)
	\item \(15.75\)
	\end{enumerate}

\item Find the mode of the data: 5, 7, 5, 8, 5, 9, 7, 7, 5.
	\begin{enumerate}[label={\Alph*.}]
	\item \(5\)
	\item \(7\)
	\item \(8\)
	\item \(9\)
	\end{enumerate}

\item The median of 21, 15, 18, 12, 24, 27 is
	\begin{enumerate}[label={\Alph*.}]
	\item \(19.5\)
	\item \(18\)
	\item \(21\)
	\item \(20\)
	\end{enumerate}

\item Find the mean of 20, 25, 30, 35, 40.
	\begin{enumerate}[label={\Alph*.}]
	\item \(30\)
	\item \(25\)
	\item \(35\)
	\item \(28\)
	\end{enumerate}

\item The data set 2, 4, 6, 8 has no mode. What can be said about this data?
	\begin{enumerate}[label={\Alph*.}]
	\item All values occur with equal frequency
	\item The data is bimodal
	\item The median is 5
	\item Both A and C
	\end{enumerate}

\item If the mean of 5, 7, 9, x, and 11 is 8, find x.
	\begin{enumerate}[label={\Alph*.}]
	\item \(8\)
	\item \(7\)
	\item \(9\)
	\item \(6\)
	\end{enumerate}

\item Find the median of 40, 35, 50, 45, 30, 55, 60.
	\begin{enumerate}[label={\Alph*.}]
	\item \(45\)
	\item \(50\)
	\item \(40\)
	\item \(47.5\)
	\end{enumerate}

\item The mode of 3, 5, 7, 5, 9, 5, 11, 7 is
	\begin{enumerate}[label={\Alph*.}]
	\item \(5\)
	\item \(7\)
	\item \(9\)
	\item \(3\)
	\end{enumerate}

\item Calculate the mean of 12, 18, 24, 30, 36.
	\begin{enumerate}[label={\Alph*.}]
	\item \(24\)
	\item \(22\)
	\item \(26\)
	\item \(20\)
	\end{enumerate}

\item The median of an even number of observations is
	\begin{enumerate}[label={\Alph*.}]
	\item The middle value
	\item The average of the two middle values
	\item The most frequent value
	\item The sum of all values
	\end{enumerate}

\item Find the mean of the squares of the first four natural numbers.
	\begin{enumerate}[label={\Alph*.}]
	\item \(7.5\)
	\item \(10\)
	\item \(5\)
	\item \(8\)
	\end{enumerate}

\item If the mean of 10 numbers is 20 and one number 25 is removed, what is the new mean?
	\begin{enumerate}[label={\Alph*.}]
	\item \(19.44\)
	\item \(20\)
	\item \(18.5\)
	\item \(21\)
	\end{enumerate}

\item Find the mode of 8, 10, 12, 10, 8, 14, 10, 8, 10.
	\begin{enumerate}[label={\Alph*.}]
	\item \(10\)
	\item \(8\)
	\item Both 8 and 10
	\item \(12\)
	\end{enumerate}

\item The median of 5, 10, 15, 20, 25, 30, 35, 40, 45 is
	\begin{enumerate}[label={\Alph*.}]
	\item \(25\)
	\item \(20\)
	\item \(30\)
	\item \(22.5\)
	\end{enumerate}

\item Find the mean of all multiples of 5 between 1 and 30.
	\begin{enumerate}[label={\Alph*.}]
	\item \(17.5\)
	\item \(15\)
	\item \(20\)
	\item \(12.5\)
	\end{enumerate}

\item The mean of 4, 6, 8, 10, 12 is increased by 3 when each number is increased by
	\begin{enumerate}[label={\Alph*.}]
	\item \(3\)
	\item \(15\)
	\item \(6\)
	\item \(9\)
	\end{enumerate}

\item Find the median of 100, 90, 80, 110, 120, 95, 105.
	\begin{enumerate}[label={\Alph*.}]
	\item \(100\)
	\item \(105\)
	\item \(95\)
	\item \(110\)
	\end{enumerate}

\item The mode of a data set with all different values is
	\begin{enumerate}[label={\Alph*.}]
	\item Zero
	\item Does not exist
	\item The mean
	\item The median
	\end{enumerate}

\item If the mean of x, x+2, x+4, x+6, and x+8 is 10, find x.
	\begin{enumerate}[label={\Alph*.}]
	\item \(6\)
	\item \(7\)
	\item \(8\)
	\item \(5\)
	\end{enumerate}

\item Find the median of 2.5, 3.5, 4.5, 5.5, 6.5, 7.5.
	\begin{enumerate}[label={\Alph*.}]
	\item \(5\)
	\item \(5.5\)
	\item \(4.5\)
	\item \(6\)
	\end{enumerate}

\item The mean of 15 observations is 32. If two observations 40 and 50 are added, what is the new mean?
	\begin{enumerate}[label={\Alph*.}]
	\item \(33.29\)
	\item \(34\)
	\item \(35\)
	\item \(32.5\)
	\end{enumerate}

\item Find the mode of 1, 2, 2, 3, 3, 3, 4, 4, 4, 4.
	\begin{enumerate}[label={\Alph*.}]
	\item \(4\)
	\item \(3\)
	\item \(2\)
	\item \(1\)
	\end{enumerate}

\item The median of 7, 14, 21, 28, 35 is
	\begin{enumerate}[label={\Alph*.}]
	\item \(21\)
	\item \(14\)
	\item \(28\)
	\item \(20\)
	\end{enumerate}

\item Calculate the mean of the first 10 even natural numbers.
	\begin{enumerate}[label={\Alph*.}]
	\item \(11\)
	\item \(10\)
	\item \(12\)
	\item \(9\)
	\end{enumerate}

\item If the mean of 20 observations is 15 and that of another 30 observations is 20, find the mean of all 50 observations.
	\begin{enumerate}[label={\Alph*.}]
	\item \(18\)
	\item \(17.5\)
	\item \(19\)
	\item \(16.5\)
	\end{enumerate}

\item Find the mode of the data where all frequencies are equal.
	\begin{enumerate}[label={\Alph*.}]
	\item Does not exist
	\item Zero
	\item The mean
	\item All values
	\end{enumerate}

\item The median of 1, 3, 5, 7, 9, 11, 13, 15, 17 is
	\begin{enumerate}[label={\Alph*.}]
	\item \(9\)
	\item \(8\)
	\item \(10\)
	\item \(7\)
	\end{enumerate}

\item Find the mean of 2.5, 3.0, 3.5, 4.0, 4.5, 5.0.
	\begin{enumerate}[label={\Alph*.}]
	\item \(3.75\)
	\item \(4\)
	\item \(3.5\)
	\item \(4.25\)
	\end{enumerate}

\item If the median of a data set is 25 and the mean is 30, the data is
	\begin{enumerate}[label={\Alph*.}]
	\item Positively skewed
	\item Negatively skewed
	\item Symmetric
	\item Normal
	\end{enumerate}

\item Find the mode of 50, 60, 70, 60, 80, 60, 90, 70.
	\begin{enumerate}[label={\Alph*.}]
	\item \(60\)
	\item \(70\)
	\item \(50\)
	\item \(80\)
	\end{enumerate}

\item The mean of three numbers is 20. If two of them are 15 and 18, find the third number.
	\begin{enumerate}[label={\Alph*.}]
	\item \(27\)
	\item \(25\)
	\item \(22\)
	\item \(20\)
	\end{enumerate}

\item Find the median of 0.5, 1.5, 2.5, 3.5, 4.5, 5.5, 6.5, 7.5.
	\begin{enumerate}[label={\Alph*.}]
	\item \(4\)
	\item \(3.5\)
	\item \(4.5\)
	\item \(5\)
	\end{enumerate}

\item The mode is most useful when data is
	\begin{enumerate}[label={\Alph*.}]
	\item Categorical
	\item Continuous
	\item Symmetric
	\item Normally distributed
	\end{enumerate}

\item Calculate the mean of 1, 4, 9, 16, 25.
	\begin{enumerate}[label={\Alph*.}]
	\item \(11\)
	\item \(10\)
	\item \(12\)
	\item \(13\)
	\end{enumerate}

\item If each observation in a data set is multiplied by 5, the mean is
	\begin{enumerate}[label={\Alph*.}]
	\item Multiplied by 5
	\item Divided by 5
	\item Increased by 5
	\item Remains the same
	\end{enumerate}

\item Find the median of 11, 22, 33, 44, 55, 66, 77, 88, 99.
	\begin{enumerate}[label={\Alph*.}]
	\item \(55\)
	\item \(44\)
	\item \(50\)
	\item \(60\)
	\end{enumerate}

\item The mean of 8 numbers is 25. If 5 is subtracted from each number, the new mean is
	\begin{enumerate}[label={\Alph*.}]
	\item \(20\)
	\item \(25\)
	\item \(30\)
	\item \(15\)
	\end{enumerate}

\item Find the mode of 15, 20, 25, 20, 30, 25, 20, 35.
	\begin{enumerate}[label={\Alph*.}]
	\item \(20\)
	\item \(25\)
	\item \(15\)
	\item \(30\)
	\end{enumerate}

\item The median of 6, 12, 18, 24, 30, 36 is
	\begin{enumerate}[label={\Alph*.}]
	\item \(21\)
	\item \(18\)
	\item \(24\)
	\item \(20\)
	\end{enumerate}

\item Find the mean of the first 7 odd natural numbers.
	\begin{enumerate}[label={\Alph*.}]
	\item \(7\)
	\item \(8\)
	\item \(6\)
	\item \(9\)
	\end{enumerate}

\item If the mean of 10, 15, 20, x, 30 is 20, find x.
	\begin{enumerate}[label={\Alph*.}]
	\item \(25\)
	\item \(20\)
	\item \(22\)
	\item \(18\)
	\end{enumerate}

\item The median is always
	\begin{enumerate}[label={\Alph*.}]
	\item A value in the data set
	\item The middle value when data is ordered
	\item Equal to the mean
	\item The most frequent value
	\end{enumerate}

\item Find the mode of 100, 200, 200, 300, 300, 300, 400.
	\begin{enumerate}[label={\Alph*.}]
	\item \(300\)
	\item \(200\)
	\item \(100\)
	\item \(400\)
	\end{enumerate}

\item The mean of 5 consecutive odd numbers is 21. Find the largest number.
	\begin{enumerate}[label={\Alph*.}]
	\item \(25\)
	\item \(23\)
	\item \(27\)
	\item \(21\)
	\end{enumerate}

\item Find the median of 50, 55, 60, 65, 70, 75, 80, 85, 90, 95.
	\begin{enumerate}[label={\Alph*.}]
	\item \(72.5\)
	\item \(70\)
	\item \(75\)
	\item \(65\)
	\end{enumerate}

\item Calculate the mean of 3, 6, 9, 12, 15, 18.
	\begin{enumerate}[label={\Alph*.}]
	\item \(10.5\)
	\item \(12\)
	\item \(9\)
	\item \(11\)
	\end{enumerate}

\item The mode of a bimodal distribution has
	\begin{enumerate}[label={\Alph*.}]
	\item Two values
	\item One value
	\item Three values
	\item No value
	\end{enumerate}

\item If the mean of x, 2x, 3x, 4x, and 5x is 30, find x.
	\begin{enumerate}[label={\Alph*.}]
	\item \(10\)
	\item \(5\)
	\item \(15\)
	\item \(20\)
	\end{enumerate}

\item Find the median of 17, 23, 19, 21, 25, 20, 22.
	\begin{enumerate}[label={\Alph*.}]
	\item \(21\)
	\item \(20\)
	\item \(22\)
	\item \(19\)
	\end{enumerate}

\item The mean of 12 observations is 15. If each observation is doubled, the new mean is
	\begin{enumerate}[label={\Alph*.}]
	\item \(30\)
	\item \(15\)
	\item \(27\)
	\item \(18\)
	\end{enumerate}

\item Find the mode of 2, 4, 6, 2, 8, 2, 10, 4, 6, 2.
	\begin{enumerate}[label={\Alph*.}]
	\item \(2\)
	\item \(4\)
	\item \(6\)
	\item \(8\)
	\end{enumerate}

\item The median of 9, 18, 27, 36, 45, 54, 63, 72 is
	\begin{enumerate}[label={\Alph*.}]
	\item \(40.5\)
	\item \(36\)
	\item \(45\)
	\item \(42\)
	\end{enumerate}

\item Find the mean of all two-digit multiples of 10.
	\begin{enumerate}[label={\Alph*.}]
	\item \(55\)
	\item \(50\)
	\item \(60\)
	\item \(45\)
	\end{enumerate}

\item If the mean of a, b, c is 15 and the mean of a, b, c, d is 20, find d.
	\begin{enumerate}[label={\Alph*.}]
	\item \(35\)
	\item \(30\)
	\item \(25\)
	\item \(40\)
	\end{enumerate}

\item Find the median of integers from 1 to 9.
	\begin{enumerate}[label={\Alph*.}]
	\item \(5\)
	\item \(4\)
	\item \(6\)
	\item \(4.5\)
	\end{enumerate}

\item The mode of the data 5, 10, 15, 10, 20, 15, 10, 25 is
	\begin{enumerate}[label={\Alph*.}]
	\item \(10\)
	\item \(15\)
	\item \(20\)
	\item \(5\)
	\end{enumerate}

\item Calculate the mean of 0, 5, 10, 15, 20, 25, 30.
	\begin{enumerate}[label={\Alph*.}]
	\item \(15\)
	\item \(12.5\)
	\item \(17.5\)
	\item \(20\)
	\end{enumerate}

\item If every value in a data set is the same, then
	\begin{enumerate}[label={\Alph*.}]
	\item Mean = Median = Mode
	\item Mean \(\neq\) Median
	\item Mode does not exist
	\item Median \(\neq\) Mode
	\end{enumerate}

\item Find the median of 1.2, 3.4, 2.3, 4.5, 3.2, 5.1.
	\begin{enumerate}[label={\Alph*.}]
	\item \(3.3\)
	\item \(3.4\)
	\item \(3.2\)
	\item \(2.85\)
	\end{enumerate}

\item The mean of 6 numbers is 18. If one number 24 is replaced by 18, the new mean is
	\begin{enumerate}[label={\Alph*.}]
	\item \(17\)
	\item \(18\)
	\item \(19\)
	\item \(16\)
	\end{enumerate}

\item Find the mode of 7, 14, 21, 14, 28, 21, 14, 35.
	\begin{enumerate}[label={\Alph*.}]
	\item \(14\)
	\item \(21\)
	\item \(7\)
	\item \(28\)
	\end{enumerate}

\item The median of 1, 2, 3, 4, 5, 6, 7, 8, 9, 10, 11 is
	\begin{enumerate}[label={\Alph*.}]
	\item \(6\)
	\item \(5\)
	\item \(7\)
	\item \(5.5\)
	\end{enumerate}

\item Find the mean of 16, 18, 20, 22, 24, 26, 28.
	\begin{enumerate}[label={\Alph*.}]
	\item \(22\)
	\item \(20\)
	\item \(24\)
	\item \(21\)
	\end{enumerate}

\item If the median is greater than the mean, the data is likely
	\begin{enumerate}[label={\Alph*.}]
	\item Negatively skewed
	\item Positively skewed
	\item Symmetric
	\item Uniform
	\end{enumerate}

\item Find the mode of 33, 44, 55, 44, 66, 55, 44, 77.
	\begin{enumerate}[label={\Alph*.}]
	\item \(44\)
	\item \(55\)
	\item \(33\)
	\item \(66\)
	\end{enumerate}

\item The mean of the first n natural numbers is
	\begin{enumerate}[label={\Alph*.}]
	\item \(\dfrac{n+1}{2}\)
	\item \(\dfrac{n}{2}\)
	\item \(n+1\)
	\item \(\dfrac{n-1}{2}\)
	\end{enumerate}

\item Find the median of 5, 15, 25, 35, 45, 55, 65, 75, 85.
	\begin{enumerate}[label={\Alph*.}]
	\item \(45\)
	\item \(40\)
	\item \(50\)
	\item \(35\)
	\end{enumerate}

\item Calculate the mean of 7, 14, 21, 28, 35, 42.
	\begin{enumerate}[label={\Alph*.}]
	\item \(24.5\)
	\item \(21\)
	\item \(28\)
	\item \(25\)
	\end{enumerate}

\item The mode is preferred over the mean when
	\begin{enumerate}[label={\Alph*.}]
	\item Data is qualitative
	\item Data has extreme values
	\item Quick calculation is needed
	\item All of the above
	\end{enumerate}

\item Find the median of 8, 16, 24, 32, 40, 48, 56.
	\begin{enumerate}[label={\Alph*.}]
	\item \(32\)
	\item \(28\)
	\item \(36\)
	\item \(24\)
	\end{enumerate}

\item If the mean of 4, 8, 12, x, 20 is 12, find x.
	\begin{enumerate}[label={\Alph*.}]
	\item \(16\)
	\item \(14\)
	\item \(12\)
	\item \(18\)
	\end{enumerate}

\item The median of 3, 6, 9, 12, 15, 18, 21, 24, 27, 30 is
	\begin{enumerate}[label={\Alph*.}]
	\item \(16.5\)
	\item \(15\)
	\item \(18\)
	\item \(15.5\)
	\end{enumerate}

\item Find the mean of 2, 4, 8, 16, 32.
	\begin{enumerate}[label={\Alph*.}]
	\item \(12.4\)
	\item \(10\)
	\item \(14\)
	\item \(16\)
	\end{enumerate}

\item Which measure of central tendency can have more than one value?
	\begin{enumerate}[label={\Alph*.}]
	\item Mode
	\item Mean
	\item Median
	\item None
	\end{enumerate}

\item Find the median of 13, 26, 39, 52, 65, 78, 91.
	\begin{enumerate}[label={\Alph*.}]
	\item \(52\)
	\item \(39\)
	\item \(45.5\)
	\item \(65\)
	\end{enumerate}

\item The mean of 5 consecutive even numbers is 16. Find the smallest number.
	\begin{enumerate}[label={\Alph*.}]
	\item \(12\)
	\item \(10\)
	\item \(14\)
	\item \(16\)
	\end{enumerate}

\item Find the mode of 1, 3, 5, 3, 7, 5, 9, 5, 11.
	\begin{enumerate}[label={\Alph*.}]
	\item \(5\)
	\item \(3\)
	\item \(7\)
	\item \(1\)
	\end{enumerate}

\item The median is less affected by outliers than the mean because it
	\begin{enumerate}[label={\Alph*.}]
	\item Depends only on position
	\item Is always smaller
	\item Is always larger
	\item Uses all data values
	\end{enumerate}

\item Find the mean of 11, 22, 33, 44, 55, 66, 77, 88, 99.
	\begin{enumerate}[label={\Alph*.}]
	\item \(55\)
	\item \(50\)
	\item \(60\)
	\item \(44\)
	\end{enumerate}

\item The sum of deviations of observations from their mean is always
	\begin{enumerate}[label={\Alph*.}]
	\item Zero
	\item Positive
	\item Negative
	\item One
	\end{enumerate}

\item The mean of 9 numbers is 40. If 4 is added to each number, the new mean is
	\begin{enumerate}[label={\Alph*.}]
	\item \(44\)
	\item \(40\)
	\item \(36\)
	\item \(45\)
	\end{enumerate}

\item Find the median of 1.5, 2.5, 3.5, 4.5, 5.5, 6.5, 7.5, 8.5.
	\begin{enumerate}[label={\Alph*.}]
	\item \(5\)
	\item \(4.5\)
	\item \(5.5\)
	\item \(6\)
	\end{enumerate}

\item For a symmetric distribution, which relationship holds?
	\begin{enumerate}[label={\Alph*.}]
	\item Mean = Median = Mode
	\item Mean \(>\) Median
	\item Median \(>\) Mean
	\item Mode \(>\) Mean
	\end{enumerate}

\item If each value in a data set is increased by k, the mean increases by
	\begin{enumerate}[label={\Alph*.}]
	\item \(k\)
	\item \(2k\)
	\item \(k^2\)
	\item \(\dfrac{k}{n}\)
	\end{enumerate}

\item The mode of a trimodal distribution has
	\begin{enumerate}[label={\Alph*.}]
	\item Three values
	\item Two values
	\item One value
	\item No value
	\end{enumerate}

\item The median divides the data into
	\begin{enumerate}[label={\Alph*.}]
	\item Two equal parts
	\item Three equal parts
	\item Four equal parts
	\item Unequal parts
	\end{enumerate}
\end{enumerate}
\end{multicols}
\chapter{Values To Memorize}
\begin{multicols}{2}
\section{Square Roots}
\begin{itemize}
    \item \(\sqrt{1} = 1\)
    \item \(\sqrt{2} = 1.4142\)
    \item \(\sqrt{3} = 1.7321\)
    \item \(\sqrt{4} = 2\)
    \item \(\sqrt{5}= 2.2361\)
    \item \(\sqrt{6} = 2.4495\)
    \item \(\sqrt{7} = 2.6458\)
    \item \(\sqrt{8} = 2.8284\)
    \item \(\sqrt{9} = 3\)
    \item \(\sqrt{10} = 3.1623\)
\end{itemize}
\section{Squares}
\begin{itemize}
    \item \(1^{2} = 1\)
    \item \(2^{2} = 4\)
    \item \(3^{2} = 9\)
    \item \(4^{2} = 16\)
    \item \(5^{2} = 25\)
    \item \(6^{2} = 36\)
    \item \(7^{2} = 49\)
    \item \(8^{2} = 64\)
    \item \(9^{2} = 81\)
    \item \({10}^{2} = 100\)
    \item \({11}^{2} = 121\)
    \item \({12}^{2} = 144\)
    \item \({13}^{2} = 169\)
    \item \({14}^{2} = 196\)
    \item \({15}^{2} = 225\)
    \item \({16}^{2} = 256\)
    \item \({17}^{2} = 289\)
    \item \({18}^{2} = 324\)
    \item \({19}^{2} = 361\)
    \item \({20}^{2} = 400\)
    \item \({21}^{2} = 441\)
    \item \({22}^{2} = 484\)
    \item \({23}^{2} = 529\)
    \item \({24}^{2} = 576\)
    \item \({25}^{2} = 625\)
    \item \({26}^{2} = 676\)
    \item \({27}^{2} = 729\)
    \item \({28}^{2} = 784\)
    \item \({29}^{2} = 841\)
    \item \({30}^{2} = 900\)
\end{itemize}
\section{Cubes}
\begin{itemize}
    \item \(1^{3} = 1\)
    \item \(2^{3} = 8\)
    \item \(3^{3} = 27\)
    \item \(4^{3} = 64\)
    \item \(5^{3} = 125\)
    \item \(6^{3} = 216\)
    \item \(7^{3} = 343\)
    \item \(8^{3} = 512\)
    \item \(9^{3} = 729\)
    \item \({10}^{3} = 1000\)
    \item \({11}^{3} = 1331\)
    \item \({12}^{3} = 1728\)
    \item \({13}^{3} = 2197\)
    \item \({14}^{3} = 2744\)
    \item \({15}^{3} = 3375\)
    \item \({16}^{3} = 4096\)
    \item \({17}^{3} = 4913\)
    \item \({18}^{3} = 5832\)
    \item \({19}^{3} = 6859\)
    \item \({20}^{3} = 8000\)
    \item \({21}^{3} = 9261\)
    \item \({22}^{3} = 10648\)
    \item \({23}^{3} = 12167\)
    \item \({24}^{3} = 13824\)
    \item \({25}^{3} = 15625\)
    \item \({26}^{3} = 17576\)
    \item \({27}^{3} = 19683\)
    \item \({28}^{3} = 21952\)
    \item \({29}^{3} = 24389\)
    \item \({30}^{3} = 27000\)
\end{itemize}
\section{Logarithms}
\begin{itemize}
    \item \(\log_{10}{1} = 0\)
    \item \(\log_{10}{2} = 0.3010\)
    \item \(\log_{10}{3} = 0.4771\)
    \item \(\log_{10}{4} = 0.6020\)
    \item \(\log_{10}{5} = 0.699\)
    \item \(\log_{10}{6} = 0.778\)
    \item \(\log_{10}{7} = 0.845\)
    \item \(\log_{10}{8} = 0.903\)
    \item \(\log_{10}{9} = 0.954\)
    \item \(\log_{10}{10} = 1\)
\end{itemize}
\end{multicols}

% --- BACKMATTER ---
\begin{backmatter}
\pagestyle{plain}

% \printbibliography % If using biblatex
\end{backmatter}

\end{document}