\chapter{Statistics}
\section{Measures of Central Tendency}
\subsection{Questions}
\begin{multicols}{2}
\begin{enumerate}[label={\arabic*.}]
\item Find the mean of the numbers 3, 5, 7, 9, and 11.
	\begin{enumerate}[label={\Alph*.}]
	\item \(7\)
	\item \(6\)
	\item \(8\)
	\item \(5\)
	\end{enumerate}

\item The median of the data set 2, 5, 8, 11, 14 is
	\begin{enumerate}[label={\Alph*.}]
	\item \(8\)
	\item \(5\)
	\item \(11\)
	\item \(7\)
	\end{enumerate}

\item Find the mode of the following data: 3, 5, 3, 7, 3, 9, 5.
	\begin{enumerate}[label={\Alph*.}]
	\item \(3\)
	\item \(5\)
	\item \(7\)
	\item \(9\)
	\end{enumerate}

\item The mean of five numbers is 12. If four of the numbers are 10, 11, 13, and 15, find the fifth number.
	\begin{enumerate}[label={\Alph*.}]
	\item \(11\)
	\item \(12\)
	\item \(10\)
	\item \(9\)
	\end{enumerate}

\item Find the median of 15, 20, 18, 12, 25, 22, 16.
	\begin{enumerate}[label={\Alph*.}]
	\item \(18\)
	\item \(20\)
	\item \(16\)
	\item \(19\)
	\end{enumerate}

\item The data set 4, 7, 7, 10, 10, 10, 13 has a mode of
	\begin{enumerate}[label={\Alph*.}]
	\item \(10\)
	\item \(7\)
	\item \(13\)
	\item \(4\)
	\end{enumerate}

\item If the mean of 8, 10, x, 14, and 16 is 12, find the value of x.
	\begin{enumerate}[label={\Alph*.}]
	\item \(12\)
	\item \(11\)
	\item \(10\)
	\item \(13\)
	\end{enumerate}

\item Find the mean of the first five prime numbers.
	\begin{enumerate}[label={\Alph*.}]
	\item \(5.6\)
	\item \(6\)
	\item \(5\)
	\item \(6.5\)
	\end{enumerate}

\item The median of 3, 7, 5, 9, 11, 13 is
	\begin{enumerate}[label={\Alph*.}]
	\item \(8\)
	\item \(7\)
	\item \(9\)
	\item \(7.5\)
	\end{enumerate}

\item Which measure of central tendency is most affected by extreme values?
	\begin{enumerate}[label={\Alph*.}]
	\item Mean
	\item Median
	\item Mode
	\item Range
	\end{enumerate}

\item Find the mode of 12, 15, 12, 18, 20, 12, 15, 18, 12.
	\begin{enumerate}[label={\Alph*.}]
	\item \(12\)
	\item \(15\)
	\item \(18\)
	\item \(20\)
	\end{enumerate}

\item The mean of 6, 8, 10, 12, and x is 10. Find x.
	\begin{enumerate}[label={\Alph*.}]
	\item \(14\)
	\item \(12\)
	\item \(10\)
	\item \(16\)
	\end{enumerate}

\item Find the median of 2, 4, 6, 8, 10, 12, 14, 16.
	\begin{enumerate}[label={\Alph*.}]
	\item \(8\)
	\item \(9\)
	\item \(10\)
	\item \(7\)
	\end{enumerate}

\item The ages of five students are 15, 16, 14, 17, and 18 years. Find their mean age.
	\begin{enumerate}[label={\Alph*.}]
	\item \(16\)
	\item \(15\)
	\item \(17\)
	\item \(14\)
	\end{enumerate}

\item If the mean of 7 numbers is 15 and the mean of 3 of them is 12, find the mean of the remaining 4 numbers.
	\begin{enumerate}[label={\Alph*.}]
	\item \(17.25\)
	\item \(16.5\)
	\item \(18\)
	\item \(15.75\)
	\end{enumerate}

\item Find the mode of the data: 5, 7, 5, 8, 5, 9, 7, 7, 5.
	\begin{enumerate}[label={\Alph*.}]
	\item \(5\)
	\item \(7\)
	\item \(8\)
	\item \(9\)
	\end{enumerate}

\item The median of 21, 15, 18, 12, 24, 27 is
	\begin{enumerate}[label={\Alph*.}]
	\item \(19.5\)
	\item \(18\)
	\item \(21\)
	\item \(20\)
	\end{enumerate}

\item Find the mean of 20, 25, 30, 35, 40.
	\begin{enumerate}[label={\Alph*.}]
	\item \(30\)
	\item \(25\)
	\item \(35\)
	\item \(28\)
	\end{enumerate}

\item The data set 2, 4, 6, 8 has no mode. What can be said about this data?
	\begin{enumerate}[label={\Alph*.}]
	\item All values occur with equal frequency
	\item The data is bimodal
	\item The median is 5
	\item Both A and C
	\end{enumerate}

\item If the mean of 5, 7, 9, x, and 11 is 8, find x.
	\begin{enumerate}[label={\Alph*.}]
	\item \(8\)
	\item \(7\)
	\item \(9\)
	\item \(6\)
	\end{enumerate}

\item Find the median of 40, 35, 50, 45, 30, 55, 60.
	\begin{enumerate}[label={\Alph*.}]
	\item \(45\)
	\item \(50\)
	\item \(40\)
	\item \(47.5\)
	\end{enumerate}

\item The mode of 3, 5, 7, 5, 9, 5, 11, 7 is
	\begin{enumerate}[label={\Alph*.}]
	\item \(5\)
	\item \(7\)
	\item \(9\)
	\item \(3\)
	\end{enumerate}

\item Calculate the mean of 12, 18, 24, 30, 36.
	\begin{enumerate}[label={\Alph*.}]
	\item \(24\)
	\item \(22\)
	\item \(26\)
	\item \(20\)
	\end{enumerate}

\item The median of an even number of observations is
	\begin{enumerate}[label={\Alph*.}]
	\item The middle value
	\item The average of the two middle values
	\item The most frequent value
	\item The sum of all values
	\end{enumerate}

\item Find the mean of the squares of the first four natural numbers.
	\begin{enumerate}[label={\Alph*.}]
	\item \(7.5\)
	\item \(10\)
	\item \(5\)
	\item \(8\)
	\end{enumerate}

\item If the mean of 10 numbers is 20 and one number 25 is removed, what is the new mean?
	\begin{enumerate}[label={\Alph*.}]
	\item \(19.44\)
	\item \(20\)
	\item \(18.5\)
	\item \(21\)
	\end{enumerate}

\item Find the mode of 8, 10, 12, 10, 8, 14, 10, 8, 10.
	\begin{enumerate}[label={\Alph*.}]
	\item \(10\)
	\item \(8\)
	\item Both 8 and 10
	\item \(12\)
	\end{enumerate}

\item The median of 5, 10, 15, 20, 25, 30, 35, 40, 45 is
	\begin{enumerate}[label={\Alph*.}]
	\item \(25\)
	\item \(20\)
	\item \(30\)
	\item \(22.5\)
	\end{enumerate}

\item Find the mean of all multiples of 5 between 1 and 30.
	\begin{enumerate}[label={\Alph*.}]
	\item \(17.5\)
	\item \(15\)
	\item \(20\)
	\item \(12.5\)
	\end{enumerate}

\item The mean of 4, 6, 8, 10, 12 is increased by 3 when each number is increased by
	\begin{enumerate}[label={\Alph*.}]
	\item \(3\)
	\item \(15\)
	\item \(6\)
	\item \(9\)
	\end{enumerate}

\item Find the median of 100, 90, 80, 110, 120, 95, 105.
	\begin{enumerate}[label={\Alph*.}]
	\item \(100\)
	\item \(105\)
	\item \(95\)
	\item \(110\)
	\end{enumerate}

\item The mode of a data set with all different values is
	\begin{enumerate}[label={\Alph*.}]
	\item Zero
	\item Does not exist
	\item The mean
	\item The median
	\end{enumerate}

\item If the mean of x, x+2, x+4, x+6, and x+8 is 10, find x.
	\begin{enumerate}[label={\Alph*.}]
	\item \(6\)
	\item \(7\)
	\item \(8\)
	\item \(5\)
	\end{enumerate}

\item Find the median of 2.5, 3.5, 4.5, 5.5, 6.5, 7.5.
	\begin{enumerate}[label={\Alph*.}]
	\item \(5\)
	\item \(5.5\)
	\item \(4.5\)
	\item \(6\)
	\end{enumerate}

\item The mean of 15 observations is 32. If two observations 40 and 50 are added, what is the new mean?
	\begin{enumerate}[label={\Alph*.}]
	\item \(33.29\)
	\item \(34\)
	\item \(35\)
	\item \(32.5\)
	\end{enumerate}

\item Find the mode of 1, 2, 2, 3, 3, 3, 4, 4, 4, 4.
	\begin{enumerate}[label={\Alph*.}]
	\item \(4\)
	\item \(3\)
	\item \(2\)
	\item \(1\)
	\end{enumerate}

\item The median of 7, 14, 21, 28, 35 is
	\begin{enumerate}[label={\Alph*.}]
	\item \(21\)
	\item \(14\)
	\item \(28\)
	\item \(20\)
	\end{enumerate}

\item Calculate the mean of the first 10 even natural numbers.
	\begin{enumerate}[label={\Alph*.}]
	\item \(11\)
	\item \(10\)
	\item \(12\)
	\item \(9\)
	\end{enumerate}

\item If the mean of 20 observations is 15 and that of another 30 observations is 20, find the mean of all 50 observations.
	\begin{enumerate}[label={\Alph*.}]
	\item \(18\)
	\item \(17.5\)
	\item \(19\)
	\item \(16.5\)
	\end{enumerate}

\item Find the mode of the data where all frequencies are equal.
	\begin{enumerate}[label={\Alph*.}]
	\item Does not exist
	\item Zero
	\item The mean
	\item All values
	\end{enumerate}

\item The median of 1, 3, 5, 7, 9, 11, 13, 15, 17 is
	\begin{enumerate}[label={\Alph*.}]
	\item \(9\)
	\item \(8\)
	\item \(10\)
	\item \(7\)
	\end{enumerate}

\item Find the mean of 2.5, 3.0, 3.5, 4.0, 4.5, 5.0.
	\begin{enumerate}[label={\Alph*.}]
	\item \(3.75\)
	\item \(4\)
	\item \(3.5\)
	\item \(4.25\)
	\end{enumerate}

\item If the median of a data set is 25 and the mean is 30, the data is
	\begin{enumerate}[label={\Alph*.}]
	\item Positively skewed
	\item Negatively skewed
	\item Symmetric
	\item Normal
	\end{enumerate}

\item Find the mode of 50, 60, 70, 60, 80, 60, 90, 70.
	\begin{enumerate}[label={\Alph*.}]
	\item \(60\)
	\item \(70\)
	\item \(50\)
	\item \(80\)
	\end{enumerate}

\item The mean of three numbers is 20. If two of them are 15 and 18, find the third number.
	\begin{enumerate}[label={\Alph*.}]
	\item \(27\)
	\item \(25\)
	\item \(22\)
	\item \(20\)
	\end{enumerate}

\item Find the median of 0.5, 1.5, 2.5, 3.5, 4.5, 5.5, 6.5, 7.5.
	\begin{enumerate}[label={\Alph*.}]
	\item \(4\)
	\item \(3.5\)
	\item \(4.5\)
	\item \(5\)
	\end{enumerate}

\item The mode is most useful when data is
	\begin{enumerate}[label={\Alph*.}]
	\item Categorical
	\item Continuous
	\item Symmetric
	\item Normally distributed
	\end{enumerate}

\item Calculate the mean of 1, 4, 9, 16, 25.
	\begin{enumerate}[label={\Alph*.}]
	\item \(11\)
	\item \(10\)
	\item \(12\)
	\item \(13\)
	\end{enumerate}

\item If each observation in a data set is multiplied by 5, the mean is
	\begin{enumerate}[label={\Alph*.}]
	\item Multiplied by 5
	\item Divided by 5
	\item Increased by 5
	\item Remains the same
	\end{enumerate}

\item Find the median of 11, 22, 33, 44, 55, 66, 77, 88, 99.
	\begin{enumerate}[label={\Alph*.}]
	\item \(55\)
	\item \(44\)
	\item \(50\)
	\item \(60\)
	\end{enumerate}

\item The mean of 8 numbers is 25. If 5 is subtracted from each number, the new mean is
	\begin{enumerate}[label={\Alph*.}]
	\item \(20\)
	\item \(25\)
	\item \(30\)
	\item \(15\)
	\end{enumerate}

\item Find the mode of 15, 20, 25, 20, 30, 25, 20, 35.
	\begin{enumerate}[label={\Alph*.}]
	\item \(20\)
	\item \(25\)
	\item \(15\)
	\item \(30\)
	\end{enumerate}

\item The median of 6, 12, 18, 24, 30, 36 is
	\begin{enumerate}[label={\Alph*.}]
	\item \(21\)
	\item \(18\)
	\item \(24\)
	\item \(20\)
	\end{enumerate}

\item Find the mean of the first 7 odd natural numbers.
	\begin{enumerate}[label={\Alph*.}]
	\item \(7\)
	\item \(8\)
	\item \(6\)
	\item \(9\)
	\end{enumerate}

\item If the mean of 10, 15, 20, x, 30 is 20, find x.
	\begin{enumerate}[label={\Alph*.}]
	\item \(25\)
	\item \(20\)
	\item \(22\)
	\item \(18\)
	\end{enumerate}

\item The median is always
	\begin{enumerate}[label={\Alph*.}]
	\item A value in the data set
	\item The middle value when data is ordered
	\item Equal to the mean
	\item The most frequent value
	\end{enumerate}

\item Find the mode of 100, 200, 200, 300, 300, 300, 400.
	\begin{enumerate}[label={\Alph*.}]
	\item \(300\)
	\item \(200\)
	\item \(100\)
	\item \(400\)
	\end{enumerate}

\item The mean of 5 consecutive odd numbers is 21. Find the largest number.
	\begin{enumerate}[label={\Alph*.}]
	\item \(25\)
	\item \(23\)
	\item \(27\)
	\item \(21\)
	\end{enumerate}

\item Find the median of 50, 55, 60, 65, 70, 75, 80, 85, 90, 95.
	\begin{enumerate}[label={\Alph*.}]
	\item \(72.5\)
	\item \(70\)
	\item \(75\)
	\item \(65\)
	\end{enumerate}

\item Calculate the mean of 3, 6, 9, 12, 15, 18.
	\begin{enumerate}[label={\Alph*.}]
	\item \(10.5\)
	\item \(12\)
	\item \(9\)
	\item \(11\)
	\end{enumerate}

\item The mode of a bimodal distribution has
	\begin{enumerate}[label={\Alph*.}]
	\item Two values
	\item One value
	\item Three values
	\item No value
	\end{enumerate}

\item If the mean of x, 2x, 3x, 4x, and 5x is 30, find x.
	\begin{enumerate}[label={\Alph*.}]
	\item \(10\)
	\item \(5\)
	\item \(15\)
	\item \(20\)
	\end{enumerate}

\item Find the median of 17, 23, 19, 21, 25, 20, 22.
	\begin{enumerate}[label={\Alph*.}]
	\item \(21\)
	\item \(20\)
	\item \(22\)
	\item \(19\)
	\end{enumerate}

\item The mean of 12 observations is 15. If each observation is doubled, the new mean is
	\begin{enumerate}[label={\Alph*.}]
	\item \(30\)
	\item \(15\)
	\item \(27\)
	\item \(18\)
	\end{enumerate}

\item Find the mode of 2, 4, 6, 2, 8, 2, 10, 4, 6, 2.
	\begin{enumerate}[label={\Alph*.}]
	\item \(2\)
	\item \(4\)
	\item \(6\)
	\item \(8\)
	\end{enumerate}

\item The median of 9, 18, 27, 36, 45, 54, 63, 72 is
	\begin{enumerate}[label={\Alph*.}]
	\item \(40.5\)
	\item \(36\)
	\item \(45\)
	\item \(42\)
	\end{enumerate}

\item Find the mean of all two-digit multiples of 10.
	\begin{enumerate}[label={\Alph*.}]
	\item \(55\)
	\item \(50\)
	\item \(60\)
	\item \(45\)
	\end{enumerate}

\item If the mean of a, b, c is 15 and the mean of a, b, c, d is 20, find d.
	\begin{enumerate}[label={\Alph*.}]
	\item \(35\)
	\item \(30\)
	\item \(25\)
	\item \(40\)
	\end{enumerate}

\item Find the median of integers from 1 to 9.
	\begin{enumerate}[label={\Alph*.}]
	\item \(5\)
	\item \(4\)
	\item \(6\)
	\item \(4.5\)
	\end{enumerate}

\item The mode of the data 5, 10, 15, 10, 20, 15, 10, 25 is
	\begin{enumerate}[label={\Alph*.}]
	\item \(10\)
	\item \(15\)
	\item \(20\)
	\item \(5\)
	\end{enumerate}

\item Calculate the mean of 0, 5, 10, 15, 20, 25, 30.
	\begin{enumerate}[label={\Alph*.}]
	\item \(15\)
	\item \(12.5\)
	\item \(17.5\)
	\item \(20\)
	\end{enumerate}

\item If every value in a data set is the same, then
	\begin{enumerate}[label={\Alph*.}]
	\item Mean = Median = Mode
	\item Mean \(\neq\) Median
	\item Mode does not exist
	\item Median \(\neq\) Mode
	\end{enumerate}

\item Find the median of 1.2, 3.4, 2.3, 4.5, 3.2, 5.1.
	\begin{enumerate}[label={\Alph*.}]
	\item \(3.3\)
	\item \(3.4\)
	\item \(3.2\)
	\item \(2.85\)
	\end{enumerate}

\item The mean of 6 numbers is 18. If one number 24 is replaced by 18, the new mean is
	\begin{enumerate}[label={\Alph*.}]
	\item \(17\)
	\item \(18\)
	\item \(19\)
	\item \(16\)
	\end{enumerate}

\item Find the mode of 7, 14, 21, 14, 28, 21, 14, 35.
	\begin{enumerate}[label={\Alph*.}]
	\item \(14\)
	\item \(21\)
	\item \(7\)
	\item \(28\)
	\end{enumerate}

\item The median of 1, 2, 3, 4, 5, 6, 7, 8, 9, 10, 11 is
	\begin{enumerate}[label={\Alph*.}]
	\item \(6\)
	\item \(5\)
	\item \(7\)
	\item \(5.5\)
	\end{enumerate}

\item Find the mean of 16, 18, 20, 22, 24, 26, 28.
	\begin{enumerate}[label={\Alph*.}]
	\item \(22\)
	\item \(20\)
	\item \(24\)
	\item \(21\)
	\end{enumerate}

\item If the median is greater than the mean, the data is likely
	\begin{enumerate}[label={\Alph*.}]
	\item Negatively skewed
	\item Positively skewed
	\item Symmetric
	\item Uniform
	\end{enumerate}

\item Find the mode of 33, 44, 55, 44, 66, 55, 44, 77.
	\begin{enumerate}[label={\Alph*.}]
	\item \(44\)
	\item \(55\)
	\item \(33\)
	\item \(66\)
	\end{enumerate}

\item The mean of the first n natural numbers is
	\begin{enumerate}[label={\Alph*.}]
	\item \(\dfrac{n+1}{2}\)
	\item \(\dfrac{n}{2}\)
	\item \(n+1\)
	\item \(\dfrac{n-1}{2}\)
	\end{enumerate}

\item Find the median of 5, 15, 25, 35, 45, 55, 65, 75, 85.
	\begin{enumerate}[label={\Alph*.}]
	\item \(45\)
	\item \(40\)
	\item \(50\)
	\item \(35\)
	\end{enumerate}

\item Calculate the mean of 7, 14, 21, 28, 35, 42.
	\begin{enumerate}[label={\Alph*.}]
	\item \(24.5\)
	\item \(21\)
	\item \(28\)
	\item \(25\)
	\end{enumerate}

\item The mode is preferred over the mean when
	\begin{enumerate}[label={\Alph*.}]
	\item Data is qualitative
	\item Data has extreme values
	\item Quick calculation is needed
	\item All of the above
	\end{enumerate}

\item Find the median of 8, 16, 24, 32, 40, 48, 56.
	\begin{enumerate}[label={\Alph*.}]
	\item \(32\)
	\item \(28\)
	\item \(36\)
	\item \(24\)
	\end{enumerate}

\item If the mean of 4, 8, 12, x, 20 is 12, find x.
	\begin{enumerate}[label={\Alph*.}]
	\item \(16\)
	\item \(14\)
	\item \(12\)
	\item \(18\)
	\end{enumerate}

\item The median of 3, 6, 9, 12, 15, 18, 21, 24, 27, 30 is
	\begin{enumerate}[label={\Alph*.}]
	\item \(16.5\)
	\item \(15\)
	\item \(18\)
	\item \(15.5\)
	\end{enumerate}

\item Find the mean of 2, 4, 8, 16, 32.
	\begin{enumerate}[label={\Alph*.}]
	\item \(12.4\)
	\item \(10\)
	\item \(14\)
	\item \(16\)
	\end{enumerate}

\item Which measure of central tendency can have more than one value?
	\begin{enumerate}[label={\Alph*.}]
	\item Mode
	\item Mean
	\item Median
	\item None
	\end{enumerate}

\item Find the median of 13, 26, 39, 52, 65, 78, 91.
	\begin{enumerate}[label={\Alph*.}]
	\item \(52\)
	\item \(39\)
	\item \(45.5\)
	\item \(65\)
	\end{enumerate}

\item The mean of 5 consecutive even numbers is 16. Find the smallest number.
	\begin{enumerate}[label={\Alph*.}]
	\item \(12\)
	\item \(10\)
	\item \(14\)
	\item \(16\)
	\end{enumerate}

\item Find the mode of 1, 3, 5, 3, 7, 5, 9, 5, 11.
	\begin{enumerate}[label={\Alph*.}]
	\item \(5\)
	\item \(3\)
	\item \(7\)
	\item \(1\)
	\end{enumerate}

\item The median is less affected by outliers than the mean because it
	\begin{enumerate}[label={\Alph*.}]
	\item Depends only on position
	\item Is always smaller
	\item Is always larger
	\item Uses all data values
	\end{enumerate}

\item Find the mean of 11, 22, 33, 44, 55, 66, 77, 88, 99.
	\begin{enumerate}[label={\Alph*.}]
	\item \(55\)
	\item \(50\)
	\item \(60\)
	\item \(44\)
	\end{enumerate}

\item The sum of deviations of observations from their mean is always
	\begin{enumerate}[label={\Alph*.}]
	\item Zero
	\item Positive
	\item Negative
	\item One
	\end{enumerate}

\item The mean of 9 numbers is 40. If 4 is added to each number, the new mean is
	\begin{enumerate}[label={\Alph*.}]
	\item \(44\)
	\item \(40\)
	\item \(36\)
	\item \(45\)
	\end{enumerate}

\item Find the median of 1.5, 2.5, 3.5, 4.5, 5.5, 6.5, 7.5, 8.5.
	\begin{enumerate}[label={\Alph*.}]
	\item \(5\)
	\item \(4.5\)
	\item \(5.5\)
	\item \(6\)
	\end{enumerate}

\item For a symmetric distribution, which relationship holds?
	\begin{enumerate}[label={\Alph*.}]
	\item Mean = Median = Mode
	\item Mean \(>\) Median
	\item Median \(>\) Mean
	\item Mode \(>\) Mean
	\end{enumerate}

\item If each value in a data set is increased by k, the mean increases by
	\begin{enumerate}[label={\Alph*.}]
	\item \(k\)
	\item \(2k\)
	\item \(k^2\)
	\item \(\dfrac{k}{n}\)
	\end{enumerate}

\item The mode of a trimodal distribution has
	\begin{enumerate}[label={\Alph*.}]
	\item Three values
	\item Two values
	\item One value
	\item No value
	\end{enumerate}

\item The median divides the data into
	\begin{enumerate}[label={\Alph*.}]
	\item Two equal parts
	\item Three equal parts
	\item Four equal parts
	\item Unequal parts
	\end{enumerate}
\end{enumerate}
\end{multicols}