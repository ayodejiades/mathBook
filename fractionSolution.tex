\subsection{Solution}
\begin{multicols}{2}
\begin{enumerate}[label={\arabic*.}]
    \item First change all to improper fraction \\
    \(3\dfrac{7}{8} = \dfrac{31}{8} \text { and } 1\dfrac{1}{3} = \dfrac{4}{3}\) and then add them up 
    Taking LCM of 8 and 3 we get 24
    \[\frac{31}{8} + \frac{4}{3} = \frac{(31 \times 3) + (8 \times 4) }{24} = \frac{93 + 32}{24} = \dfrac{125}{24} \]
    We do same for, where the LCM is still 24: 
    \[1\frac{2}{3} - \frac{3}{8} \Rightarrow \frac{5}{3} - \frac{3}{8}  \frac{(5 \times 8) - (3 \times 3)}{24} = \frac{31}{24}\]
    Now substract \(\dfrac{31}{24} \text { from } \dfrac{125}{24}\)
    \[\frac{125}{24} - \frac{31}{24} = \frac{125 - 31}{24} = \frac{94}{24} = 3\frac{11}{12} \text{(D)} \]
    \item \textbf{Amount he earned before the increase} i.e his Initial Salary \\
    Let initial Salary = \(x\) \hspace {10px}  new salary = N345\\
    So, listen He had 15\% of his initial salary (x) added to his initial earn (x)
    \[\text{initial salary + 15\% of initial = new salary}\]
    \[x + \dfrac{15}{100}x = 345 \Rightarrow \dfrac{100x + 15x}{100} = 345\]
    \[115x = 345 \times 100 \Rightarrow x = \left(\frac{345}{115}\right) \times 100\]
    \(\therefore 3 \times 100 = 300 \text{ (B)}\)\\
    \textbf{Note:} when give this: \( \dfrac{x \times 100}{y}\)its advisable to group in the form
    \(\left(\frac{x}{y}\right) \times 100\) , incase the fraction part becomes decimal the 100 only shift the
    decimal point. 

    \item
    \item The \textbf{time taken (t)} to complete a job is inversely proportional to \textbf{numbers ot men (n)} 
    i.e as the number of men increase the job is done faster. 
    \begin{equation}
        \text{time taken} \propto \dfrac{1}{\text{number of men}} \Rightarrow t \propto \dfrac{1}{n}
    \end {equation}
    Removing proportionality sign
    \[t = \dfrac{k}{n} \Rightarrow tn = k\]
    Generally: \(t_{1}n_{1} = t_{2}n_2 = \cdots = t_{k}n_{k}\)\\
    given \(t_{1} = 9\ , n_{1} = 12 , t_{2} = 6 \text{ and } n_{2} = ?\) \\
    Using: \(t_{1}n_{1} = t_{2}n_{2} \Rightarrow 9 \times 12 = 6 \times n_{2}\) \\
    taking the advantage that 6 $\mid$ 12 
    \[n_{2} = \left(\frac{12}{6}\right) \times 9 = 18 \text { men (E)}\]

    \item Remember convert to improper fraction first
    \[2\frac{5}{12} - 1\frac{7}{8} \times \frac{6}{5} \Rightarrow \frac{29}{12} - \frac{15}{8} \times \frac{6}{5}\]
    Applying BODMAS multilplication comes before substraction\\
    \[\therefore \hspace{10px} \dfrac{29}{12} - \left(\dfrac{\cancelto{3}{15} \times \cancelto{3}{6}}{\cancelto{4}{8} \times \cancelto{1}{5}}\right) \Rightarrow \dfrac{29}{12} - \dfrac{9}{4} = \dfrac{29 - (9 \times 3)}{12} = \frac{2}{12} = \frac{1}{6} \text{(C)}\]
    \item Selling Price = N45.00, profit\% = 8\%, let cost price = \(x\)
       \[ \%profit = \frac{Selling Price - Cost Price }{Cost Price} \times 100 \]
       \[8 = \frac{45 - x }{x} \times 100 \Rightarrow 8x = 4500 -100x\]
       \[108x = 4500 \Rightarrow x = \dfrac{4500}{108}\]
    \item
    \item
    \item 
    \item 
    \item 
    \item 
    \item
    \item
    \item
    \item 
    \item
    \item
    \item 
    \item 
    \item 
    \item 
    \item
    \item
    \item
    \item 
    \item
    \item
    \item 
    \item 
    \item 
    \item 
    \item
    \item
    \item
    \item 
    \item
    \item
    \item 
    \item 
    \item 
    \item 
    \item
    \item
    \item
    \item 
    \item
    \item
    \item 
    \item 
    \item 
    \item 
    \item
    \item
    \item
    \item 
    \item
    \item
    \item 
    \item 
    \item 
    \item 
    \item
    \item
    \item
    \item 
    \item
    \item
    \item 
    \item 
    \item 
    \item 
    \item
    \item
    \item
    \item 
    \item
    \item
    \item 
    \item 
    \item 
    \item 
    \item
    \item
    \item
    \item 
    \item
    \item
    \item 
    \item 
    \item 
    \item 
    \item
    \item
    \item
    \item 
    \item
    \item
    \item 
    \item 
\end{enumerate}
\end {multicols}