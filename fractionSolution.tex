\rhead{\textit{Fraction Solution}}
\chead{\textbf{\textit{Mathematics for UTME}}}
\subsection{Solution}
\begin{multicols}{2}
\begin{enumerate}[label={\textbf{\arabic*.}}]
    \item \textbf{(D)} We begin by converting all the mixed fractions to improper fractions and add them up.
    \[3\dfrac{7}{8} = \dfrac{31}{8} \text { and } 1\dfrac{1}{3} = \dfrac{4}{3}\] 
    Taking LCM of \(8\) and \(3\) we get \(24\),
    \begin{align*}
    \frac{31}{8} + \frac{4}{3} &= \frac{(31 \times 3) + (8 \times 4) }{24} \\
    &= \frac{93 + 32}{24} = \dfrac{125}{24}
    \end{align*}
    Similarly, 
    \begin{align*}
    1\dfrac{2}{3} - \dfrac{3}{8} &\Rightarrow \dfrac{5}{3} - \dfrac{3}{8} \\ 
    &= \dfrac{(5 \times 8) - (3 \times 3)}{24} \\ &= \dfrac{31}{24}
    \end{align*}
    Now subtract  \(\dfrac{31}{24} \text { from } \dfrac{125}{24}\) \\\\
    \[\dfrac{125}{24} - \dfrac{31}{24} = \dfrac{125 - 31}{24} = \dfrac{94}{24} = 3\dfrac{11}{12} \]
    
    \item \textbf{(B)} \textbf{Amount he earned before the increase} i.e his initial salary
    Let initial salary = \(x\) and his new salary = N\(345\)\\
    From the question, he had 15\% of his initial salary $(x)$ added to his initial salary $(x)$ \\
    initial salary + $15\%$ of initial = new salary
    \begin{align*}
    x + \dfrac{15}{100}x = 345 &\Rightarrow \dfrac{100x + 15x}{100} = 345 \\
    115x = 345 \times 100 &\Rightarrow x = \left(\frac{345}{115}\right) \times 100 \\
    \therefore \hspace{70px} 3 \times 100 &= 300 \\
    \end{align*}
    \textbf{Note:} when give this: \( \dfrac{x \times 100}{y}\)its advisable to group in the form
    \(\left(\dfrac{x}{y}\right) \times 100\) , incase the fraction part becomes decimal the 100 only shift the
    decimal point. 

    \item \textbf{(C) } $ \sqrt{41830}$ is not a perfect square, so we can't evaulate using tables
    \[ \sqrt{41830} \approx 204.5 = 205 \hspace{30px} \text{\{ to 3d.p\}}\]
    \item \textbf{(E)} The \textbf{time taken (t)} to complete a job is inversely proportional to \textbf{numbers ot men (n)} 
    i.e as the number of men increase the job is done faster. 
        \begin{equation}
            \text{time taken} \propto \dfrac{1}{\text{number of men}} \Rightarrow t \propto \dfrac{1}{n} 
        \end{equation}
    Removing proportionality sign
    \(t = \dfrac{k}{n} \Rightarrow tn = k\) \\
    Generally:  \hspace{10px}\(t_{1}n_{1} = t_{2}n_2 = \cdots = t_{k}n_{k}\)\\
    given \(t_{1} = 9\ , n_{1} = 12 , t_{2} = 6 \text{ and } n_{2} = ?\) \\
    Using: \(t_{1}n_{1} = t_{2}n_{2} \Rightarrow 9 \times 12 = 6 \times n_{2}\) \\
    taking the advantage that 6 $\mid$ 12 
    \[n_{2} = \left(\frac{12}{6}\right) \times 9 = 18\]

    \item \textbf{(C)} Remember convert to improper fraction first
    \[2\frac{5}{12} - 1\frac{7}{8} \times \frac{6}{5} \Rightarrow \frac{29}{12} - \frac{15}{8} \times \frac{6}{5}\]
    Applying BODMAS multilplication comes before substraction\\
    \[\dfrac{29}{12} - \left(\dfrac{\cancelto{3}{15} \times \cancelto{3}{6}}{\cancelto{4}{8} \times \cancelto{1}{5}}\right) \Rightarrow \dfrac{29}{12} - \dfrac{9}{4} = \dfrac{29 - (9 \times 3)}{12} = \frac{2}{12} = \frac{1}{6}\]

    \item \textbf{(B)} Selling Price = N45.00, profit\% = 8\%, \\let cost price = \(x\)
    \begin{align*}
    \%\text{profit} &= \frac{\text{Selling Price - Cost Price} }{\text{Cost Price}} \times 100 \\
         8 &= \frac{45 - x }{x} \times 100 \Rightarrow 8x = 4500 -100x\\
       108x &= 4500 \Rightarrow x = \dfrac{4500}{108}  \\
       \therefore \hspace{10px}  \text{ cost price }  &= \dfrac{4500}{108} 
    \end{align*}
       Now the question says He had made a \%profit of 32\%, how much did he sell it? \\
       Using Formular, We have: \\
       \begin{align*}
        32 &= \dfrac{Sp - (4500/108)}{(4500/108)} \times 100 \\
        &= \dfrac{108\left[Sp - (4500/108)\right]}{4500} \times 100 \\
        &=  \dfrac{108Sp - 4500}{4500} \times 100 \\
        \dfrac{32 \times 4500}{100} &= 108Sp - 4500 \\
        Sp &= \dfrac{\left(32 \times 45\right) + 4500}{108} = \dfrac{5940}{108} = 55 
    \end{align*}

    \item \textbf{(B) }The easiest ways is to compare $\dfrac{1}{3}$ with every other item \\
        \begin{align*}
            \dfrac{4}{11} \text{ vs } \dfrac{1}{3} &\Rightarrow \dfrac{4}{11} \text{ vs } \dfrac{4}{12} \\
            \dfrac{122}{383} \text{ vs } \dfrac{1}{3} &\Rightarrow \dfrac{122}{383} \text{ vs } \dfrac{122}{366} \\
            \dfrac{15}{44} \text{ vs } \dfrac{1}{3} &\Rightarrow \dfrac{15}{44} \text{ vs } \dfrac{15}{45} \\
            \dfrac{22}{63} \text{ vs } \dfrac{1}{3} &\Rightarrow \dfrac{22}{63} \text{ vs } \dfrac{22}{66} \\
            \dfrac{6}{14} \text{ vs } \dfrac{1}{3} &\Rightarrow \dfrac{6}{14} \text{ vs } \dfrac{6}{18} \\
        \end{align*}
        Since the denominator in $\dfrac{122}{383}$ is greater than $\dfrac{122}{366} = \dfrac{1}{3}$ therefore $\dfrac{122}{383}$ is lessthan one-third

    \item \textbf{(A)} In 1980: The intial price \(a : x\) rose by 25\% and 10\% respectively, hence: 
        \begin{align*}
        a:x &\Rightarrow  a + \dfrac{25}{100}a : x + \dfrac{10}{100}x  \\
        & = \dfrac{125}{\cancel{100}}a : \dfrac{110}{\cancel{100}}x \\ 
        &= {125}a : 110x \\
    \end{align*}
    \(\text{dividing by 5 } \hspace{10px} \dfrac{125}{5} a : \dfrac{110}{5} x  \Rightarrow 25a : 22x\) \\\\
    \(\therefore \text{ Multiplying by 2 } \hspace{20px}  50a : 44x \)

    \item \textbf{(B)} For Questions like this its best to start from the bottom \\
    \[5 + \dfrac{6}{7} = \dfrac{35 + 6}{7} = \dfrac{41}{7} \] 
     \[\Rightarrow \cfrac{4}{5 + \cfrac{6}{7}} = \cfrac{4}{41/7} = \dfrac{4 \cdot 7}{41} = \dfrac{28}{41}\]
    now, \(3 + \cfrac{4}{5 + \cfrac{6}{7}} = 3 + \dfrac{28}{41} = \dfrac{123 + 28}{41} = \dfrac{151}{41} \)  \vspace{10pt} \\
    \(\Rightarrow \hspace{20px} \dfrac{2}{3 + \cfrac{4}{5 + \cfrac{6}{7}}} = \dfrac{2}{151/41} = \dfrac{2 \cdot 41}{151} = \dfrac{82}{151} \) \vspace{10pt} \\
    \(\Rightarrow \hspace{20px} 1 + \dfrac{2}{3 + \cfrac{4}{5 + \cfrac{6}{7}}} = 1 + \dfrac{82}{151} = \dfrac{151 + 82}{151} = \dfrac{233}{151} \)
    
    \item \textbf{(C)}\begin{align*}
        827.51 \times 15 \times 10^{-3} &= 12.41265 \\
        &= 12.4127  &\text{\{ to 4d.p\}}
        \end{align*}
    \item Notice the difference between million and million\textbf{th} 
    \begin{align*}
     1 \text{micrometer} &= 10^{-6}\text{millimeter} \\
    12,000 \text{ micrometer} &= x \text{ millimeter} 
    \end{align*}
    \begin{align*}
    x = 12000 \times 10^{-6}  = 1.2 \times 10^{-4 -6 } &= 1.2 \times 10^{-10} \\ &= 0.00000000012 \text{m}
    \end{align*}

    \item \textbf{(C)} Converting to improper fraction 
        \begin{align*}
        4\,\dfrac{5}{7} - 2\dfrac{1}{4} \Rightarrow \dfrac{33}{7} - \dfrac{9}{4} &= \dfrac{(33\times 4) - (9 \times 7) }{28}  \\
        &= \dfrac{132 - 63}{28} = \dfrac{69}{28} 
        \end{align*}
        Solving for the second:
        \[\dfrac{1}{14} + 1\dfrac{1}{2} = \dfrac{1}{14} + \dfrac{3}{2} = \dfrac{1 + (7 \times 3)}{14} = \dfrac{22}{14}\] 
        Now substracting $\dfrac{22}{14}$ from $\dfrac{69}{28}$: 
        \[\dfrac{69}{28} - \dfrac{22}{14} = \dfrac{69 - (22 \times 2)}{28} =  \dfrac{25}{28} = \dfrac{50}{56}\]

    \item \textbf{(A)} SP = N81, \%profit = 8\% \\
    What did he pay for the bicycle = \textbf{CP} = x \\
    \begin{align*}
    \%\text{profit} &= \dfrac{ \text{SP - CP}}{\text{CP}} \times 100  \\ 
    \Rightarrow \hspace{55px} 8 &= \dfrac{81 - x }{x} \times 100 \\
     \hspace{30px} 8x &= 100(81 - x)  \\
     \hspace{30px} 2x &= 25(81 - x ) = (81\times25) -25x \\
     \hspace{30px} 27x &= 81\times 25 \\
    \therefore \hspace{60px}  x &= \dfrac{81 \times 25}{27} = \text{N}75
    \end{align*}

    \item \textbf{(B)}  Let the cost price be CP, hence CP = 400, woman's money = 10\% of 400 = N40 \\
    the remainder is supposed to be N360, but the man has 40\% of 360 = \(\dfrac{4\cancel{0}}{1\cancel{0}\cancel{0}} \times 36\cancel{0} = 144 \) \vspace {5px}\\
    So, answer is 40 + 144 = 184

    \item \textbf{(A)} Note that the same amount was invested in both cases which is also the principal (P) 
    \( \therefore \text{P}_1 = \text{P}_2\)
    \begin{align*} 
        \Rightarrow \hspace{40px} A_1 = P_1 + I_1 &= P_1 + \dfrac{P_1 \times R_1 \times T_1}{100} \\
        285.20&= P_1 + \dfrac{P_1 \times 5 \times 3}{100} \\
         28520 &= 100P_1 + 15P_1 = 115P_1 \\
         \dfrac{28520}{115}&= P_1 = 248 \\
        \Rightarrow \hspace{40px}  A_2 = P_2 + I_2 &= P_2 + \dfrac{P_2 \times R_2 \times T_2}{100} \\
        434&= P_2 + \dfrac{P_2 \times 7\dfrac{1}{2} \times T_2}{100} \\
        \{ \because P_1 = P_2\} \hspace{40px} 434&= 248 + \dfrac{248 \times 7\dfrac{1}{2} \times T_2}{100} \\
        186 &= \dfrac{248 \times 15 \times T_2}{100 \times 2} \\
        \therefore \hspace{70px} T_2 & = 10 \text{ years}
    \end{align*}
    
    \item \textbf{(A) } Given that, \(B = A \text{ + 5000 and } \dfrac{A}{B} = \dfrac{4}{5}\), we are to find the sum 
    of the total profit $\left( A + B\right)$
    \[B = \dfrac{5}{9}(A + B) \text{ and } A = \dfrac{4}{9}(A + B)\]
    \begin{align*}
        B - A &= 5000 \\
        \dfrac{5}{9}(A + B) - \dfrac{4}{9}(A + B) &= 5000 \\
        \therefore \hspace{80px}  A + B &= 45000
    \end{align*}
    \item 
    \item Start by resolving \\
    \[ \dfrac{4}{5} + \dfrac{1}{2} = \dfrac{13}{10}\text{ now }  \dfrac{2}{13/10} = \dfrac{20}{13} \] 
    then, \\
    \[3 - \dfrac{20}{13} = \dfrac{39 - 20}{13} = \dfrac{19}{13} = 1\dfrac{6}{13} \]
    \item \textbf{(B)} Given $\dfrac{x}{y} = \dfrac{1/3}{1/2} = \dfrac{2}{3} = \dfrac{28}{42}$ and $\dfrac{\psi}{\theta} = \dfrac{2/5}{4/7} = \dfrac{42}{60}$ \\
    \[ \hspace{10px} \dfrac{x/y}{\theta / \psi} = \left(\dfrac{x}{y}\right) \cdot \left(\dfrac{\psi}{\theta}\right) = \dfrac{28 \cdot \cancel{42}}{\cancel{42} \cdot 60} = \dfrac{7}{15}  \]
    \(\therefore \hspace{10px} x:\theta = 7:15\) \\\\
    \textbf{Note} I multiplied $y$ and $\psi$ by 14 and 3 respectively, so that they cancel each other

    \item \textbf{(C)} The smallest share is one with the least ratio value \\
    \[ \therefore \hspace{40px} \dfrac{1}{7+2+1} \times 560 = \dfrac{1}{10} \times 560 = 56\]
    \item Start form bottom: \\
    \[\cfrac{1}{4 + \cfrac{1}{5}}  = \cfrac{1}{ 21/5} = \cfrac{5}{21} \Rightarrow 2 - \dfrac{5}{21} = \dfrac{37}{21}\] \\
    now we have that the bottom \(2 - \cfrac{1}{4 + \cfrac{1}{5}} = \dfrac{37}{21}\) \vspace{3pt}
    \[2 + \cfrac{ 1 }{37/21} = 2 + \dfrac{21}{37}  = \dfrac{74 + 7}{12} = \dfrac{81}{12} = \dfrac{27}{4}\]
    Notice that \(2 + \cfrac{1}{2 - \cfrac{1}{4 + \cfrac{1}{5}}} = \dfrac{27}{4}\) \vspace{3pt}
    \[\dfrac{1}{2} + \cfrac{1}{27/4} = \dfrac{1}{2} + \dfrac{4}{27} = \dfrac{27 + 8}{54} = \dfrac{35}{54} \]

    \item \textbf{(B)} $22 \dfrac{1}{2} \% = \dfrac{45}{2} \% \text{ and } 17\dfrac{1}{10} \% = \dfrac{171}{10} \%$, let Naira = $N$, we can ignore the \% sign since they both carry it \\
        \begin{align*}
        \dfrac{45}{2}\,N = \dfrac{171}{10}\,M  &\Rightarrow 45 \times 10 \,N = 171 \times 2 \,M \\
        &\Rightarrow \left(\dfrac{45 \times 10 }{171 \times 2}\right)N = M = \dfrac{225}{171}
        \end{align*}
    \(\therefore \hspace{20px}  M = 1\dfrac{54}{171} = 1\dfrac{18}{57} \) \\\\  
    \textbf{Hint:} How did i know a large number like 171 is divisible by 3. \\
    For a number to be divisible by 3, the sum of the individual number must be divisible by 3 like 171, the sum of the digit is 9 like wise 54

    \item \textbf{(E)} Express these number in index form \\
    \(48 = 2^3 \times 3^1, 64 = 2^5 \text{ and } 80 = 2^4 \times 5\) \\\\
    The LCM is \textbf{is the number with the largest index} while HCF is the number with the \textbf{least index} 
        \begin{align*}
            \therefore \text{LCM } &=  2^5 \times 3^1 \times 5^1 \text{ while HCF = } 2^3 \\
            \dfrac{\text{LCM}}{\text{HCF}} &= \dfrac{2^5 \times 3^1 \times 5^1 }{2^3} = 60 
        \end{align*}

    \item given that rate = 8\%, time = 4, amount = 330 
        \begin{align*}
            \text{A} &= \text{P} + \text{I} = \text{P} + \dfrac{\text{P} \cdot \text{R} \cdot \text{T}}{100} = \text{P} + \dfrac{\text{P} \cdot 8 \cdot 4}{100} \\
            100\text{A} &= 100\text{P} + 32\text{P} = 132\text{P} \\
            \therefore  \text{P} &=\dfrac{100}{132}
        \end{align*}
    
    \item \textbf{(E)} let $C_p$ = cost price for $P$ and $\{S_p = C_q\}$ = selling price for $P$ equals the cost price for $Q$, likewise the \{$S_q = C_r = \text{N}209 $\} the selling price for $Q$ equals the cost price for, $R$ The motive is to find $C_p$ \{cost price for $P$\} \\
    Given \% profit of $P$ = 10\% and \% loss $Q$ = 5\% \\\\
    \( \Rightarrow \% \text{profit} = \dfrac{\text{Selling price} - \text{Cost price}}{\text{Cost price}} \times 100\) \\\\
    \( \Rightarrow \% \text{loss} = \dfrac{\text{Cost price} - \text{Selling price}}{\text{Cost price}} \times 100\) \\
        \begin{align*}
            \text{for } q: \hspace{10px} 5 &= \dfrac{C_q - S_q}{C_q} \times 100 = \dfrac{C_q - C_r}{C_q}  \times 100 \\
            5 &=  \dfrac{C_q - 209}{C_q} \times 100 \\
            5C_q &= (C_q - 209)100 \Rightarrow -95C_q= -20900 \\
            C_q &= 220 \\
            \text{for } p: \hspace {10px} 10 &= \dfrac{S_p - C_p}{C_p} \times 100 = \dfrac{C_q - C_p}{C_p}  \times 100 \\
            10 &= \dfrac{220 - C_p}{C_p} \times 100 \Rightarrow 110 C_p = 22000 \\
            \therefore  \hspace{20px}C_p &= 200 
        \end{align*}
        the cost price for $P$ = N200

    \item \textbf{(D)} Let that number = $x$, and expressing 252 in index 
    \begin{tabular}{c|c} 
        2&252 \\ \hline
        2 & 126 \\ \hline
        3&63 \\ \hline
        3&21 \\ \hline 
        7&7 \\ \hline
        &1
    \end{tabular} \\
    $\therefore 252 = 2^2 \times 3^2 \times 7^1 = ( 2 \times 3)^2 \times 7 $ so inorder for the expression to be a perfect square we would have to multiply it by 7 

    \item \textbf{(B)} Resolving the denominator $ \dfrac{1}{2} + \dfrac{1}{3} = \dfrac{3 + 2}{6} = \dfrac{5}{6}$  \\\\
    now the question $ \dfrac{\dfrac{2}{3}}{\dfrac{1}{2} + \dfrac{1}{3}} = \dfrac{2/3}{5/6} = \dfrac{2}{\cancel{3}} \cdot \dfrac{\cancelto{2}{6}}{5} = \dfrac{4}{5}$ \\
    Lets say $a$ is a number, the reciprocal of $a$ is $\dfrac{1}{a}$ \\\\
    $\therefore \text{Reciprocal of the question } \hspace{10px} 1 \divisionsymbol \dfrac{4}{5} = \dfrac{5}{4}$

    \item \textbf{(C)} let the number of oranges share =  $x$. \\
    The first boy received $\dfrac{1}{3}x$, now the remiander is \\ 
    \[x - \dfrac{1}{3}x = \dfrac{3x - x }{3} = \dfrac{2x}{3}, \hspace{10px} \therefore \text{remainder} = \dfrac{2x}{3}\]
    The second boy received $\dfrac{2}{3} \times \dfrac{2x}{3} = \dfrac{4}{9}x$ 
    \begin{align*}
    \text{Now remainder} &= \text{Old remainder - second boy's share} \\
    &= \dfrac{2x}{3} - \dfrac{4x}{9} = \dfrac{6x - 4x}{9} = \dfrac{2x}{9}
    \end{align*}
    The remaining orange = $12 = \dfrac{2x}{9} = \dfrac{\cancelto{6}{12} \cdot 9}{\cancel{2}} = 54 \text{ oranges } $
    \item \textbf{(A)} $P$ = N150.00, $T$ = 5 years, $I$ = N55.00 and $R$ = $x$ \\
    \[ I = \dfrac{P \times R \times T}{100} \Rightarrow 55 =  \dfrac{150 \times R \times 5}{100} \Rightarrow \dfrac{5500}{150 \times 5} = R\]
    $ \therefore \hspace{10px} R = 7\,\dfrac{250}{750} = 7 \dfrac{1}{3} $
    
    \item let the total pencil share = $p$, Desmond's portion of the ratio is 2 and the sum of the ratio = 2 + 3 + 5 = 10 \\\\
    Desmond's share $ = \dfrac{2}{10} \times p = 5 \hspace{10px} \therefore p = 25$ \\
    The total number of pencil shared is 25
    \item 
    \item \textbf{(B)} Given, $1m$ = $1.1yard$ and $ l = 50m $  \\
    $\therefore 1 yard = \dfrac{1}{1.1}m $
    \begin{align*}
         l \times w = 1815y^2 \hspace{5px} \Rightarrow
         50m \times w &= 1815\left(\dfrac{1}{1.1}\right)^2m \\
         w &= \dfrac{1815}{50 \times 1.1^2} = 30m 
    \end{align*}

    \item $0.02147 = 0.023, \, 1.2047 = 1.2 \text{ and } 0.023789 = 0.024$ \\
    \(\therefore \text{ we have the expression as, } \) \\
    \[\dfrac{0.023 \times 1.2}{0.024} = \dfrac{0.023 \times \cancel{1.2}}{\cancelto{2}{2.4} \times 10^{-2}} = \dfrac{2.3}{2} = 1.15 \approx 1.2\] 
    \item let number of people in cinema = $x$ 
    \begin{align*}
    x &= 27\,\dfrac{1}{2}\%x + 47\,\dfrac{1}{2}\%x + 84  \\
    &= \dfrac{55}{200}x + \dfrac{95}{200}x + 84 \\
     200x &= 55x + 95x + (84 \cdot 200) \\
     200x - 150x &= 84 \cdot 200 
    \Rightarrow \hspace{5px} x = \dfrac{84 \cdot \cancelto{4}{200}}{\cancel{50}}  = 336 
    \end{align*}
    $\text{Finding for Men:} \hspace{10px} \dfrac{95}{200} \times 336$
    \item
    \item
    \item 
    \item
    \item
    \item 
    \item 
    \item 
    \item 
    \item
    \item
    \item
    \item 
    \item
    \item
    \item 
    \item 
    \item 
    \item 
    \item
    \item
    \item
    \item 
    \item
    \item
    \item 
    \item 
    \item 
    \item 
    \item
    \item
    \item
    \item 
    \item
    \item
    \item 
    \item 
    \item 
    \item 
    \item
    \item
    \item
    \item 
    \item
    \item
    \item 
    \item 
    \item 
    \item 
    \item
    \item
    \item
    \item 
    \item
    \item
    \item 
    \item 
    \item 
    \item 
    \item
    \item
    \item
    \item 
    \item
    \item
    \item 
    \item 
\end{enumerate}
\end {multicols}