<<<<<<< HEAD








\documentclass[a4paper]{book}
=======
\documentclass[a4paper, flenq]{book}
>>>>>>> e5d91a4063d0202e7da7b019d0c8fdb0ba605f83
\usepackage{multicol}
%\usepackage {stix}
\usepackage {cancel}
\usepackage{amsmath, amsfonts, amssymb}
\usepackage{tgtermes}




% Page layout customization
\usepackage{geometry}
\geometry{margin=.7in} % Adjust margins for better readability

% Header and footer customization
\usepackage{fancyhdr}
\pagestyle{fancy}
\fancyhf{} % Clear header and footer
\renewcommand{\headrulewidth}{0pt} % Remove header rule
\fancyhead[LE,RO]{\thepage} % Page number on outer corners

% Section title customization
\usepackage{titlesec}

% Table of contents customization
\usepackage{tocloft}
\renewcommand{\cftchapleader}{\cftdotfill{\cftdotsep}} % Dots between chapters in TOC
\renewcommand{\cftbeforechapskip}{20pt} % Increase space before chapter entries

% Image inclusion
\usepackage{graphicx}

% Hyperlinks
\usepackage{hyperref}
\hypersetup{
colorlinks=true,
linkcolor=blue,
urlcolor=blue,
citecolor=purple,
}

% Bibliography management
\usepackage{biblatex}
%\addbibresource{your_bibliography_file.bib} % Replace with your bibliography file

% Professional-looking tables
\usepackage{booktabs}

% Customize lists
\usepackage{enumitem}
\setlist[enumerate]{itemsep=3pt,topsep=0pt,partopsep=0pt} % Improve spacing for lists

% Mathematics packages
\usepackage{amsmath, amsthm}
\usepackage{physics}
\AtBeginDocument{\RenewCommandCopy\qty\SI}

% Additional math fonts (optional)
%\usepackage{newtxtext,newtxmath}

% SI units (optional)
\usepackage{siunitx}

% Code listings (if needed)
\usepackage{listings}

% Title page customization
\usepackage{titling}

% Clever referencing
\usepackage{cleveref}

% Lines (optional)
\usepackage{tikz}
\usetikzlibrary{positioning}


\begin{document}

\begin{titlingpage}
\centering
\vspace*{\fill}
\Huge\textbf{Ace Your UTME Mathematics}\\
\vspace{1cm}
\Large{A Comprehensive Guide with Past Questions and Solutions}\\
\vspace{2cm}
\Large\textbf{By}\\
\vspace{0.5cm}
\Large\textbf{Ayodeji Adesegun and Chimobi Nwafor}\\
\vspace{\fill}
\Large{March 2024}\\
\vspace*{\fill}
\end{titlingpage}

\begin{frontmatter}
\chapter*{Dedication and Acknowledgements}

This work is dedicated to our families, whose unwavering support has been our greatest strength throughout this journey. Their constant encouragement and belief in our abilities have fueled our passion and perseverance in creating this resource.

We would also like to express our sincere gratitude to the following individuals and institutions for their invaluable contributions:
\begin{itemize}
\item Our mentors and teachers, who instilled in us a love for mathematics and equipped us with the knowledge and skills needed to succeed.
\item The examiners and administrators of the UTME, whose dedication to educational standards ensures a fair and effective assessment process.
\item  Our colleagues and friends, who provided feedback and support throughout the development of this book.
\item The wider academic community, whose research and publications have laid the foundation for our understanding of mathematics.
\end{itemize}

We are truly grateful for the collective effort that has made this book possible. We hope that it will be a valuable resource for students preparing for the UTME and beyond.

\section*{How We Wrote This Book}
This book was written using the  \LaTeX \ document processing package. Specifically, this book was prepared using the MikTex installation of pdflatex on a PC running Windows 11. We must thank the authors of the various packages we used. The diagrams were prepared using METAPOST, a graphics programming language based on Donald Knuth's METAFONT.

\section*{About the Authors}
\begin{description}
\item \href{https://linkedin.com/in/ayodejiades}{\textbf{Ayodeji Adesegun}}: He is a teenager. In his spare time, he enjoys solving mathematical problems from various contests not excluding the Olympiads. You can find more about him \href{https://ayodejiades.vercel.app}{here.}
\item \href{https://linkedin.com/in/jeremaih-nwafor}{\textbf{Chimobi Nwafor}}: He is a software engineer trainee at INITS. 
\end{description}

\clearpage

\tableofcontents

\chapter*{Preface}

Welcome to "Ace Your UTME Mathematics," a comprehensive guide designed to help you conquer the upcoming UTME Mathematics exam. This book provides you with a wealth of past questions, detailed solutions, and insightful strategies to enhance your understanding and preparation.

This book is organized into chapters that follow the official UTME Mathematics syllabus, covering all key topics and subtopics. Each chapter includes a variety of past questions carefully selected to reflect the types and difficulty levels encountered in the actual exam.

In addition to past questions, we have provided detailed solutions that explain the reasoning behind each step and highlight common mistakes to avoid. We encourage you to work through these solutions carefully and utilize them as learning tools to improve your problem-solving skills.

Furthermore, we have incorporated valuable strategies throughout the book, offering tips and techniques to maximize your efficiency and performance on the exam. These strategies will help you manage your time effectively, approach different question types confidently, and overcome any challenges you may encounter.

We are confident that "Ace Your UTME Mathematics" will be your ultimate companion on your journey to success. By diligently working through the material and utilizing the resources provided, you will gain the knowledge, skills, and confidence needed to achieve your desired score on the UTME Mathematics exam.\\

Best of luck!\\

Ayodeji Adesegun and Chimobi Nwafor
\clearpage
\end{frontmatter}

\mainmatter

% Chapter organization
\chapter{Number and Numeration}
\section{Number Bases}
\subsection{Questions}
\begin{multicols}{2}
\begin{enumerate}[label={\arabic*.}]
\item The number \(25\) when converted from the tens and units base to the binary base (base \(2\)) is one of the following
	\begin{enumerate}[label={\Alph*.}]
	\item \(10011\)
	\item \(1111011\)
	\item \(111000\)
	\item \(11001\)
	\item \(110011\)
	\end{enumerate}
\item The currency used in a country bought \(4\) bags of rice at \(N56\) per bag and \(3\) tins of milk at \(N4\) per tin. What is the total cost of the items she bought?
  \begin{enumerate}[label={\Alph*.}]
    \item \(N245\)
    \item \(N242\)
    \item \(N236\)
    \item \(N341\)
    \item \(N338\)
  \end{enumerate}
\item Evaluate \(212_{3} - 121_{3} + 222_{3}\).
  \begin{enumerate}[label={\Alph*.}]
    \item \(313_3\)
    \item \(1000_3\)
    \item \(1020_3\)
    \item \(1222_3\)
    \item \(1213_3\)
  \end{enumerate}
\item A trader in a country where their currency 'MONT' (\(M\)) is in base five bought \(103_5\) oranges at \(M14_5\) each. If he sold the oranges at \(M24_5\) each, what would be his gain?
  \begin{enumerate}[label={\Alph*.}]
    \item \(M103_5\)
    \item \(M1030_5\)
    \item \(M102_5\)
    \item \(M2002_5\)
    \item \(M3024_5\)
  \end{enumerate}  
\item Find \(x\) if \((x_{4})^2 = (100100)_{2}\)
	\begin{enumerate}[label={\Alph*.}]
	\item \(6\)
	\item \(12\)
	\item \(100\)
	\item \(210\)
 	\item \(10042\)
	\end{enumerate}
\item Convert \(241_5\) to base \(8\).
	\begin{enumerate}[label={\Alph*.}]
	\item \(71_8\)
	\item \(107_8\)
	\item \(176_8\)
	\item \(241_8\)
	\end{enumerate}
\item In the equation \(\dfrac{11_2}{x_2}  = \dfrac{1000_2}{{x}_2 + {101}_{2}}\), solve for \(x\).  
	\begin{enumerate}[label={\Alph*.}]
	\item \(101\)
	\item \(11\)
	\item \(110\)
	\item \(111\)
 	\item \(10\)
	\end{enumerate}
\item  \({4243}_5 - {12x4}_5 = y344_5\). What is the difference between \(x\) and \(y\)?
	\begin{enumerate}[label={\Alph*.}]
	\item \(4\)
	\item \(2\)
	\item \(1\)
	\item \(3\)
 	\item \(5\)
	\end{enumerate}
\item In base ten, the number \(101101_2\) equals?
	\begin{enumerate}[label={\Alph*.}]
	\item \(15\)
	\item \(45\)
	\item \(23\)
	\item \(12\)
	\end{enumerate}
\item Convert the number \(39\) to base \(2\).
	\begin{enumerate}[label={\Alph*.}]
	\item \(100111\)
	\item \(111001\)
	\item \(110111\)
	\item \(111111\)
 	\item \(100101\)
	\end{enumerate}
\item Find \(n\) if \({34}_n = 10110_2\). 
	\begin{enumerate}[label={\Alph*.}]
	\item \(5\)
	\item \(6\)
	\item \(7\)
	\item \(8\)
 	\item \(9\)
	\end{enumerate}
\item If \(2_9 \times {(Y3)}_9 = {3}_5 \times {(Y3)}_5\). Find the value of \(Y\).
	\begin{enumerate}[label={\Alph*.}]
	\item \(4\)
	\item \(3\)
	\item \(2\)
	\item \(1\)
 	\item \(5\)
	\end{enumerate}
\item Simplify \({213}_4 \times {23}_4\).		
	\begin{enumerate}[label={\Alph*.}]
	\item \({10321}_4\)
	\item \({12231}_4\)
	\item \({13221}_4\)
	\item \({10311}_4\)
 	\item \({13021}_4\)
	\end{enumerate}
\item \({55}_x + {52}_x = {77}_{10}\), find \(x\).
	\begin{enumerate}[label={\Alph*.}]
	\item \(5\)
	\item \(6\)
	\item \(7\)
	\item \(8\)
 	\item \(10\)
	\end{enumerate}
\item If \(x_{10} = {23}_5\), find \(x\). 
	\begin{enumerate}[label={\Alph*.}]
	\item \( 15\)
	\item \(12\)
	\item \(14\)
	\item \(13\)
 	\item \(16\)
	\end{enumerate}
\item Find the sum of \(25_{6}\), \(52_{6}\), \(43_{6}\) in base \(8\).
	\begin{enumerate}[label={\Alph*.}]
	\item \(411\)
	\item \(141\)
	\item \(114\)
	\item \(417\)
	\end{enumerate}
\item \({2A3}_3\) = \({77}_8\), find \(A\).
	\begin{enumerate}[label={\Alph*.}]
	\item \(1\)
	\item \(2\)
	\item \(0\)
	\item \(4\)
	\end{enumerate}
\item Evaluate \(({202}_3)^2\) - \(({112}_3)^2\)
	\begin{enumerate}[label={\Alph*.}]
	\item \(21112\)
	\item \(21121\)
	\item \(21011\)
	\item \(21120\)
	\end{enumerate}
\item If \({321}_4\) is divided by \({23}_4\) and leaves a remainder \(r\), what is the value of \(r\) ?
	\begin{enumerate}[label={\Alph*.}]
	\item \(4\)
	\item \(2\)
	\item \(3\)
	\item \(0\)
	\item \(1\)
	\end{enumerate}
\item Convert \(521_{10}\) to a number in base 5
	\begin{enumerate}[label={\Alph*.}]
	\item \({1404}_5\)
	\item \({4041}_5\)
	\item \({4140}_5\)
	\item \({4014}_5\)
	\item \({4104}_5\)
	\end{enumerate}
\item If \({6R7}_8\) = \({511}_9\), find \(R\).
	\begin{enumerate}[label={\Alph*.}]
	\item \(6\)
	\item \(5\)
	\item \(3\)
	\item \(2\)
	\item \(8\)
	\end{enumerate}
\item Find the value of \(x\) if \({121}_x\) + \({112}_x\) = \({30}_{10}\).
	\begin{enumerate}[label={\Alph*.}]
	\item \(5\)
	\item \(7\)
	\item \(-{\dfrac{9}{2}}\)
	\item \(3\)
	\item \(4\)
	\end{enumerate}
\item Evaluate \(({1011}_2)^2\) - \({1012}_2\).
	\begin{enumerate}[label={\Alph*.}]
	\item \({110000}_2\)
	\item \({110000}_2\)
	\item \({101011}_2\)
	\item \({110110}_2\)
	\end{enumerate}
\item Add \({1101}_2\), \({11011}_2\) and \({111}_2\).
	\begin{enumerate}[label={\Alph*.}]
	\item \({110110}_2\)
	\item \({101011}_2\)
	\item \({111011}_2\)
	\item \({101010}_2\)
	\item \({110011}_2\)
	\end{enumerate}
\item Find the value of \(m\) if \({13}_m\) + \({24}_m\) = \({41}_m\)
	\begin{enumerate}[label={\Alph*.}]
	\item \(8\)
	\item \(5\)
	\item \(4\)
	\item \(6\)
	\item \(3\)
	\end{enumerate}
\item If \({125}_x\) = \({20}_10\), find \(x\).
	\begin{enumerate}[label={\Alph*.}]
	\item \(2\)
	\item \(3\)
	\item \(4\)
	\item \(6\)
	\item \(5\)
	\end{enumerate}
\item If \((K2)_6\) \(\times\) \({3}_6\) = \({3}_5(K4)_5\), what is the value of \(k\)?
	\begin{enumerate}[label={\Alph*.}]
	\item \(2\)
	\item \(1\)
	\item \(3\)
	\item \(4\)
	\item \(5\)
	\end{enumerate}
\item Find \(P\), if \({451}_6\) - \({P}_7\) = \({305}_6\)
	\begin{enumerate}[label={\Alph*.}]
	\item \({116}_7\)
	\item \({62}_7\)
	\item \({611}_7\)
	\item \({142}_7\)
	\end{enumerate}
\item The sum of four numbers is \({1214}_5\). What is the average expressed in base 5? 
	\begin{enumerate}[label={\Alph*.}]
	\item \(141\)
	\item \(411\)
	\item \(417\)
	\item \(114\)
	\item \(471\)
	\end{enumerate}
\item \((1P03)_4\) = \(115_{10}\), find \(P\).
	\begin{enumerate}[label={\Alph*.}]
	\item \(2\)
	\item \(0\)
	\item \(1\)
	\item \(4\)
	\item \(3\)
	\end{enumerate}
\item \((P344)_6\) - \((23P2)_6\) = \((2PP2)_6\), find the digit \(P\).
	\begin{enumerate}[label={\Alph*.}]
	\item \(1\)
	\item \(2\)
	\item \(3\)
	\item \(4\)
	\item \(5\)
	\end{enumerate}
\item \(4243_5\) - \((12X4)_5\) = \(Y344\). What is the difference between \(X\) and \(Y\)?
	\begin{enumerate}[label={\Alph*.}]
	\item \(1\)
	\item \(2\)
	\item \(3\)
	\item \(4\)
	\item \(5\)
	\end{enumerate}
\item
	\begin{enumerate}[label={\Alph*.}]
	\item \(\)
	\item \(\)
	\item \(\)
	\item \(\)
	\end{enumerate}
\item
	\begin{enumerate}[label={\Alph*.}]
	\item \(\)
	\item \(\)
	\item \(\)
	\item \(\)
	\end{enumerate}
\item
	\begin{enumerate}[label={\Alph*.}]
	\item \(\)
	\item \(\)
	\item \(\)
	\item \(\)
	\end{enumerate}
\item
	\begin{enumerate}[label={\Alph*.}]
	\item \(\)
	\item \(\)
	\item \(\)
	\item \(\)
	\end{enumerate}
\item
	\begin{enumerate}[label={\Alph*.}]
	\item \(\)
	\item \(\)
	\item \(\)
	\item \(\)
	\end{enumerate}
\item
	\begin{enumerate}[label={\Alph*.}]
	\item \(\)
	\item \(\)
	\item \(\)
	\item \(\)
	\end{enumerate}
\item
	\begin{enumerate}[label={\Alph*.}]
	\item \(\)
	\item \(\)
	\item \(\)
	\item \(\)
	\end{enumerate}
\item
	\begin{enumerate}[label={\Alph*.}]
	\item \(\)
	\item \(\)
	\item \(\)
	\item \(\)
	\end{enumerate}
\item
	\begin{enumerate}[label={\Alph*.}]
	\item \(\)
	\item \(\)
	\item \(\)
	\item \(\)
	\end{enumerate}
\item
	\begin{enumerate}[label={\Alph*.}]
	\item \(\)
	\item \(\)
	\item \(\)
	\item \(\)
	\end{enumerate}
\item
	\begin{enumerate}[label={\Alph*.}]
	\item \(\)
	\item \(\)
	\item \(\)
	\item \(\)
	\end{enumerate}
\item
	\begin{enumerate}[label={\Alph*.}]
	\item \(\)
	\item \(\)
	\item \(\)
	\item \(\)
	\end{enumerate}
\item
	\begin{enumerate}[label={\Alph*.}]
	\item \(\)
	\item \(\)
	\item \(\)
	\item \(\)
	\end{enumerate}
\item
	\begin{enumerate}[label={\Alph*.}]
	\item \(\)
	\item \(\)
	\item \(\)
	\item \(\)
	\end{enumerate}
\item
	\begin{enumerate}[label={\Alph*.}]
	\item \(\)
	\item \(\)
	\item \(\)
	\item \(\)
	\end{enumerate}
\item
	\begin{enumerate}[label={\Alph*.}]
	\item \(\)
	\item \(\)
	\item \(\)
	\item \(\)
	\end{enumerate}
\item
	\begin{enumerate}[label={\Alph*.}]
	\item \(\)
	\item \(\)
	\item \(\)
	\item \(\)
	\end{enumerate}
\item
	\begin{enumerate}[label={\Alph*.}]
	\item \(\)
	\item \(\)
	\item \(\)
	\item \(\)
	\end{enumerate}
\item
	\begin{enumerate}[label={\Alph*.}]
	\item \(\)
	\item \(\)
	\item \(\)
	\item \(\)
	\end{enumerate}
\item
	\begin{enumerate}[label={\Alph*.}]
	\item \(\)
	\item \(\)
	\item \(\)
	\item \(\)
	\end{enumerate}
\item
	\begin{enumerate}[label={\Alph*.}]
	\item \(\)
	\item \(\)
	\item \(\)
	\item \(\)
	\end{enumerate}
\item
	\begin{enumerate}[label={\Alph*.}]
	\item \(\)
	\item \(\)
	\item \(\)
	\item \(\)
	\end{enumerate}
\item
	\begin{enumerate}[label={\Alph*.}]
	\item \(\)
	\item \(\)
	\item \(\)
	\item \(\)
	\end{enumerate}
\item
	\begin{enumerate}[label={\Alph*.}]
	\item \(\)
	\item \(\)
	\item \(\)
	\item \(\)
	\end{enumerate}
\item
	\begin{enumerate}[label={\Alph*.}]
	\item \(\)
	\item \(\)
	\item \(\)
	\item \(\)
	\end{enumerate}
\item
	\begin{enumerate}[label={\Alph*.}]
	\item \(\)
	\item \(\)
	\item \(\)
	\item \(\)
	\end{enumerate}
\item
	\begin{enumerate}[label={\Alph*.}]
	\item \(\)
	\item \(\)
	\item \(\)
	\item \(\)
	\end{enumerate}
\item
	\begin{enumerate}[label={\Alph*.}]
	\item \(\)
	\item \(\)
	\item \(\)
	\item \(\)
	\end{enumerate}
\item
	\begin{enumerate}[label={\Alph*.}]
	\item \(\)
	\item \(\)
	\item \(\)
	\item \(\)
	\end{enumerate}
\item
	\begin{enumerate}[label={\Alph*.}]
	\item \(\)
	\item \(\)
	\item \(\)
	\item \(\)
	\end{enumerate}
\item
	\begin{enumerate}[label={\Alph*.}]
	\item \(\)
	\item \(\)
	\item \(\)
	\item \(\)
	\end{enumerate}
\item
	\begin{enumerate}[label={\Alph*.}]
	\item \(\)
	\item \(\)
	\item \(\)
	\item \(\)
	\end{enumerate}
\item
	\begin{enumerate}[label={\Alph*.}]
	\item \(\)
	\item \(\)
	\item \(\)
	\item \(\)
	\end{enumerate}
\item
	\begin{enumerate}[label={\Alph*.}]
	\item \(\)
	\item \(\)
	\item \(\)
	\item \(\)
	\end{enumerate}
\item
	\begin{enumerate}[label={\Alph*.}]
	\item \(\)
	\item \(\)
	\item \(\)
	\item \(\)
	\end{enumerate}
\item
	\begin{enumerate}[label={\Alph*.}]
	\item \(\)
	\item \(\)
	\item \(\)
	\item \(\)
	\end{enumerate}
\item
	\begin{enumerate}[label={\Alph*.}]
	\item \(\)
	\item \(\)
	\item \(\)
	\item \(\)
	\end{enumerate}
\item
	\begin{enumerate}[label={\Alph*.}]
	\item \(\)
	\item \(\)
	\item \(\)
	\item \(\)
	\end{enumerate}
\item
	\begin{enumerate}[label={\Alph*.}]
	\item \(\)
	\item \(\)
	\item \(\)
	\item \(\)
	\end{enumerate}
\item
	\begin{enumerate}[label={\Alph*.}]
	\item \(\)
	\item \(\)
	\item \(\)
	\item \(\)
	\end{enumerate}
\item
	\begin{enumerate}[label={\Alph*.}]
	\item \(\)
	\item \(\)
	\item \(\)
	\item \(\)
	\end{enumerate}
\item
	\begin{enumerate}[label={\Alph*.}]
	\item \(\)
	\item \(\)
	\item \(\)
	\item \(\)
	\end{enumerate}


\item
	\begin{enumerate}[label={\Alph*.}]
	\item \(\)
	\item \(\)
	\item \(\)
	\item \(\)
	\end{enumerate}
\item
	\begin{enumerate}[label={\Alph*.}]
	\item \(\)
	\item \(\)
	\item \(\)
	\item \(\)
	\end{enumerate}
\item
	\begin{enumerate}[label={\Alph*.}]
	\item \(\)
	\item \(\)
	\item \(\)
	\item \(\)
	\end{enumerate}
\item
	\begin{enumerate}[label={\Alph*.}]
	\item \(\)
	\item \(\)
	\item \(\)
	\item \(\)
	\end{enumerate}
\item
	\begin{enumerate}[label={\Alph*.}]
	\item \(\)
	\item \(\)
	\item \(\)
	\item \(\)
	\end{enumerate}
\item
	\begin{enumerate}[label={\Alph*.}]
	\item \(\)
	\item \(\)
	\item \(\)
	\item \(\)
	\end{enumerate}
\item
	\begin{enumerate}[label={\Alph*.}]
	\item \(\)
	\item \(\)
	\item \(\)
	\item \(\)
	\end{enumerate}
\item
	\begin{enumerate}[label={\Alph*.}]
	\item \(\)
	\item \(\)
	\item \(\)
	\item \(\)
	\end{enumerate}
\item
	\begin{enumerate}[label={\Alph*.}]
	\item \(\)
	\item \(\)
	\item \(\)
	\item \(\)
	\end{enumerate}
\item
	\begin{enumerate}[label={\Alph*.}]
	\item \(\)
	\item \(\)
	\item \(\)
	\item \(\)
	\end{enumerate}
\item
	\begin{enumerate}[label={\Alph*.}]
	\item \(\)
	\item \(\)
	\item \(\)
	\item \(\)
	\end{enumerate}
\item
	\begin{enumerate}[label={\Alph*.}]
	\item \(\)
	\item \(\)
	\item \(\)
	\item \(\)
	\end{enumerate}
\item
	\begin{enumerate}[label={\Alph*.}]
	\item \(\)
	\item \(\)
	\item \(\)
	\item \(\)
	\end{enumerate}
\item
	\begin{enumerate}[label={\Alph*.}]
	\item \(\)
	\item \(\)
	\item \(\)
	\item \(\)
	\end{enumerate}
\item
	\begin{enumerate}[label={\Alph*.}]
	\item \(\)
	\item \(\)
	\item \(\)
	\item \(\)
	\end{enumerate}
\item
	\begin{enumerate}[label={\Alph*.}]
	\item \(\)
	\item \(\)
	\item \(\)
	\item \(\)
	\end{enumerate}
\item
	\begin{enumerate}[label={\Alph*.}]
	\item \(\)
	\item \(\)
	\item \(\)
	\item \(\)
	\end{enumerate}
\item
	\begin{enumerate}[label={\Alph*.}]
	\item \(\)
	\item \(\)
	\item \(\)
	\item \(\)
	\end{enumerate}
\item
	\begin{enumerate}[label={\Alph*.}]
	\item \(\)
	\item \(\)
	\item \(\)
	\item \(\)
	\end{enumerate}
\item
	\begin{enumerate}[label={\Alph*.}]
	\item \(\)
	\item \(\)
	\item \(\)
	\item \(\)
	\end{enumerate}
\item
	\begin{enumerate}[label={\Alph*.}]
	\item \(\)
	\item \(\)
	\item \(\)
	\item \(\)
	\end{enumerate}
\item
	\begin{enumerate}[label={\Alph*.}]
	\item \(\)
	\item \(\)
	\item \(\)
	\item \(\)
	\end{enumerate}
\item
	\begin{enumerate}[label={\Alph*.}]
	\item \(\)
	\item \(\)
	\item \(\)
	\item \(\)
	\end{enumerate}
\item
	\begin{enumerate}[label={\Alph*.}]
	\item \(\)
	\item \(\)
	\item \(\)
	\item \(\)
	\end{enumerate}
\item 
	\begin{enumerate}[label={\Alph*.}]
	\item \(\)
	\item \(\)
	\item \(\)
	\item \(\)
	\end{enumerate}
\item 
	\begin{enumerate}[label={\Alph*.}]
	\item \(\)
	\item \(\)
	\item \(\)
	\item \(\)
	\end{enumerate}
\end{enumerate}
\end{multicols}
\subsection{Answers}
\begin{multicols}{2}
\begin{enumerate}[label={\arabic*.}]
    \item \begin{tabular}{c|cc}
        5 & 25 & \textit{rem} \\ \hline
        5 & 5 & 0 \\ \hline
        5 & 1 & 0 \\ \hline
         & 0 & 1
    \end{tabular}
    \item 
    \item Your results are in base $3$, so changing all the number to base $10$ before solving would, is time wastage, 
    rather evaluate while they are in base $3$
    \item
    \item
    \item \textbf{(B)} convert $241_5$ to base $8$ first
    \begin{align*}
    241_5 & = 2 \times 5^2 + 4 \times 5^1 + 1 \times 5^0 \\
        & = 2 \times 25 + 4 \times 5 + 1 \times 1 \\
        & = 50 + 20 + 1 = 71_{10}
    \end{align*}
    then, convert $71_{10}$ to base  $8$ \\

    \begin{tabular}{c|cc}
        $8$ & $71$ & \textit{rem}\\ \hline
        $8$ & $8$ & $7$ \\ \hline 
        $8$ & $1$ & $0$ \\ \hline 
        & $0$ & $1$
    \end{tabular}
    $ \hspace{.2in}\therefore \,\, 241_5 = 107_8$
    \item It's easier to work with number while they are in base $10$ 
    \begin{itemize}
        \item $11_2  = 1 \times 2^1 + 1 \times 2^0 = 2 + 1 = 3_{10} = 3$ 
        \item$101_2 = 1\times 2^2 + 0\times 2^1 + 1\times 2^0 = 4 + 0 + 1 = 5_{10} = 5$ 
        \item $1000_2 = 1\times 2^3 = 8_{10} = 8$
    \end{itemize}
    you should have notice every other part with zero's will automatically be zero, so just ignore those parts
    $$\cfrac{3}{x_2} = \cfrac{8}{x_2 + 5} $$ 
    $3(x_2 + 5) = 8x_2 \,\, \Rightarrow \,\, 3x_2 + 15 = 8x_2$ \\
$\therefore \hspace{10pt} x_2 = 3_{10} = 11_{2}$
    \item If your first thought was to convert everything to base ten, that's really going to be a pain in the ass my friend, rather make an arrangement like this \\
    $\begin{array}{c}
        4243 
       -\cr \underline{12x4}\\
       \underline{\hspace{.3in}}
    \end{array}$
    \item \textbf{(B)} \begin{align*} 
        101101_2 &= 1 \times 2^5 + 0 \times 2^4 + 1 \times 2^3 + 1 \times 2^2  \\ & \hspace{10pt} + 0 \times 2^1 + 1\times 2^0 \\
        & = 32 + 0 + 8 + 4 + 0 + 1 = 45
    \end{align*}
    \item \textbf{(A)} \begin{tabular}{c|cc}
        2&39& \textit{rem} \\ \hline
        2&19& 1 \\ \hline
        2&9&1 \\ \hline
        2&4&1 \\ \hline
        2&2&0  \\ \hline
        2&1&0 \\ \hline
        &0&1 
    \end{tabular} $ \> \therefore 39_{10} = 100111 \text{   in base 2}$
    \item Convert all to base 10, To speed things up ignore all the zero's 
    \begin{align*}
   34_n &= 10110_2 \\
    3 \times n + 4 &= 1\times 2^4 + 1\times 2^2 + 1 \times 2^1 = 16 + 4 + 2 \\
    3n + 4 & = 19 \hspace{20pt} \Rightarrow 3n = 23 
    \end{align*}
    \item 
    \item
    \item
    \item
    \item 
    \item
    \item
    \item 
    \item 
    \item 
    \item 
    \item
    \item
    \item
    \item 
    \item
    \item
    \item 
    \item 
    \item 
    \item 
    \item
    \item
    \item
    \item 
    \item
    \item
    \item 
    \item 
    \item 
    \item 
    \item
    \item
    \item
    \item 
    \item
    \item
    \item 
    \item 
    \item 
    \item 
    \item
    \item
    \item
    \item 
    \item
    \item
    \item 
    \item 
    \item 
    \item 
    \item
    \item
    \item
    \item 
    \item
    \item
    \item 
    \item 
    \item 
    \item 
    \item
    \item
    \item
    \item 
    \item
    \item
    \item 
    \item 
    \item 
    \item 
    \item
    \item
    \item
    \item 
    \item
    \item
    \item 
    \item 
    \item 
    \item 
    \item
    \item
    \item
    \item 
    \item
    \item
    \item 
    \item 
\end{enumerate}
\end{multicols}
\section{Fraction and Decimals}
\subsection{Questions}
\begin{multicols}{2}
\begin{enumerate}[label={\arabic*.}]
\item The sum of \(3 {\dfrac {7}{8}}\) and \(1{\dfrac{1}{3}}\) is greater than the difference between \(\dfrac {3}{8}\) and \(1{\dfrac{2}{3}}\) by.
    \begin{enumerate}[label={\Alph*.}]
    \item \(3\dfrac{2}{3}\)
    \item \(1\dfrac{1}{2}\)
    \item \(8\dfrac{1}{8}\)
    \item \(3\frac{11}{12}\)
    \item \(5\dfrac{1}{4}\)
    \end{enumerate}
\item  After getting a rise of \(15\%\), a man's new monthly salary is N \(345\). How much per month did he earn before the increase? 
    \begin{enumerate}[label={\Alph*.}]
    \item N\(360\)
    \item N\(300\)
    \item N\(293.25\)
    \item N\(330\)
    \item N\(396.75\)
    \end{enumerate}
\item Find correct to \(3\) significant figures, the value of \(\sqrt{41830}\)
    \begin{enumerate}[label={\Alph*.}]
    \item \(647\)
    \item \(2050\)
    \item \(205\)
    \item \(647\)
    \item \(6470\)
    \end{enumerate}
\item \(12\) men complete a job in \(9\) days. How many men working at the same rate, would be required to complete the job in \(6\) days?. 
    \begin{enumerate}[label={\Alph*.}]
    \item \(24\)
    \item \(9\)
    \item \(8\)
    \item \(12\)
    \item \(18\)
    \end{enumerate}
\item Simplify \(2{\dfrac{5}{12}} - 1{\dfrac{7}{8}} \times \dfrac{6}{5}\).
    \begin{enumerate}[label={\Alph*.}]
    \item \(\dfrac{11}{30}\)
    \item \(\dfrac{9}{4}\)
    \item \(\dfrac{1}{6}\)
    \item \(\dfrac{5}{3}\)
    \item \(\dfrac{13}{20}\)
    \end{enumerate}
\item By selling an article for N45.00 a man makes a profit of \(8\%\). For how much should he have sold it in order to make a profit of 32\%? 
    \begin{enumerate}[label={\Alph*.}]
    \item N\(59.00\)
    \item N\(55.00\)
    \item N\(180.00\)
    \item N\(63.00\)
    \item N\(42.00\)
    \end{enumerate}
\item Which of the following fractions is less than one-third?
    \begin{enumerate}[label={\Alph*.}]
    \item \(\dfrac{4}{11}\)
    \item \(\dfrac{122}{383}\)
    \item \(\dfrac{15}{46}\)
    \item \(\dfrac{22}{63}\)
    \item \(\dfrac{6}{25}\)
    \end{enumerate}
\item The ratio of the price of loaf of bread to the price of a packet of sugar in \(1975\) was \(a:x\). In \(1980\), the price of a loaf of bread went up by \(25\%\) and that of a packet of sugar by \(10\%\). Their new ratio is now ?
    \begin{enumerate}[label={\Alph*.}]
    \item \(50a:44x\)
    \item \(44a:50x\)
    \item \(40a:44x\)
    \item \(55a:44x\)
    \item \(44a:55x\)
    \end{enumerate}
\item Simplify: \(1 + {\cfrac{2}{3 + \cfrac{4}{5+ \cfrac{6}{7}}}}\)
    \begin{enumerate}[label={\Alph*.}]
    \item \(\dfrac{7}{95}\)
    \item \(\dfrac{177}{95}\)
    \item \(\dfrac{233}{151}\)
    \item \(\dfrac{17}{10}\)
    \item \(\dfrac{3}{10}\)
    \end{enumerate}
\item Evaluate and correct to \(4\) decimal places \(827.51 \times 0.015\) .
    \begin{enumerate}[label={\Alph*.}]
    \item \(124.1265\)
    \item \(8.8415\)
    \item \(12.4127\)
    \item \(12.4120\)
    \item \(124.1265\)
    \end{enumerate}
\item A micrometer is defined as one millionth of a millimeter. A length of of \(12,000\) micrometer may be represented as
    \begin{enumerate}[label={\Alph*.}]
    \item \(0.000012m\)
    \item \(0.12m\)
    \item \(0.00000012m\)
    \item \(0.00000000012m\)
    \item \(0.0000012m\)
    \end{enumerate}
\item The difference between \(4{\dfrac{5}{7}}\) and \(2{\dfrac{1}{4}}\) is greater than the sum of \(\dfrac{1}{14}\) and 1\(\dfrac{1}{2}\) by.
    \begin{enumerate}[label={\Alph*.}]
    \item \(\dfrac{27}{28}\)
    \item \(\dfrac{23}{28}\)
    \item \(\dfrac{50}{56}\)
    \item \(\dfrac{48}{56}\)
    \item \(\dfrac{24}{48}\)
    \end{enumerate}
\item When a dealer sells a bicycle for N\(81\), he makes a profit of \(8\%\). What did he pay for the bicycle. 
    \begin{enumerate}[label={\Alph*.}]
    \item N\(75\)
    \item N\(75.52\)
    \item N\(74.52\)
    \item N\(87.48\)
    \item N\(73\)
    \end{enumerate}
\item A man and wife went to by an article costing N\(400\). The woman had \(10\%\) of the cost and the man \(40\%\) of the remainder. How much did they have altogether?
    \begin{enumerate}[label={\Alph*.}]
    \item N\(186\)
    \item N\(184\)
    \item N\(200\)
    \item N\(144\)
    \item N\(100\)
    \end{enumerate}
\item A sum of money invested at \(5\%\) per annum simple interest amount to N\(285.20\) after \(3\) years. How long will it take the same sum to amount to N\(434.00\) at \(7{\dfrac{1}{2}}\)\% per annum simple interest?
    \begin{enumerate}[label={\Alph*.}]
    \item \(10\) years
    \item \(12\) years
    \item \(7\dfrac{1}{2}\) years
    \item \(14\) years
    \item \(5\) years
    \end{enumerate}
\item A construction company is owned by two partners \(A\) and \(B\) and it is agreed that their profit will be divided in ratio \(4:5\), at the end of the year, B recieved N\(5,000\) more than A. What is the total profit of the company for the year? 
    \begin{enumerate}[label={\Alph*.}]
    \item N\(45,000\)
    \item N\(30,000\)
    \item N\(150,000\)
    \item N\(25,000\)
    \item N\(30,000\)
    \end{enumerate}
\item The diameter of metal rod is meased as \(23.40\)cm to 4 significant figures. What isthe maximum error in the measurement?
    \begin{enumerate}[label={\Alph*.}]
    \item \(0.0004\)cm
    \item \(0.05\)cm
    \item \(0.005\)cm
    \item \(0.5\)cm
    \item \(0.45\)cm
    \end{enumerate}
\item Simplify: \(3 - {\cfrac{2}{\dfrac{4}{5} + \dfrac{1}{2}}}\)
    \begin{enumerate}[label={\Alph*.}]
    \item \(1\dfrac{9}{10}\)
    \item \(1\dfrac{3}{10}\)
    \item \(1\dfrac{3}{4}\)
    \item \(-1\)
    \item \(1\)
    \end{enumerate}
\item Given that \(x:y\) = \(\dfrac{1}{3}:\dfrac{1}{2}\) and \(\psi:\theta\) = \(\dfrac{2}{5}:\dfrac{4}{7}\), find \(x:\theta\).
    \begin{enumerate}[label={\Alph*.}]
    \item \(20:21\)
    \item \(7:15\)
    \item \(3:20\)
    \item \(2:35\)
    \item \(4:105\)
    \end{enumerate}
\item If N\(560\) is shared in the ratio \(7:2:1\), what is the smallest share?
    \begin{enumerate}[label={\Alph*.}]
    \item N\(392\)
    \item N\(113.40\)
    \item N\(56.70\)
    \item N\(87.48\)
    \item N\(126.41\)
    \end{enumerate}
\item Simplify: \(\dfrac{1}{2} + \cfrac{1}{2 + \cfrac{1}{2 - \cfrac{1}{4 + \cfrac{1}{5}}}}\)
    \begin{enumerate}[label={\Alph*.}]
    \item \(\dfrac{169}{190}\)
    \item \(-\dfrac{1}{3}\)
    \item \(\dfrac{13}{15}\)
    \item \(-\dfrac{3}{4}\)
    \item \(-\dfrac{14}{27}\)
    \end{enumerate}
\item \(22{\dfrac{1}{2}}\%\) of the Nigerian Naira equals \(17{\dfrac{1}{10}}\%\) of a foreign currency \(M\). What is the conversion rate of \(M\) to Naira?
    \begin{enumerate}[label={\Alph*.}]
    \item \(2\dfrac{11}{57}\) Naria
    \item \(1\dfrac{18}{57}\) Naria
    \item \(\dfrac{15}{59}\) Naria
    \item \(\dfrac{15}{57}\) Naria
    \item \(38\dfrac{1}{4}\) Naria
    \end{enumerate}
\item Divide the LCM of \(48\), \(64\), and \(80\) by their HCF.
    \begin{enumerate}[label={\Alph*.}]
    \item \(30\)
    \item \(48\)
    \item \(52\)
    \item \(20\)
    \item \(60\)
    \end{enumerate}
\item A sum of money was invested at \(8\%\) per annum simple interest. If after \(4\) years the money amounts to N330.00, find the amount originally invested.
    \begin{enumerate}[label={\Alph*.}]
    \item N\(150\)
    \item N\(200\)
    \item N\(165\)
    \item N\(180\)
    \item N\(250\)
    \end{enumerate}
\item \(P\) sold his bicycle to \(Q\) at a profit of \(10\%\). \(Q\) sold to \(R\) for N209 at a loss of 5\%. How much did the bicycle cost P? 
    \begin{enumerate}[label={\Alph*.}]
    \item N\(150\)
    \item N\(205\)
    \item N\(180\)
    \item N\(196\)
    \item N\(200\)
    \end{enumerate}
\item Find the smallest number by which \(252\) can be multiplied to obtain a perfect square. 
    \begin{enumerate}[label={\Alph*.}]
    \item \(2\)
    \item \(3\)
    \item \(5\)
    \item \(7\)
    \item \(9\)
    \end{enumerate}
\item Find the reciprocal of: \(\dfrac{\dfrac{2}{3}}{\dfrac{1}{2} + \dfrac{1}{3}}\)
    \begin{enumerate}[label={\Alph*.}]
    \item \(\dfrac{4}{5}\)
    \item \(\dfrac{2}{5}\)
    \item \(\dfrac{6}{9}\)
    \item \(\dfrac{5}{4}\)
    \item \(\dfrac{3}{4}\)
    \end{enumerate}
\item Three boys shared some oranges, the first recieved \(\dfrac{1}{3}\) of the oranges, the second received \(\dfrac{2}{3}\) of the remainder, if the third boy
    \begin{enumerate}[label={\Alph*.}]
    \item \(48\)
    \item \(72\)
    \item \(54\)
    \item \(42\)
    \item \(60\)
    \end{enumerate}
\item Udoh deposited N\(150.00\) in the bank. At the end of 5 years, the simple interest on the principal was N\(55.00\). At what rate per annum was the interest paid 
    \begin{enumerate}[label={\Alph*.}]
    \item \(7\dfrac{1}{3}\%\)
    \item \(5\%\)
    \item \(11\%\)
    \item \(3\dfrac{1}{2}\%\)
    \item \(4\dfrac{2}{5}\%\)
    \end{enumerate}
\item A number of pencil were shared among Desmond, Florence, and Kevin in ratio \(2:3:5\) respectively. If Desmond got \(5\), how many were shared out?
    \begin{enumerate}[label={\Alph*.}]
    \item \(30\)
    \item \(15\)
    \item \(25\)
    \item \(20\)
    \item \(35\)
    \end{enumerate}
\item Find the least length of a rod which can be cut into exactly equal strips, each of \(40\) cm or \(48\) cm in length.
    \begin{enumerate}[label={\Alph*.}]
    \item \(240 \) cm
    \item \(480\) cm
    \item \(360\) cm
    \item \(120\) cm
    \item \(480\) cm
    \end{enumerate}
\item A rectangular lawn has an area of \(1815\) square yards. If its length is \(50\) metres, find its width in meters. Given that 1 metre equals 1.1 yard.
    \begin{enumerate}[label={\Alph*.}]
    \item \(\SI{30.00}{\meter}\)
    \item \(\SI{33.00}{\meter}\)
    \item \(\SI{32.00}{\meter}\)
    \item \(\SI{39.93}{\meter}\)
    \item \(\SI{36.45}{\meter}\)
    \end{enumerate}
\item Reduce each number to two significant figures and then evaluate \(\dfrac{0.021741 \times 1.2047}{0.023789}\)
    \begin{enumerate}[label={\Alph*.}]
    \item \(0.8\)
    \item \(1.2\)
    \item \(1.1\)
    \item \(0.9\)
    \item \(0.6\)
    \end{enumerate}
\item A cinema hall contains a certain number of people. If 27\(\dfrac{1}{2}\)\% are children, 47\(\dfrac{1}{2}\)\% are men and 84 are 
women, find the number of men in the hall
    \begin{enumerate}[label={\Alph*.}]
    \item \(133\)
    \item \(84\)
    \item \(63\)
    \item \(113\)
    \end{enumerate}
\item A woman buys \(270\) oranges for N\(1,800\) and sells at 5 for N\(40\). What is her profit?
    \begin{enumerate}[label={\Alph*.}]
    \item N \(1,620\)
    \item N \(630\)
    \item N \(360\)
    \item N \(2,160\)
    \end{enumerate}
\item If a car travels \(120\)km on \(45\) litres of petrol, how much petrol is needed for a journey of \(600\)km?
    \begin{enumerate}[label={\Alph*.}]
    \item \(720\) litres
    \item \(225\) litres
    \item \(960\) litres
    \item \(160\) litres
    \end{enumerate}
\item Simplify \(1 - \left(\dfrac{1}{7} \times 3\dfrac{1}{2}\right) \divisionsymbol \dfrac{3}{4}\)
    \begin{enumerate}[label={\Alph*.}]
    \item \(2\)
    \item \(1\)
    \item \(\dfrac{1}{3}\)
    \item \(\dfrac{2}{3}\)
    \end{enumerate}
\item Evaluate: \(\dfrac{12.02 \times 20.06}{26.04 \times 60.06}\), correct to 3 significant figures
    \begin{enumerate}[label={\Alph*.}]
    \item \(0.154\)
    \item \(0.155\)
    \item \(0.158\)
    \item \(0.157\)
    \end{enumerate}
\item Evaluate: \(\dfrac{0.8 \times 0.43 \times 0.031}{0.05 \times 0.72 \times 0.021}\), correct to 3 significant figures
    \begin{enumerate}[label={\Alph*.}]
    \item \(14.1\)
    \item \(14.09\)
    \item \(14.12\)
    \item \(14.11\)
    \end{enumerate}
\item A man bought a car for N\(500, 000\) and was able to sell it for N\(350, 000\), what was his percentage loss?
    \begin{enumerate}[label={\Alph*.}]
    \item \(50\%\)
    \item \(30\%\)
    \item \(70\%\)
    \item \(60\%\)
    \end{enumerate}
\item Simplify: \(1{\dfrac{2}{3}} + 4{\dfrac{1}{4}} + 1{\dfrac{5}{12}}\)
    \begin{enumerate}[label={\Alph*.}]
    \item \(4\dfrac{1}{3}\)
    \item \(4\dfrac{2}{3}\)
    \item \(4\dfrac{12}{17}\)
    \item \(4\dfrac{3}{17}\)
    \end{enumerate}
\item A man donates \(16\%\) of his monthly net earning to the church. If it amounts to N\(4,500\), 
what is his monthly income? 
    \begin{enumerate}[label={\Alph*.}]
    \item N\(40,500\)
    \item N\(52,000\)
    \item N\(52,500\)
    \item N\(45,000\)
    \end{enumerate}
\item If a student measured the length of a table to be \(2.30\) m insted of \(2.50\) m. 
What was his percentage error in measuring the length?
    \begin{enumerate}[label={\Alph*.}]
    \item \(7\)\%
    \item \(10\)\%
    \item \(9\)\%
    \item \(8\)\%
    \end{enumerate}
\item A man bought a second-hand photocopy machine for \(34,000\). He serviced it at a cost of N\(2,000\) and then sold it at a profit of \(15\%\). What was the selling price? 
    \begin{enumerate}[label={\Alph*.}]
    \item \(37,550\)
    \item \(40,000\)
    \item \(41,400\)
    \item \(42,400\)
    \end{enumerate}
\item A student spent \(\dfrac{1}{5}\) of his allowance on books, \(\dfrac{1}{3}\) of the remainder on food and kept the rest for contingencies. 
What fraction was kept?
    \begin{enumerate}[label={\Alph*.}]
    \item \(\dfrac{8}{15}\)
    \item \(\dfrac{4}{5}\)
    \item \(\dfrac{2}{3}\)
    \item \(\dfrac{7}{15}\)
    \end{enumerate}
\item If \(p:q\) = \(\dfrac{2}{3}:\dfrac{5}{6}\) and \(\dfrac{3}{4}:\dfrac{1}{2}\), find \(p:q:r\)
    \begin{enumerate}[label={\Alph*.}]
    \item \(12:15:10\)
    \item \(10:15:24\)
    \item \(9:10:15\)
    \item \(12:15:16\)
    \end{enumerate}
\item Simplify: \(\dfrac{3\dfrac{2}{3} \times \dfrac{5}{6} \times \dfrac{2}{3}}{\dfrac{11}{25} \times \dfrac{3}{4} \times \dfrac{2}{27}}\)
    \begin{enumerate}[label={\Alph*.}]
    \item \(4\dfrac{1}{3}\)
    \item \(30\)
    \item \(5\dfrac{2}{3}\)
    \item \(50\)
    \end{enumerate}
\item A man earns N\(3,500\) per month out of which he spend 15\% on his children's education. If he spends additional N\(1,950\) on food, how much does he have left? 
    \begin{enumerate}[label={\Alph*.}]
    \item N\(2,975\)
    \item N\(1,950\)
    \item N\(525\)
    \item N\(1025\)
    \end{enumerate}
\item Evaluate \(\dfrac{21}{9}\) to \(3\) significant figures
    \begin{enumerate}[label={\Alph*.}]
    \item \(2.30\)
    \item \(2.31\)
    \item \(2.32\)
    \item \(2.33\)
    \end{enumerate}
\item A girl shares a number of apples in the ratio \(5:3:2\). If the highest share is \(40\), find the smallest share.
    \begin{enumerate}[label={\Alph*.}]
    \item \(74\)
    \item \(38\)
    \item \(36\)
    \item \(16\)
    \end{enumerate}
\item Calculate the time taken for N\(3,000\) to earn N\(600\) at 8\% simple interest.
    \begin{enumerate}[label={\Alph*.}]
    \item \(3\) years
    \item \(2\dfrac{1}{2}\) years
    \item \(1\dfrac{1}{2}\) years
    \item \(3\dfrac{1}{2}\) years
    \end{enumerate}
\item Find the tax on an income of N\(20,000\) if no tax is paid on the first N\(10,000\) and tax is paid at N\(50\) and in N\(1,000\)
on the next N\(5,000\) and at N\(55\) and N\(1000\) on the remainder. 
    \begin{enumerate}[label={\Alph*.}]
    \item N\(225\)
    \item N\(525\)
    \item N\(552\)
    \item N\(500\)
    \end{enumerate}
\item The time taken to do a piece of work is inversely proportional to the number of men employed. If it takes \(30\) men to do a piece of work in \(6\)days, how many men are required to do the work in 4 days?
    \begin{enumerate}[label={\Alph*.}]
    \item \(35\)
    \item \(45\)
    \item \(25\)
    \item \(60\)
    \end{enumerate}
\item Three boys shared oranges. The first received \(\dfrac{1}{3}\) of the oranges and the second received \(\dfrac{2}{3}\)
of the remainder. If the third boy received the remaining \(12\) oranges, how much oranges did they share?
    \begin{enumerate}[label={\Alph*.}]
    \item \(42\)
    \item \(60\)
    \item \(54\)
    \item \(48\)
    \end{enumerate}
\item A farmer planted \(5,000\) grains of maize and harvested \(5,000\) cobs, each bearing \(500\) grains. What is the ratio of the number of grains sowed 
to the number harvested? 
    \begin{enumerate}[label={\Alph*.}]
    \item \(1:5,000\)
    \item \(1:25,000\)
    \item \(1:500\)
    \item \(1:250,000\)
    \end{enumerate}
\item Evaluate: \(\dfrac{0.21 \times 0.072 \times 0.00054}{0.006 \times 1.68 \times 0.063}\)
    \begin{enumerate}[label={\Alph*.}]
    \item \(0.1286\)
    \item \(0.01285\)
    \item \(0.01286\)
    \item \(0.1285\)
    \end{enumerate}
\item A man's initial salary is N\(540\) a month and increases after a period of six months by N\(36\) a
month. Find his salary in the eight month of the third year. 
    \begin{enumerate}[label={\Alph*.}]
    \item \(828\)
    \item \(756\)
    \item \(720\)
    \item \(684\)
    \end{enumerate}
<<<<<<< HEAD
\item Find correct to 3 decimal places: \[\left(\dfrac{1}{0.05}\right)\divisionsymbol \left(\dfrac{1}{5.005}\right) -(0.05 \times 2.05)\]
=======
\item Find correct to \(3\) decimal places: \(\left(\dfrac{1}{0.05}\right) \divisionsymbol \left(\dfrac{1}{5.005}\right) - 0.05 \times 2.05\)
>>>>>>> e2cf38e1f1daffdb98f60694e3750b7850cf3ced
    \begin{enumerate}[label={\Alph*.}]
    \item \(\)
    \item \(\)
    \item \(\)
    \item \(\)
    \end{enumerate}
\item Simplify \(\dfrac{x^{2} - y^{2}}{2x^{2}+xy - y^{2}}\)
    \begin{enumerate}[label={\Alph*.}]
    \item \(\)
    \item \(\)
    \item \(\)
    \item \(\)
    \end{enumerate}
\item 
    \begin{enumerate}[label={\Alph*.}]
    \item \(\)
    \item \(\)
    \item \(\)
    \item \(\)
    \end{enumerate}
\item 
    \begin{enumerate}[label={\Alph*.}]
    \item \(\)
    \item \(\)
    \item \(\)
    \item \(\)

    \end{enumerate}
\item 
    \begin{enumerate}[label={\Alph*.}]
    \item \(\)
    \item \(\)
    \item \(\)
    \item \(\)

    \end{enumerate}
\item 
    \begin{enumerate}[label={\Alph*.}]
    \item \(\)
    \item \(\)
    \item \(\)
    \item \(\)

    \end{enumerate}
\item 
    \begin{enumerate}[label={\Alph*.}]
    \item \(\)
    \item \(\)
    \item \(\)
    \item \(\)

    \end{enumerate}
\item 
    \begin{enumerate}[label={\Alph*.}]
    \item \(\)
    \item \(\)
    \item \(\)
    \item \(\)

    \end{enumerate}
\item 
    \begin{enumerate}[label={\Alph*.}]
    \item \(\)
    \item \(\)
    \item \(\)
    \item \(\)

    \end{enumerate}
\item 
    \begin{enumerate}[label={\Alph*.}]
    \item \(\)
    \item \(\)
    \item \(\)
    \item \(\)

    \end{enumerate}
\item 
    \begin{enumerate}[label={\Alph*.}]
    \item \(\)
    \item \(\)
    \item \(\)
    \item \(\)

    \end{enumerate}
\item 
    \begin{enumerate}[label={\Alph*.}]
    \item \(\)
    \item \(\)
    \item \(\)
    \item \(\)

    \end{enumerate}
\item 
    \begin{enumerate}[label={\Alph*.}]
    \item \(\)
    \item \(\)
    \item \(\)
    \item \(\)

    \end{enumerate}
\item 
    \begin{enumerate}[label={\Alph*.}]
    \item \(\)
    \item \(\)
    \item \(\)
    \item \(\)

    \end{enumerate}
\item 
    \begin{enumerate}[label={\Alph*.}]
    \item \(\)
    \item \(\)
    \item \(\)
    \item \(\)

    \end{enumerate}
\item 
    \begin{enumerate}[label={\Alph*.}]
    \item \(\)
    \item \(\)
    \item \(\)
    \item \(\)

    \end{enumerate}
\item 
    \begin{enumerate}[label={\Alph*.}]
    \item \(\)
    \item \(\)
    \item \(\)
    \item \(\)

    \end{enumerate}
\item 
    \begin{enumerate}[label={\Alph*.}]
    \item \(\)
    \item \(\)
    \item \(\)
    \item \(\)

    \end{enumerate}
\item 
    \begin{enumerate}[label={\Alph*.}]
    \item \(\)
    \item \(\)
    \item \(\)
    \item \(\)

    \end{enumerate}
\item 
    \begin{enumerate}[label={\Alph*.}]
    \item \(\)
    \item \(\)
    \item \(\)
    \item \(\)

    \end{enumerate}
\item 
    \begin{enumerate}[label={\Alph*.}]
    \item \(\)
    \item \(\)
    \item \(\)
    \item \(\)

    \end{enumerate}
\item 
    \begin{enumerate}[label={\Alph*.}]
    \item \(\)
    \item \(\)
    \item \(\)
    \item \(\)

    \end{enumerate}
\item 
    \begin{enumerate}[label={\Alph*.}]
    \item \(\)
    \item \(\)
    \item \(\)
    \item \(\)

    \end{enumerate}
\item 
    \begin{enumerate}[label={\Alph*.}]
    \item \(\)
    \item \(\)
    \item \(\)
    \item \(\)

    \end{enumerate}
\item 
    \begin{enumerate}[label={\Alph*.}]
    \item \(\)
    \item \(\)
    \item \(\)
    \item \(\)

    \end{enumerate}
\item 
    \begin{enumerate}[label={\Alph*.}]
    \item \(\)
    \item \(\)
    \item \(\)
    \item \(\)

    \end{enumerate}
\item 
    \begin{enumerate}[label={\Alph*.}]
    \item \(\)
    \item \(\)
    \item \(\)
    \item \(\)

    \end{enumerate}
\item 
    \begin{enumerate}[label={\Alph*.}]
    \item \(\)
    \item \(\)
    \item \(\)
    \item \(\)

    \end{enumerate}
\item 
    \begin{enumerate}[label={\Alph*.}]
    \item \(\)
    \item \(\)
    \item \(\)
    \item \(\)

    \end{enumerate}
\item 
    \begin{enumerate}[label={\Alph*.}]
    \item \(\)
    \item \(\)
    \item \(\)
    \item \(\)

    \end{enumerate}
\item 
    \begin{enumerate}[label={\Alph*.}]
    \item \(\)
    \item \(\)
    \item \(\)
    \item \(\)

    \end{enumerate}
\item 
    \begin{enumerate}[label={\Alph*.}]
    \item \(\)
    \item \(\)
    \item \(\)
    \item \(\)

    \end{enumerate}
\item 
    \begin{enumerate}[label={\Alph*.}]
    \item \(\)
    \item \(\)
    \item \(\)
    \item \(\)

    \end{enumerate}
\item 
    \begin{enumerate}[label={\Alph*.}]
    \item \(\)
    \item \(\)
    \item \(\)
    \item \(\)

    \end{enumerate}
\item 
    \begin{enumerate}[label={\Alph*.}]
    \item \(\)
    \item \(\)
    \item \(\)
    \item \(\)

    \end{enumerate}
\item 
    \begin{enumerate}[label={\Alph*.}]
    \item \(\)
    \item \(\)
    \item \(\)
    \item \(\)

    \end{enumerate}
\item 
    \begin{enumerate}[label={\Alph*.}]
    \item \(\)
    \item \(\)
    \item \(\)
    \item \(\)
    \end{enumerate}
\item 
    \begin{enumerate}[label={\Alph*.}]
    \item \(\)
    \item \(\)
    \item \(\)
    \item \(\)
    \end{enumerate}
\item 
    \begin{enumerate}[label={\Alph*.}]
    \item \(\)
    \item \(\)
    \item \(\)
    \item \(\)
    \end{enumerate}
\end{enumerate}
\end{multicols}

\subsection{Solution}
\begin{enumerate}[label={\arabic*.}]
    \item 
    \item 
    \item
    \item
    \item
    \item 
    \item
    \item
    \item 
    \item 
    \item 
    \item 
    \item
    \item
    \item
    \item 
    \item
    \item
    \item 
    \item 
    \item 
    \item 
    \item
    \item
    \item
    \item 
    \item
    \item
    \item 
    \item 
    \item 
    \item 
    \item
    \item
    \item
    \item 
    \item
    \item
    \item 
    \item 
    \item 
    \item 
    \item
    \item
    \item
    \item 
    \item
    \item
    \item 
    \item 
    \item 
    \item 
    \item
    \item
    \item
    \item 
    \item
    \item
    \item 
    \item 
    \item 
    \item 
    \item
    \item
    \item
    \item 
    \item
    \item
    \item 
    \item 
    \item 
    \item 
    \item
    \item
    \item
    \item 
    \item
    \item
    \item 
    \item 
    \item 
    \item 
    \item
    \item
    \item
    \item 
    \item
    \item
    \item 
    \item 
    \item 
    \item 
    \item
    \item
    \item
    \item 
    \item
    \item
    \item 
    \item 
\end{enumerate}
\chapter{Algebra}
\section{Factorization and Remainder Theorem}
\subsection{Questions}
\begin{multicols}{2}
\begin{enumerate}[label={\arabic*.}]
\item If the function \(f\) is defined by \(f(x+2)\) = \(2{x}^{2} + 7x -5\), find \(f(-1)\)
	\begin{enumerate}[label={\Alph*.}]
	\item \(-8\)
	\item \(4\)
	\item \(10\)
	\item \(-10\)
	\end{enumerate}
\item Factorize \(a^2x - b^2y - b^2x + a^2y\)
	\begin{enumerate}[label={\Alph*.}]
	\item \((y-x)(a-b)(a+b)\)
	\item \((a-b)(x+y)\)
	\item \((x-y)(a-b)\)
	\item \((x+y)(a-b)(a+b)\)
	\end{enumerate}
\item if \(x - 1\) and \(x + 1\) are both factors of the equation: \({x}^{3} + p{x}^{2} + qx + 6 \) = 0, 
evaluate p and q
	\begin{enumerate}[label={\Alph*.}]
	\item \(-6, -1\)
	\item \(1, -1\)
	\item \(6, 1\)
	\item \(6, -6\)
	\end{enumerate}
\item if \(f(x)\) = \(\dfrac{1}{x - 1} + \dfrac{x - 1}{{x}^{2} - 1}\), find \(f(1 - x)\)
	\begin{enumerate}[label={\Alph*.}]
	\item \(\dfrac{1}{x} + \dfrac{1}{x - 2}\)
	\item \( -\dfrac{1}{x} - \dfrac{1}{x - 2}\)
	\item \(x + \dfrac{1}{2x - 1}\)
	\item \(\dfrac{1}{x} + \dfrac{1}{{x}^{2} - 1}\)
	\item \(-\dfrac{1}{x} - \dfrac{1}{2x - 1}\)
	\end{enumerate}
\item Multiply (\(x + 3y + 5\)) by (\(2{x}^{2} + 5y + 2\))
	\begin{enumerate}[label={\Alph*.}]
	\item \(2{x}^{3} + 3{x}^{2}y + 10xy^2 + 13y + 10{x}^{2} + 2x + 10\)
	\item \(2{x}^{3} + 2{x}^{2}y + 10xy + 10y^2 +  31y + 5{x}^{2} + 2x + 10\)
	\item \(2{x}^{3} + 6{x}^{2}y + 5xy + 10y^2 + 13y + 5{x}^{2} + 2x + 10\)
	\item \(2{x}^{3} + 6{x}^{2}y + 5xy + 15y^2 + 31y + 5{x}^{2} + 2x + 10\)
	\item \(2{x}^{3} + 3{x}^{2}y + 5xy + 10y^2 + 13y + 5{x}^{2} + 2x + 10\)
	\end{enumerate}
\item If \(k{x}^{3} + 	10{x}^{2} + lx - 3\) is divisible by (\(x - 1\)) and if when it is divided by (\(x + 2\)) the remainder is 27,
find the constant \(k\) and \(l\).
	\begin{enumerate}[label={\Alph*.}]
	\item \(\)
	\item \(\)
	\item \(\)
	\item \(\)
	\end{enumerate}
\item Factorize \(3{x}^{3} + 4{x}^{2} - 13x + 6\) completely, given that \(x - 1\) is a factor
	\begin{enumerate}[label={\Alph*.}]
	\item \((x-1)(x-3)(x+2)\)
	\item \((x-1)(x-3)(3x+2)\)
	\item \((x-1)(x-2)(x+3)\)
	\item \((x-1)(x+3)(3x+2)\)
	\item \((x-1)(x+3)(3x-2)\)
	\end{enumerate}
\item Multiply \({x}^{2}+x+1\) by \({x}^{2}-x+1\)
	\begin{enumerate}[label={\Alph*.}]
	\item \(x^4+3{x}^{2}+x+1\)
	\item \(x^4+4{x}^{2}-6x+1\)
	\item \(x^4+4{x}^{2}+1\)
	\item \(x^4-{x}^{3}-{x}^{2}+x+1\)
	\item \(x^4-6{x}^{2}-4x+1\)
	\end{enumerate}
\item If \(x=1\) is a root of the equation: \({x}^{3}-2{x}^{2}-5x+6\), find the other roots.
	\begin{enumerate}[label={\Alph*.}]
	\item -3 and 2
	\item 1 and 3
	\item -2 and 2
	\item 3 and -2
	\item -3 and 1
	\end{enumerate}
\item If \(x+2\) and \(x-1\) are factors of the expression: \(l{x}^{3}+2k{x}^{2}+24\), find the values of \(l\) and \(k\)
	\begin{enumerate}[label={\Alph*.}]
	\item \(l=-6, k=-9\)
	\item \(l=-2, k=-1\)
	\item \(l=-2, k=1\)
	\item \(l=0, k=1\)
	\item \(l=6, k=0\)
	\end{enumerate}
\item Factorize completely: \(81a^4-16b^4\)
	\begin{enumerate}[label={\Alph*.}]
	\item \((3a+2b)(2a-3b)(9a^2+4b^2)\)
	\item \((3a-2b)(2a-3b)(9a^2-4b^2)\)
	\item \((3a-2b)(3a-2b)(9a^2+4b^2)\)
	\item \((3a-2b)(2a-3b)(4a^2-9b^2)\)
	\item \((3a-2b)(3a+2b)(9a^2+4b^2)\)
	\end{enumerate}
\item The factor which is common to all three bionomial expressions: \(4a^2-9b^2\), \(8a^2+27b^3\), \((4a+6b)^2\)
	\begin{enumerate}[label={\Alph*.}]
	\item \(4a-6b\)
	\item \(4a+6b\)
	\item \(2a-3b\)
	\item \(2a+3b\)
	\item \(3a-2b\)
	\end{enumerate}
\item If \(x-2\) and \(x+1\) are factors of the expression: \({x}^{3}+p{x}^{2}+qx+1\)
	\begin{enumerate}[label={\Alph*.}]
	\item \(-3\)
	\item \(0\)
	\item \(-\dfrac{17}{3}\)
	\item \(-\dfrac{2}{3}\)
	\item \(3\)
	\end{enumerate}
\item The factors of \(9-({x}^{2}-3x-1)^2\) are. 
	\begin{enumerate}[label={\Alph*.}]
	\item \((x-4)(x-1)(x-1)(x+2)\)
	\item \((x-4)(x+1)(x-2)(x-1)\)
	\item \((x-2)(x+2)(x+1)(x+4)\)
	\item \((x-2)(x+2)(x+1)(x-1)\)
	\item \((x-4)(x-3)(x-2)(x+1)\)
	\end{enumerate}
\item If \(f(x-2) = 4{x}^{2} + x + 7 \) find \(f(1)\)
	\begin{enumerate}[label={\Alph*.}]
	\item \(27\)
	\item \(7\)
	\item \(17\)
	\item \(46\)
	\item \(12\)
	\end{enumerate}
\item If \(g(y) = \dfrac{y - 3}{11} + \dfrac{11}{y^2-9}\) what is \(g(y+3)\)
	\begin{enumerate}[label={\Alph*.}]
	\item \(\dfrac{y}{11} + \dfrac{11}{y(y+5)}\)
	\item \(\dfrac{y+30}{11} + \dfrac{11}{y(y+3)}\)
	\item \(\dfrac{y}{11} + \dfrac{11}{y(y+3)}\)
	\item \(\dfrac{y+3}{11}+\dfrac{11}{y(y-6)}\)
	\end{enumerate}
\item Factorize completely \(3a+125a^3\)
	\begin{enumerate}[label={\Alph*.}]
	\item \((2a+5x)(4+10ax + 25a{x}^{2})\)
	\item \((2a+5{x}^{2})(4+25ax)\)
	\item \(a(2+5x)(4-10x+25a{x}^{2})\)
	\item \(a(2+5x)(4+10ax+25a{x}^{2})\)
	\end{enumerate}
\item Factorize \({x}^{2}+2a+ax+2x\)
	\begin{enumerate}[label={\Alph*.}]
	\item \(({x}^{2}-1)(x+a)\)
	\item \((x+2)(x+a)\)
	\item \((x+2a)(x+3)\)
	\item \((x+2a)(x-1)\)
	\end{enumerate}
\item The graphical method of solving the equation: \({x}^{3}+3{x}^{2}+4x-28 = 0\) is by drawing the graphs of the curves
	\begin{enumerate}[label={\Alph*.}]
	\item \(y = {x}^{3}\) and \(y=3{x}^{2} + x -28\)
	\item \(y={x}^{3}+3{x}^{2}+4x\) and \(y\)
	\item \(y={x}^{3}+3{x}^{2}+4x-28\) and the line \(y=\dfrac{28}{x}\)
	\item \(y={x}^{2}+3x+4\) and \(y=\dfrac{28}{x}\)
	\item \(y={x}^{2}+3x+4\) and line \(y=28x\)
	\end{enumerate}
\item Factorize \((4a+3)^2-(3a-2)^2\)
	\begin{enumerate}[label={\Alph*.}]
	\item \((x+2a)(x-1)\)
	\item \((x+1)(x+2a)\)
	\item \((x+2)(x+a)\)
	\item \(({x}^{2}-1)(x+a)\)
	\end{enumerate}
\item If \({x}^{3} - 12x - 16 - 0\) has \(x-2\) as a solution, then the equation has
	\begin{enumerate}[label={\Alph*.}]
	\item 3 roots all different
	\item \(x-4\) as a solution also
	\item 3 roots all equal
	\item 3 roots with two equal and the third different
	\item only one root
	\end{enumerate}
\item The expression: \({x}^{3}-4{x}^{2}+cx+d\) is such that \(x+1\) is a factor and its values is 1 when x is -2. find \(c\) and \(d\)
	\begin{enumerate}[label={\Alph*.}]
	\item \(c=-4\) and \(d=9\)
	\item \(c=20\) and \(d =9\)
	\item \(c=-20\) and \(d=15\)
	\item \(c=-20\) and \(d=-15\)
	\item \(c=20\) and \(d=-15\)
	\end{enumerate}
\item What factor is common to all the expressions: \(x+1, 2{x}^{2}+x+1\) and \({x}^{2}-1\)
	\begin{enumerate}[label={\Alph*.}]
	\item \(x+1\)
	\item \(1\)
	\item No common factor
	\item \(2x-1\)
	\item \(x\)
	\end{enumerate}
\item Facorize completely: \(({x}^{2}+x)^2-(2x+2)^2\)
	\begin{enumerate}[label={\Alph*.}]
	\item \((x+y)(x+2)(x-2)\)
	\item \((x+1)^2(x+2(x-2))\)
	\item \((x+y)^2(y-2)^2\)
	\item \((x+1)^2(x+2)^2\)
	\end{enumerate}
\item if \(f(x) = 2{x}^{2}+5x+3\), find \(f(x+1)\)
	\begin{enumerate}[label={\Alph*.}]
	\item \(2{x}^{2}-x+10\)
	\item \(2{x}^{2}-x\)
	\item \(4{x}^{2}+3x+12\)
	\item \(4{x}^{2}+3x+2\)
	\end{enumerate}
\item If one factor of \({x}^{3}-8^{-1}\) is \(x-2^{-1}\), the other factor is
	\begin{enumerate}[label={\Alph*.}]
	\item \({x}^{2}+2^{-1}x-4^{-1}\)
	\item \({x}^{2}+2^{-1}x-4^{-1}\)
	\item \({x}^{2}-2^{-1}x-4^{-1}\)
	\item \({x}^{2}+4^{-1}x-2^{-1}\)
	\end{enumerate}
\item Factorize \(9(x+y)^2 - 4(x-y)^2\)
	\begin{enumerate}[label={\Alph*.}]
	\item \((x+y)(5x+y)\)
	\item \((x+y)^2\)
	\item \((5(x+y)^2)\)
	\item \((x+5y)(5x+y)\)
	\end{enumerate}
\item Factorize \(4a^2-12ab-c^2+9b^2\)
	\begin{enumerate}[label={\Alph*.}]
	\item \((2a+3b-c)(2a+3b+c)\)
	\item \(4a(a-3b)+(3b-c)^2\)
	\item \((2a-3b-c)(2a-3b+c)\)
	\item \(4a(a-3b) + (3b+c)^2\)
	\end{enumerate}
\item What are \(k\) and \(l\) respectively if: \(\dfrac{1}{2}(3y-4x)^2 = (8{x}^{2}+kxy+ly^2)\)
	\begin{enumerate}[label={\Alph*.}]
	\item \(12, \dfrac{9}{2}\)
	\item \(-12, \dfrac{9}{2}\)
	\item \(6, 9\)
	\item \(-6, 9\)
	\end{enumerate}
\item if \(f(x-4) = {x}^{2}+2x+3\), find \(f(2)\)
	\begin{enumerate}[label={\Alph*.}]
	\item \(11\)
	\item \(6\)
	\item \(51\)
	\item \(27\)
	\end{enumerate}
\item Factorize completely: \(y^3-4xy+xy^3-4y\)
	\begin{enumerate}[label={\Alph*.}]
	\item \(y(1-x)(y+2)(y-2)\)
	\item \(y(1+x)(y-2)(y-2)\)
	\item \((y+xy)(y+2)(y-2)\)
	\item \((y+xy)(y+2)(y-2)\)
	\end{enumerate}
\item If \(g(x) = {x}^{2}+3x+4, find g(x+1) -g(x)\)
	\begin{enumerate}[label={\Alph*.}]
	\item \(2(x+2)\)
	\item \((x+2)\)
	\item \((2x+1)\)
	\item \({x}^{2}+4\)
	\end{enumerate}
\item Factorize: \(m^3-2m^2-m+2\)
	\begin{enumerate}[label={\Alph*.}]
	\item \((m+1)(m+1)(m+2)\)
	\item \((m^2+1)(m-2)\)
	\item \((m^2+2)(m-1)\)
	\item \((m-2)(m+1)(m-1)\)
	\end{enumerate}
\item Which of the following is a factor of \(rs+tr-pt-ps\)
	\begin{enumerate}[label={\Alph*.}]
	\item \((p-s)\)
	\item \((r-p)\)
	\item \((s-p)\)
	\item \((r+p)\)
	\end{enumerate}
\item If \(x+1\) is a factor of: \({x}^{3}+3{x}^{2}+kx+4\), find the value of k
	\begin{enumerate}[label={\Alph*.}]
	\item \(-6\)
	\item \(6\)
	\item \(8\)
	\item \(-8\)
	\end{enumerate}
\item Factorize: \(9p^2-q^2+6pr-9r^2\)
	\begin{enumerate}[label={\Alph*.}]
	\item \((3p-3q+r)(3p-q)\)
	\item \((6p-3q-3r)(3p-q-4r)\)
	\item \((3p-3q+r)(3p-q-3r)\)
	\item \(\)
	\end{enumerate}
\item
	\begin{enumerate}[label={\Alph*.}]
	\item \(\)
	\item \(\)
	\item \(\)
	\item \(\)
	\end{enumerate}
\item
	\begin{enumerate}[label={\Alph*.}]
	\item \(\)
	\item \(\)
	\item \(\)
	\item \(\)
	\end{enumerate}
\item
	\begin{enumerate}[label={\Alph*.}]
	\item \(\)
	\item \(\)
	\item \(\)
	\item \(\)
	\end{enumerate}
\item
	\begin{enumerate}[label={\Alph*.}]
	\item \(\)
	\item \(\)
	\item \(\)
	\item \(\)
	\end{enumerate}
\item
	\begin{enumerate}[label={\Alph*.}]
	\item \(\)
	\item \(\)
	\item \(\)
	\item \(\)
	\end{enumerate}
\item
	\begin{enumerate}[label={\Alph*.}]
	\item \(\)
	\item \(\)
	\item \(\)
	\item \(\)
	\end{enumerate}
\item
	\begin{enumerate}[label={\Alph*.}]
	\item \(\)
	\item \(\)
	\item \(\)
	\item \(\)
	\end{enumerate}
\item
	\begin{enumerate}[label={\Alph*.}]
	\item \(\)
	\item \(\)
	\item \(\)
	\item \(\)
	\end{enumerate}
\item
	\begin{enumerate}[label={\Alph*.}]
	\item \(\)
	\item \(\)
	\item \(\)
	\item \(\)
	\end{enumerate}
\item
	\begin{enumerate}[label={\Alph*.}]
	\item \(\)
	\item \(\)
	\item \(\)
	\item \(\)
	\end{enumerate}
\item
	\begin{enumerate}[label={\Alph*.}]
	\item \(\)
	\item \(\)
	\item \(\)
	\item \(\)
	\end{enumerate}
\item
	\begin{enumerate}[label={\Alph*.}]
	\item \(\)
	\item \(\)
	\item \(\)
	\item \(\)
	\end{enumerate}
\item
	\begin{enumerate}[label={\Alph*.}]
	\item \(\)
	\item \(\)
	\item \(\)
	\item \(\)
	\end{enumerate}
\item
	\begin{enumerate}[label={\Alph*.}]
	\item \(\)
	\item \(\)
	\item \(\)
	\item \(\)
	\end{enumerate}
\item
	\begin{enumerate}[label={\Alph*.}]
	\item \(\)
	\item \(\)
	\item \(\)
	\item \(\)
	\end{enumerate}
\item
	\begin{enumerate}[label={\Alph*.}]
	\item \(\)
	\item \(\)
	\item \(\)
	\item \(\)
	\end{enumerate}
\item
	\begin{enumerate}[label={\Alph*.}]
	\item \(\)
	\item \(\)
	\item \(\)
	\item \(\)
	\end{enumerate}
\item
	\begin{enumerate}[label={\Alph*.}]
	\item \(\)
	\item \(\)
	\item \(\)
	\item \(\)
	\end{enumerate}
\item
	\begin{enumerate}[label={\Alph*.}]
	\item \(\)
	\item \(\)
	\item \(\)
	\item \(\)
	\end{enumerate}
\item
	\begin{enumerate}[label={\Alph*.}]
	\item \(\)
	\item \(\)
	\item \(\)
	\item \(\)
	\end{enumerate}
\item
	\begin{enumerate}[label={\Alph*.}]
	\item \(\)
	\item \(\)
	\item \(\)
	\item \(\)
	\end{enumerate}
\item
	\begin{enumerate}[label={\Alph*.}]
	\item \(\)
	\item \(\)
	\item \(\)
	\item \(\)
	\end{enumerate}
\item
	\begin{enumerate}[label={\Alph*.}]
	\item \(\)
	\item \(\)
	\item \(\)
	\item \(\)
	\end{enumerate}
\item
	\begin{enumerate}[label={\Alph*.}]
	\item \(\)
	\item \(\)
	\item \(\)
	\item \(\)
	\end{enumerate}
\item
	\begin{enumerate}[label={\Alph*.}]
	\item \(\)
	\item \(\)
	\item \(\)
	\item \(\)
	\end{enumerate}
\item
	\begin{enumerate}[label={\Alph*.}]
	\item \(\)
	\item \(\)
	\item \(\)
	\item \(\)
	\end{enumerate}
\item
	\begin{enumerate}[label={\Alph*.}]
	\item \(\)
	\item \(\)
	\item \(\)
	\item \(\)
	\end{enumerate}
\item
	\begin{enumerate}[label={\Alph*.}]
	\item \(\)
	\item \(\)
	\item \(\)
	\item \(\)
	\end{enumerate}
\item
	\begin{enumerate}[label={\Alph*.}]
	\item \(\)
	\item \(\)
	\item \(\)
	\item \(\)
	\end{enumerate}
\item
	\begin{enumerate}[label={\Alph*.}]
	\item \(\)
	\item \(\)
	\item \(\)
	\item \(\)
	\end{enumerate}
\item
	\begin{enumerate}[label={\Alph*.}]
	\item \(\)
	\item \(\)
	\item \(\)
	\item \(\)
	\end{enumerate}
\item
	\begin{enumerate}[label={\Alph*.}]
	\item \(\)
	\item \(\)
	\item \(\)
	\item \(\)
	\end{enumerate}
\item
	\begin{enumerate}[label={\Alph*.}]
	\item \(\)
	\item \(\)
	\item \(\)
	\item \(\)
	\end{enumerate}
\item
	\begin{enumerate}[label={\Alph*.}]
	\item \(\)
	\item \(\)
	\item \(\)
	\item \(\)
	\end{enumerate}
\item
	\begin{enumerate}[label={\Alph*.}]
	\item \(\)
	\item \(\)
	\item \(\)
	\item \(\)
	\end{enumerate}
\item
	\begin{enumerate}[label={\Alph*.}]
	\item \(\)
	\item \(\)
	\item \(\)
	\item \(\)
	\end{enumerate}
\item
	\begin{enumerate}[label={\Alph*.}]
	\item \(\)
	\item \(\)
	\item \(\)
	\item \(\)
	\end{enumerate}
\item
	\begin{enumerate}[label={\Alph*.}]
	\item \(\)
	\item \(\)
	\item \(\)
	\item \(\)
	\end{enumerate}


\item
	\begin{enumerate}[label={\Alph*.}]
	\item \(\)
	\item \(\)
	\item \(\)
	\item \(\)
	\end{enumerate}
\item
	\begin{enumerate}[label={\Alph*.}]
	\item \(\)
	\item \(\)
	\item \(\)
	\item \(\)
	\end{enumerate}
\item
	\begin{enumerate}[label={\Alph*.}]
	\item \(\)
	\item \(\)
	\item \(\)
	\item \(\)
	\end{enumerate}
\item
	\begin{enumerate}[label={\Alph*.}]
	\item \(\)
	\item \(\)
	\item \(\)
	\item \(\)
	\end{enumerate}
\item
	\begin{enumerate}[label={\Alph*.}]
	\item \(\)
	\item \(\)
	\item \(\)
	\item \(\)
	\end{enumerate}
\item
	\begin{enumerate}[label={\Alph*.}]
	\item \(\)
	\item \(\)
	\item \(\)
	\item \(\)
	\end{enumerate}
\item
	\begin{enumerate}[label={\Alph*.}]
	\item \(\)
	\item \(\)
	\item \(\)
	\item \(\)
	\end{enumerate}
\item
	\begin{enumerate}[label={\Alph*.}]
	\item \(\)
	\item \(\)
	\item \(\)
	\item \(\)
	\end{enumerate}
\item
	\begin{enumerate}[label={\Alph*.}]
	\item \(\)
	\item \(\)
	\item \(\)
	\item \(\)
	\end{enumerate}
\item
	\begin{enumerate}[label={\Alph*.}]
	\item \(\)
	\item \(\)
	\item \(\)
	\item \(\)
	\end{enumerate}
\item
	\begin{enumerate}[label={\Alph*.}]
	\item \(\)
	\item \(\)
	\item \(\)
	\item \(\)
	\end{enumerate}
\item
	\begin{enumerate}[label={\Alph*.}]
	\item \(\)
	\item \(\)
	\item \(\)
	\item \(\)
	\end{enumerate}
\item
	\begin{enumerate}[label={\Alph*.}]
	\item \(\)
	\item \(\)
	\item \(\)
	\item \(\)
	\end{enumerate}
\item
	\begin{enumerate}[label={\Alph*.}]
	\item \(\)
	\item \(\)
	\item \(\)
	\item \(\)
	\end{enumerate}
\item
	\begin{enumerate}[label={\Alph*.}]
	\item \(\)
	\item \(\)
	\item \(\)
	\item \(\)
	\end{enumerate}
\item
	\begin{enumerate}[label={\Alph*.}]
	\item \(\)
	\item \(\)
	\item \(\)
	\item \(\)
	\end{enumerate}
\item
	\begin{enumerate}[label={\Alph*.}]
	\item \(\)
	\item \(\)
	\item \(\)
	\item \(\)
	\end{enumerate}
\item
	\begin{enumerate}[label={\Alph*.}]
	\item \(\)
	\item \(\)
	\item \(\)
	\item \(\)
	\end{enumerate}
\item
	\begin{enumerate}[label={\Alph*.}]
	\item \(\)
	\item \(\)
	\item \(\)
	\item \(\)
	\end{enumerate}
\item
	\begin{enumerate}[label={\Alph*.}]
	\item \(\)
	\item \(\)
	\item \(\)
	\item \(\)
	\end{enumerate}
\item
	\begin{enumerate}[label={\Alph*.}]
	\item \(\)
	\item \(\)
	\item \(\)
	\item \(\)
	\end{enumerate}
\item
	\begin{enumerate}[label={\Alph*.}]
	\item \(\)
	\item \(\)
	\item \(\)
	\item \(\)
	\end{enumerate}
\item
	\begin{enumerate}[label={\Alph*.}]
	\item \(\)
	\item \(\)
	\item \(\)
	\item \(\)
	\end{enumerate}
\item
	\begin{enumerate}[label={\Alph*.}]
	\item \(\)
	\item \(\)
	\item \(\)
	\item \(\)
	\end{enumerate}
\item
	\begin{enumerate}[label={\Alph*.}]
	\item \(\)
	\item \(\)
	\item \(\)
	\item \(\)
	\end{enumerate}
\end{enumerate}
\end{multicols}
\subsection{Solution}
\begin{enumerate}[label={\arabic*.}]
    \item 
    \item 
    \item
    \item
    \item
    \item 
    \item
    \item
    \item 
    \item 
    \item 
    \item 
    \item
    \item
    \item
    \item 
    \item
    \item
    \item 
    \item 
    \item 
    \item 
    \item
    \item
    \item
    \item 
    \item
    \item
    \item 
    \item 
    \item 
    \item 
    \item
    \item
    \item
    \item 
    \item
    \item
    \item 
    \item 
    \item 
    \item 
    \item
    \item
    \item
    \item 
    \item
    \item
    \item 
    \item 
    \item 
    \item 
    \item
    \item
    \item
    \item 
    \item
    \item
    \item 
    \item 
    \item 
    \item 
    \item
    \item
    \item
    \item 
    \item
    \item
    \item 
    \item 
    \item 
    \item 
    \item
    \item
    \item
    \item 
    \item
    \item
    \item 
    \item 
    \item 
    \item 
    \item
    \item
    \item
    \item 
    \item
    \item
    \item 
    \item 
    \item 
    \item 
    \item
    \item
    \item
    \item 
    \item
    \item
    \item 
    \item 
\end{enumerate}
\section{Indices and Standard Form}
\subsection{Questions}
\begin{multicols}{2}
\begin{enumerate}[label={\arabic*.}] 
\item
	\begin{enumerate}[label={\Alph*.}]
	\item \(\)
	\item \(\)
	\item \(\)
	\item \(\)
	\end{enumerate}
\item \(\dfrac{1.28 * {10}^4 }{6.4 *{10}^2}\) equals
	\begin{enumerate}[label={\Alph*.}]
	\item \(\)
	\item \(\)
	\item \(\)
	\item \(\)
	\end{enumerate}
\item
	\begin{enumerate}[label={\Alph*.}]
	\item \(\)
	\item \(\)
	\item \(\)
	\item \(\)
	\end{enumerate}
\item
	\begin{enumerate}[label={\Alph*.}]
	\item \(\)
	\item \(\)
	\item \(\)
	\item \(\)
	\end{enumerate}
\item
	\begin{enumerate}[label={\Alph*.}]
	\item \(\)
	\item \(\)
	\item \(\)
	\item \(\)
	\end{enumerate}
\item
	\begin{enumerate}[label={\Alph*.}]
	\item \(\)
	\item \(\)
	\item \(\)
	\item \(\)
	\end{enumerate}
\item
	\begin{enumerate}[label={\Alph*.}]
	\item \(\)
	\item \(\)
	\item \(\)
	\item \(\)
	\end{enumerate}
\item
	\begin{enumerate}[label={\Alph*.}]
	\item \(\)
	\item \(\)
	\item \(\)
	\item \(\)
	\end{enumerate}
\item
	\begin{enumerate}[label={\Alph*.}]
	\item \(\)
	\item \(\)
	\item \(\)
	\item \(\)
	\end{enumerate}
\item
	\begin{enumerate}[label={\Alph*.}]
	\item \(\)
	\item \(\)
	\item \(\)
	\item \(\)
	\end{enumerate}
\item
	\begin{enumerate}[label={\Alph*.}]
	\item \(\)
	\item \(\)
	\item \(\)
	\item \(\)
	\end{enumerate}


\item
	\begin{enumerate}[label={\Alph*.}]
	\item \(\)
	\item \(\)
	\item \(\)
	\item \(\)
	\end{enumerate}
\item
	\begin{enumerate}[label={\Alph*.}]
	\item \(\)
	\item \(\)
	\item \(\)
	\item \(\)
	\end{enumerate}
\item
	\begin{enumerate}[label={\Alph*.}]
	\item \(\)
	\item \(\)
	\item \(\)
	\item \(\)
	\end{enumerate}
\item
	\begin{enumerate}[label={\Alph*.}]
	\item \(\)
	\item \(\)
	\item \(\)
	\item \(\)
	\end{enumerate}
\item
	\begin{enumerate}[label={\Alph*.}]
	\item \(\)
	\item \(\)
	\item \(\)
	\item \(\)
	\end{enumerate}
\item 
	\begin{enumerate}[label={\Alph*.}]
	\item \(\)
	\item \(\)
	\item \(\)
	\item \(\)
	\end{enumerate}
\item
	\begin{enumerate}[label={\Alph*.}]
	\item \(\)
	\item \(\)
	\item \(\)
	\item \(\)
	\end{enumerate}
\item
	\begin{enumerate}[label={\Alph*.}]
	\item \(\)
	\item \(\)
	\item \(\)
	\item \(\)
	\end{enumerate}
\item
	\begin{enumerate}[label={\Alph*.}]
	\item \(\)
	\item \(\)
	\item \(\)
	\item \(\)
	\end{enumerate}
\item
	\begin{enumerate}[label={\Alph*.}]
	\item \(\)
	\item \(\)
	\item \(\)
	\item \(\)
	\end{enumerate}
\item
	\begin{enumerate}[label={\Alph*.}]
	\item \(\)
	\item \(\)
	\item \(\)
	\item \(\)
	\end{enumerate}
\item
	\begin{enumerate}[label={\Alph*.}]
	\item \(\)
	\item \(\)
	\item \(\)
	\item \(\)
	\end{enumerate}
\item
	\begin{enumerate}[label={\Alph*.}]
	\item \(\)
	\item \(\)
	\item \(\)
	\item \(\)
	\end{enumerate}
\item
	\begin{enumerate}[label={\Alph*.}]
	\item \(\)
	\item \(\)
	\item \(\)
	\item \(\)
	\end{enumerate}
\item
	\begin{enumerate}[label={\Alph*.}]
	\item \(\)
	\item \(\)
	\item \(\)
	\item \(\)
	\end{enumerate}
\item
	\begin{enumerate}[label={\Alph*.}]
	\item \(\)
	\item \(\)
	\item \(\)
	\item \(\)
	\end{enumerate}
\item
	\begin{enumerate}[label={\Alph*.}]
	\item \(\)
	\item \(\)
	\item \(\)
	\item \(\)
	\end{enumerate}
\item
	\begin{enumerate}[label={\Alph*.}]
	\item \(\)
	\item \(\)
	\item \(\)
	\item \(\)
	\end{enumerate}
\item
	\begin{enumerate}[label={\Alph*.}]
	\item \(\)
	\item \(\)
	\item \(\)
	\item \(\)
	\end{enumerate}
\item
	\begin{enumerate}[label={\Alph*.}]
	\item \(\)
	\item \(\)
	\item \(\)
	\item \(\)
	\end{enumerate}
\item
	\begin{enumerate}[label={\Alph*.}]
	\item \(\)
	\item \(\)
	\item \(\)
	\item \(\)
	\end{enumerate}
\item
	\begin{enumerate}[label={\Alph*.}]
	\item \(\)
	\item \(\)
	\item \(\)
	\item \(\)
	\end{enumerate}
\item
	\begin{enumerate}[label={\Alph*.}]
	\item \(\)
	\item \(\)
	\item \(\)
	\item \(\)
	\end{enumerate}
\item
	\begin{enumerate}[label={\Alph*.}]
	\item \(\)
	\item \(\)
	\item \(\)
	\item \(\)
	\end{enumerate}
\item
	\begin{enumerate}[label={\Alph*.}]
	\item \(\)
	\item \(\)
	\item \(\)
	\item \(\)
	\end{enumerate}
\item
	\begin{enumerate}[label={\Alph*.}]
	\item \(\)
	\item \(\)
	\item \(\)
	\item \(\)
	\end{enumerate}
\item
	\begin{enumerate}[label={\Alph*.}]
	\item \(\)
	\item \(\)
	\item \(\)
	\item \(\)
	\end{enumerate}
\item
	\begin{enumerate}[label={\Alph*.}]
	\item \(\)
	\item \(\)
	\item \(\)
	\item \(\)
	\end{enumerate}
\item
	\begin{enumerate}[label={\Alph*.}]
	\item \(\)
	\item \(\)
	\item \(\)
	\item \(\)
	\end{enumerate}
\item
	\begin{enumerate}[label={\Alph*.}]
	\item \(\)
	\item \(\)
	\item \(\)
	\item \(\)
	\end{enumerate}
\item
	\begin{enumerate}[label={\Alph*.}]
	\item \(\)
	\item \(\)
	\item \(\)
	\item \(\)
	\end{enumerate}
\item
	\begin{enumerate}[label={\Alph*.}]
	\item \(\)
	\item \(\)
	\item \(\)
	\item \(\)
	\end{enumerate}
\item
	\begin{enumerate}[label={\Alph*.}]
	\item \(\)
	\item \(\)
	\item \(\)
	\item \(\)
	\end{enumerate}
\item
	\begin{enumerate}[label={\Alph*.}]
	\item \(\)
	\item \(\)
	\item \(\)
	\item \(\)
	\end{enumerate}
\item
	\begin{enumerate}[label={\Alph*.}]
	\item \(\)
	\item \(\)
	\item \(\)
	\item \(\)
	\end{enumerate}
\item
	\begin{enumerate}[label={\Alph*.}]
	\item \(\)
	\item \(\)
	\item \(\)
	\item \(\)
	\end{enumerate}
\item
	\begin{enumerate}[label={\Alph*.}]
	\item \(\)
	\item \(\)
	\item \(\)
	\item \(\)
	\end{enumerate}
\item
	\begin{enumerate}[label={\Alph*.}]
	\item \(\)
	\item \(\)
	\item \(\)
	\item \(\)
	\end{enumerate}
\item
	\begin{enumerate}[label={\Alph*.}]
	\item \(\)
	\item \(\)
	\item \(\)
	\item \(\)
	\end{enumerate}
\item
	\begin{enumerate}[label={\Alph*.}]
	\item \(\)
	\item \(\)
	\item \(\)
	\item \(\)
	\end{enumerate}
\item
	\begin{enumerate}[label={\Alph*.}]
	\item \(\)
	\item \(\)
	\item \(\)
	\item \(\)
	\end{enumerate}
\item
	\begin{enumerate}[label={\Alph*.}]
	\item \(\)
	\item \(\)
	\item \(\)
	\item \(\)
	\end{enumerate}
\item
	\begin{enumerate}[label={\Alph*.}]
	\item \(\)
	\item \(\)
	\item \(\)
	\item \(\)
	\end{enumerate}
\item
	\begin{enumerate}[label={\Alph*.}]
	\item \(\)
	\item \(\)
	\item \(\)
	\item \(\)
	\end{enumerate}
\item
	\begin{enumerate}[label={\Alph*.}]
	\item \(\)
	\item \(\)
	\item \(\)
	\item \(\)
	\end{enumerate}
\item
	\begin{enumerate}[label={\Alph*.}]
	\item \(\)
	\item \(\)
	\item \(\)
	\item \(\)
	\end{enumerate}
\item
	\begin{enumerate}[label={\Alph*.}]
	\item \(\)
	\item \(\)
	\item \(\)
	\item \(\)
	\end{enumerate}
\item
	\begin{enumerate}[label={\Alph*.}]
	\item \(\)
	\item \(\)
	\item \(\)
	\item \(\)
	\end{enumerate}
\item
	\begin{enumerate}[label={\Alph*.}]
	\item \(\)
	\item \(\)
	\item \(\)
	\item \(\)
	\end{enumerate}
\item
	\begin{enumerate}[label={\Alph*.}]
	\item \(\)
	\item \(\)
	\item \(\)
	\item \(\)
	\end{enumerate}
\item
	\begin{enumerate}[label={\Alph*.}]
	\item \(\)
	\item \(\)
	\item \(\)
	\item \(\)
	\end{enumerate}
\item
	\begin{enumerate}[label={\Alph*.}]
	\item \(\)
	\item \(\)
	\item \(\)
	\item \(\)
	\end{enumerate}
\item
	\begin{enumerate}[label={\Alph*.}]
	\item \(\)
	\item \(\)
	\item \(\)
	\item \(\)
	\end{enumerate}
\item
	\begin{enumerate}[label={\Alph*.}]
	\item \(\)
	\item \(\)
	\item \(\)
	\item \(\)
	\end{enumerate}
\item
	\begin{enumerate}[label={\Alph*.}]
	\item \(\)
	\item \(\)
	\item \(\)
	\item \(\)
	\end{enumerate}
\item
	\begin{enumerate}[label={\Alph*.}]
	\item \(\)
	\item \(\)
	\item \(\)
	\item \(\)
	\end{enumerate}
\item
	\begin{enumerate}[label={\Alph*.}]
	\item \(\)
	\item \(\)
	\item \(\)
	\item \(\)
	\end{enumerate}
\item
	\begin{enumerate}[label={\Alph*.}]
	\item \(\)
	\item \(\)
	\item \(\)
	\item \(\)
	\end{enumerate}
\item
	\begin{enumerate}[label={\Alph*.}]
	\item \(\)
	\item \(\)
	\item \(\)
	\item \(\)
	\end{enumerate}
\item
	\begin{enumerate}[label={\Alph*.}]
	\item \(\)
	\item \(\)
	\item \(\)
	\item \(\)
	\end{enumerate}
\item
	\begin{enumerate}[label={\Alph*.}]
	\item \(\)
	\item \(\)
	\item \(\)
	\item \(\)
	\end{enumerate}
\item
	\begin{enumerate}[label={\Alph*.}]
	\item \(\)
	\item \(\)
	\item \(\)
	\item \(\)
	\end{enumerate}
\item
	\begin{enumerate}[label={\Alph*.}]
	\item \(\)
	\item \(\)
	\item \(\)
	\item \(\)
	\end{enumerate}
\item
	\begin{enumerate}[label={\Alph*.}]
	\item \(\)
	\item \(\)
	\item \(\)
	\item \(\)
	\end{enumerate}


\item
	\begin{enumerate}[label={\Alph*.}]
	\item \(\)
	\item \(\)
	\item \(\)
	\item \(\)
	\end{enumerate}
\item
	\begin{enumerate}[label={\Alph*.}]
	\item \(\)
	\item \(\)
	\item \(\)
	\item \(\)
	\end{enumerate}
\item
	\begin{enumerate}[label={\Alph*.}]
	\item \(\)
	\item \(\)
	\item \(\)
	\item \(\)
	\end{enumerate}
\item
	\begin{enumerate}[label={\Alph*.}]
	\item \(\)
	\item \(\)
	\item \(\)
	\item \(\)
	\end{enumerate}
\item
	\begin{enumerate}[label={\Alph*.}]
	\item \(\)
	\item \(\)
	\item \(\)
	\item \(\)
	\end{enumerate}
\item
	\begin{enumerate}[label={\Alph*.}]
	\item \(\)
	\item \(\)
	\item \(\)
	\item \(\)
	\end{enumerate}
\item
	\begin{enumerate}[label={\Alph*.}]
	\item \(\)
	\item \(\)
	\item \(\)
	\item \(\)
	\end{enumerate}
\item
	\begin{enumerate}[label={\Alph*.}]
	\item \(\)
	\item \(\)
	\item \(\)
	\item \(\)
	\end{enumerate}
\item
	\begin{enumerate}[label={\Alph*.}]
	\item \(\)
	\item \(\)
	\item \(\)
	\item \(\)
	\end{enumerate}
\item
	\begin{enumerate}[label={\Alph*.}]
	\item \(\)
	\item \(\)
	\item \(\)
	\item \(\)
	\end{enumerate}
\item
	\begin{enumerate}[label={\Alph*.}]
	\item \(\)
	\item \(\)
	\item \(\)
	\item \(\)
	\end{enumerate}
\item
	\begin{enumerate}[label={\Alph*.}]
	\item \(\)
	\item \(\)
	\item \(\)
	\item \(\)
	\end{enumerate}
\item
	\begin{enumerate}[label={\Alph*.}]
	\item \(\)
	\item \(\)
	\item \(\)
	\item \(\)
	\end{enumerate}
\item
	\begin{enumerate}[label={\Alph*.}]
	\item \(\)
	\item \(\)
	\item \(\)
	\item \(\)
	\end{enumerate}
\item
	\begin{enumerate}[label={\Alph*.}]
	\item \(\)
	\item \(\)
	\item \(\)
	\item \(\)
	\end{enumerate}
\item
	\begin{enumerate}[label={\Alph*.}]
	\item \(\)
	\item \(\)
	\item \(\)
	\item \(\)
	\end{enumerate}
\item
	\begin{enumerate}[label={\Alph*.}]
	\item \(\)
	\item \(\)
	\item \(\)
	\item \(\)
	\end{enumerate}
\item
	\begin{enumerate}[label={\Alph*.}]
	\item \(\)
	\item \(\)
	\item \(\)
	\item \(\)
	\end{enumerate}
\item
	\begin{enumerate}[label={\Alph*.}]
	\item \(\)
	\item \(\)
	\item \(\)
	\item \(\)
	\end{enumerate}
\item
	\begin{enumerate}[label={\Alph*.}]
	\item \(\)
	\item \(\)
	\item \(\)
	\item \(\)
	\end{enumerate}
\item
	\begin{enumerate}[label={\Alph*.}]
	\item \(\)
	\item \(\)
	\item \(\)
	\item \(\)
	\end{enumerate}
\item
	\begin{enumerate}[label={\Alph*.}]
	\item \(\)
	\item \(\)
	\item \(\)
	\item \(\)
	\end{enumerate}
\item
	\begin{enumerate}[label={\Alph*.}]
	\item \(\)
	\item \(\)
	\item \(\)
	\item \(\)
	\end{enumerate}
\item
	\begin{enumerate}[label={\Alph*.}]
	\item \(\)
	\item \(\)
	\item \(\)
	\item \(\)
	\end{enumerate}
\item
	\begin{enumerate}[label={\Alph*.}]
	\item \(\)
	\item \(\)
	\item \(\)
	\item \(\)
	\end{enumerate}
\end{enumerate}
\end{multicols}
\subsection{Solutions}
\begin{multicols}{2}
\begin{enumerate}[label={\arabic*.}]

    \item \textbf{(B)} First we separate the powers, and notice $2^6 = 64$, so it cancels out \\
    $25^{x-1} = \cancel{64} \left(\cfrac{5^6}{\cancel{2^6}}\right) \rightarrow 25^{x-1} = 5^6$, rewrite 25 as $5^2$ and resolve $(5^2)^{x-1} = 5^{2(x-1)}$
    \begin{align*} 
        \cancel{5}^{2(x-1)} &= \cancel{5}^6 \\
        2x - 2 &= 6 \rightarrow x = 4
    \end{align*}

    \item \textbf{(E)} The least number for which all the numbers can be written is 5:
    \[25^{x-1} = (5^2)^{x-1} = 5^{2(x-1)} = 5^{2x - 2}\]
    \[125^{x+1} = (5^3)^{x+1} = 5^{3(x+1)} = 5^{3x + 3}\]
    \[\cfrac{5^x \times 5^{2x-2}}{5^{3x+3}} = \cfrac{5^{x + 2x - 2}}{5^{3x + 3}} = 5^{3x - 2 - (3x + 3)} = 5^{-5}\]

    \item \textbf{(D)} Express them individually in standard form 
    $$(3.705 \times 10^1) \times (4.2 \times 10^{-3})$$ 
    rearrange and combine powers
    \begin{align*} 
        3.705 \times 4.2  \times 10^{-2} &= 15.561 \times 10^{-2} \\
        & = 1.5561 \times 10^{-1}
    \end{align*}

    \item \textbf{(B)} notice $64 = 2^6$ making it possible to group $64$ and $r^{-6}$, $64r^{-6} = 2^6\cdot r^{-6} = 2^6 \cdot (r^{-1})^6 $ \\
    Using the property: $$a^k \cdot b^k = (ab)^k \Rightarrow 2^6 \cdot (r^{-1})^6 = (2r^{-1})^6$$ \\
    hence, $\sqrt[3]{(64r^{-6})^{\frac{1}{2}}} = \sqrt[3]{((2r^{-1})^6)^{\frac{1}{2}}} = \sqrt[3]{(2r^{-1})^{\frac{6}{2}}}  = \sqrt[3]{(2r^{-1})^3}$ \\
    $$\sqrt[m]{a^{n}} = a^{\frac{m}{n}} \Rightarrow (2r^{-1})^{\frac{3}{3}} = 2r^{-1} = \frac{2}{r} $$

    \item \textbf{(C)} Rewrite $9^y$ as $(3^2)^y = (3^y)^2$ let $3^y = p$
    \begin{align*}
        9^y - 4(3^y) + 3 &\Rightarrow p^2 - 4p + 3 = 0 \\
        & = p^2 -3p - p + 3 \\ 
        &= p(p-3)-1(p-3) \\ 
        &= (p-1)(p-3) \\ 
        & = (3^y -1)(3^y -3) = 0
    \end{align*}
    $$\begin{tabular}{c|c}
        for $3^y - 1 = 0$ &  for $3^y - 3 = 0$ \\
        $\cancel{3}^y = 1 = \cancel{3}^0$ &  $\cancel{3}^y = \cancel{3}^1$ \\
        $\therefore \hspace{10pt} y = 0 $ & $\therefore \hspace{10pt} y = 1$
    \end{tabular} $$

    \item Working on the numerator:
    \begin{align*}
        9^{\frac{1}{3}} \times 27^{\frac{1}{2}} &= 3^{\frac{2}{3}} \times 3^{\frac{3}{2}} = 3^{\frac{2}{3} + \frac{3}{2}}
    \end{align*}
    \item \textbf{(B)} $ 3^{\frac{x}{2}} = 9^{\frac{1}{3}} \Rightarrow \cancel{3}^{\frac{x}{2}} = \cancel{3}^{\frac{2}{3}} \therefore x = \dfrac{4}{3} $
    \item $4^{\frac{1}{2}} = \sqrt{4} = 2 \hspace{15pt} \therefore 2^6 = 64$
    \item \textbf{(D)} If we let $2^{n-1} = k$ then the numerator $$3k - 4k  = -k = -2^{n-1} = -2^n\cdot 2^{-1}  $$We can also resolve the denominator if rewrite $2^{n+1}$ as $2\cdot 2^n$ and let $2^n = p$ 
    $$2\cdot2^n - 2^n = 2p - p = p = 2^n$$
    $\therefore \hspace{15pt} \dfrac{-\cancel{2^n} \cdot 2^{-1}}{\cancel{2^n}} = -2^{-1}$

    \item \textbf{(A)} $27^{\frac{1}{3}} = 3, 8^{\frac{2}{3}} = (2^3)^{\frac{2}{3}} = 4, 16^{\frac{2}{4}} = 4$ 
    $$\frac{27^{\frac{1}{3}} - 8^{\frac{2}{3}}}{16^{\frac{2}{4}} \times 2} = \dfrac{3 - 4}{4 \times 2} = -\dfrac{1}{8}$$
    \item \textbf{(E)} $16^{2x+3} = 4^{2(2x-3)}$ \\
    $\dfrac{4^{x+3}}{4^{2(2x - 3)}} = 1 \Rightarrow \cancel{4}^{x+3 - (4x - 6)} = \cancel{4}^0$ 
    \begin{align*}
    \Rightarrow \hspace{10pt} x + 3 - 4x + 6 &= 0\\
        -3x + 9 &= 0 & \therefore x  =-3 
    \end{align*}

    \item 
    \begin{align*} 
        (0.008)^{-\frac{1}{3}} = (8 \times 10^{-3})^{-\frac{1}{3}} &= 8^{-\frac{1}{3}} \times (10^{-3})^{-\frac{1}{3}} \\
        & = (8^{\frac{1}{3}})^{-1} \times 10 \\
        & = 2^{-1} \times 10  = \dfrac{10}{2}\\
        (0.16)^{-\frac{3}{2}} = (16 \times 10^{-2})^{-\frac{3}{2}} &= 16^{-\frac{3}{2}} \times (10^{-2})^{-\frac{3}{2}} \\
        &= (16^{-\frac{1}{2}})^3 \times 10^3 \\
        & = 4^{-3} \times 10^3 =\cfrac{1000}{4}
    \end{align*}

    \item \textbf{(E)} On the numerator: $3^{n-1} = \cfrac{3^n}{3}$,  can be simplified to
    $$3^n - 3^{n-1}= 3^n - \cfrac{3^n}{3} = 3^n\left(1- \cfrac{1}{3}\right) = 3^n \cdot \cfrac{2}{3}$$
    On the denominator: \(27 = 3^3\)
    \[3^3 \times 3^n - 3^3 \times 3^n \cdot \cfrac{2}{3} = 3^n \times 3^3\left(1-\cfrac{2}{3}\right) = 3^n \times 3^3 \cfrac{1}{3}\] 
    \(= \cfrac{\cancel{3^n} \cdot \frac{2}{3}}{\cancel{3^n} \times 3^3 \times \frac{1}{3}}  = \cfrac{1}{3^{3}} \times \cfrac{\frac{2}{3}}{\frac{1}{3}} = \cfrac{2}{27}\)

    \item \textbf{(B)} Express all in index form \\
    $0.0048 = 48 \times 10^{-4} = 2^4 \times 3 \times 10^{-4}$, $0.81 = 81 \times 10^{-2} = 3^4 \times 10^{-2}$, $0.027 = 27 \times 10^{-3} = 3^3 \times 10^{-3}$ and $0.04 = 4 \times 10^{-2} = 2^2 \times 10^{-2}$ \\
    The expression as it appears can be rewritten as: 
    \begin{align*}
        &= \sqrt{\cfrac{(2^4 \times 3 \times 10^{-4}) \times (3^4 \times 10^{-2}) \times 10^{-7}}{(3^3 \times 10^{-3}) \times (2^2 \times 10^{-2}) \times 10^6 }} \\
        & = \sqrt{\cfrac{2^4 }{2^2}\left( \cfrac{3^4 \times 3}{3^3} \right) \left(\cfrac{10^{-4} \times \cancel{10^{-2}} \times 10^{-7}}{10^{-3} \times \cancel{10^{-2}} \times 10^6}\right) } \\
        & = \sqrt{2^2 \times 3^2 \times 10^{-14} } = 2 \times 3 \times 10^{-7} = 6 \times 10^{-7}
    \end{align*}

    \item \textbf{(C)} \begin{align*} 
        3^{2y} -6(3^y) & = (3^y)^2 - 6(3^y) = 27 
    \end{align*}
    Let $3^y = k$ , $k^2 - 6k = 27 $
    \begin{align*} 
            & \Rightarrow k^2 - 6k -27 = 0 \\
            & = k^2 - 9k + 3k - 27 = 0 \\
            & = k(k-9)+3(k-9) = 0 \\
        &= (k+3)(k-9) = (3^y + 3)(3^y - 9) = 0
    \end{align*}
    \begin{tabular}{c|c}
        for $3^y + 3 = 0$ & for $3^y - 9 = 0$ \\
        $3^y = -3 $& $3^y = 9 = 3^2$ \\
            & $y = 2$
    \end{tabular} \\

    \item \textbf{(A)} Rewrite \( 5^{x+1} = 5^x \cdot 5 \) and factor the expression: 
    \[\Rightarrow \hspace{5pt} 5^x \cdot 5 + 5^x = 5^x (5 + 1) = \cfrac{5^x \times \cancel{6}}{\cancel{6}} = \cfrac{150}{6}\]
    \[5^x = 25 = 5^2 \, \hspace{5pt} \therefore x = 2 \]

    \item \textbf{(E)} Just count the number of zeros you see: \\
    \( 0.0000001 = 10^{-7} \) \(\hspace{5pt}\therefore 2n + 1 = -7, n = -4 \)

    \item 
    \item \textbf{(A)} Rewrite \( 81 \) using \( 3 \): \\ 
    \(\sqrt[3]{81} = 81^{\frac{1}{3}} = (3^4)^{\frac{1}{3}} = 3^{\frac{4}{3}} = 3^x, \, x = \dfrac{4}{3}\)
    \item \textbf{(B)} Simplifying:    
    \[x(x+1)^{-\frac{1}{2}} = \cfrac{x}{(x+1)^{\frac{1}{2}}} = \cfrac{x}{\sqrt{x+1}}\]
    \begin{align*}
    \text{Rationalizing:} \cfrac{x}{\sqrt{x+1}} &= \cfrac{x}{\sqrt{x+1}} \times \cfrac{\sqrt{x+1}}{\sqrt{x+1}} \\
    &= \cfrac{x\sqrt{x+1}}{x+1} = \cfrac{x(x+1)^{\frac{1}{2}}}{x+1}
    \end{align*} 
    The question can be rewritten as: 
    \begin{align*} 
        &= \cfrac{\cfrac{x(x+1)^{\frac{1}{2}}}{x+1} - (x+1)^{\frac{1}{2}}}{(x+1)^{\frac{1}{2}}} \\
        &= \cfrac{\cancel{(x+1)^{\frac{1}{2}}}\left(\cfrac{x}{x+1} - 1\right)}{\cancel{(x+1)^{\frac{1}{2}}}} \\
        &= \cfrac{x - (x+1)}{x+1} = -\cfrac{1}{x+1}
    \end{align*}

    \item \textbf{(C)} Approaching this problem directly is really slow and time-consuming. Instead, you'll want to let \( 0.8 = a \) and \( 0.5 = b \):
    \[\Rightarrow \cfrac{a \times a \times a - b \times b \times b}{a \times a + a \times b + b \times b} = \cfrac{a^3 - b^3}{a^2 + ab + b^2} = a - b\]
    \[\therefore \hspace{5pt} 0.8 - 0.5 = 0.3 = 3 \times 10^{-1}\]
        \textbf{Workings:} \vspace{-10pt}
        \begin{align*} 
            & a - b\\
            a^2 + ab + b^2 \,\, &\overline{ \hspace{3pt} a^3 - b^3 \hspace{50pt}} \\
            -& \hspace{5pt} \underline{a^3 + a^2b + ab^2} \\ 
            & \hspace{20pt} -a^2b - ab^2 - b^3 \\
            & \hspace{22pt} \underline{-a^2b - ab^2 - b^3} \\
            & \hspace{22pt} \underline{-------}
        \end{align*}
        On the day of the exam, you would avoid the long division if you remember it just like \( x^2 - y^2 = (x+y)(x-y) \). 

    \item \textbf{(E)} Apply same technique as before, let $a = 69842$ and $b = 30158$ \\
    $$\cfrac{a \times a - b \times b }{a - b} = \cfrac{a^2 - b^2}{a-b} = \cfrac{(a-b)(a+b)}{(a-b)} = a+b $$
    $ a + b = 69842 + 30158  = 100000 = 10^5$

    \item \textbf{(D)} \, \,$\cfrac{(3^2)^2 \times (2 \times 3^2)^4}{3^{16}} = 2^4 \cdot \cfrac{3^4 \times 3^8}{3^{16}} = \cfrac{2^4}{3^4} = \cfrac{16}{81}$

    \item \textbf{(C)} $121$ can be expressed in only two forms 
    \begin{tabular}{c|c}
        $m^n = 121 = 121^1$ & $m^n = 121 = 11^2$ \\
        $m = 121, n = 1$& $m = 11, n = 2$ \\
        $(m-1)^{n+1}$ & \\
            $= (121-1)^{1+1} = 120^2$ & $=(11 - 1)^{2+1} = 10^3$ \\    
    \end{tabular}

    \item Rewrite \( a^{-\frac{1}{2}} = \cfrac{1}{a^{\frac{1}{2}}} = \cfrac{1}{\sqrt{a}} \), then rationalize:
    \[\cfrac{1}{\sqrt{a}} \cdot \cfrac{\sqrt{a}}{\sqrt{a}} = \cfrac{\sqrt{a}}{a}\]
    Rationalize:
    \[\cfrac{1 - \sqrt{a}}{1 + \sqrt{a}} \cdot \cfrac{1 - \sqrt{a}}{1 - \sqrt{a}} = \cfrac{(1 - \sqrt{a})^2}{1 - a} = \cfrac{1 - 2\sqrt{a} + a}{1 - a}\]
    Since both expressions share the same base, we can rewrite the expression as:
    \[\cfrac{\left(\sqrt{a} + \cfrac{\sqrt{a}}{a}\right) + (1 - 2\sqrt{a} + a)}{1 - a}\]    

    \item \textbf{(A)}
    \begin{itemize} 
    \item $\left(\cfrac{1}{64}\right)^0 = 1$\\
    \item $64^{-\frac{1}{2}} = (64^{\frac{1}{2}})^{-1} = (\sqrt{64})^{-1}= 8^{-1} = \cfrac{1}{8}$ \\
    \end{itemize}

\item \textbf{(A)} If \( a^{-1} = \cfrac{1}{a} \) and \( \left(\cfrac{1}{b}\right)^{-1} = b \),
    \[\left(\cfrac{a}{b}\right)^{-1} = \left(a \times \cfrac{1}{b}\right)^{-1} = a^{-1} \times \left(\cfrac{1}{b}\right)^{-1} = \cfrac{1}{a} \times b = \cfrac{b}{a}\]
    \[\boxed{\therefore \left(\cfrac{a}{b}\right)^{-1} = \cfrac{b}{a}}\]
    \[\left(\cfrac{x}{y}\right)^{5a - 3} = \left(\cfrac{x}{y}\right)^{-(17 - 3a)}, \, 5a - 3 = 3a - 17 \implies a = -7\]

    \item When this kind of expression is being posed, I've said your best bet isn't to solve it directly. Rather, you should think of a way to cancel out some numbers using a possible identity, with the most basic one being:
    \[a^2 - b^2 = (a + b)(a - b)\]
    With nothing to think of, my eyes took me to \( (0.4 + 0.2)^3 \). Because of the index 3, I decided to look at \( 0.064 - 0.008 \), where \( 64 = 4^3 \) and \( 8 = 2^3 \):
    \begin{align*}
    (0.064 - 0.008) &= (64 \times 10^{-3}) - (8 \times 10^{-3}) \\
    &= (4^3 \times (10^{-1})^3) - (2^3 \times (10^{-1})^3) \\
    &= (4 \times 10^{-1})^3 - (2 \times 10^{-1})^3 \\
    &= 0.4^3 - 0.2^3
    \end{align*}
    So, I just believed every other number contains at least \( 0.4 \) or \( 0.2 \):
    \[0.16 = (0.4)^2, \, 0.08 = 0.4 \times 0.2, \, 0.04 = (0.2)^2\]
    Rewrite the question as:
    \[\cfrac{(0.4^3 - 0.2^3)(0.4^2 - 0.2^2)}{(0.4^2 + 0.4 \times 0.2 + 0.2^2)(0.4 + 0.2)^3}\]
    Let \( 0.4 = a \) and \( 0.2 = b \), then:
    \[\cfrac{(a^3 - b^3)(a^2 - b^2)}{(a^2 + ab + b^2)(a + b)^3}\]
    \[= \cfrac{\cancel{(a^2 + ab + b^2)}(a - b)}{\cancel{a^2 + ab + b^2}} \cdot \cfrac{\cancel{(a + b)}(a - b)}{\cancel{(a + b)^3}} \]
    \[ = \cfrac{(a - b)(a - b)}{(a + b)^2} = \cfrac{(a - b)^2}{(a + b)^2} = \cfrac{a^2 - 2ab + b^2}{a^2 + 2ab + b^2} = \cfrac{0.16 - 2(0.08) + 0.04}{0.16 + 2(0.08) + 0.04} = \cfrac{0.04}{0.36} = \cfrac{1}{9}\]  

    \item \textbf{(C)}
    \begin{align*} 
    \text{ from inside}  \sqrt[n]{x^2} &= x^{\frac{2}{n}} \\
    (\sqrt[n]{x^2})^{\frac{n}{2}} & = (x^{\frac{2}{n}})^{\frac{n}{2}} = x^{\frac{2}{n} \cdot \frac{n}{2}} = x 
    \end{align*}
    then just raise what's inside to the power of $2 \rightarrow x^2$
    \item \textbf{(B)} \begin{align*} 
        3^x - 3^{x-1} &= 3^x -\cfrac{3^x}{3} = 3^x\left(1- \cfrac{1}{3}\right) =3^x \cdot \cfrac{2}{3} =  486 \\ \rightarrow \hspace{5pt}  3^x &=  \cfrac{\cancel{486}\times 3}{2} = 243 \times 3 = 3^5 \times 3 = 3^6, x = 6
    \end{align*}
    \item
    \begin{itemize} 
    \item $5\sqrt{5} = 5^1 \times 5^{\frac{1}{2}} = 5^{\frac{3}{2}}$
    \item $5^3 \divisionsymbol 5^{-\frac{3}{2}} = \cfrac{5^3}{5^{-\frac{3}{2}}} = 5^3 \times \cfrac{1}{5^{-\frac{3}{2}}} = 5$
    \end{itemize}

    \item \textbf{(D)} $(\sqrt{3})^5 = (3^{\frac{1}{2}})^5 = 3^{\frac{5}{2}}, \hspace{5pt} 9^2 =(3^2)^2 = 3^4  $ and $3\sqrt{3} = 3^1 \times 3^{\frac{1}{2}} = 3^{\frac{3}{2}}$ \\
        $$3^{\frac{5}{2}} \times 3^4 = 3^n \times 3^{\frac{3}{2}}\hspace{5pt} \Rightarrow 3^{\frac{5}{2} + 4} = 3^{n + \frac{3}{2}} $$ 
        $ \cfrac{5}{2} + 4 = n + \cfrac{3}{2} \Rightarrow \hspace{5pt} \therefore \hspace{5pt} n = \cfrac{5}{2} - \cfrac{3}{2} + 4 = 1 + 4 = 5  $
    \item \textbf{(C)} $243^{\frac{n}{5}} = (3^5)^{\frac{n}{5}} = 3^n$ and $9^n = (3^2)^n = 3^{2n} = \cfrac{3^n \times 3^{2n + 1}}{3^{2n} \times 3^{n-1}} = \cfrac{3^{n + 2n + 1}}{3^{2n + n -1}} = \cfrac{3^{3n + 1}}{3^{3n - 1}} = 3^{3n + 1 - (3n - 1)} = 3^2 = 9 $

    \item \( k^a k^b k^c = \cancel{k}^{a + b + c} = 1 = \cancel{k}^0, \,\, \therefore \, a + b + c = 0 \)
    \begin{align*} 
        (a + b + c)^3 &= a^3 + 3a^2(b + c) + 3a(b + c)^2 + (b + c)^3 \\
        &= a^3 + 3a^2(b + c) + 3a(b^2 + 2bc + c^2) + b^3 + 3b^2c + 3bc^2 + c^3 \\
        &= a^3 + 3a^2b + 3a^2c + 3ab^2 + 6abc + 3ac^2 + b^3 + 3b^2c + 3bc^2 + c^3 \\
        &= a^3 + b^3 + c^3 + (3a^2c + 3abc + 3ac^2) + (3a^2b + 3ab^2 + 3abc) + (3b^2c + 3bc^2) \\
        &= a^3 + b^3 + c^3 + 3ac(a + b + c) + 3ab(a + b + c) + 3bc(a + b + c) - 3abc
    \end{align*}
    \vspace{-20pt}
    \begin{align*}
    (a + b + c)^3 - 3ac(a + b + c) &= a^3 + b^3 + c^3 \\
    - 3ab(a + b + c) - 3bc(a + b + c) + 3abc
    \end{align*}
    So, if \( a + b + c = 0 \):
    \[0^3 - 3ac(0) - 3ab(0) - 3bc(0) + 3abc = 3abc\]

    \item \textbf{(C)}
        \begin{itemize} 
        \item \( 81^{3.6} = (3^4)^{3.6} = 3^{4 \times 3.6} = 3^{14.4} \)
        \item \( 9^{2.7} = (3^2)^{2.7} = 3^{2 \times 2.7} = 3^{5.4} \)
        \end{itemize}
        \[\therefore \hspace{10pt} 3^{14.4} \times 3^{5.4} = 3^{14.4 + 5.4} = 3^{19.8}\]
        \begin{itemize} 
        \item \( 81^{4.2} = (3^4)^{4.2} = 3^{16.8} \)
        \end{itemize}
        \[\therefore \hspace{10pt} 3^{16.8} \times 3^{1} = 3^{16.8 + 1} = 3^{17.8}\]
        \[\cfrac{3^{19.8}}{3^{17.8}} = 3^{19.8 - 17.8} = 3^2 = 9\]

    \item
        \begin{align*}
        6^{2n + 1} &= (2 \times 3)^{2n + 1} = 2^{2n + 1} \times 3^{2n + 1} \\
            &= 2^{2n} \cdot 2 \times 3^{2n} \cdot 3
        \end{align*}
        \( 18^n = (9 \times 2)^n = (3^2 \times 2)^n = 3^{2n} \times 2^n \) \\
        \( 12^{2n} = (2 \times 3)^{2n} = 2^{2n} \times 3^{2n} \)
        \[\therefore \cfrac{(\cancel{2^{2n}} \times 2 \times \cancel{3^{2n}} \times 3) \times \cancel{3^{2n}} \times 2^{4n}}{(\cancel{3^{2n}} \times 2^n) \times 2^n \times (\cancel{2^{2n}} \times \cancel{3^{2n}})} = \cfrac{2^{4n + 1} \times 3}{2^{2n}}\]
        \[= 2^{4n - 2n + 1} \times 3 = 2^{2n + 1} \times 3\]

    \item \textbf{(C)}
    \begin{tabular}{c|c}
        $\cancel{2}^{x+y} = \cancel{2}^5$, $x + y = 5$ & $\cancel{3}^{3y - x} = \cancel{3}^3$, $3y - x = 3$
    \end{tabular}
    \begin{align*}
        x + y &= 5 \hspace{.2in} ...(i) \\
        3y - x &= 3 \hspace{.2in} ...(ii) \\
        x + y &= 5 \\
        3y - x &= 3 \\
        4y &= 8 \implies y = 2
    \end{align*}
    If \( x + y = 5 \), and \( y = 2 \), then:    
    \[x + 2 = 5 \implies x = 3 \]
    \item 
    \item
    \item
    \item 
    \item 
    \item 
    \item 
    \item
    \item
    \item
    \item 
    \item
    \item
    \item 
    \item 
    \item 
    \item 
    \item
    \item
    \item
    \item 
    \item
    \item
    \item 
    \item 
    \item 
    \item 
    \item
    \item
    \item
    \item 
    \item
    \item
    \item 
    \item 
    \item 
    \item 
    \item
    \item
    \item
    \item 
    \item
    \item
    \item 
    \item 
    \item 
    \item 
    \item
    \item
    \item
    \item 
    \item
    \item
    \item 
    \item 
    \item 
    \item 
    \item
    \item
    \item
    \item 
    \item
    \item
    \item 
\end{enumerate}
\end{multicols}
\section{Logarithms}
\subsection{Questions}
\begin{multicols}{2}
\begin{enumerate}[label={\arabic*.}] 
\item
	\begin{enumerate}[label={\Alph*.}]
	\item \(\)
	\item \(\)
	\item \(\)
	\item \(\)
	\end{enumerate}
\item 
	\begin{enumerate}[label={\Alph*.}]
	\item \(\)
	\item \(\)
	\item \(\)
	\item \(\)
	\end{enumerate}
\item
	\begin{enumerate}[label={\Alph*.}]
	\item \(\)
	\item \(\)
	\item \(\)
	\item \(\)
	\end{enumerate}
\item
	\begin{enumerate}[label={\Alph*.}]
	\item \(\)
	\item \(\)
	\item \(\)
	\item \(\)
	\end{enumerate}
\item
	\begin{enumerate}[label={\Alph*.}]
	\item \(\)
	\item \(\)
	\item \(\)
	\item \(\)
	\end{enumerate}
\item
	\begin{enumerate}[label={\Alph*.}]
	\item \(\)
	\item \(\)
	\item \(\)
	\item \(\)
	\end{enumerate}
\item
	\begin{enumerate}[label={\Alph*.}]
	\item \(\)
	\item \(\)
	\item \(\)
	\item \(\)
	\end{enumerate}
\item
	\begin{enumerate}[label={\Alph*.}]
	\item \(\)
	\item \(\)
	\item \(\)
	\item \(\)
	\end{enumerate}
\item
	\begin{enumerate}[label={\Alph*.}]
	\item \(\)
	\item \(\)
	\item \(\)
	\item \(\)
	\end{enumerate}
\item
	\begin{enumerate}[label={\Alph*.}]
	\item \(\)
	\item \(\)
	\item \(\)
	\item \(\)
	\end{enumerate}
\item
	\begin{enumerate}[label={\Alph*.}]
	\item \(\)
	\item \(\)
	\item \(\)
	\item \(\)
	\end{enumerate}


\item
	\begin{enumerate}[label={\Alph*.}]
	\item \(\)
	\item \(\)
	\item \(\)
	\item \(\)
	\end{enumerate}
\item
	\begin{enumerate}[label={\Alph*.}]
	\item \(\)
	\item \(\)
	\item \(\)
	\item \(\)
	\end{enumerate}
\item
	\begin{enumerate}[label={\Alph*.}]
	\item \(\)
	\item \(\)
	\item \(\)
	\item \(\)
	\end{enumerate}
\item
	\begin{enumerate}[label={\Alph*.}]
	\item \(\)
	\item \(\)
	\item \(\)
	\item \(\)
	\end{enumerate}
\item
	\begin{enumerate}[label={\Alph*.}]
	\item \(\)
	\item \(\)
	\item \(\)
	\item \(\)
	\end{enumerate}
\item 
	\begin{enumerate}[label={\Alph*.}]
	\item \(\)
	\item \(\)
	\item \(\)
	\item \(\)
	\end{enumerate}
\item
	\begin{enumerate}[label={\Alph*.}]
	\item \(\)
	\item \(\)
	\item \(\)
	\item \(\)
	\end{enumerate}
\item
	\begin{enumerate}[label={\Alph*.}]
	\item \(\)
	\item \(\)
	\item \(\)
	\item \(\)
	\end{enumerate}
\item
	\begin{enumerate}[label={\Alph*.}]
	\item \(\)
	\item \(\)
	\item \(\)
	\item \(\)
	\end{enumerate}
\item
	\begin{enumerate}[label={\Alph*.}]
	\item \(\)
	\item \(\)
	\item \(\)
	\item \(\)
	\end{enumerate}
\item
	\begin{enumerate}[label={\Alph*.}]
	\item \(\)
	\item \(\)
	\item \(\)
	\item \(\)
	\end{enumerate}
\item
	\begin{enumerate}[label={\Alph*.}]
	\item \(\)
	\item \(\)
	\item \(\)
	\item \(\)
	\end{enumerate}
\item
	\begin{enumerate}[label={\Alph*.}]
	\item \(\)
	\item \(\)
	\item \(\)
	\item \(\)
	\end{enumerate}
\item
	\begin{enumerate}[label={\Alph*.}]
	\item \(\)
	\item \(\)
	\item \(\)
	\item \(\)
	\end{enumerate}
\item
	\begin{enumerate}[label={\Alph*.}]
	\item \(\)
	\item \(\)
	\item \(\)
	\item \(\)
	\end{enumerate}
\item
	\begin{enumerate}[label={\Alph*.}]
	\item \(\)
	\item \(\)
	\item \(\)
	\item \(\)
	\end{enumerate}
\item
	\begin{enumerate}[label={\Alph*.}]
	\item \(\)
	\item \(\)
	\item \(\)
	\item \(\)
	\end{enumerate}
\item
	\begin{enumerate}[label={\Alph*.}]
	\item \(\)
	\item \(\)
	\item \(\)
	\item \(\)
	\end{enumerate}
\item
	\begin{enumerate}[label={\Alph*.}]
	\item \(\)
	\item \(\)
	\item \(\)
	\item \(\)
	\end{enumerate}
\item
	\begin{enumerate}[label={\Alph*.}]
	\item \(\)
	\item \(\)
	\item \(\)
	\item \(\)
	\end{enumerate}
\item
	\begin{enumerate}[label={\Alph*.}]
	\item \(\)
	\item \(\)
	\item \(\)
	\item \(\)
	\end{enumerate}
\item
	\begin{enumerate}[label={\Alph*.}]
	\item \(\)
	\item \(\)
	\item \(\)
	\item \(\)
	\end{enumerate}
\item
	\begin{enumerate}[label={\Alph*.}]
	\item \(\)
	\item \(\)
	\item \(\)
	\item \(\)
	\end{enumerate}
\item
	\begin{enumerate}[label={\Alph*.}]
	\item \(\)
	\item \(\)
	\item \(\)
	\item \(\)
	\end{enumerate}
\item
	\begin{enumerate}[label={\Alph*.}]
	\item \(\)
	\item \(\)
	\item \(\)
	\item \(\)
	\end{enumerate}
\item
	\begin{enumerate}[label={\Alph*.}]
	\item \(\)
	\item \(\)
	\item \(\)
	\item \(\)
	\end{enumerate}
\item
	\begin{enumerate}[label={\Alph*.}]
	\item \(\)
	\item \(\)
	\item \(\)
	\item \(\)
	\end{enumerate}
\item
	\begin{enumerate}[label={\Alph*.}]
	\item \(\)
	\item \(\)
	\item \(\)
	\item \(\)
	\end{enumerate}
\item
	\begin{enumerate}[label={\Alph*.}]
	\item \(\)
	\item \(\)
	\item \(\)
	\item \(\)
	\end{enumerate}
\item
	\begin{enumerate}[label={\Alph*.}]
	\item \(\)
	\item \(\)
	\item \(\)
	\item \(\)
	\end{enumerate}
\item
	\begin{enumerate}[label={\Alph*.}]
	\item \(\)
	\item \(\)
	\item \(\)
	\item \(\)
	\end{enumerate}
\item
	\begin{enumerate}[label={\Alph*.}]
	\item \(\)
	\item \(\)
	\item \(\)
	\item \(\)
	\end{enumerate}
\item
	\begin{enumerate}[label={\Alph*.}]
	\item \(\)
	\item \(\)
	\item \(\)
	\item \(\)
	\end{enumerate}
\item
	\begin{enumerate}[label={\Alph*.}]
	\item \(\)
	\item \(\)
	\item \(\)
	\item \(\)
	\end{enumerate}
\item
	\begin{enumerate}[label={\Alph*.}]
	\item \(\)
	\item \(\)
	\item \(\)
	\item \(\)
	\end{enumerate}
\item
	\begin{enumerate}[label={\Alph*.}]
	\item \(\)
	\item \(\)
	\item \(\)
	\item \(\)
	\end{enumerate}
\item
	\begin{enumerate}[label={\Alph*.}]
	\item \(\)
	\item \(\)
	\item \(\)
	\item \(\)
	\end{enumerate}
\item
	\begin{enumerate}[label={\Alph*.}]
	\item \(\)
	\item \(\)
	\item \(\)
	\item \(\)
	\end{enumerate}
\item
	\begin{enumerate}[label={\Alph*.}]
	\item \(\)
	\item \(\)
	\item \(\)
	\item \(\)
	\end{enumerate}
\item
	\begin{enumerate}[label={\Alph*.}]
	\item \(\)
	\item \(\)
	\item \(\)
	\item \(\)
	\end{enumerate}
\item
	\begin{enumerate}[label={\Alph*.}]
	\item \(\)
	\item \(\)
	\item \(\)
	\item \(\)
	\end{enumerate}
\item
	\begin{enumerate}[label={\Alph*.}]
	\item \(\)
	\item \(\)
	\item \(\)
	\item \(\)
	\end{enumerate}
\item
	\begin{enumerate}[label={\Alph*.}]
	\item \(\)
	\item \(\)
	\item \(\)
	\item \(\)
	\end{enumerate}
\item
	\begin{enumerate}[label={\Alph*.}]
	\item \(\)
	\item \(\)
	\item \(\)
	\item \(\)
	\end{enumerate}
\item
	\begin{enumerate}[label={\Alph*.}]
	\item \(\)
	\item \(\)
	\item \(\)
	\item \(\)
	\end{enumerate}
\item
	\begin{enumerate}[label={\Alph*.}]
	\item \(\)
	\item \(\)
	\item \(\)
	\item \(\)
	\end{enumerate}
\item
	\begin{enumerate}[label={\Alph*.}]
	\item \(\)
	\item \(\)
	\item \(\)
	\item \(\)
	\end{enumerate}
\item
	\begin{enumerate}[label={\Alph*.}]
	\item \(\)
	\item \(\)
	\item \(\)
	\item \(\)
	\end{enumerate}
\item
	\begin{enumerate}[label={\Alph*.}]
	\item \(\)
	\item \(\)
	\item \(\)
	\item \(\)
	\end{enumerate}
\item
	\begin{enumerate}[label={\Alph*.}]
	\item \(\)
	\item \(\)
	\item \(\)
	\item \(\)
	\end{enumerate}
\item
	\begin{enumerate}[label={\Alph*.}]
	\item \(\)
	\item \(\)
	\item \(\)
	\item \(\)
	\end{enumerate}
\item
	\begin{enumerate}[label={\Alph*.}]
	\item \(\)
	\item \(\)
	\item \(\)
	\item \(\)
	\end{enumerate}
\item
	\begin{enumerate}[label={\Alph*.}]
	\item \(\)
	\item \(\)
	\item \(\)
	\item \(\)
	\end{enumerate}
\item
	\begin{enumerate}[label={\Alph*.}]
	\item \(\)
	\item \(\)
	\item \(\)
	\item \(\)
	\end{enumerate}
\item
	\begin{enumerate}[label={\Alph*.}]
	\item \(\)
	\item \(\)
	\item \(\)
	\item \(\)
	\end{enumerate}
\item
	\begin{enumerate}[label={\Alph*.}]
	\item \(\)
	\item \(\)
	\item \(\)
	\item \(\)
	\end{enumerate}
\item
	\begin{enumerate}[label={\Alph*.}]
	\item \(\)
	\item \(\)
	\item \(\)
	\item \(\)
	\end{enumerate}
\item
	\begin{enumerate}[label={\Alph*.}]
	\item \(\)
	\item \(\)
	\item \(\)
	\item \(\)
	\end{enumerate}
\item
	\begin{enumerate}[label={\Alph*.}]
	\item \(\)
	\item \(\)
	\item \(\)
	\item \(\)
	\end{enumerate}
\item
	\begin{enumerate}[label={\Alph*.}]
	\item \(\)
	\item \(\)
	\item \(\)
	\item \(\)
	\end{enumerate}
\item
	\begin{enumerate}[label={\Alph*.}]
	\item \(\)
	\item \(\)
	\item \(\)
	\item \(\)
	\end{enumerate}
\item
	\begin{enumerate}[label={\Alph*.}]
	\item \(\)
	\item \(\)
	\item \(\)
	\item \(\)
	\end{enumerate}
\item
	\begin{enumerate}[label={\Alph*.}]
	\item \(\)
	\item \(\)
	\item \(\)
	\item \(\)
	\end{enumerate}
\item
	\begin{enumerate}[label={\Alph*.}]
	\item \(\)
	\item \(\)
	\item \(\)
	\item \(\)
	\end{enumerate}


\item
	\begin{enumerate}[label={\Alph*.}]
	\item \(\)
	\item \(\)
	\item \(\)
	\item \(\)
	\end{enumerate}
\item
	\begin{enumerate}[label={\Alph*.}]
	\item \(\)
	\item \(\)
	\item \(\)
	\item \(\)
	\end{enumerate}
\item
	\begin{enumerate}[label={\Alph*.}]
	\item \(\)
	\item \(\)
	\item \(\)
	\item \(\)
	\end{enumerate}
\item
	\begin{enumerate}[label={\Alph*.}]
	\item \(\)
	\item \(\)
	\item \(\)
	\item \(\)
	\end{enumerate}
\item
	\begin{enumerate}[label={\Alph*.}]
	\item \(\)
	\item \(\)
	\item \(\)
	\item \(\)
	\end{enumerate}
\item
	\begin{enumerate}[label={\Alph*.}]
	\item \(\)
	\item \(\)
	\item \(\)
	\item \(\)
	\end{enumerate}
\item
	\begin{enumerate}[label={\Alph*.}]
	\item \(\)
	\item \(\)
	\item \(\)
	\item \(\)
	\end{enumerate}
\item
	\begin{enumerate}[label={\Alph*.}]
	\item \(\)
	\item \(\)
	\item \(\)
	\item \(\)
	\end{enumerate}
\item
	\begin{enumerate}[label={\Alph*.}]
	\item \(\)
	\item \(\)
	\item \(\)
	\item \(\)
	\end{enumerate}
\item
	\begin{enumerate}[label={\Alph*.}]
	\item \(\)
	\item \(\)
	\item \(\)
	\item \(\)
	\end{enumerate}
\item
	\begin{enumerate}[label={\Alph*.}]
	\item \(\)
	\item \(\)
	\item \(\)
	\item \(\)
	\end{enumerate}
\item
	\begin{enumerate}[label={\Alph*.}]
	\item \(\)
	\item \(\)
	\item \(\)
	\item \(\)
	\end{enumerate}
\item
	\begin{enumerate}[label={\Alph*.}]
	\item \(\)
	\item \(\)
	\item \(\)
	\item \(\)
	\end{enumerate}
\item
	\begin{enumerate}[label={\Alph*.}]
	\item \(\)
	\item \(\)
	\item \(\)
	\item \(\)
	\end{enumerate}
\item
	\begin{enumerate}[label={\Alph*.}]
	\item \(\)
	\item \(\)
	\item \(\)
	\item \(\)
	\end{enumerate}
\item
	\begin{enumerate}[label={\Alph*.}]
	\item \(\)
	\item \(\)
	\item \(\)
	\item \(\)
	\end{enumerate}
\item
	\begin{enumerate}[label={\Alph*.}]
	\item \(\)
	\item \(\)
	\item \(\)
	\item \(\)
	\end{enumerate}
\item
	\begin{enumerate}[label={\Alph*.}]
	\item \(\)
	\item \(\)
	\item \(\)
	\item \(\)
	\end{enumerate}
\item
	\begin{enumerate}[label={\Alph*.}]
	\item \(\)
	\item \(\)
	\item \(\)
	\item \(\)
	\end{enumerate}
\item
	\begin{enumerate}[label={\Alph*.}]
	\item \(\)
	\item \(\)
	\item \(\)
	\item \(\)
	\end{enumerate}
\item
	\begin{enumerate}[label={\Alph*.}]
	\item \(\)
	\item \(\)
	\item \(\)
	\item \(\)
	\end{enumerate}
\item
	\begin{enumerate}[label={\Alph*.}]
	\item \(\)
	\item \(\)
	\item \(\)
	\item \(\)
	\end{enumerate}
\item
	\begin{enumerate}[label={\Alph*.}]
	\item \(\)
	\item \(\)
	\item \(\)
	\item \(\)
	\end{enumerate}
\item
	\begin{enumerate}[label={\Alph*.}]
	\item \(\)
	\item \(\)
	\item \(\)
	\item \(\)
	\end{enumerate}
\item
	\begin{enumerate}[label={\Alph*.}]
	\item \(\)
	\item \(\)
	\item \(\)
	\item \(\)
	\end{enumerate}
\end{enumerate}
\end{multicols}
\subsection{Solutions}
\begin{enumerate}[label={\arabic*.}]
    \item 
    \item 
    \item
    \item
    \item
    \item 
    \item
    \item
    \item 
    \item 
    \item 
    \item 
    \item
    \item
    \item
    \item 
    \item
    \item
    \item 
    \item 
    \item 
    \item 
    \item
    \item
    \item
    \item 
    \item
    \item
    \item 
    \item 
    \item 
    \item 
    \item
    \item
    \item
    \item 
    \item
    \item
    \item 
    \item 
    \item 
    \item 
    \item
    \item
    \item
    \item 
    \item
    \item
    \item 
    \item 
    \item 
    \item 
    \item
    \item
    \item
    \item 
    \item
    \item
    \item 
    \item 
    \item 
    \item 
    \item
    \item
    \item
    \item 
    \item
    \item
    \item 
    \item 
    \item 
    \item 
    \item
    \item
    \item
    \item 
    \item
    \item
    \item 
    \item 
    \item 
    \item 
    \item
    \item
    \item
    \item 
    \item
    \item
    \item 
    \item 
    \item 
    \item 
    \item
    \item
    \item
    \item 
    \item
    \item
    \item 
    \item 
\end{enumerate}
\chapter{Geometry}
\section{}
\subsection{Questions}
\begin{multicols}{2}
\begin{enumerate}[label={\arabic*.}]
\item
	\begin{enumerate}[label={\Alph*.}]
	\item \(\)
	\item \(\)
	\item \(\)
	\item \(\)
	\end{enumerate}
\item
	\begin{enumerate}[label={\Alph*.}]
	\item \(\)
	\item \(\)
	\item \(\)
	\item \(\)
	\end{enumerate}
\item
	\begin{enumerate}[label={\Alph*.}]
	\item \(\)
	\item \(\)
	\item \(\)
	\item \(\)
	\end{enumerate}
\item
	\begin{enumerate}[label={\Alph*.}]
	\item \(\)
	\item \(\)
	\item \(\)
	\item \(\)
	\end{enumerate}
\item
	\begin{enumerate}[label={\Alph*.}]
	\item \(\)
	\item \(\)
	\item \(\)
	\item \(\)
	\end{enumerate}
\item
	\begin{enumerate}[label={\Alph*.}]
	\item \(\)
	\item \(\)
	\item \(\)
	\item \(\)
	\end{enumerate}
\item
	\begin{enumerate}[label={\Alph*.}]
	\item \(\)
	\item \(\)
	\item \(\)
	\item \(\)
	\end{enumerate}
\item
	\begin{enumerate}[label={\Alph*.}]
	\item \(\)
	\item \(\)
	\item \(\)
	\item \(\)
	\end{enumerate}
\item
	\begin{enumerate}[label={\Alph*.}]
	\item \(\)
	\item \(\)
	\item \(\)
	\item \(\)
	\end{enumerate}
\item
	\begin{enumerate}[label={\Alph*.}]
	\item \(\)
	\item \(\)
	\item \(\)
	\item \(\)
	\end{enumerate}
\item
	\begin{enumerate}[label={\Alph*.}]
	\item \(\)
	\item \(\)
	\item \(\)
	\item \(\)
	\end{enumerate}


\item
	\begin{enumerate}[label={\Alph*.}]
	\item \(\)
	\item \(\)
	\item \(\)
	\item \(\)
	\end{enumerate}
\item
	\begin{enumerate}[label={\Alph*.}]
	\item \(\)
	\item \(\)
	\item \(\)
	\item \(\)
	\end{enumerate}
\item
	\begin{enumerate}[label={\Alph*.}]
	\item \(\)
	\item \(\)
	\item \(\)
	\item \(\)
	\end{enumerate}
\item
	\begin{enumerate}[label={\Alph*.}]
	\item \(\)
	\item \(\)
	\item \(\)
	\item \(\)
	\end{enumerate}
\item
	\begin{enumerate}[label={\Alph*.}]
	\item \(\)
	\item \(\)
	\item \(\)
	\item \(\)
	\end{enumerate}
\item
	\begin{enumerate}[label={\Alph*.}]
	\item \(\)
	\item \(\)
	\item \(\)
	\item \(\)
	\end{enumerate}
\item
	\begin{enumerate}[label={\Alph*.}]
	\item \(\)
	\item \(\)
	\item \(\)
	\item \(\)
	\end{enumerate}
\item
	\begin{enumerate}[label={\Alph*.}]
	\item \(\)
	\item \(\)
	\item \(\)
	\item \(\)
	\end{enumerate}
\item
	\begin{enumerate}[label={\Alph*.}]
	\item \(\)
	\item \(\)
	\item \(\)
	\item \(\)
	\end{enumerate}
\item
	\begin{enumerate}[label={\Alph*.}]
	\item \(\)
	\item \(\)
	\item \(\)
	\item \(\)
	\end{enumerate}
\item
	\begin{enumerate}[label={\Alph*.}]
	\item \(\)
	\item \(\)
	\item \(\)
	\item \(\)
	\end{enumerate}
\item
	\begin{enumerate}[label={\Alph*.}]
	\item \(\)
	\item \(\)
	\item \(\)
	\item \(\)
	\end{enumerate}
\item
	\begin{enumerate}[label={\Alph*.}]
	\item \(\)
	\item \(\)
	\item \(\)
	\item \(\)
	\end{enumerate}
\item
	\begin{enumerate}[label={\Alph*.}]
	\item \(\)
	\item \(\)
	\item \(\)
	\item \(\)
	\end{enumerate}
\item
	\begin{enumerate}[label={\Alph*.}]
	\item \(\)
	\item \(\)
	\item \(\)
	\item \(\)
	\end{enumerate}
\item
	\begin{enumerate}[label={\Alph*.}]
	\item \(\)
	\item \(\)
	\item \(\)
	\item \(\)
	\end{enumerate}
\item
	\begin{enumerate}[label={\Alph*.}]
	\item \(\)
	\item \(\)
	\item \(\)
	\item \(\)
	\end{enumerate}
\item
	\begin{enumerate}[label={\Alph*.}]
	\item \(\)
	\item \(\)
	\item \(\)
	\item \(\)
	\end{enumerate}
\item
	\begin{enumerate}[label={\Alph*.}]
	\item \(\)
	\item \(\)
	\item \(\)
	\item \(\)
	\end{enumerate}
\item
	\begin{enumerate}[label={\Alph*.}]
	\item \(\)
	\item \(\)
	\item \(\)
	\item \(\)
	\end{enumerate}
\item
	\begin{enumerate}[label={\Alph*.}]
	\item \(\)
	\item \(\)
	\item \(\)
	\item \(\)
	\end{enumerate}
\item
	\begin{enumerate}[label={\Alph*.}]
	\item \(\)
	\item \(\)
	\item \(\)
	\item \(\)
	\end{enumerate}
\item
	\begin{enumerate}[label={\Alph*.}]
	\item \(\)
	\item \(\)
	\item \(\)
	\item \(\)
	\end{enumerate}
\item
	\begin{enumerate}[label={\Alph*.}]
	\item \(\)
	\item \(\)
	\item \(\)
	\item \(\)
	\end{enumerate}
\item
	\begin{enumerate}[label={\Alph*.}]
	\item \(\)
	\item \(\)
	\item \(\)
	\item \(\)
	\end{enumerate}
\item
	\begin{enumerate}[label={\Alph*.}]
	\item \(\)
	\item \(\)
	\item \(\)
	\item \(\)
	\end{enumerate}
\item
	\begin{enumerate}[label={\Alph*.}]
	\item \(\)
	\item \(\)
	\item \(\)
	\item \(\)
	\end{enumerate}
\item
	\begin{enumerate}[label={\Alph*.}]
	\item \(\)
	\item \(\)
	\item \(\)
	\item \(\)
	\end{enumerate}
\item
	\begin{enumerate}[label={\Alph*.}]
	\item \(\)
	\item \(\)
	\item \(\)
	\item \(\)
	\end{enumerate}
\item
	\begin{enumerate}[label={\Alph*.}]
	\item \(\)
	\item \(\)
	\item \(\)
	\item \(\)
	\end{enumerate}
\item
	\begin{enumerate}[label={\Alph*.}]
	\item \(\)
	\item \(\)
	\item \(\)
	\item \(\)
	\end{enumerate}
\item
	\begin{enumerate}[label={\Alph*.}]
	\item \(\)
	\item \(\)
	\item \(\)
	\item \(\)
	\end{enumerate}
\item
	\begin{enumerate}[label={\Alph*.}]
	\item \(\)
	\item \(\)
	\item \(\)
	\item \(\)
	\end{enumerate}
\item
	\begin{enumerate}[label={\Alph*.}]
	\item \(\)
	\item \(\)
	\item \(\)
	\item \(\)
	\end{enumerate}
\item
	\begin{enumerate}[label={\Alph*.}]
	\item \(\)
	\item \(\)
	\item \(\)
	\item \(\)
	\end{enumerate}
\item
	\begin{enumerate}[label={\Alph*.}]
	\item \(\)
	\item \(\)
	\item \(\)
	\item \(\)
	\end{enumerate}
\item
	\begin{enumerate}[label={\Alph*.}]
	\item \(\)
	\item \(\)
	\item \(\)
	\item \(\)
	\end{enumerate}
\item
	\begin{enumerate}[label={\Alph*.}]
	\item \(\)
	\item \(\)
	\item \(\)
	\item \(\)
	\end{enumerate}
\item
	\begin{enumerate}[label={\Alph*.}]
	\item \(\)
	\item \(\)
	\item \(\)
	\item \(\)
	\end{enumerate}
\item
	\begin{enumerate}[label={\Alph*.}]
	\item \(\)
	\item \(\)
	\item \(\)
	\item \(\)
	\end{enumerate}
\item
	\begin{enumerate}[label={\Alph*.}]
	\item \(\)
	\item \(\)
	\item \(\)
	\item \(\)
	\end{enumerate}
\item
	\begin{enumerate}[label={\Alph*.}]
	\item \(\)
	\item \(\)
	\item \(\)
	\item \(\)
	\end{enumerate}
\item
	\begin{enumerate}[label={\Alph*.}]
	\item \(\)
	\item \(\)
	\item \(\)
	\item \(\)
	\end{enumerate}
\item
	\begin{enumerate}[label={\Alph*.}]
	\item \(\)
	\item \(\)
	\item \(\)
	\item \(\)
	\end{enumerate}
\item
	\begin{enumerate}[label={\Alph*.}]
	\item \(\)
	\item \(\)
	\item \(\)
	\item \(\)
	\end{enumerate}
\item
	\begin{enumerate}[label={\Alph*.}]
	\item \(\)
	\item \(\)
	\item \(\)
	\item \(\)
	\end{enumerate}
\item
	\begin{enumerate}[label={\Alph*.}]
	\item \(\)
	\item \(\)
	\item \(\)
	\item \(\)
	\end{enumerate}
\item
	\begin{enumerate}[label={\Alph*.}]
	\item \(\)
	\item \(\)
	\item \(\)
	\item \(\)
	\end{enumerate}
\item
	\begin{enumerate}[label={\Alph*.}]
	\item \(\)
	\item \(\)
	\item \(\)
	\item \(\)
	\end{enumerate}
\item
	\begin{enumerate}[label={\Alph*.}]
	\item \(\)
	\item \(\)
	\item \(\)
	\item \(\)
	\end{enumerate}
\item
	\begin{enumerate}[label={\Alph*.}]
	\item \(\)
	\item \(\)
	\item \(\)
	\item \(\)
	\end{enumerate}
\item
	\begin{enumerate}[label={\Alph*.}]
	\item \(\)
	\item \(\)
	\item \(\)
	\item \(\)
	\end{enumerate}
\item
	\begin{enumerate}[label={\Alph*.}]
	\item \(\)
	\item \(\)
	\item \(\)
	\item \(\)
	\end{enumerate}
\item
	\begin{enumerate}[label={\Alph*.}]
	\item \(\)
	\item \(\)
	\item \(\)
	\item \(\)
	\end{enumerate}
\item
	\begin{enumerate}[label={\Alph*.}]
	\item \(\)
	\item \(\)
	\item \(\)
	\item \(\)
	\end{enumerate}
\item
	\begin{enumerate}[label={\Alph*.}]
	\item \(\)
	\item \(\)
	\item \(\)
	\item \(\)
	\end{enumerate}
\item
	\begin{enumerate}[label={\Alph*.}]
	\item \(\)
	\item \(\)
	\item \(\)
	\item \(\)
	\end{enumerate}
\item
	\begin{enumerate}[label={\Alph*.}]
	\item \(\)
	\item \(\)
	\item \(\)
	\item \(\)
	\end{enumerate}
\item
	\begin{enumerate}[label={\Alph*.}]
	\item \(\)
	\item \(\)
	\item \(\)
	\item \(\)
	\end{enumerate}
\item
	\begin{enumerate}[label={\Alph*.}]
	\item \(\)
	\item \(\)
	\item \(\)
	\item \(\)
	\end{enumerate}
\item
	\begin{enumerate}[label={\Alph*.}]
	\item \(\)
	\item \(\)
	\item \(\)
	\item \(\)
	\end{enumerate}
\item
	\begin{enumerate}[label={\Alph*.}]
	\item \(\)
	\item \(\)
	\item \(\)
	\item \(\)
	\end{enumerate}
\item
	\begin{enumerate}[label={\Alph*.}]
	\item \(\)
	\item \(\)
	\item \(\)
	\item \(\)
	\end{enumerate}
\item
	\begin{enumerate}[label={\Alph*.}]
	\item \(\)
	\item \(\)
	\item \(\)
	\item \(\)
	\end{enumerate}


\item
	\begin{enumerate}[label={\Alph*.}]
	\item \(\)
	\item \(\)
	\item \(\)
	\item \(\)
	\end{enumerate}
\item
	\begin{enumerate}[label={\Alph*.}]
	\item \(\)
	\item \(\)
	\item \(\)
	\item \(\)
	\end{enumerate}
\item
	\begin{enumerate}[label={\Alph*.}]
	\item \(\)
	\item \(\)
	\item \(\)
	\item \(\)
	\end{enumerate}
\item
	\begin{enumerate}[label={\Alph*.}]
	\item \(\)
	\item \(\)
	\item \(\)
	\item \(\)
	\end{enumerate}
\item
	\begin{enumerate}[label={\Alph*.}]
	\item \(\)
	\item \(\)
	\item \(\)
	\item \(\)
	\end{enumerate}
\item
	\begin{enumerate}[label={\Alph*.}]
	\item \(\)
	\item \(\)
	\item \(\)
	\item \(\)
	\end{enumerate}
\item
	\begin{enumerate}[label={\Alph*.}]
	\item \(\)
	\item \(\)
	\item \(\)
	\item \(\)
	\end{enumerate}
\item
	\begin{enumerate}[label={\Alph*.}]
	\item \(\)
	\item \(\)
	\item \(\)
	\item \(\)
	\end{enumerate}
\item
	\begin{enumerate}[label={\Alph*.}]
	\item \(\)
	\item \(\)
	\item \(\)
	\item \(\)
	\end{enumerate}
\item
	\begin{enumerate}[label={\Alph*.}]
	\item \(\)
	\item \(\)
	\item \(\)
	\item \(\)
	\end{enumerate}
\item
	\begin{enumerate}[label={\Alph*.}]
	\item \(\)
	\item \(\)
	\item \(\)
	\item \(\)
	\end{enumerate}
\item
	\begin{enumerate}[label={\Alph*.}]
	\item \(\)
	\item \(\)
	\item \(\)
	\item \(\)
	\end{enumerate}
\item
	\begin{enumerate}[label={\Alph*.}]
	\item \(\)
	\item \(\)
	\item \(\)
	\item \(\)
	\end{enumerate}
\item
	\begin{enumerate}[label={\Alph*.}]
	\item \(\)
	\item \(\)
	\item \(\)
	\item \(\)
	\end{enumerate}
\item
	\begin{enumerate}[label={\Alph*.}]
	\item \(\)
	\item \(\)
	\item \(\)
	\item \(\)
	\end{enumerate}
\item
	\begin{enumerate}[label={\Alph*.}]
	\item \(\)
	\item \(\)
	\item \(\)
	\item \(\)
	\end{enumerate}
\item
	\begin{enumerate}[label={\Alph*.}]
	\item \(\)
	\item \(\)
	\item \(\)
	\item \(\)
	\end{enumerate}
\item
	\begin{enumerate}[label={\Alph*.}]
	\item \(\)
	\item \(\)
	\item \(\)
	\item \(\)
	\end{enumerate}
\item
	\begin{enumerate}[label={\Alph*.}]
	\item \(\)
	\item \(\)
	\item \(\)
	\item \(\)
	\end{enumerate}
\item
	\begin{enumerate}[label={\Alph*.}]
	\item \(\)
	\item \(\)
	\item \(\)
	\item \(\)
	\end{enumerate}
\item
	\begin{enumerate}[label={\Alph*.}]
	\item \(\)
	\item \(\)
	\item \(\)
	\item \(\)
	\end{enumerate}
\item
	\begin{enumerate}[label={\Alph*.}]
	\item \(\)
	\item \(\)
	\item \(\)
	\item \(\)
	\end{enumerate}
\item
	\begin{enumerate}[label={\Alph*.}]
	\item \(\)
	\item \(\)
	\item \(\)
	\item \(\)
	\end{enumerate}
\item
	\begin{enumerate}[label={\Alph*.}]
	\item \(\)
	\item \(\)
	\item \(\)
	\item \(\)
	\end{enumerate}
\item
	\begin{enumerate}[label={\Alph*.}]
	\item \(\)
	\item \(\)
	\item \(\)
	\item \(\)
	\end{enumerate}
\end{enumerate}
\end{multicols}
\subsection{Solution}
\begin{enumerate}[label={\arabic*.}]
    \item 
    \item 
    \item
    \item
    \item
    \item 
    \item
    \item
    \item 
    \item 
    \item 
    \item 
    \item
    \item
    \item
    \item 
    \item
    \item
    \item 
    \item 
    \item 
    \item 
    \item
    \item
    \item
    \item 
    \item
    \item
    \item 
    \item 
    \item 
    \item 
    \item
    \item
    \item
    \item 
    \item
    \item
    \item 
    \item 
    \item 
    \item 
    \item
    \item
    \item
    \item 
    \item
    \item
    \item 
    \item 
    \item 
    \item 
    \item
    \item
    \item
    \item 
    \item
    \item
    \item 
    \item 
    \item 
    \item 
    \item
    \item
    \item
    \item 
    \item
    \item
    \item 
    \item 
    \item 
    \item 
    \item
    \item
    \item
    \item 
    \item
    \item
    \item 
    \item 
    \item 
    \item 
    \item
    \item
    \item
    \item 
    \item
    \item
    \item 
    \item 
    \item 
    \item 
    \item
    \item
    \item
    \item 
    \item
    \item
    \item 
    \item 
\end{enumerate}
\section{Co-ordinate Geometry}
\subsection{Questions}
\begin{multicols}{2}
\begin{enumerate}[label={\arabic*.}]
\item
	\begin{enumerate}[label={\Alph*.}]
	\item \(\)
	\item \(\)
	\item \(\)
	\item \(\)
	\end{enumerate}
\item
	\begin{enumerate}[label={\Alph*.}]
	\item \(\)
	\item \(\)
	\item \(\)
	\item \(\)
	\end{enumerate}
\item
	\begin{enumerate}[label={\Alph*.}]
	\item \(\)
	\item \(\)
	\item \(\)
	\item \(\)
	\end{enumerate}
\item
	\begin{enumerate}[label={\Alph*.}]
	\item \(\)
	\item \(\)
	\item \(\)
	\item \(\)
	\end{enumerate}
\item
	\begin{enumerate}[label={\Alph*.}]
	\item \(\)
	\item \(\)
	\item \(\)
	\item \(\)
	\end{enumerate}
\item
	\begin{enumerate}[label={\Alph*.}]
	\item \(\)
	\item \(\)
	\item \(\)
	\item \(\)
	\end{enumerate}
\item
	\begin{enumerate}[label={\Alph*.}]
	\item \(\)
	\item \(\)
	\item \(\)
	\item \(\)
	\end{enumerate}
\item
	\begin{enumerate}[label={\Alph*.}]
	\item \(\)
	\item \(\)
	\item \(\)
	\item \(\)
	\end{enumerate}
\item
	\begin{enumerate}[label={\Alph*.}]
	\item \(\)
	\item \(\)
	\item \(\)
	\item \(\)
	\end{enumerate}
\item
	\begin{enumerate}[label={\Alph*.}]
	\item \(\)
	\item \(\)
	\item \(\)
	\item \(\)
	\end{enumerate}
\item
	\begin{enumerate}[label={\Alph*.}]
	\item \(\)
	\item \(\)
	\item \(\)
	\item \(\)
	\end{enumerate}


\item
	\begin{enumerate}[label={\Alph*.}]
	\item \(\)
	\item \(\)
	\item \(\)
	\item \(\)
	\end{enumerate}
\item
	\begin{enumerate}[label={\Alph*.}]
	\item \(\)
	\item \(\)
	\item \(\)
	\item \(\)
	\end{enumerate}
\item
	\begin{enumerate}[label={\Alph*.}]
	\item \(\)
	\item \(\)
	\item \(\)
	\item \(\)
	\end{enumerate}
\item
	\begin{enumerate}[label={\Alph*.}]
	\item \(\)
	\item \(\)
	\item \(\)
	\item \(\)
	\end{enumerate}
\item
	\begin{enumerate}[label={\Alph*.}]
	\item \(\)
	\item \(\)
	\item \(\)
	\item \(\)
	\end{enumerate}
\item
	\begin{enumerate}[label={\Alph*.}]
	\item \(\)
	\item \(\)
	\item \(\)
	\item \(\)
	\end{enumerate}
\item
	\begin{enumerate}[label={\Alph*.}]
	\item \(\)
	\item \(\)
	\item \(\)
	\item \(\)
	\end{enumerate}
\item
	\begin{enumerate}[label={\Alph*.}]
	\item \(\)
	\item \(\)
	\item \(\)
	\item \(\)
	\end{enumerate}
\item
	\begin{enumerate}[label={\Alph*.}]
	\item \(\)
	\item \(\)
	\item \(\)
	\item \(\)
	\end{enumerate}
\item
	\begin{enumerate}[label={\Alph*.}]
	\item \(\)
	\item \(\)
	\item \(\)
	\item \(\)
	\end{enumerate}
\item
	\begin{enumerate}[label={\Alph*.}]
	\item \(\)
	\item \(\)
	\item \(\)
	\item \(\)
	\end{enumerate}
\item
	\begin{enumerate}[label={\Alph*.}]
	\item \(\)
	\item \(\)
	\item \(\)
	\item \(\)
	\end{enumerate}
\item
	\begin{enumerate}[label={\Alph*.}]
	\item \(\)
	\item \(\)
	\item \(\)
	\item \(\)
	\end{enumerate}
\item
	\begin{enumerate}[label={\Alph*.}]
	\item \(\)
	\item \(\)
	\item \(\)
	\item \(\)
	\end{enumerate}
\item
	\begin{enumerate}[label={\Alph*.}]
	\item \(\)
	\item \(\)
	\item \(\)
	\item \(\)
	\end{enumerate}
\item
	\begin{enumerate}[label={\Alph*.}]
	\item \(\)
	\item \(\)
	\item \(\)
	\item \(\)
	\end{enumerate}
\item
	\begin{enumerate}[label={\Alph*.}]
	\item \(\)
	\item \(\)
	\item \(\)
	\item \(\)
	\end{enumerate}
\item
	\begin{enumerate}[label={\Alph*.}]
	\item \(\)
	\item \(\)
	\item \(\)
	\item \(\)
	\end{enumerate}
\item
	\begin{enumerate}[label={\Alph*.}]
	\item \(\)
	\item \(\)
	\item \(\)
	\item \(\)
	\end{enumerate}
\item
	\begin{enumerate}[label={\Alph*.}]
	\item \(\)
	\item \(\)
	\item \(\)
	\item \(\)
	\end{enumerate}
\item
	\begin{enumerate}[label={\Alph*.}]
	\item \(\)
	\item \(\)
	\item \(\)
	\item \(\)
	\end{enumerate}
\item
	\begin{enumerate}[label={\Alph*.}]
	\item \(\)
	\item \(\)
	\item \(\)
	\item \(\)
	\end{enumerate}
\item
	\begin{enumerate}[label={\Alph*.}]
	\item \(\)
	\item \(\)
	\item \(\)
	\item \(\)
	\end{enumerate}
\item
	\begin{enumerate}[label={\Alph*.}]
	\item \(\)
	\item \(\)
	\item \(\)
	\item \(\)
	\end{enumerate}
\item
	\begin{enumerate}[label={\Alph*.}]
	\item \(\)
	\item \(\)
	\item \(\)
	\item \(\)
	\end{enumerate}
\item
	\begin{enumerate}[label={\Alph*.}]
	\item \(\)
	\item \(\)
	\item \(\)
	\item \(\)
	\end{enumerate}
\item
	\begin{enumerate}[label={\Alph*.}]
	\item \(\)
	\item \(\)
	\item \(\)
	\item \(\)
	\end{enumerate}
\item
	\begin{enumerate}[label={\Alph*.}]
	\item \(\)
	\item \(\)
	\item \(\)
	\item \(\)
	\end{enumerate}
\item
	\begin{enumerate}[label={\Alph*.}]
	\item \(\)
	\item \(\)
	\item \(\)
	\item \(\)
	\end{enumerate}
\item
	\begin{enumerate}[label={\Alph*.}]
	\item \(\)
	\item \(\)
	\item \(\)
	\item \(\)
	\end{enumerate}
\item
	\begin{enumerate}[label={\Alph*.}]
	\item \(\)
	\item \(\)
	\item \(\)
	\item \(\)
	\end{enumerate}
\item
	\begin{enumerate}[label={\Alph*.}]
	\item \(\)
	\item \(\)
	\item \(\)
	\item \(\)
	\end{enumerate}
\item
	\begin{enumerate}[label={\Alph*.}]
	\item \(\)
	\item \(\)
	\item \(\)
	\item \(\)
	\end{enumerate}
\item
	\begin{enumerate}[label={\Alph*.}]
	\item \(\)
	\item \(\)
	\item \(\)
	\item \(\)
	\end{enumerate}
\item
	\begin{enumerate}[label={\Alph*.}]
	\item \(\)
	\item \(\)
	\item \(\)
	\item \(\)
	\end{enumerate}
\item
	\begin{enumerate}[label={\Alph*.}]
	\item \(\)
	\item \(\)
	\item \(\)
	\item \(\)
	\end{enumerate}
\item
	\begin{enumerate}[label={\Alph*.}]
	\item \(\)
	\item \(\)
	\item \(\)
	\item \(\)
	\end{enumerate}
\item
	\begin{enumerate}[label={\Alph*.}]
	\item \(\)
	\item \(\)
	\item \(\)
	\item \(\)
	\end{enumerate}
\item
	\begin{enumerate}[label={\Alph*.}]
	\item \(\)
	\item \(\)
	\item \(\)
	\item \(\)
	\end{enumerate}
\item
	\begin{enumerate}[label={\Alph*.}]
	\item \(\)
	\item \(\)
	\item \(\)
	\item \(\)
	\end{enumerate}
\item
	\begin{enumerate}[label={\Alph*.}]
	\item \(\)
	\item \(\)
	\item \(\)
	\item \(\)
	\end{enumerate}
\item
	\begin{enumerate}[label={\Alph*.}]
	\item \(\)
	\item \(\)
	\item \(\)
	\item \(\)
	\end{enumerate}
\item
	\begin{enumerate}[label={\Alph*.}]
	\item \(\)
	\item \(\)
	\item \(\)
	\item \(\)
	\end{enumerate}
\item
	\begin{enumerate}[label={\Alph*.}]
	\item \(\)
	\item \(\)
	\item \(\)
	\item \(\)
	\end{enumerate}
\item
	\begin{enumerate}[label={\Alph*.}]
	\item \(\)
	\item \(\)
	\item \(\)
	\item \(\)
	\end{enumerate}
\item
	\begin{enumerate}[label={\Alph*.}]
	\item \(\)
	\item \(\)
	\item \(\)
	\item \(\)
	\end{enumerate}
\item
	\begin{enumerate}[label={\Alph*.}]
	\item \(\)
	\item \(\)
	\item \(\)
	\item \(\)
	\end{enumerate}
\item
	\begin{enumerate}[label={\Alph*.}]
	\item \(\)
	\item \(\)
	\item \(\)
	\item \(\)
	\end{enumerate}
\item
	\begin{enumerate}[label={\Alph*.}]
	\item \(\)
	\item \(\)
	\item \(\)
	\item \(\)
	\end{enumerate}
\item
	\begin{enumerate}[label={\Alph*.}]
	\item \(\)
	\item \(\)
	\item \(\)
	\item \(\)
	\end{enumerate}
\item
	\begin{enumerate}[label={\Alph*.}]
	\item \(\)
	\item \(\)
	\item \(\)
	\item \(\)
	\end{enumerate}
\item
	\begin{enumerate}[label={\Alph*.}]
	\item \(\)
	\item \(\)
	\item \(\)
	\item \(\)
	\end{enumerate}
\item
	\begin{enumerate}[label={\Alph*.}]
	\item \(\)
	\item \(\)
	\item \(\)
	\item \(\)
	\end{enumerate}
\item
	\begin{enumerate}[label={\Alph*.}]
	\item \(\)
	\item \(\)
	\item \(\)
	\item \(\)
	\end{enumerate}
\item
	\begin{enumerate}[label={\Alph*.}]
	\item \(\)
	\item \(\)
	\item \(\)
	\item \(\)
	\end{enumerate}
\item
	\begin{enumerate}[label={\Alph*.}]
	\item \(\)
	\item \(\)
	\item \(\)
	\item \(\)
	\end{enumerate}
\item
	\begin{enumerate}[label={\Alph*.}]
	\item \(\)
	\item \(\)
	\item \(\)
	\item \(\)
	\end{enumerate}
\item
	\begin{enumerate}[label={\Alph*.}]
	\item \(\)
	\item \(\)
	\item \(\)
	\item \(\)
	\end{enumerate}
\item
	\begin{enumerate}[label={\Alph*.}]
	\item \(\)
	\item \(\)
	\item \(\)
	\item \(\)
	\end{enumerate}
\item
	\begin{enumerate}[label={\Alph*.}]
	\item \(\)
	\item \(\)
	\item \(\)
	\item \(\)
	\end{enumerate}
\item
	\begin{enumerate}[label={\Alph*.}]
	\item \(\)
	\item \(\)
	\item \(\)
	\item \(\)
	\end{enumerate}
\item
	\begin{enumerate}[label={\Alph*.}]
	\item \(\)
	\item \(\)
	\item \(\)
	\item \(\)
	\end{enumerate}
\item
	\begin{enumerate}[label={\Alph*.}]
	\item \(\)
	\item \(\)
	\item \(\)
	\item \(\)
	\end{enumerate}
\item
	\begin{enumerate}[label={\Alph*.}]
	\item \(\)
	\item \(\)
	\item \(\)
	\item \(\)
	\end{enumerate}


\item
	\begin{enumerate}[label={\Alph*.}]
	\item \(\)
	\item \(\)
	\item \(\)
	\item \(\)
	\end{enumerate}
\item
	\begin{enumerate}[label={\Alph*.}]
	\item \(\)
	\item \(\)
	\item \(\)
	\item \(\)
	\end{enumerate}
\item
	\begin{enumerate}[label={\Alph*.}]
	\item \(\)
	\item \(\)
	\item \(\)
	\item \(\)
	\end{enumerate}
\item
	\begin{enumerate}[label={\Alph*.}]
	\item \(\)
	\item \(\)
	\item \(\)
	\item \(\)
	\end{enumerate}
\item
	\begin{enumerate}[label={\Alph*.}]
	\item \(\)
	\item \(\)
	\item \(\)
	\item \(\)
	\end{enumerate}
\item
	\begin{enumerate}[label={\Alph*.}]
	\item \(\)
	\item \(\)
	\item \(\)
	\item \(\)
	\end{enumerate}
\item
	\begin{enumerate}[label={\Alph*.}]
	\item \(\)
	\item \(\)
	\item \(\)
	\item \(\)
	\end{enumerate}
\item
	\begin{enumerate}[label={\Alph*.}]
	\item \(\)
	\item \(\)
	\item \(\)
	\item \(\)
	\end{enumerate}
\item
	\begin{enumerate}[label={\Alph*.}]
	\item \(\)
	\item \(\)
	\item \(\)
	\item \(\)
	\end{enumerate}
\item
	\begin{enumerate}[label={\Alph*.}]
	\item \(\)
	\item \(\)
	\item \(\)
	\item \(\)
	\end{enumerate}
\item
	\begin{enumerate}[label={\Alph*.}]
	\item \(\)
	\item \(\)
	\item \(\)
	\item \(\)
	\end{enumerate}
\item
	\begin{enumerate}[label={\Alph*.}]
	\item \(\)
	\item \(\)
	\item \(\)
	\item \(\)
	\end{enumerate}
\item
	\begin{enumerate}[label={\Alph*.}]
	\item \(\)
	\item \(\)
	\item \(\)
	\item \(\)
	\end{enumerate}
\item
	\begin{enumerate}[label={\Alph*.}]
	\item \(\)
	\item \(\)
	\item \(\)
	\item \(\)
	\end{enumerate}
\item
	\begin{enumerate}[label={\Alph*.}]
	\item \(\)
	\item \(\)
	\item \(\)
	\item \(\)
	\end{enumerate}
\item
	\begin{enumerate}[label={\Alph*.}]
	\item \(\)
	\item \(\)
	\item \(\)
	\item \(\)
	\end{enumerate}
\item
	\begin{enumerate}[label={\Alph*.}]
	\item \(\)
	\item \(\)
	\item \(\)
	\item \(\)
	\end{enumerate}
\item
	\begin{enumerate}[label={\Alph*.}]
	\item \(\)
	\item \(\)
	\item \(\)
	\item \(\)
	\end{enumerate}
\item
	\begin{enumerate}[label={\Alph*.}]
	\item \(\)
	\item \(\)
	\item \(\)
	\item \(\)
	\end{enumerate}
\item
	\begin{enumerate}[label={\Alph*.}]
	\item \(\)
	\item \(\)
	\item \(\)
	\item \(\)
	\end{enumerate}
\item
	\begin{enumerate}[label={\Alph*.}]
	\item \(\)
	\item \(\)
	\item \(\)
	\item \(\)
	\end{enumerate}
\item
	\begin{enumerate}[label={\Alph*.}]
	\item \(\)
	\item \(\)
	\item \(\)
	\item \(\)
	\end{enumerate}
\item
	\begin{enumerate}[label={\Alph*.}]
	\item \(\)
	\item \(\)
	\item \(\)
	\item \(\)
	\end{enumerate}
\item
	\begin{enumerate}[label={\Alph*.}]
	\item \(\)
	\item \(\)
	\item \(\)
	\item \(\)
	\end{enumerate}
\item
	\begin{enumerate}[label={\Alph*.}]
	\item \(\)
	\item \(\)
	\item \(\)
	\item \(\)
	\end{enumerate}
\end{enumerate}
\end{multicols}
\subsection{Solution}
\begin{enumerate}[label={\arabic*.}]
    \item 
    \item 
    \item
    \item
    \item
    \item 
    \item
    \item
    \item 
    \item 
    \item 
    \item 
    \item
    \item
    \item
    \item 
    \item
    \item
    \item 
    \item 
    \item 
    \item 
    \item
    \item
    \item
    \item 
    \item
    \item
    \item 
    \item 
    \item 
    \item 
    \item
    \item
    \item
    \item 
    \item
    \item
    \item 
    \item 
    \item 
    \item 
    \item
    \item
    \item
    \item 
    \item
    \item
    \item 
    \item 
    \item 
    \item 
    \item
    \item
    \item
    \item 
    \item
    \item
    \item 
    \item 
    \item 
    \item 
    \item
    \item
    \item
    \item 
    \item
    \item
    \item 
    \item 
    \item 
    \item 
    \item
    \item
    \item
    \item 
    \item
    \item
    \item 
    \item 
    \item 
    \item 
    \item
    \item
    \item
    \item 
    \item
    \item
    \item 
    \item 
    \item 
    \item 
    \item
    \item
    \item
    \item 
    \item
    \item
    \item 
    \item 
\end{enumerate}
\chapter{Calculus}
\section{Differentiation}
\subsection{Questions}
\begin{multicols}{2}
\begin{enumerate}[label={\arabic*.}]
  \item The minimum point on the curve \(y = x^2 - 6x + 5\) is at?
        \begin{enumerate}[label={\Alph*.}]
            \item (1,5)
            \item (2,3)
            \item (3,4)
            \item (-3,4)
            \item (3,-4)
        \end{enumerate}
  \item At what value of \(x\) is the function \(y = x^2 + x + 1\) minimum? 
        \begin{enumerate}[label={\Alph*.}]
            \item \(-1\)
            \item \(-\frac{1}{2}\)
            \item \(\frac{1}{2}\)
            \item \(1\)
        \end{enumerate}
  \item At what value of \(x\) is the function \(y = x^2 - 2x - 3\) minimum?
    \begin{enumerate}[label={\Alph*.}]
            \item \(1\) 
            \item  \(-1\) 
            \item \(-4\)
            \item \(4\)
        \end{enumerate}
  \item Find the maximum value of \(y = x^2 - 2x - 3\)
        \begin{enumerate}[label={\Alph*.}]
			\item \(\)
			\item \(\)
			\item \(\)
			\item \(\)
			\item \(\)
        \end{enumerate}
  \item Find the maximum value of \(y = 3x^2 - x^3\)
        \begin{enumerate}[label={\Alph*.}]
            \item \(2\)
            \item \(4\)
            \item \(6\)
            \item \(0\)
        \end{enumerate}
  \item Find the minimum value of \(y = x^3 + x^2 - x + 1\)
        \begin{enumerate}[label={\Alph*.}]
            \item \(\)
			\item \(\)
			\item \(\)
			\item \(\)
			\item \(\)
        \end{enumerate}
  \item Find the value of \(x\) for which the function \(f(x) = 2x^3 - x^2 -4x + 4\) has a maximum value.
        \begin{enumerate}[label={\Alph*.}]
            \item \(\frac{2}{3}\)
            \item \(1\)
            \item \(-1\)
            \item \(-\frac{2}{3}\)
        \end{enumerate}
  \item Find the value of \(x\) for which the function \(f(x) = 3x^3-9x^2\) is minimum
        \begin{enumerate}[label={\Alph*.}]
            \item \(2\)
            \item \(0\)
            \item \(5\)
            \item \(3\)
        \end{enumerate}
  \item Find the maximum value of the function \(f(x) = 2 + x - x^2\)
        \begin{enumerate}[label={\Alph*.}]
            \item \(\frac{9}{4}\)
            \item \(\frac{7}{4}\)
            \item \(\frac{3}{2}\)
            \item \(\frac{1}{2}\)
        \end{enumerate}
  \item Find the maximum value of \(y\) in the equation: \(y = 1 - 2 - 3x^2\)
        \begin{enumerate}[label={\Alph*.}]
            \item \(\frac{4}{3}\)
            \item \(\frac{5}{4}\)
            \item \(\frac{3}{4}\)
            \item \(\frac{5}{3}\)
        \end{enumerate}
  \item The minimum value of \(y\) in the equation: \(y = x^2 - 6x + 8\) is
        \begin{enumerate}[label={\Alph*.}]
            \item \(8\)
            \item \(3\)
            \item \(0\)
            \item \(-1\)
        \end{enumerate}
  \item Obtain a maximum value of the function: \(f(x) = x^3 - 12x + 11\)
        \begin{enumerate}[label={\Alph*.}]
            \item \(-5\)
            \item \(-2\)
            \item \(2\)
            \item \(27\)
        \end{enumerate}
  \item Find the value of \(h\) if the maximum value of \(y = 1 + hx - 3x^2\) is 13.
        \begin{enumerate}[label={\Alph*.}]
            \item \(10\)
            \item \(11\)
            \item \(12\)
            \item \(13\)
        \end{enumerate}
  \item A trader realizes \(10x - x^2\) naira profit from the sale of \(x\) bags of corn. How many bags will give him the maximum profit?
        \begin{enumerate}[label={\Alph*.}]
            \item \(4\)
            \item \(5\)
            \item \(6\)
            \item \(7\)
        \end{enumerate}
  \item Find the value of \(x\) for which the function \(y = x^3 - x\) has a minimum value.
        \begin{enumerate}[label={\Alph*.}]
            \item \(\frac{\sqrt{3}}{3}\)
            \item -\(\frac{\sqrt{3}}{3}\)
            \item \(\sqrt{3}\)
            \item -\(\sqrt{3}\)
        \end{enumerate}
  \item If \(f(x) = x^2 - 2x - 3\), find the least value of \(f(x)\) and the corresponding value of \(x\).
        \begin{enumerate}[label={\Alph*.}]
            \item \(f(x) = -3, x = 1\)
            \item \(f(x) = -3, x = 3\)
            \item \(f(x) = 1, x = -4\)
            \item \(f(x) = 1, x = -4\)
        \end{enumerate}
  \item If \(y = 3 \cos\left(\frac{x}{3}\right)\), find \(\dv{y}{x}\) when \(x = \frac{3\pi}{2}\).
        \begin{enumerate}[label={\Alph*.}]
            \item \(-1\)
            \item \(1\)
            \item \(2\)
            \item  \(3\)
        \end{enumerate}
  \item What is the rate of change of the volume \(v\) of a hemisphere with respect to its radius \(r\) when \(r = 2\)?
        \begin{enumerate}[label={\Alph*.}]
            \item  \(2\pi\)
            \item  \(4\pi\)
            \item  \(8\pi\)
            \item  \(16\pi\)
        \end{enumerate}
  \item If \(y = (1-2x)^3\), find the value of \(\dv{y}{x}\) at \(x = -1\).
         \begin{enumerate}[label={\Alph*.}]
            \item  \(22\)
            \item \(57\)
            \item \(-6\)
            \item \(-54\)
        \end{enumerate}
  \item Find the derivative of \(y = \sin(2x^3+3x-4)\).
      \begin{enumerate}[label={\Alph*.}]
            \item \(\cos(2x^2+3x-4)\)
            \item \(-\cos(2x^2+3x-4)\)
            \item  \(-(6x^2+3)\cos(2x^2+3x-4)\)
            \item \((6x^2 + 3) \cos(2x^2+3x-4)\)
        \end{enumerate}
  \item The radius \(r\) of a circular disc is increasing at the rate of \(0.5\ \text{cm/sec}\). At what rate is the area of the disc increasing when its radius is \(6\ \text{cm}\)?
       \begin{enumerate}[label={\Alph*.}]
            \item  \(3\pi\ \text{cm}^2/\text{sec}\)
            \item  \(18\pi\ \text{cm}^2/\text{sec}\)
            \item  \(6\pi\ \text{cm}^2/\text{sec}\)
            \item \(36\pi\ \text{cm}^2/\text{sec}\)
        \end{enumerate}
  \item Find \(\dv{y}{x}\), if \(y = \cos x\).
       \begin{enumerate}[label={\Alph*.}]
            \item  \(\sin x\)
            \item  \(-\sin x\)
            \item  \(\tan x\)
            \item  \(-\tan x\)
        \end{enumerate}
  \item Differentiate: \((\cos \theta - \sin \theta)^2\) with respect to \(\theta\).
     \begin{enumerate}[label={\Alph*.}]
            \item \(1-2\cos 2\theta\)
            \item  \(-2\sin 2\theta\)
            \item  \(-2\cos 2\theta\)
            \item  \(1-2\sin 2\theta\)
        \end{enumerate}
  \item Differentiate: \(\left(x^2 + \frac{1}{x}\right)^2\) with respect to \(x\).
       \begin{enumerate}[label={\Alph*.}]
            \item \(4x^3 - 2 + \frac{2}{x^3}\)
            \item  \(4x^3 - 2 - \frac{2}{x^3}\)
            \item   \(4x^3 - 4x - \frac{2}{x}\)
            \item    \(4x^3 - 3x + \frac{2}{x}\)
        \end{enumerate}
\item Find the point \(x,y\) on the Euclidean plane where the curve \(y = 2x^{2} - 2x +3\) has \(2\) as the gradient.
    \begin{enumerate}[label={\Alph*.}]
        \item \((1, 4)\)
        \item \((2, 2)\)
        \item \((3, 4)\)
        \item \((3, 2)\)
    \end{enumerate}
    
\item For what value of \(x\) is the tangent to the curve \(y = x^{2} - 4x + 3\) parallel to the \(x\)-axis?
    \begin{enumerate}[label={\Alph*.}]
        \item \(0\)
        \item \(1\)
        \item \(2\)
        \item \(3\)
    \end{enumerate}   
\item If \(y = x \sin x\), find \( \frac{d^2y}{dx^2} \).
	\begin{enumerate}[label={\Alph*.}]
	\item \(2\cos x - \sin x\)
	\item \(\sin x + \cos x\)
	\item \(\sin x  - \cos  x\)
	\item \( \cos x  - 2\sin x \)
	\end{enumerate}
\item Differentiate: \(\frac{6x^{3} - 5x^{2} + 1}{3x^{2}}\) with respect to \(x\)
	\begin{enumerate}[label={\Alph*.}]
	\item \(2 + \frac{2}{3{x}^{3}}\)
	\item \(2 + \frac{1}{6x}\)
	\item \(\sin x - \cos x\)
	\item \(\cos x - 2\sin x\)
	\end{enumerate}
\item If \(y = (1+x)^{2}\), find \( \dv{y}{x} \).
	\begin{enumerate}[label={\Alph*.}]
	\item \(x+1\)
	\item \(2x-1\)
	\item \(2 + 2x\)
	\item \(1+2x\)
	\end{enumerate}
\item Differentiate: \(3x^3+2x^2+3x+1\) with respect to \(x\)
	\begin{enumerate}[label={\Alph*.}]
	\item \(9x^2+4x+3\)
	\item \(9x^2+4x-3\)
	\item \(9x^2-4x-3\)
	\item \(9x^2-4x+3\)
	\end{enumerate}
\item Find the derivative of: \(y=(\frac{1}{3}x + 6)^{2}\).
	\begin{enumerate}[label={\Alph*.}]
	\item \(\frac{1}{3}(\frac{1}{3}x + 6)^2\)
	\item \(2(\frac{1}{3}x + 6)\)
	\item \(\frac{2}{3}(\frac{1}{3}x + 6)\)
	\item \(\frac{2}{3}(\frac{1}{3}x + 6)^2\)
	\end{enumerate}
\item Differentiate: \(\frac{2}{3}x^{3}-\frac{4}{x}\)
	\begin{enumerate}[label={\Alph*.}]
	\item \(2x^2 + \frac{4}{x^2}\)
	\item \(2x^{2}-\frac{4}{x}\)
	\item \(3x^{2}-\frac{4}{x}\)
	\item \(3x^{2}+\frac{4}{x^2}\)
	\end{enumerate}
\item If \(y=x^2-3x+4\), find \(\dv{y}{x}\) at \(x=5\).
	\begin{enumerate}[label={\Alph*.}]
	\item \(9\)
	\item \(7\)
	\item \(5\)
	\item \(3\)
	\end{enumerate}
\item If \(y= 2x\cos{2x} -\sin{2x}\), find \(\dv{y}{x}\) when \(x=\frac{\pi}{4}\).
	\begin{enumerate}[label={\Alph*.}]
	\item \(9\)
	\item \(7\)
	\item \(5\)
	\item \(3\)
	\end{enumerate}
\item If \(y=3\cos{4x}\), find \(\dv{y}{x}\)
	\begin{enumerate}[label={\Alph*.}]
	\item \(-24\sin{4x}\)
	\item \(12\sin{4x}\)
	\item \(-12 \sin{4x}\)
	\item \(6\sin{8x}\)
	\end{enumerate}
\item Find the derivative of \((2+3x)(1-x)\) with respect to \(x\)
	\begin{enumerate}[label={\Alph*.}]
	\item \(6x-1\)
	\item \(1-6x\)
	\item \(-3\)
	\item \(6\)
	\end{enumerate}
\item
	\begin{enumerate}[label={\Alph*.}]
	\item \(\)
	\item \(\)
	\item \(\)
	\item \(\)
	\end{enumerate}
\item
	\begin{enumerate}[label={\Alph*.}]
	\item \(\)
	\item \(\)
	\item \(\)
	\item \(\)
	\end{enumerate}
\item
	\begin{enumerate}[label={\Alph*.}]
	\item \(\)
	\item \(\)
	\item \(\)
	\item \(\)
	\end{enumerate}
\item
	\begin{enumerate}[label={\Alph*.}]
	\item \(\)
	\item \(\)
	\item \(\)
	\item \(\)
	\end{enumerate}
\item
	\begin{enumerate}[label={\Alph*.}]
	\item \(\)
	\item \(\)
	\item \(\)
	\item \(\)
	\end{enumerate}
\item
	\begin{enumerate}[label={\Alph*.}]
	\item \(\)
	\item \(\)
	\item \(\)
	\item \(\)
	\end{enumerate}
\item
	\begin{enumerate}[label={\Alph*.}]
	\item \(\)
	\item \(\)
	\item \(\)
	\item \(\)
	\end{enumerate}
\item
	\begin{enumerate}[label={\Alph*.}]
	\item \(\)
	\item \(\)
	\item \(\)
	\item \(\)
	\end{enumerate}
\item
	\begin{enumerate}[label={\Alph*.}]
	\item \(\)
	\item \(\)
	\item \(\)
	\item \(\)
	\end{enumerate}
\item
	\begin{enumerate}[label={\Alph*.}]
	\item \(\)
	\item \(\)
	\item \(\)
	\item \(\)
	\end{enumerate}
\item
	\begin{enumerate}[label={\Alph*.}]
	\item \(\)
	\item \(\)
	\item \(\)
	\item \(\)
	\end{enumerate}


\item
	\begin{enumerate}[label={\Alph*.}]
	\item \(\)
	\item \(\)
	\item \(\)
	\item \(\)
	\end{enumerate}
\item
	\begin{enumerate}[label={\Alph*.}]
	\item \(\)
	\item \(\)
	\item \(\)
	\item \(\)
	\end{enumerate}
\item
	\begin{enumerate}[label={\Alph*.}]
	\item \(\)
	\item \(\)
	\item \(\)
	\item \(\)
	\end{enumerate}
\item
	\begin{enumerate}[label={\Alph*.}]
	\item \(\)
	\item \(\)
	\item \(\)
	\item \(\)
	\end{enumerate}
\item
	\begin{enumerate}[label={\Alph*.}]
	\item \(\)
	\item \(\)
	\item \(\)
	\item \(\)
	\end{enumerate}
\item
	\begin{enumerate}[label={\Alph*.}]
	\item \(\)
	\item \(\)
	\item \(\)
	\item \(\)
	\end{enumerate}
\item
	\begin{enumerate}[label={\Alph*.}]
	\item \(\)
	\item \(\)
	\item \(\)
	\item \(\)
	\end{enumerate}
\item
	\begin{enumerate}[label={\Alph*.}]
	\item \(\)
	\item \(\)
	\item \(\)
	\item \(\)
	\end{enumerate}
\item
	\begin{enumerate}[label={\Alph*.}]
	\item \(\)
	\item \(\)
	\item \(\)
	\item \(\)
	\end{enumerate}
\item
	\begin{enumerate}[label={\Alph*.}]
	\item \(\)
	\item \(\)
	\item \(\)
	\item \(\)
	\end{enumerate}
\item
	\begin{enumerate}[label={\Alph*.}]
	\item \(\)
	\item \(\)
	\item \(\)
	\item \(\)
	\end{enumerate}
\item
	\begin{enumerate}[label={\Alph*.}]
	\item \(\)
	\item \(\)
	\item \(\)
	\item \(\)
	\end{enumerate}
\item
	\begin{enumerate}[label={\Alph*.}]
	\item \(\)
	\item \(\)
	\item \(\)
	\item \(\)
	\end{enumerate}
\item
	\begin{enumerate}[label={\Alph*.}]
	\item \(\)
	\item \(\)
	\item \(\)
	\item \(\)
	\end{enumerate}
\item
	\begin{enumerate}[label={\Alph*.}]
	\item \(\)
	\item \(\)
	\item \(\)
	\item \(\)
	\end{enumerate}
\item
	\begin{enumerate}[label={\Alph*.}]
	\item \(\)
	\item \(\)
	\item \(\)
	\item \(\)
	\end{enumerate}
\item
	\begin{enumerate}[label={\Alph*.}]
	\item \(\)
	\item \(\)
	\item \(\)
	\item \(\)
	\end{enumerate}
\item
	\begin{enumerate}[label={\Alph*.}]
	\item \(\)
	\item \(\)
	\item \(\)
	\item \(\)
	\end{enumerate}
\item
	\begin{enumerate}[label={\Alph*.}]
	\item \(\)
	\item \(\)
	\item \(\)
	\item \(\)
	\end{enumerate}
\item
	\begin{enumerate}[label={\Alph*.}]
	\item \(\)
	\item \(\)
	\item \(\)
	\item \(\)
	\end{enumerate}
\item
	\begin{enumerate}[label={\Alph*.}]
	\item \(\)
	\item \(\)
	\item \(\)
	\item \(\)
	\end{enumerate}
\item
	\begin{enumerate}[label={\Alph*.}]
	\item \(\)
	\item \(\)
	\item \(\)
	\item \(\)
	\end{enumerate}
\item
	\begin{enumerate}[label={\Alph*.}]
	\item \(\)
	\item \(\)
	\item \(\)
	\item \(\)
	\end{enumerate}
\item
	\begin{enumerate}[label={\Alph*.}]
	\item \(\)
	\item \(\)
	\item \(\)
	\item \(\)
	\end{enumerate}
\item
	\begin{enumerate}[label={\Alph*.}]
	\item \(\)
	\item \(\)
	\item \(\)
	\item \(\)
	\end{enumerate}
\item
	\begin{enumerate}[label={\Alph*.}]
	\item \(\)
	\item \(\)
	\item \(\)
	\item \(\)
	\end{enumerate}
\item
	\begin{enumerate}[label={\Alph*.}]
	\item \(\)
	\item \(\)
	\item \(\)
	\item \(\)
	\end{enumerate}
\item
	\begin{enumerate}[label={\Alph*.}]
	\item \(\)
	\item \(\)
	\item \(\)
	\item \(\)
	\end{enumerate}
\item
	\begin{enumerate}[label={\Alph*.}]
	\item \(\)
	\item \(\)
	\item \(\)
	\item \(\)
	\end{enumerate}
\item
	\begin{enumerate}[label={\Alph*.}]
	\item \(\)
	\item \(\)
	\item \(\)
	\item \(\)
	\end{enumerate}
\item
	\begin{enumerate}[label={\Alph*.}]
	\item \(\)
	\item \(\)
	\item \(\)
	\item \(\)
	\end{enumerate}
\item
	\begin{enumerate}[label={\Alph*.}]
	\item \(\)
	\item \(\)
	\item \(\)
	\item \(\)
	\end{enumerate}
\item
	\begin{enumerate}[label={\Alph*.}]
	\item \(\)
	\item \(\)
	\item \(\)
	\item \(\)
	\end{enumerate}
\item
	\begin{enumerate}[label={\Alph*.}]
	\item \(\)
	\item \(\)
	\item \(\)
	\item \(\)
	\end{enumerate}
\item
	\begin{enumerate}[label={\Alph*.}]
	\item \(\)
	\item \(\)
	\item \(\)
	\item \(\)
	\end{enumerate}
\item
	\begin{enumerate}[label={\Alph*.}]
	\item \(\)
	\item \(\)
	\item \(\)
	\item \(\)
	\end{enumerate}
\item
	\begin{enumerate}[label={\Alph*.}]
	\item \(\)
	\item \(\)
	\item \(\)
	\item \(\)
	\end{enumerate}
\item
	\begin{enumerate}[label={\Alph*.}]
	\item \(\)
	\item \(\)
	\item \(\)
	\item \(\)
	\end{enumerate}
\item
	\begin{enumerate}[label={\Alph*.}]
	\item \(\)
	\item \(\)
	\item \(\)
	\item \(\)
	\end{enumerate}
\item
	\begin{enumerate}[label={\Alph*.}]
	\item \(\)
	\item \(\)
	\item \(\)
	\item \(\)
	\end{enumerate}
\item
	\begin{enumerate}[label={\Alph*.}]
	\item \(\)
	\item \(\)
	\item \(\)
	\item \(\)
	\end{enumerate}
\item
	\begin{enumerate}[label={\Alph*.}]
	\item \(\)
	\item \(\)
	\item \(\)
	\item \(\)
	\end{enumerate}
\item
	\begin{enumerate}[label={\Alph*.}]
	\item \(\)
	\item \(\)
	\item \(\)
	\item \(\)
	\end{enumerate}
\item
	\begin{enumerate}[label={\Alph*.}]
	\item \(\)
	\item \(\)
	\item \(\)
	\item \(\)
	\end{enumerate}
\item
	\begin{enumerate}[label={\Alph*.}]
	\item \(\)
	\item \(\)
	\item \(\)
	\item \(\)
	\end{enumerate}
\item
	\begin{enumerate}[label={\Alph*.}]
	\item \(\)
	\item \(\)
	\item \(\)
	\item \(\)
	\end{enumerate}
\item
	\begin{enumerate}[label={\Alph*.}]
	\item \(\)
	\item \(\)
	\item \(\)
	\item \(\)
	\end{enumerate}
\item
	\begin{enumerate}[label={\Alph*.}]
	\item \(\)
	\item \(\)
	\item \(\)
	\item \(\)
	\end{enumerate}
\item
	\begin{enumerate}[label={\Alph*.}]
	\item \(\)
	\item \(\)
	\item \(\)
	\item \(\)
	\end{enumerate}
\item
	\begin{enumerate}[label={\Alph*.}]
	\item \(\)
	\item \(\)
	\item \(\)
	\item \(\)
	\end{enumerate}
\item
	\begin{enumerate}[label={\Alph*.}]
	\item \(\)
	\item \(\)
	\item \(\)
	\item \(\)
	\end{enumerate}
\item
	\begin{enumerate}[label={\Alph*.}]
	\item \(\)
	\item \(\)
	\item \(\)
	\item \(\)
	\end{enumerate}
\item
	\begin{enumerate}[label={\Alph*.}]
	\item \(\)
	\item \(\)
	\item \(\)
	\item \(\)
	\end{enumerate}
\end{enumerate}
\end{multicols}
\subsection{Solution}
\begin{enumerate}[label={\arabic*.}]
    \item The minimum point can be gotten by 
    \item 
    \item
    \item
    \item
    \item 
    \item
    \item
    \item 
    \item 
    \item 
    \item 
    \item
    \item
    \item
    \item 
    \item
    \item
    \item 
    \item 
    \item 
    \item 
    \item
    \item
    \item
    \item 
    \item
    \item
    \item 
    \item 
    \item 
    \item 
    \item
    \item
    \item
    \item 
    \item
    \item
    \item 
    \item 
    \item 
    \item 
    \item
    \item
    \item
    \item 
    \item
    \item
    \item 
    \item 
    \item 
    \item 
    \item
    \item
    \item
    \item 
    \item
    \item
    \item 
    \item 
    \item 
    \item 
    \item
    \item
    \item
    \item 
    \item
    \item
    \item 
    \item 
    \item 
    \item 
    \item
    \item
    \item
    \item 
    \item
    \item
    \item 
    \item 
    \item 
    \item 
    \item
    \item
    \item
    \item 
    \item
    \item
    \item 
    \item 
    \item 
    \item 
    \item
    \item
    \item
    \item 
    \item
    \item
    \item 
    \item 
\end{enumerate}
\section{Integration}
\subsection{Questions}
\begin{multicols}{2}
\begin{enumerate}[label={\arabic*.}]
\item Find the integral of \(y = 3{x}^{2}-2x-1\)
	\begin{enumerate}[label={\Alph*.}]
	\item \({x}^{3} - {x}^{2} - x\)
	\item \({x}^{3} + {x}^{2} - x\)
	\item \({x}^{3} + {x}^{2} + x\)
	\item \({x}^{3} - {x}^{2} + x\)
	\end{enumerate}
\item Integrate the expression \(6{x}^{2} - 2x + 1\)
	\begin{enumerate}[label={\Alph*.}]
	\item \(3{x}^{3} - 2{x}^{2} + x + c\)
	\item \(2{x}^{3} - x^{2} + x + c\)
	\item \(2{x}^{3} - 3{x}^{2} + c\)
	\item \({x}^{3} + {x}^{2} - x + c\)
	\end{enumerate}
\item Integrate \(\dfrac{1}{x} + \cos {x}\) with respect to \(x\)
	\begin{enumerate}[label={\Alph*.}]
	\item \(x - \sin {x} + k\)
	\item \(x + \sin {x} - k\)
	\item \(-{\dfrac{1}{x^2}} + \sin {x} + k\)
	\item \(-{\dfrac{1}{x^2}} - \sin {x} + k\)
	\end{enumerate}
\item If the expression \(a{x}^{2} + bx + c\) equals \(5\) at \(x = 1\). If its derivative is \(2x + 1\), what are the values of \(a\),\(b\), \(c\) respectively?
	\begin{enumerate}[label={\Alph*.}]
	\item \(1, 1, 3\)
	\item \(1, 3, 1\)
	\item \(1, 2, 1\)
	\item \(2, 1, 1\)
	\end{enumerate}
\item Integrate the expression \((2x+1)^{3}\)
	\begin{enumerate}[label={\Alph*.}]
	\item \(\dfrac{(2x+1)^{3}}{8} + k\)
	\item \(\dfrac{(2x+1)^{4}}{8} + k\)
	\item \(\dfrac{(2x+1)^{4}}{6} + k\)
	\item \(\dfrac{(2x+1)^{2}}{8} + k\)
	\end{enumerate}
\item Evaluate \(\displaystyle \int 4{x}^{-3} - 7{x}^{2} + 5x - 6\ \mathrm{d}x\)
	\begin{enumerate}[label={\Alph*.}]
	\item \(-2x^{-2}-\dfrac{7}{3}x^{3}+\dfrac{5}{2}x^2-6x\)
	\item \(2x^{2}+\dfrac{7}{3}x^{3}+5x^{2}-6\)
	\item \(12x^{2}+14x-5\)
	\item \(-12x^{-4}-14x+5\)
	\end{enumerate}
\item Evaluate \(\displaystyle \int_{-1}^{2}\left(2x^{2} + x\right)\ \mathrm{d}x\)
	\begin{enumerate}[label={\Alph*.}]
	\item \(4\frac{1}{2}\)
	\item \(3\frac{1}{2}\)
	\item \(7\frac{1}{2}\)
	\item \(5\frac{1}{4}\)
	\end{enumerate}
\item Integrate \(\dfrac{x^{2}-\sqrt{x}}{x}\) with respect to \(x\)
	\begin{enumerate}[label={\Alph*.}]
	\item \(\dfrac{x^{2}}{2} - 2\sqrt{x} + k\)
	\item \(\dfrac{2(x^{2}-x)}{3x} + k\)
	\item \(\dfrac{x^{2}}{2} - \sqrt{x} + k\)
	\item \(\dfrac{x^{2}-x}{3x} + k\)
	\end{enumerate}
\item Evaluate \(\displaystyle \int_{-1}^{1} \left(2x + 1\right)^{2}\ \mathrm{d}x\)
	\begin{enumerate}[label={\Alph*.}]
	\item \(3\frac{2}{3}\)
	\item \(4\)
	\item \(4\frac{1}{3}\)
	\item \(4\frac{2}{3}\)
	\end{enumerate}
\item Evaluate \(\displaystyle \int \left(\cos{4x} + \sin{3x}\right)\ \mathrm{d}x\)
	\begin{enumerate}[label={\Alph*.}]
	\item \(\sin{4x} - \cos{3x} + k\)
	\item \(\sin{4x} + \cos{3x} + k\)
	\item \(\cfrac{1}{4}\sin{4x} - \cfrac{1}{3}\cos{3x} + k\)
	\item \(\cfrac{1}{4}\sin{4x} + \cfrac{1}{3}\cos{3x} + k\)
	\end{enumerate}
\item Evaluate \(\displaystyle \int_{0}^{\frac{\pi}{2}} \sin{x}\ \mathrm{d}x\)
	\begin{enumerate}[label={\Alph*.}]
	\item \(-2\)
	\item \(-1\)
	\item \(1\)
	\item \(2\)
	\end{enumerate}
\item Evaluate \(\displaystyle \int_{1}^{2}\frac{5}{x}\ \mathrm{d}x\)
	\begin{enumerate}[label={\Alph*.}]
	\item \(1.47\)
	\item \(2.67\)
	\item \(3.23\)
	\item \(3.47\)
	\end{enumerate}
\item Evaluate the integral \(\displaystyle \int_{\frac{\pi}{12}}^{\frac{\pi}{4}} 2\cos{2x}\ \mathrm{d}x\)
	\begin{enumerate}[label={\Alph*.}]
	\item \(-{\cfrac{1}{2}}\)
	\item \(-1\)
	\item \(\cfrac{1}{2}\)
	\item \(1\)
	\end{enumerate}
\item Evaluate \(\displaystyle \int \left(2x+3\right)^{\frac{1}{2}}\ \mathrm{d}x\)
	\begin{enumerate}[label={\Alph*.}]
	\item \(\cfrac{1}{12}(2x+3)^6 + k\)
	\item \(\cfrac{1}{3}(2x+3)^{\frac{1}{2}} + k\)
	\item \(\cfrac{1}{3}(2x+3)^{\frac{3}{2}} + k\)
	\item \(\cfrac{1}{12}(2x+3)^{\frac{3}{4}} + k\)
	\end{enumerate}
\item Evaluate \(\displaystyle \int \left(\sin{x} - 5{x}^{2}\right)\ \mathrm{d}x\)
	\begin{enumerate}[label={\Alph*.}]
	\item \(-\cos{x} - 10x + k\)
	\item \(\cos{x} - \cfrac{5x^3}{3} + k\)
	\item \(-\cos{x} - \cfrac{5x^3}{3} + k\)
	\item \(\cos{x} - 10x + k\)
	\end{enumerate}
\item Evaluate \(\displaystyle \int \sin{2x}\ \mathrm{d}x\)
	\begin{enumerate}[label={\Alph*.}]
	\item \(\cos{2x} + k\)
	\item \(\cfrac{1}{2}\cos{2x} + k\)
	\item \(-\cfrac{1}{2}\cos{2x} + k\)
	\item \(-\cos{2x} + k\)
	\end{enumerate}
\item If \(y = x(x^4 + x + 1)\), evaluate \(\displaystyle \int_{0}^{1} y\ \mathrm{d}x\)
	\begin{enumerate}[label={\Alph*.}]
	\item \(\cfrac{11}{12}\)
	\item \(1\)
	\item \(\cfrac{5}{6}\)
	\item \(0\)
	\end{enumerate}
\item Evaluate \(\displaystyle \int_{2}^{\pi} \sec^{2}{x} - \tan^{2}{x}\ \mathrm{d}x\)
	\begin{enumerate}[label={\Alph*.}]
	\item \(\cfrac{\pi}{2}\)
	\item \(\cfrac{\pi}{3}\)
	\item \(\pi - 2\)
	\item \(\pi + 2\)
	\end{enumerate}
\item Evaluate \(\displaystyle \int_{2}^{\frac{\pi}{4}} \sin{x} - \cos{x}\ \mathrm{d}x\)
	\begin{enumerate}[label={\Alph*.}]
	\item \(\sqrt{2} + 1\)
	\item \(\sqrt{2} - 1\)
	\item \(-\sqrt{2} + 1\)
	\item \(-\sqrt{2} - 1\)
	\end{enumerate}
\item Evaluate \(\displaystyle \int_{-2}^{1} \left(x - 1\right)^2\ \mathrm{d}x\)
	\begin{enumerate}[label={\Alph*.}]
	\item \(-{\cfrac{10}{3}}\)
	\item \(7\)
	\item \(9\)
	\item \(11\)
	\end{enumerate}
\item A function \(f(x)\) passes through the origin and its first derivative is \(3x + 2\). What is \(f(x)\)?
	\begin{enumerate}[label={\Alph*.}]
	\item \(f(x) = \cfrac{3{x}^{2}}{2} + 2x \)
	\item \(f(x) = \cfrac{3{x}^{2}}{2} + x\)
	\item \(f(x) = 3{x}^{2} + \cfrac{x}{2}\)
	\item \(f(x) = 3{x}^{2} + 2x\)
	\end{enumerate}
\item Evaluate \(\displaystyle \int_{3}^{2} \left(x^2-2x\right)\ \mathrm{d}x\)
	\begin{enumerate}[label={\Alph*.}]
	\item \(4\)
	\item \(2\)
	\item \(\cfrac{4}{3}\)
	\item \(\cfrac{1}{3}\)
	\end{enumerate}
\item Evaluate \(\displaystyle \int_{-4}^{0} \left(1-2x\right)\ \mathrm{d}x\)
	\begin{enumerate}[label={\Alph*.}]
	\item \(-20\)
	\item \(-16\)
	\item \(10\)
	\item \(20\)
	\end{enumerate}
\item Evaluate \(\displaystyle \int_{1}^{2} \left(6x^2-2x\right)\ \mathrm{d}x\)
	\begin{enumerate}[label={\Alph*.}]
	\item \(11\)
	\item \(12\)
	\item \(13\)
	\item \(16\)
	\end{enumerate}
\item Evaluate \(\displaystyle \int_{-{\frac{\pi}{2}}}^{\frac{\pi}{2}} \cos{x}\ \mathrm{d}x\)
	\begin{enumerate}[label={\Alph*.}]
	\item \(0\)
	\item \(1\)
	\item \(2\)
	\item \(3\)
	\end{enumerate}
\item Evaluate \(\displaystyle \int_{0}^{2} \left({x}^{3}+{x}^{2}\right)\ \mathrm{d}x\)
	\begin{enumerate}[label={\Alph*.}]
	\item \(4\frac{5}{6}\)
	\item \(6\frac{2}{3}\)
	\item \(1\frac{5}{6}\)
	\item \(2\frac{5}{6}\)
	\end{enumerate}
\item Evaluate \(\displaystyle \int_{2}^{1} \left(3-2x\right)\ \mathrm{d}x\)
	\begin{enumerate}[label={\Alph*.}]
	\item \(3\)
	\item \(5\)
	\item \(2\)
	\item \(6\)
	\end{enumerate}
\item Evaluate \(\displaystyle \int_{1}^{2} \left({x}^{2}-4x\right)\ \mathrm{d}x\)
	\begin{enumerate}[label={\Alph*.}]
	\item \(\cfrac{11}{3}\)
	\item \(\cfrac{3}{11}\)
	\item \(-{\cfrac{3}{11}}\)
	\item \(-{\cfrac{11}{3}}\)
	\end{enumerate}
\item Evaluate \(\displaystyle \int \left(\sin{x}+2\right)\ \mathrm{d}x\)
	\begin{enumerate}[label={\Alph*.}]
	\item \(\cos{x} + {x}^{2} + k\)
	\item \(\cos{x} + 2x + k\)
	\item \(-{\cos{x}} + {x}^{2} + k\)
	\item \(-{\cos{x}} + 2x + k\)
	\end{enumerate}
\item Evaluate \(\displaystyle \int_{0}^{1} \left(\cos{4x}\right)\ \mathrm{d}x\)
	\begin{enumerate}[label={\Alph*.}]
	\item \(\cfrac{3}{4}\sin{4x} + k\)
	\item \(-{\cfrac{1}{4}\sin{4x}} + k\)
	\item \(-{\cfrac{3}{4}\sin{4x}} + k\)
	\item \({\cfrac{1}{4}\sin{4x}} + k\)
	\end{enumerate}
\item Integrate \(\dfrac{1+x}{{x}^{3}}\) with respect to \(x\)
	\begin{enumerate}[label={\Alph*.}]
	\item \(2{x}^{2} -\cfrac{1}{x} + k \)
	\item \({x}^{2} -\cfrac{1}{x} + k \)
	\item \(-{\cfrac{{x}^{2}}{2}} -\cfrac{1}{x} + k \)
	\item \(-{\cfrac{1}{{2x}^{2}}} -\cfrac{1}{x} + k \)
	\end{enumerate}
\item Evaluate \(\displaystyle \int \left({x}^{2}+3x-5\right)\ \mathrm{d}x\)
	\begin{enumerate}[label={\Alph*.}]
	\item \({\cfrac{{x}^{3}}{3}} - {\cfrac{3{x}^{2}}{2}} - 5x + k \)
	\item \({\cfrac{{x}^{3}}{3}} - {\cfrac{3{x}^{2}}{2}} + 5x + k \)
	\item \({\cfrac{{x}^{3}}{3}} + {\cfrac{3{x}^{2}}{2}} - 5x + k \)
	\item \({\cfrac{{x}^{3}}{3}} + {\cfrac{3{x}^{2}}{2}} + 5x + k \)
	\end{enumerate}
\item Integrate \(\dfrac{2x^3+2x}{x}\) with respect to \(x\)
	\begin{enumerate}[label={\Alph*.}]
	\item \({\cfrac{2{x}^{3}}{3}} - 2x + k \)
	\item \({\cfrac{2{x}^{3}}{3}} + 2x + k \)
	\item \({x}^{3} - 2x + k \)
	\item \({x}^{3} + 2x + k \)
	\end{enumerate}
\item Evaluate \(\displaystyle \int \left(5{x}^{3} + 7{x}^{2} -2x + 5\right)\ \mathrm{d}x\)
	\begin{enumerate}[label={\Alph*.}]
	\item \({\cfrac{5{x}^{4}}{4}} + {\cfrac{7{x}^{3}}{3}} + 2x + C \)
	\item \({\cfrac{5{x}^{4}}{4}} + {\cfrac{7{x}^{3}}{3}} - {x}^{2} + 5x + C \)
	\item \({\cfrac{5{x}^{3}}{3}} + {\cfrac{7{x}^{2}}{2}} - {x} + C \)
	\item \({\cfrac{2{x}^{2}}{3}} + {\cfrac{x}{5}} - C \)
	\end{enumerate}
\item
	\begin{enumerate}[label={\Alph*.}]
	\item \(\)
	\item \(\)
	\item \(\)
	\item \(\)
	\end{enumerate}
\item
	\begin{enumerate}[label={\Alph*.}]
	\item \(\)
	\item \(\)
	\item \(\)
	\item \(\)
	\end{enumerate}
\item
	\begin{enumerate}[label={\Alph*.}]
	\item \(\)
	\item \(\)
	\item \(\)
	\item \(\)
	\end{enumerate}
\item
	\begin{enumerate}[label={\Alph*.}]
	\item \(\)
	\item \(\)
	\item \(\)
	\item \(\)
	\end{enumerate}
\item
	\begin{enumerate}[label={\Alph*.}]
	\item \(\)
	\item \(\)
	\item \(\)
	\item \(\)
	\end{enumerate}
\item
	\begin{enumerate}[label={\Alph*.}]
	\item \(\)
	\item \(\)
	\item \(\)
	\item \(\)
	\end{enumerate}
\item
	\begin{enumerate}[label={\Alph*.}]
	\item \(\)
	\item \(\)
	\item \(\)
	\item \(\)
	\end{enumerate}
\item
	\begin{enumerate}[label={\Alph*.}]
	\item \(\)
	\item \(\)
	\item \(\)
	\item \(\)
	\end{enumerate}
\item
	\begin{enumerate}[label={\Alph*.}]
	\item \(\)
	\item \(\)
	\item \(\)
	\item \(\)
	\end{enumerate}
\item
	\begin{enumerate}[label={\Alph*.}]
	\item \(\)
	\item \(\)
	\item \(\)
	\item \(\)
	\end{enumerate}
\item
	\begin{enumerate}[label={\Alph*.}]
	\item \(\)
	\item \(\)
	\item \(\)
	\item \(\)
	\end{enumerate}
\item
	\begin{enumerate}[label={\Alph*.}]
	\item \(\)
	\item \(\)
	\item \(\)
	\item \(\)
	\end{enumerate}
\item
	\begin{enumerate}[label={\Alph*.}]
	\item \(\)
	\item \(\)
	\item \(\)
	\item \(\)
	\end{enumerate}
\item
	\begin{enumerate}[label={\Alph*.}]
	\item \(\)
	\item \(\)
	\item \(\)
	\item \(\)
	\end{enumerate}
\item
	\begin{enumerate}[label={\Alph*.}]
	\item \(\)
	\item \(\)
	\item \(\)
	\item \(\)
	\end{enumerate}
\item
	\begin{enumerate}[label={\Alph*.}]
	\item \(\)
	\item \(\)
	\item \(\)
	\item \(\)
	\end{enumerate}
\item
	\begin{enumerate}[label={\Alph*.}]
	\item \(\)
	\item \(\)
	\item \(\)
	\item \(\)
	\end{enumerate}
\item
	\begin{enumerate}[label={\Alph*.}]
	\item \(\)
	\item \(\)
	\item \(\)
	\item \(\)
	\end{enumerate}
\item
	\begin{enumerate}[label={\Alph*.}]
	\item \(\)
	\item \(\)
	\item \(\)
	\item \(\)
	\end{enumerate}
\item
	\begin{enumerate}[label={\Alph*.}]
	\item \(\)
	\item \(\)
	\item \(\)
	\item \(\)
	\end{enumerate}
\item
	\begin{enumerate}[label={\Alph*.}]
	\item \(\)
	\item \(\)
	\item \(\)
	\item \(\)
	\end{enumerate}
\item
	\begin{enumerate}[label={\Alph*.}]
	\item \(\)
	\item \(\)
	\item \(\)
	\item \(\)
	\end{enumerate}
\item
	\begin{enumerate}[label={\Alph*.}]
	\item \(\)
	\item \(\)
	\item \(\)
	\item \(\)
	\end{enumerate}
\item
	\begin{enumerate}[label={\Alph*.}]
	\item \(\)
	\item \(\)
	\item \(\)
	\item \(\)
	\end{enumerate}
\item
	\begin{enumerate}[label={\Alph*.}]
	\item \(\)
	\item \(\)
	\item \(\)
	\item \(\)
	\end{enumerate}
\item
	\begin{enumerate}[label={\Alph*.}]
	\item \(\)
	\item \(\)
	\item \(\)
	\item \(\)
	\end{enumerate}
\item
	\begin{enumerate}[label={\Alph*.}]
	\item \(\)
	\item \(\)
	\item \(\)
	\item \(\)
	\end{enumerate}
\item
	\begin{enumerate}[label={\Alph*.}]
	\item \(\)
	\item \(\)
	\item \(\)
	\item \(\)
	\end{enumerate}
\item
	\begin{enumerate}[label={\Alph*.}]
	\item \(\)
	\item \(\)
	\item \(\)
	\item \(\)
	\end{enumerate}
\item
	\begin{enumerate}[label={\Alph*.}]
	\item \(\)
	\item \(\)
	\item \(\)
	\item \(\)
	\end{enumerate}
\item
	\begin{enumerate}[label={\Alph*.}]
	\item \(\)
	\item \(\)
	\item \(\)
	\item \(\)
	\end{enumerate}
\item
	\begin{enumerate}[label={\Alph*.}]
	\item \(\)
	\item \(\)
	\item \(\)
	\item \(\)
	\end{enumerate}
\item
	\begin{enumerate}[label={\Alph*.}]
	\item \(\)
	\item \(\)
	\item \(\)
	\item \(\)
	\end{enumerate}
\item
	\begin{enumerate}[label={\Alph*.}]
	\item \(\)
	\item \(\)
	\item \(\)
	\item \(\)
	\end{enumerate}
\item
	\begin{enumerate}[label={\Alph*.}]
	\item \(\)
	\item \(\)
	\item \(\)
	\item \(\)
	\end{enumerate}
\item
	\begin{enumerate}[label={\Alph*.}]
	\item \(\)
	\item \(\)
	\item \(\)
	\item \(\)
	\end{enumerate}
\item
	\begin{enumerate}[label={\Alph*.}]
	\item \(\)
	\item \(\)
	\item \(\)
	\item \(\)
	\end{enumerate}
\item
	\begin{enumerate}[label={\Alph*.}]
	\item \(\)
	\item \(\)
	\item \(\)
	\item \(\)
	\end{enumerate}
\item
	\begin{enumerate}[label={\Alph*.}]
	\item \(\)
	\item \(\)
	\item \(\)
	\item \(\)
	\end{enumerate}
\item
	\begin{enumerate}[label={\Alph*.}]
	\item \(\)
	\item \(\)
	\item \(\)
	\item \(\)
	\end{enumerate}
\item
	\begin{enumerate}[label={\Alph*.}]
	\item \(\)
	\item \(\)
	\item \(\)
	\item \(\)
	\end{enumerate}


\item
	\begin{enumerate}[label={\Alph*.}]
	\item \(\)
	\item \(\)
	\item \(\)
	\item \(\)
	\end{enumerate}
\item
	\begin{enumerate}[label={\Alph*.}]
	\item \(\)
	\item \(\)
	\item \(\)
	\item \(\)
	\end{enumerate}
\item
	\begin{enumerate}[label={\Alph*.}]
	\item \(\)
	\item \(\)
	\item \(\)
	\item \(\)
	\end{enumerate}
\item
	\begin{enumerate}[label={\Alph*.}]
	\item \(\)
	\item \(\)
	\item \(\)
	\item \(\)
	\end{enumerate}
\item
	\begin{enumerate}[label={\Alph*.}]
	\item \(\)
	\item \(\)
	\item \(\)
	\item \(\)
	\end{enumerate}
\item
	\begin{enumerate}[label={\Alph*.}]
	\item \(\)
	\item \(\)
	\item \(\)
	\item \(\)
	\end{enumerate}
\item
	\begin{enumerate}[label={\Alph*.}]
	\item \(\)
	\item \(\)
	\item \(\)
	\item \(\)
	\end{enumerate}
\item
	\begin{enumerate}[label={\Alph*.}]
	\item \(\)
	\item \(\)
	\item \(\)
	\item \(\)
	\end{enumerate}
\item
	\begin{enumerate}[label={\Alph*.}]
	\item \(\)
	\item \(\)
	\item \(\)
	\item \(\)
	\end{enumerate}
\item
	\begin{enumerate}[label={\Alph*.}]
	\item \(\)
	\item \(\)
	\item \(\)
	\item \(\)
	\end{enumerate}
\item
	\begin{enumerate}[label={\Alph*.}]
	\item \(\)
	\item \(\)
	\item \(\)
	\item \(\)
	\end{enumerate}
\item
	\begin{enumerate}[label={\Alph*.}]
	\item \(\)
	\item \(\)
	\item \(\)
	\item \(\)
	\end{enumerate}
\item
	\begin{enumerate}[label={\Alph*.}]
	\item \(\)
	\item \(\)
	\item \(\)
	\item \(\)
	\end{enumerate}
\item
	\begin{enumerate}[label={\Alph*.}]
	\item \(\)
	\item \(\)
	\item \(\)
	\item \(\)
	\end{enumerate}
\item
	\begin{enumerate}[label={\Alph*.}]
	\item \(\)
	\item \(\)
	\item \(\)
	\item \(\)
	\end{enumerate}
\item
	\begin{enumerate}[label={\Alph*.}]
	\item \(\)
	\item \(\)
	\item \(\)
	\item \(\)
	\end{enumerate}
\item
	\begin{enumerate}[label={\Alph*.}]
	\item \(\)
	\item \(\)
	\item \(\)
	\item \(\)
	\end{enumerate}
\item
	\begin{enumerate}[label={\Alph*.}]
	\item \(\)
	\item \(\)
	\item \(\)
	\item \(\)
	\end{enumerate}
\item
	\begin{enumerate}[label={\Alph*.}]
	\item \(\)
	\item \(\)
	\item \(\)
	\item \(\)
	\end{enumerate}
\item
	\begin{enumerate}[label={\Alph*.}]
	\item \(\)
	\item \(\)
	\item \(\)
	\item \(\)
	\end{enumerate}
\item
	\begin{enumerate}[label={\Alph*.}]
	\item \(\)
	\item \(\)
	\item \(\)
	\item \(\)
	\end{enumerate}
\item
	\begin{enumerate}[label={\Alph*.}]
	\item \(\)
	\item \(\)
	\item \(\)
	\item \(\)
	\end{enumerate}
\item
	\begin{enumerate}[label={\Alph*.}]
	\item \(\)
	\item \(\)
	\item \(\)
	\item \(\)
	\end{enumerate}
\item
	\begin{enumerate}[label={\Alph*.}]
	\item \(\)
	\item \(\)
	\item \(\)
	\item \(\)
	\end{enumerate}
\item
	\begin{enumerate}[label={\Alph*.}]
	\item \(\)
	\item \(\)
	\item \(\)
	\item \(\)
	\end{enumerate}
\end{enumerate}
\end{multicols}
\subsection{Solution}
\begin{enumerate}[label={\arabic*.}]
    \item 
    \item 
    \item
    \item
    \item
    \item 
    \item
    \item
    \item 
    \item 
    \item 
    \item 
    \item
    \item
    \item
    \item 
    \item
    \item
    \item 
    \item 
    \item 
    \item 
    \item
    \item
    \item
    \item 
    \item
    \item
    \item 
    \item 
    \item 
    \item 
    \item
    \item
    \item
    \item 
    \item
    \item
    \item 
    \item 
    \item 
    \item 
    \item
    \item
    \item
    \item 
    \item
    \item
    \item 
    \item 
    \item 
    \item 
    \item
    \item
    \item
    \item 
    \item
    \item
    \item 
    \item 
    \item 
    \item 
    \item
    \item
    \item
    \item 
    \item
    \item
    \item 
    \item 
    \item 
    \item 
    \item
    \item
    \item
    \item 
    \item
    \item
    \item 
    \item 
    \item 
    \item 
    \item
    \item
    \item
    \item 
    \item
    \item
    \item 
    \item 
    \item 
    \item 
    \item
    \item
    \item
    \item 
    \item
    \item
    \item 
    \item 
\end{enumerate}
\chapter{Combinatorics}
\section{Combination \& Permutation}
\subsection{Questions}
\begin{enumerate}[label={\arabic*.}]
\item
	\begin{enumerate}[label={\Alph*.}]
	\item \(\)
	\item \(\)
	\item \(\)
	\item \(\)
	\end{enumerate}
\item
	\begin{enumerate}[label={\Alph*.}]
	\item \(\)
	\item \(\)
	\item \(\)
	\item \(\)
	\end{enumerate}
\item
	\begin{enumerate}[label={\Alph*.}]
	\item \(\)
	\item \(\)
	\item \(\)
	\item \(\)
	\end{enumerate}
\item
	\begin{enumerate}[label={\Alph*.}]
	\item \(\)
	\item \(\)
	\item \(\)
	\item \(\)
	\end{enumerate}
\item
	\begin{enumerate}[label={\Alph*.}]
	\item \(\)
	\item \(\)
	\item \(\)
	\item \(\)
	\end{enumerate}
\item
	\begin{enumerate}[label={\Alph*.}]
	\item \(\)
	\item \(\)
	\item \(\)
	\item \(\)
	\end{enumerate}
\item
	\begin{enumerate}[label={\Alph*.}]
	\item \(\)
	\item \(\)
	\item \(\)
	\item \(\)
	\end{enumerate}
\item
	\begin{enumerate}[label={\Alph*.}]
	\item \(\)
	\item \(\)
	\item \(\)
	\item \(\)
	\end{enumerate}
\item
	\begin{enumerate}[label={\Alph*.}]
	\item \(\)
	\item \(\)
	\item \(\)
	\item \(\)
	\end{enumerate}
\item
	\begin{enumerate}[label={\Alph*.}]
	\item \(\)
	\item \(\)
	\item \(\)
	\item \(\)
	\end{enumerate}
\item
	\begin{enumerate}[label={\Alph*.}]
	\item \(\)
	\item \(\)
	\item \(\)
	\item \(\)
	\end{enumerate}


\item
	\begin{enumerate}[label={\Alph*.}]
	\item \(\)
	\item \(\)
	\item \(\)
	\item \(\)
	\end{enumerate}
\item
	\begin{enumerate}[label={\Alph*.}]
	\item \(\)
	\item \(\)
	\item \(\)
	\item \(\)
	\end{enumerate}
\item
	\begin{enumerate}[label={\Alph*.}]
	\item \(\)
	\item \(\)
	\item \(\)
	\item \(\)
	\end{enumerate}
\item
	\begin{enumerate}[label={\Alph*.}]
	\item \(\)
	\item \(\)
	\item \(\)
	\item \(\)
	\end{enumerate}
\item
	\begin{enumerate}[label={\Alph*.}]
	\item \(\)
	\item \(\)
	\item \(\)
	\item \(\)
	\end{enumerate}
\item
	\begin{enumerate}[label={\Alph*.}]
	\item \(\)
	\item \(\)
	\item \(\)
	\item \(\)
	\end{enumerate}
\item
	\begin{enumerate}[label={\Alph*.}]
	\item \(\)
	\item \(\)
	\item \(\)
	\item \(\)
	\end{enumerate}
\item
	\begin{enumerate}[label={\Alph*.}]
	\item \(\)
	\item \(\)
	\item \(\)
	\item \(\)
	\end{enumerate}
\item
	\begin{enumerate}[label={\Alph*.}]
	\item \(\)
	\item \(\)
	\item \(\)
	\item \(\)
	\end{enumerate}
\item
	\begin{enumerate}[label={\Alph*.}]
	\item \(\)
	\item \(\)
	\item \(\)
	\item \(\)
	\end{enumerate}
\item
	\begin{enumerate}[label={\Alph*.}]
	\item \(\)
	\item \(\)
	\item \(\)
	\item \(\)
	\end{enumerate}
\item
	\begin{enumerate}[label={\Alph*.}]
	\item \(\)
	\item \(\)
	\item \(\)
	\item \(\)
	\end{enumerate}
\item
	\begin{enumerate}[label={\Alph*.}]
	\item \(\)
	\item \(\)
	\item \(\)
	\item \(\)
	\end{enumerate}
\item
	\begin{enumerate}[label={\Alph*.}]
	\item \(\)
	\item \(\)
	\item \(\)
	\item \(\)
	\end{enumerate}
\item
	\begin{enumerate}[label={\Alph*.}]
	\item \(\)
	\item \(\)
	\item \(\)
	\item \(\)
	\end{enumerate}
\item
	\begin{enumerate}[label={\Alph*.}]
	\item \(\)
	\item \(\)
	\item \(\)
	\item \(\)
	\end{enumerate}
\item
	\begin{enumerate}[label={\Alph*.}]
	\item \(\)
	\item \(\)
	\item \(\)
	\item \(\)
	\end{enumerate}
\item
	\begin{enumerate}[label={\Alph*.}]
	\item \(\)
	\item \(\)
	\item \(\)
	\item \(\)
	\end{enumerate}
\item
	\begin{enumerate}[label={\Alph*.}]
	\item \(\)
	\item \(\)
	\item \(\)
	\item \(\)
	\end{enumerate}
\item
	\begin{enumerate}[label={\Alph*.}]
	\item \(\)
	\item \(\)
	\item \(\)
	\item \(\)
	\end{enumerate}
\item
	\begin{enumerate}[label={\Alph*.}]
	\item \(\)
	\item \(\)
	\item \(\)
	\item \(\)
	\end{enumerate}
\item
	\begin{enumerate}[label={\Alph*.}]
	\item \(\)
	\item \(\)
	\item \(\)
	\item \(\)
	\end{enumerate}
\item
	\begin{enumerate}[label={\Alph*.}]
	\item \(\)
	\item \(\)
	\item \(\)
	\item \(\)
	\end{enumerate}
\item
	\begin{enumerate}[label={\Alph*.}]
	\item \(\)
	\item \(\)
	\item \(\)
	\item \(\)
	\end{enumerate}
\item
	\begin{enumerate}[label={\Alph*.}]
	\item \(\)
	\item \(\)
	\item \(\)
	\item \(\)
	\end{enumerate}
\item
	\begin{enumerate}[label={\Alph*.}]
	\item \(\)
	\item \(\)
	\item \(\)
	\item \(\)
	\end{enumerate}
\item
	\begin{enumerate}[label={\Alph*.}]
	\item \(\)
	\item \(\)
	\item \(\)
	\item \(\)
	\end{enumerate}
\item
	\begin{enumerate}[label={\Alph*.}]
	\item \(\)
	\item \(\)
	\item \(\)
	\item \(\)
	\end{enumerate}
\item
	\begin{enumerate}[label={\Alph*.}]
	\item \(\)
	\item \(\)
	\item \(\)
	\item \(\)
	\end{enumerate}
\item
	\begin{enumerate}[label={\Alph*.}]
	\item \(\)
	\item \(\)
	\item \(\)
	\item \(\)
	\end{enumerate}
\item
	\begin{enumerate}[label={\Alph*.}]
	\item \(\)
	\item \(\)
	\item \(\)
	\item \(\)
	\end{enumerate}
\item
	\begin{enumerate}[label={\Alph*.}]
	\item \(\)
	\item \(\)
	\item \(\)
	\item \(\)
	\end{enumerate}
\item
	\begin{enumerate}[label={\Alph*.}]
	\item \(\)
	\item \(\)
	\item \(\)
	\item \(\)
	\end{enumerate}
\item
	\begin{enumerate}[label={\Alph*.}]
	\item \(\)
	\item \(\)
	\item \(\)
	\item \(\)
	\end{enumerate}
\item
	\begin{enumerate}[label={\Alph*.}]
	\item \(\)
	\item \(\)
	\item \(\)
	\item \(\)
	\end{enumerate}
\item
	\begin{enumerate}[label={\Alph*.}]
	\item \(\)
	\item \(\)
	\item \(\)
	\item \(\)
	\end{enumerate}
\item
	\begin{enumerate}[label={\Alph*.}]
	\item \(\)
	\item \(\)
	\item \(\)
	\item \(\)
	\end{enumerate}
\item
	\begin{enumerate}[label={\Alph*.}]
	\item \(\)
	\item \(\)
	\item \(\)
	\item \(\)
	\end{enumerate}
\item
	\begin{enumerate}[label={\Alph*.}]
	\item \(\)
	\item \(\)
	\item \(\)
	\item \(\)
	\end{enumerate}
\item
	\begin{enumerate}[label={\Alph*.}]
	\item \(\)
	\item \(\)
	\item \(\)
	\item \(\)
	\end{enumerate}
\item
	\begin{enumerate}[label={\Alph*.}]
	\item \(\)
	\item \(\)
	\item \(\)
	\item \(\)
	\end{enumerate}
\item
	\begin{enumerate}[label={\Alph*.}]
	\item \(\)
	\item \(\)
	\item \(\)
	\item \(\)
	\end{enumerate}
\item
	\begin{enumerate}[label={\Alph*.}]
	\item \(\)
	\item \(\)
	\item \(\)
	\item \(\)
	\end{enumerate}
\item
	\begin{enumerate}[label={\Alph*.}]
	\item \(\)
	\item \(\)
	\item \(\)
	\item \(\)
	\end{enumerate}
\item
	\begin{enumerate}[label={\Alph*.}]
	\item \(\)
	\item \(\)
	\item \(\)
	\item \(\)
	\end{enumerate}
\item
	\begin{enumerate}[label={\Alph*.}]
	\item \(\)
	\item \(\)
	\item \(\)
	\item \(\)
	\end{enumerate}
\item
	\begin{enumerate}[label={\Alph*.}]
	\item \(\)
	\item \(\)
	\item \(\)
	\item \(\)
	\end{enumerate}
\item
	\begin{enumerate}[label={\Alph*.}]
	\item \(\)
	\item \(\)
	\item \(\)
	\item \(\)
	\end{enumerate}
\item
	\begin{enumerate}[label={\Alph*.}]
	\item \(\)
	\item \(\)
	\item \(\)
	\item \(\)
	\end{enumerate}
\item
	\begin{enumerate}[label={\Alph*.}]
	\item \(\)
	\item \(\)
	\item \(\)
	\item \(\)
	\end{enumerate}
\item
	\begin{enumerate}[label={\Alph*.}]
	\item \(\)
	\item \(\)
	\item \(\)
	\item \(\)
	\end{enumerate}
\item
	\begin{enumerate}[label={\Alph*.}]
	\item \(\)
	\item \(\)
	\item \(\)
	\item \(\)
	\end{enumerate}
\item
	\begin{enumerate}[label={\Alph*.}]
	\item \(\)
	\item \(\)
	\item \(\)
	\item \(\)
	\end{enumerate}
\item
	\begin{enumerate}[label={\Alph*.}]
	\item \(\)
	\item \(\)
	\item \(\)
	\item \(\)
	\end{enumerate}
\item
	\begin{enumerate}[label={\Alph*.}]
	\item \(\)
	\item \(\)
	\item \(\)
	\item \(\)
	\end{enumerate}
\item
	\begin{enumerate}[label={\Alph*.}]
	\item \(\)
	\item \(\)
	\item \(\)
	\item \(\)
	\end{enumerate}
\item
	\begin{enumerate}[label={\Alph*.}]
	\item \(\)
	\item \(\)
	\item \(\)
	\item \(\)
	\end{enumerate}
\item
	\begin{enumerate}[label={\Alph*.}]
	\item \(\)
	\item \(\)
	\item \(\)
	\item \(\)
	\end{enumerate}
\item
	\begin{enumerate}[label={\Alph*.}]
	\item \(\)
	\item \(\)
	\item \(\)
	\item \(\)
	\end{enumerate}
\item
	\begin{enumerate}[label={\Alph*.}]
	\item \(\)
	\item \(\)
	\item \(\)
	\item \(\)
	\end{enumerate}
\item
	\begin{enumerate}[label={\Alph*.}]
	\item \(\)
	\item \(\)
	\item \(\)
	\item \(\)
	\end{enumerate}
\item
	\begin{enumerate}[label={\Alph*.}]
	\item \(\)
	\item \(\)
	\item \(\)
	\item \(\)
	\end{enumerate}
\item
	\begin{enumerate}[label={\Alph*.}]
	\item \(\)
	\item \(\)
	\item \(\)
	\item \(\)
	\end{enumerate}
\item
	\begin{enumerate}[label={\Alph*.}]
	\item \(\)
	\item \(\)
	\item \(\)
	\item \(\)
	\end{enumerate}


\item
	\begin{enumerate}[label={\Alph*.}]
	\item \(\)
	\item \(\)
	\item \(\)
	\item \(\)
	\end{enumerate}
\item
	\begin{enumerate}[label={\Alph*.}]
	\item \(\)
	\item \(\)
	\item \(\)
	\item \(\)
	\end{enumerate}
\item
	\begin{enumerate}[label={\Alph*.}]
	\item \(\)
	\item \(\)
	\item \(\)
	\item \(\)
	\end{enumerate}
\item
	\begin{enumerate}[label={\Alph*.}]
	\item \(\)
	\item \(\)
	\item \(\)
	\item \(\)
	\end{enumerate}
\item
	\begin{enumerate}[label={\Alph*.}]
	\item \(\)
	\item \(\)
	\item \(\)
	\item \(\)
	\end{enumerate}
\item
	\begin{enumerate}[label={\Alph*.}]
	\item \(\)
	\item \(\)
	\item \(\)
	\item \(\)
	\end{enumerate}
\item
	\begin{enumerate}[label={\Alph*.}]
	\item \(\)
	\item \(\)
	\item \(\)
	\item \(\)
	\end{enumerate}
\item
	\begin{enumerate}[label={\Alph*.}]
	\item \(\)
	\item \(\)
	\item \(\)
	\item \(\)
	\end{enumerate}
\item
	\begin{enumerate}[label={\Alph*.}]
	\item \(\)
	\item \(\)
	\item \(\)
	\item \(\)
	\end{enumerate}
\item
	\begin{enumerate}[label={\Alph*.}]
	\item \(\)
	\item \(\)
	\item \(\)
	\item \(\)
	\end{enumerate}
\item
	\begin{enumerate}[label={\Alph*.}]
	\item \(\)
	\item \(\)
	\item \(\)
	\item \(\)
	\end{enumerate}
\item
	\begin{enumerate}[label={\Alph*.}]
	\item \(\)
	\item \(\)
	\item \(\)
	\item \(\)
	\end{enumerate}
\item
	\begin{enumerate}[label={\Alph*.}]
	\item \(\)
	\item \(\)
	\item \(\)
	\item \(\)
	\end{enumerate}
\item
	\begin{enumerate}[label={\Alph*.}]
	\item \(\)
	\item \(\)
	\item \(\)
	\item \(\)
	\end{enumerate}
\item
	\begin{enumerate}[label={\Alph*.}]
	\item \(\)
	\item \(\)
	\item \(\)
	\item \(\)
	\end{enumerate}
\item
	\begin{enumerate}[label={\Alph*.}]
	\item \(\)
	\item \(\)
	\item \(\)
	\item \(\)
	\end{enumerate}
\item
	\begin{enumerate}[label={\Alph*.}]
	\item \(\)
	\item \(\)
	\item \(\)
	\item \(\)
	\end{enumerate}
\item
	\begin{enumerate}[label={\Alph*.}]
	\item \(\)
	\item \(\)
	\item \(\)
	\item \(\)
	\end{enumerate}
\item
	\begin{enumerate}[label={\Alph*.}]
	\item \(\)
	\item \(\)
	\item \(\)
	\item \(\)
	\end{enumerate}
\item
	\begin{enumerate}[label={\Alph*.}]
	\item \(\)
	\item \(\)
	\item \(\)
	\item \(\)
	\end{enumerate}
\item
	\begin{enumerate}[label={\Alph*.}]
	\item \(\)
	\item \(\)
	\item \(\)
	\item \(\)
	\end{enumerate}
\item
	\begin{enumerate}[label={\Alph*.}]
	\item \(\)
	\item \(\)
	\item \(\)
	\item \(\)
	\end{enumerate}
\item
	\begin{enumerate}[label={\Alph*.}]
	\item \(\)
	\item \(\)
	\item \(\)
	\item \(\)
	\end{enumerate}
\item
	\begin{enumerate}[label={\Alph*.}]
	\item \(\)
	\item \(\)
	\item \(\)
	\item \(\)
	\end{enumerate}
\item
	\begin{enumerate}[label={\Alph*.}]
	\item \(\)
	\item \(\)
	\item \(\)
	\item \(\)
	\end{enumerate}
\end{enumerate}
\subsection{Solution}
\begin{enumerate}[label={\arabic*.}]
    \item 
    \item 
    \item
    \item
    \item
    \item 
    \item
    \item
    \item 
    \item 
    \item 
    \item 
    \item
    \item
    \item
    \item 
    \item
    \item
    \item 
    \item 
    \item 
    \item 
    \item
    \item
    \item
    \item 
    \item
    \item
    \item 
    \item 
    \item 
    \item 
    \item
    \item
    \item
    \item 
    \item
    \item
    \item 
    \item 
    \item 
    \item 
    \item
    \item
    \item
    \item 
    \item
    \item
    \item 
    \item 
    \item 
    \item 
    \item
    \item
    \item
    \item 
    \item
    \item
    \item 
    \item 
    \item 
    \item 
    \item
    \item
    \item
    \item 
    \item
    \item
    \item 
    \item 
    \item 
    \item 
    \item
    \item
    \item
    \item 
    \item
    \item
    \item 
    \item 
    \item 
    \item 
    \item
    \item
    \item
    \item 
    \item
    \item
    \item 
    \item 
    \item 
    \item 
    \item
    \item
    \item
    \item 
    \item
    \item
    \item 
    \item 
\end{enumerate}
\chapter{Statistics}
\section{Integration}
\subsection{Questions}
\begin{enumerate}[label={\arabic*.}]
\item
	\begin{enumerate}[label={\Alph*.}]
	\item \(\)
	\item \(\)
	\item \(\)
	\item \(\)
	\end{enumerate}
\item
	\begin{enumerate}[label={\Alph*.}]
	\item \(\)
	\item \(\)
	\item \(\)
	\item \(\)
	\end{enumerate}
\item
	\begin{enumerate}[label={\Alph*.}]
	\item \(\)
	\item \(\)
	\item \(\)
	\item \(\)
	\end{enumerate}
\item
	\begin{enumerate}[label={\Alph*.}]
	\item \(\)
	\item \(\)
	\item \(\)
	\item \(\)
	\end{enumerate}
\item
	\begin{enumerate}[label={\Alph*.}]
	\item \(\)
	\item \(\)
	\item \(\)
	\item \(\)
	\end{enumerate}
\item
	\begin{enumerate}[label={\Alph*.}]
	\item \(\)
	\item \(\)
	\item \(\)
	\item \(\)
	\end{enumerate}
\item
	\begin{enumerate}[label={\Alph*.}]
	\item \(\)
	\item \(\)
	\item \(\)
	\item \(\)
	\end{enumerate}
\item
	\begin{enumerate}[label={\Alph*.}]
	\item \(\)
	\item \(\)
	\item \(\)
	\item \(\)
	\end{enumerate}
\item
	\begin{enumerate}[label={\Alph*.}]
	\item \(\)
	\item \(\)
	\item \(\)
	\item \(\)
	\end{enumerate}
\item
	\begin{enumerate}[label={\Alph*.}]
	\item \(\)
	\item \(\)
	\item \(\)
	\item \(\)
	\end{enumerate}
\item
	\begin{enumerate}[label={\Alph*.}]
	\item \(\)
	\item \(\)
	\item \(\)
	\item \(\)
	\end{enumerate}


\item
	\begin{enumerate}[label={\Alph*.}]
	\item \(\)
	\item \(\)
	\item \(\)
	\item \(\)
	\end{enumerate}
\item
	\begin{enumerate}[label={\Alph*.}]
	\item \(\)
	\item \(\)
	\item \(\)
	\item \(\)
	\end{enumerate}
\item
	\begin{enumerate}[label={\Alph*.}]
	\item \(\)
	\item \(\)
	\item \(\)
	\item \(\)
	\end{enumerate}
\item
	\begin{enumerate}[label={\Alph*.}]
	\item \(\)
	\item \(\)
	\item \(\)
	\item \(\)
	\end{enumerate}
\item
	\begin{enumerate}[label={\Alph*.}]
	\item \(\)
	\item \(\)
	\item \(\)
	\item \(\)
	\end{enumerate}
\item
	\begin{enumerate}[label={\Alph*.}]
	\item \(\)
	\item \(\)
	\item \(\)
	\item \(\)
	\end{enumerate}
\item
	\begin{enumerate}[label={\Alph*.}]
	\item \(\)
	\item \(\)
	\item \(\)
	\item \(\)
	\end{enumerate}
\item
	\begin{enumerate}[label={\Alph*.}]
	\item \(\)
	\item \(\)
	\item \(\)
	\item \(\)
	\end{enumerate}
\item
	\begin{enumerate}[label={\Alph*.}]
	\item \(\)
	\item \(\)
	\item \(\)
	\item \(\)
	\end{enumerate}
\item
	\begin{enumerate}[label={\Alph*.}]
	\item \(\)
	\item \(\)
	\item \(\)
	\item \(\)
	\end{enumerate}
\item
	\begin{enumerate}[label={\Alph*.}]
	\item \(\)
	\item \(\)
	\item \(\)
	\item \(\)
	\end{enumerate}
\item
	\begin{enumerate}[label={\Alph*.}]
	\item \(\)
	\item \(\)
	\item \(\)
	\item \(\)
	\end{enumerate}
\item
	\begin{enumerate}[label={\Alph*.}]
	\item \(\)
	\item \(\)
	\item \(\)
	\item \(\)
	\end{enumerate}
\item
	\begin{enumerate}[label={\Alph*.}]
	\item \(\)
	\item \(\)
	\item \(\)
	\item \(\)
	\end{enumerate}
\item
	\begin{enumerate}[label={\Alph*.}]
	\item \(\)
	\item \(\)
	\item \(\)
	\item \(\)
	\end{enumerate}
\item
	\begin{enumerate}[label={\Alph*.}]
	\item \(\)
	\item \(\)
	\item \(\)
	\item \(\)
	\end{enumerate}
\item
	\begin{enumerate}[label={\Alph*.}]
	\item \(\)
	\item \(\)
	\item \(\)
	\item \(\)
	\end{enumerate}
\item
	\begin{enumerate}[label={\Alph*.}]
	\item \(\)
	\item \(\)
	\item \(\)
	\item \(\)
	\end{enumerate}
\item
	\begin{enumerate}[label={\Alph*.}]
	\item \(\)
	\item \(\)
	\item \(\)
	\item \(\)
	\end{enumerate}
\item
	\begin{enumerate}[label={\Alph*.}]
	\item \(\)
	\item \(\)
	\item \(\)
	\item \(\)
	\end{enumerate}
\item
	\begin{enumerate}[label={\Alph*.}]
	\item \(\)
	\item \(\)
	\item \(\)
	\item \(\)
	\end{enumerate}
\item
	\begin{enumerate}[label={\Alph*.}]
	\item \(\)
	\item \(\)
	\item \(\)
	\item \(\)
	\end{enumerate}
\item
	\begin{enumerate}[label={\Alph*.}]
	\item \(\)
	\item \(\)
	\item \(\)
	\item \(\)
	\end{enumerate}
\item
	\begin{enumerate}[label={\Alph*.}]
	\item \(\)
	\item \(\)
	\item \(\)
	\item \(\)
	\end{enumerate}
\item
	\begin{enumerate}[label={\Alph*.}]
	\item \(\)
	\item \(\)
	\item \(\)
	\item \(\)
	\end{enumerate}
\item
	\begin{enumerate}[label={\Alph*.}]
	\item \(\)
	\item \(\)
	\item \(\)
	\item \(\)
	\end{enumerate}
\item
	\begin{enumerate}[label={\Alph*.}]
	\item \(\)
	\item \(\)
	\item \(\)
	\item \(\)
	\end{enumerate}
\item
	\begin{enumerate}[label={\Alph*.}]
	\item \(\)
	\item \(\)
	\item \(\)
	\item \(\)
	\end{enumerate}
\item
	\begin{enumerate}[label={\Alph*.}]
	\item \(\)
	\item \(\)
	\item \(\)
	\item \(\)
	\end{enumerate}
\item
	\begin{enumerate}[label={\Alph*.}]
	\item \(\)
	\item \(\)
	\item \(\)
	\item \(\)
	\end{enumerate}
\item
	\begin{enumerate}[label={\Alph*.}]
	\item \(\)
	\item \(\)
	\item \(\)
	\item \(\)
	\end{enumerate}
\item
	\begin{enumerate}[label={\Alph*.}]
	\item \(\)
	\item \(\)
	\item \(\)
	\item \(\)
	\end{enumerate}
\item
	\begin{enumerate}[label={\Alph*.}]
	\item \(\)
	\item \(\)
	\item \(\)
	\item \(\)
	\end{enumerate}
\item
	\begin{enumerate}[label={\Alph*.}]
	\item \(\)
	\item \(\)
	\item \(\)
	\item \(\)
	\end{enumerate}
\item
	\begin{enumerate}[label={\Alph*.}]
	\item \(\)
	\item \(\)
	\item \(\)
	\item \(\)
	\end{enumerate}
\item
	\begin{enumerate}[label={\Alph*.}]
	\item \(\)
	\item \(\)
	\item \(\)
	\item \(\)
	\end{enumerate}
\item
	\begin{enumerate}[label={\Alph*.}]
	\item \(\)
	\item \(\)
	\item \(\)
	\item \(\)
	\end{enumerate}
\item
	\begin{enumerate}[label={\Alph*.}]
	\item \(\)
	\item \(\)
	\item \(\)
	\item \(\)
	\end{enumerate}
\item
	\begin{enumerate}[label={\Alph*.}]
	\item \(\)
	\item \(\)
	\item \(\)
	\item \(\)
	\end{enumerate}
\item
	\begin{enumerate}[label={\Alph*.}]
	\item \(\)
	\item \(\)
	\item \(\)
	\item \(\)
	\end{enumerate}
\item
	\begin{enumerate}[label={\Alph*.}]
	\item \(\)
	\item \(\)
	\item \(\)
	\item \(\)
	\end{enumerate}
\item
	\begin{enumerate}[label={\Alph*.}]
	\item \(\)
	\item \(\)
	\item \(\)
	\item \(\)
	\end{enumerate}
\item
	\begin{enumerate}[label={\Alph*.}]
	\item \(\)
	\item \(\)
	\item \(\)
	\item \(\)
	\end{enumerate}
\item
	\begin{enumerate}[label={\Alph*.}]
	\item \(\)
	\item \(\)
	\item \(\)
	\item \(\)
	\end{enumerate}
\item
	\begin{enumerate}[label={\Alph*.}]
	\item \(\)
	\item \(\)
	\item \(\)
	\item \(\)
	\end{enumerate}
\item
	\begin{enumerate}[label={\Alph*.}]
	\item \(\)
	\item \(\)
	\item \(\)
	\item \(\)
	\end{enumerate}
\item
	\begin{enumerate}[label={\Alph*.}]
	\item \(\)
	\item \(\)
	\item \(\)
	\item \(\)
	\end{enumerate}
\item
	\begin{enumerate}[label={\Alph*.}]
	\item \(\)
	\item \(\)
	\item \(\)
	\item \(\)
	\end{enumerate}
\item
	\begin{enumerate}[label={\Alph*.}]
	\item \(\)
	\item \(\)
	\item \(\)
	\item \(\)
	\end{enumerate}
\item
	\begin{enumerate}[label={\Alph*.}]
	\item \(\)
	\item \(\)
	\item \(\)
	\item \(\)
	\end{enumerate}
\item
	\begin{enumerate}[label={\Alph*.}]
	\item \(\)
	\item \(\)
	\item \(\)
	\item \(\)
	\end{enumerate}
\item
	\begin{enumerate}[label={\Alph*.}]
	\item \(\)
	\item \(\)
	\item \(\)
	\item \(\)
	\end{enumerate}
\item
	\begin{enumerate}[label={\Alph*.}]
	\item \(\)
	\item \(\)
	\item \(\)
	\item \(\)
	\end{enumerate}
\item
	\begin{enumerate}[label={\Alph*.}]
	\item \(\)
	\item \(\)
	\item \(\)
	\item \(\)
	\end{enumerate}
\item
	\begin{enumerate}[label={\Alph*.}]
	\item \(\)
	\item \(\)
	\item \(\)
	\item \(\)
	\end{enumerate}
\item
	\begin{enumerate}[label={\Alph*.}]
	\item \(\)
	\item \(\)
	\item \(\)
	\item \(\)
	\end{enumerate}
\item
	\begin{enumerate}[label={\Alph*.}]
	\item \(\)
	\item \(\)
	\item \(\)
	\item \(\)
	\end{enumerate}
\item
	\begin{enumerate}[label={\Alph*.}]
	\item \(\)
	\item \(\)
	\item \(\)
	\item \(\)
	\end{enumerate}
\item
	\begin{enumerate}[label={\Alph*.}]
	\item \(\)
	\item \(\)
	\item \(\)
	\item \(\)
	\end{enumerate}
\item
	\begin{enumerate}[label={\Alph*.}]
	\item \(\)
	\item \(\)
	\item \(\)
	\item \(\)
	\end{enumerate}
\item
	\begin{enumerate}[label={\Alph*.}]
	\item \(\)
	\item \(\)
	\item \(\)
	\item \(\)
	\end{enumerate}
\item
	\begin{enumerate}[label={\Alph*.}]
	\item \(\)
	\item \(\)
	\item \(\)
	\item \(\)
	\end{enumerate}
\item
	\begin{enumerate}[label={\Alph*.}]
	\item \(\)
	\item \(\)
	\item \(\)
	\item \(\)
	\end{enumerate}
\item
	\begin{enumerate}[label={\Alph*.}]
	\item \(\)
	\item \(\)
	\item \(\)
	\item \(\)
	\end{enumerate}


\item
	\begin{enumerate}[label={\Alph*.}]
	\item \(\)
	\item \(\)
	\item \(\)
	\item \(\)
	\end{enumerate}
\item
	\begin{enumerate}[label={\Alph*.}]
	\item \(\)
	\item \(\)
	\item \(\)
	\item \(\)
	\end{enumerate}
\item
	\begin{enumerate}[label={\Alph*.}]
	\item \(\)
	\item \(\)
	\item \(\)
	\item \(\)
	\end{enumerate}
\item
	\begin{enumerate}[label={\Alph*.}]
	\item \(\)
	\item \(\)
	\item \(\)
	\item \(\)
	\end{enumerate}
\item
	\begin{enumerate}[label={\Alph*.}]
	\item \(\)
	\item \(\)
	\item \(\)
	\item \(\)
	\end{enumerate}
\item
	\begin{enumerate}[label={\Alph*.}]
	\item \(\)
	\item \(\)
	\item \(\)
	\item \(\)
	\end{enumerate}
\item
	\begin{enumerate}[label={\Alph*.}]
	\item \(\)
	\item \(\)
	\item \(\)
	\item \(\)
	\end{enumerate}
\item
	\begin{enumerate}[label={\Alph*.}]
	\item \(\)
	\item \(\)
	\item \(\)
	\item \(\)
	\end{enumerate}
\item
	\begin{enumerate}[label={\Alph*.}]
	\item \(\)
	\item \(\)
	\item \(\)
	\item \(\)
	\end{enumerate}
\item
	\begin{enumerate}[label={\Alph*.}]
	\item \(\)
	\item \(\)
	\item \(\)
	\item \(\)
	\end{enumerate}
\item
	\begin{enumerate}[label={\Alph*.}]
	\item \(\)
	\item \(\)
	\item \(\)
	\item \(\)
	\end{enumerate}
\item
	\begin{enumerate}[label={\Alph*.}]
	\item \(\)
	\item \(\)
	\item \(\)
	\item \(\)
	\end{enumerate}
\item
	\begin{enumerate}[label={\Alph*.}]
	\item \(\)
	\item \(\)
	\item \(\)
	\item \(\)
	\end{enumerate}
\item
	\begin{enumerate}[label={\Alph*.}]
	\item \(\)
	\item \(\)
	\item \(\)
	\item \(\)
	\end{enumerate}
\item
	\begin{enumerate}[label={\Alph*.}]
	\item \(\)
	\item \(\)
	\item \(\)
	\item \(\)
	\end{enumerate}
\item
	\begin{enumerate}[label={\Alph*.}]
	\item \(\)
	\item \(\)
	\item \(\)
	\item \(\)
	\end{enumerate}
\item
	\begin{enumerate}[label={\Alph*.}]
	\item \(\)
	\item \(\)
	\item \(\)
	\item \(\)
	\end{enumerate}
\item
	\begin{enumerate}[label={\Alph*.}]
	\item \(\)
	\item \(\)
	\item \(\)
	\item \(\)
	\end{enumerate}
\item
	\begin{enumerate}[label={\Alph*.}]
	\item \(\)
	\item \(\)
	\item \(\)
	\item \(\)
	\end{enumerate}
\item
	\begin{enumerate}[label={\Alph*.}]
	\item \(\)
	\item \(\)
	\item \(\)
	\item \(\)
	\end{enumerate}
\item
	\begin{enumerate}[label={\Alph*.}]
	\item \(\)
	\item \(\)
	\item \(\)
	\item \(\)
	\end{enumerate}
\item
	\begin{enumerate}[label={\Alph*.}]
	\item \(\)
	\item \(\)
	\item \(\)
	\item \(\)
	\end{enumerate}
\item
	\begin{enumerate}[label={\Alph*.}]
	\item \(\)
	\item \(\)
	\item \(\)
	\item \(\)
	\end{enumerate}
\item
	\begin{enumerate}[label={\Alph*.}]
	\item \(\)
	\item \(\)
	\item \(\)
	\item \(\)
	\end{enumerate}
\item
	\begin{enumerate}[label={\Alph*.}]
	\item \(\)
	\item \(\)
	\item \(\)
	\item \(\)
	\end{enumerate}
\end{enumerate}
\subsection{Solution}
\begin{enumerate}[label={\arabic*.}]
    \item 
    \item 
    \item
    \item
    \item
    \item 
    \item
    \item
    \item 
    \item 
    \item 
    \item 
    \item
    \item
    \item
    \item 
    \item
    \item
    \item 
    \item 
    \item 
    \item 
    \item
    \item
    \item
    \item 
    \item
    \item
    \item 
    \item 
    \item 
    \item 
    \item
    \item
    \item
    \item 
    \item
    \item
    \item 
    \item 
    \item 
    \item 
    \item
    \item
    \item
    \item 
    \item
    \item
    \item 
    \item 
    \item 
    \item 
    \item
    \item
    \item
    \item 
    \item
    \item
    \item 
    \item 
    \item 
    \item 
    \item
    \item
    \item
    \item 
    \item
    \item
    \item 
    \item 
    \item 
    \item 
    \item
    \item
    \item
    \item 
    \item
    \item
    \item 
    \item 
    \item 
    \item 
    \item
    \item
    \item
    \item 
    \item
    \item
    \item 
    \item 
    \item 
    \item 
    \item
    \item
    \item
    \item 
    \item
    \item
    \item 
    \item 
\end{enumerate}
\chapter{Values To Memorize}
\begin{multicols}{2}
\section{Square Roots}
\begin{itemize}
    \item \(\sqrt{1} = 1\)
    \item \(\sqrt{2} = 1.4142\)
    \item \(\sqrt{3} = 1.7321\)
    \item \(\sqrt{4} = 2\)
    \item \(\sqrt{5}= 2.2361\)
    \item \(\sqrt{6} = 2.4495\)
    \item \(\sqrt{7} = 2.6458\)
    \item \(\sqrt{8} = 2.8284\)
    \item \(\sqrt{9} = 3\)
    \item \(\sqrt{10} = 3.1623\)
\end{itemize}
\section{Squares}
\begin{itemize}
    \item \(1^{2} = 1\)
    \item \(2^{2} = 4\)
    \item \(3^{2} = 9\)
    \item \(4^{2} = 16\)
    \item \(5^{2} = 25\)
    \item \(6^{2} = 36\)
    \item \(7^{2} = 49\)
    \item \(8^{2} = 64\)
    \item \(9^{2} = 81\)
    \item \({10}^{2} = 100\)
    \item \({11}^{2} = 121\)
    \item \({12}^{2} = 144\)
    \item \({13}^{2} = 169\)
    \item \({14}^{2} = 196\)
    \item \({15}^{2} = 225\)
    \item \({16}^{2} = 256\)
    \item \({17}^{2} = 289\)
    \item \({18}^{2} = 324\)
    \item \({19}^{2} = 361\)
    \item \({20}^{2} = 400\)
    \item \({21}^{2} = 441\)
    \item \({22}^{2} = 484\)
    \item \({23}^{2} = 529\)
    \item \({24}^{2} = 576\)
    \item \({25}^{2} = 625\)
    \item \({26}^{2} = 676\)
    \item \({27}^{2} = 729\)
    \item \({28}^{2} = 784\)
    \item \({29}^{2} = 841\)
    \item \({30}^{2} = 900\)
\end{itemize}
\section{Cubes}
\begin{itemize}
    \item \(1^{3} = 1\)
    \item \(2^{3} = 8\)
    \item \(3^{3} = 27\)
    \item \(4^{3} = 64\)
    \item \(5^{3} = 125\)
    \item \(6^{3} = 216\)
    \item \(7^{3} = 343\)
    \item \(8^{3} = 512\)
    \item \(9^{3} = 729\)
    \item \({10}^{3} = 1000\)
    \item \({11}^{3} = 1331\)
    \item \({12}^{3} = 1728\)
    \item \({13}^{3} = 2197\)
    \item \({14}^{3} = 2744\)
    \item \({15}^{3} = 3375\)
    \item \({16}^{3} = 4096\)
    \item \({17}^{3} = 4913\)
    \item \({18}^{3} = 5832\)
    \item \({19}^{3} = 6859\)
    \item \({20}^{3} = 8000\)
    \item \({21}^{3} = 9261\)
    \item \({22}^{3} = 10648\)
    \item \({23}^{3} = 12167\)
    \item \({24}^{3} = 13824\)
    \item \({25}^{3} = 15625\)
    \item \({26}^{3} = 17576\)
    \item \({27}^{3} = 19683\)
    \item \({28}^{3} = 21952\)
    \item \({29}^{3} = 24389\)
    \item \({30}^{3} = 27000\)
\end{itemize}
\section{Logarithms}
\begin{itemize}
    \item \(\log_{10}{1} = 0\)
    \item \(\log_{10}{2} = 0.3010\)
    \item \(\log_{10}{3} = 0.4771\)
    \item \(\log_{10}{4} = 0.6020\)
    \item \(\log_{10}{5} = 0.699\)
    \item \(\log_{10}{6} = 0.778\)
    \item \(\log_{10}{7} = 0.845\)
    \item \(\log_{10}{8} = 0.903\)
    \item \(\log_{10}{9} = 0.954\)
    \item \(\log_{10}{10} = 1\)
\end{itemize}
\end{multicols}

\begin{backmatter} % Backmatter pages (e.g., bibliography, index, etc.)

\end{backmatter}
\end{document}
