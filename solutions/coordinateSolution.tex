\subsection{Solutions}
\begin{enumerate}[label={\textbf{\arabic*.}}]
  \item \textbf{(C)} Distance \(\sqrt{(6-3)^2 + (8-4)^2} = \sqrt{9+16} = 5\).
  \item \textbf{(A)} Midpoint \(((-2+4)/2, (5-3)/2) = (1, 1)\).
  \item \textbf{(B)} Gradient \((9-3)/(5-2) = 6/3 = 2\).
  \item \textbf{(A)} Gradient \((-4-0)/(0-(-2)) = -4/2 = -2\).
  \item \textbf{(C)} Distance \(\sqrt{3^2 + 4^2} = 5\).
  \item \textbf{(A)} Midpoint \((6+2)/2, (8+4)/2 = (4, 6)\).
  \item \textbf{(B)} \(2y = 4x + 6 \implies y = 2x + 3\), so \(m = 2\).
  \item \textbf{(C)} Distance \(\sqrt{(4-1)^2 + (6-2)^2} = \sqrt{3^2 + 4^2} = 5\).
  \item \textbf{(A)} \(y - 2 = 3(x - 0) \implies y = 3x + 2\).
  \item \textbf{(A)} Midpoint \((10+6)/2, (14+2)/2 = (8, 8)\).
  \item \textbf{(B)} Gradient \((11-5)/(3-0) = 6/3 = 2\).
  \item \textbf{(C)} Distance \(\sqrt{(3-(-3))^2 + (4-4)^2} = 6\).
  \item \textbf{(A)} \(3y = -6x + 9 \implies y = -2x + 3\), so \(m = -2\).
  \item \textbf{(A)} Midpoint \((5+9)/2, (7+11)/2 = (7, 9)\).
  \item \textbf{(C)} Distance \(\sqrt{(5-2)^2 + (5-1)^2} = 5\).
  \item \textbf{(C)} Parallel to \(y = 4x + 1 \implies m = 4\).
  \item \textbf{(A)} Midpoint \((0+8)/2, (0+6)/2 = (4, 3)\).
  \item \textbf{(B)} Gradient \((7-1)/(4-1) = 6/3 = 2\).
  \item \textbf{(C)} Distance \(\sqrt{(2-(-1))^2 + (3-(-1))^2} = 5\).
  \item \textbf{(A)} \(y = 2x - 3\).
  \item \textbf{(A)} Midpoint \((-4+10)/2, (6-2)/2 = (3, 2)\).
  \item \textbf{(A)} From \(y = -3x + 7, m = -3\).
  \item \textbf{(C)} Distance \(\sqrt{5^2+12^2} = 13\).
  \item \textbf{(D)} Perpendicular to \(m = 2 \implies m = -1/2\).
  \item \textbf{(A)} Midpoint \((7+3)/2, (9+5)/2 = (5, 7)\).
  \item \textbf{(C)} Gradient \((9-5)/(2-2) = 4/0 \implies\) Undefined.
  \item \textbf{(C)} Distance \(\sqrt{(6-6)^2 + (8-2)^2} = 6\).
  \item \textbf{(A)} \(y - 4 = -1(x - 3) \implies y = -x + 7\).
  \item \textbf{(A)} Midpoint \((12+4)/2, (8+16)/2 = (8, 12)\).
  \item \textbf{(A)} Horizontal line gradient = 0.
  \item \textbf{(C)} Distance \(\sqrt{(8-8)^2 + (15-3)^2} = 12\).
  \item \textbf{(B)} \(4y = 8x - 12 \implies y = 2x - 3, m = 2\).
  \item \textbf{(A)} Midpoint \((-5+7)/2, (-3+9)/2 = (1, 3)\).
  \item \textbf{(A)} Gradient \((2-2)/(7-3) = 0\).
  \item \textbf{(C)} Distance \(\sqrt{12^2+5^2} = 13\).
  \item \textbf{(A)} \(y - 3 = 1/2(x - 4) \implies y = 1/2x + 1\).
  \item \textbf{(A)} Midpoint \((9+15)/2, (12+4)/2 = (12, 8)\).
  \item \textbf{(D)} Vertical line gradient = Undefined.
  \item \textbf{(C)} Distance \(\sqrt{(4-(-2))^2 + (5-(-3))^2} = 10\).
  \item \textbf{(B)} \(5y = 10x + 15 \implies m = 2\).
  \item \textbf{(A)} Midpoint \((20+10)/2, (30+10)/2 = (15, 20)\).
  \item \textbf{(B)} Gradient \((10-4)/(2-(-1)) = 2\).
  \item \textbf{(B)} Distance \(\sqrt{7^2+24^2} = 25\).
  \item \textbf{(D)} Perpendicular to \(m = -1/3 \implies m = 3\).
  \item \textbf{(A)} Midpoint \((8+12)/2, (14+6)/2 = (10, 10)\).
  \item \textbf{(B)} \(y = 3x + 6 \implies m = 3\).
  \item \textbf{(B)} Distance \(\sqrt{9^2+40^2} = 41\).
  \item \textbf{(A)} Passing through origin with gradient 5: \(y = 5x\).
  \item \textbf{(A)} Midpoint \((-6+14)/2, (8-4)/2 = (4, 2)\).
  \item \textbf{(B)} Gradient \((5-1)/(1-5) = -1\).
  \item \textbf{(C)} Distance \((3,7)\) to \((6,11)\): \(\sqrt{3^2 + 4^2} = 5\).
  \item \textbf{(A)} Parallel to \(y = -5x + 2\), \(m = -5\).
  \item \textbf{(A)} Midpoint \(((18+6)/2, (24+8)/2) = (12, 16)\).
  \item \textbf{(A)} \(6y = -12x + 18 \implies y = -2x + 3\), \(m = -2\).
  \item \textbf{(C)} Distance between \((10,24)\) and \((10,4)\) is \(24 - 4 = 20\).
  \item \textbf{(A)} \(y = -3x + 5\).
  \item \textbf{(A)} Midpoint \(((25+15)/2, (35+15)/2) = (20, 25)\).
  \item \textbf{(B)} Gradient \((11-3)/(4-0) = 8/4 = 2\).
  \item \textbf{(C)} Distance \(\sqrt{(8-2)^2 + (6-(-2))^2} = \sqrt{6^2 + 8^2} = 10\).
  \item \textbf{(D)} Perpendicular to \(m = 2/3 \implies -3/2\).
  \item \textbf{(A)} Midpoint \(((5+11)/2, (13+7)/2) = (8, 10)\).
  \item \textbf{(A)} \(2y = -6x + 10 \implies m = -3\).
  \item \textbf{(B)} Distance \((15, 20)\) and \((15, 8)\) is \(20 - 8 = 12\).
  \item \textbf{(A)} \(y - 1 = 4(x - 2) \implies y = 4x - 7\).
  \item \textbf{(A)} Midpoint \(((30+10)/2, (40+20)/2) = (20, 30)\).
  \item \textbf{(B)} Gradient \((10-2)/(10-6) = 8/4 = 2\).
  \item \textbf{(B)} Distance \((5, 12)\) to origin: \(\sqrt{5^2+12^2} = 13\).
  \item \textbf{(A)} Parallel to \(y = 1/4x - 3\), \(m = 1/4\).
  \item \textbf{(A)} Midpoint \(((22+14)/2, (18+10)/2) = (18, 14)\).
  \item \textbf{(A)} \(3y = -9x + 12 \implies m = -3\).
  \item \textbf{(B)} Distance between \((20, 21)\) and \((20, 5)\) is \(21 - 5 = 16\).
  \item \textbf{(A)} \(y - 5 = 3/4(x - 8) \implies y = 3/4x - 1\).
  \item \textbf{(A)} Midpoint \((12, 18)\).
  \item \textbf{(D)} Gradient \((15-5)/(7-7)\) is undefined.
  \item \textbf{(B)} Distance \((9, 12)\) to origin: \(\sqrt{9^2+12^2} = 15\).
  \item \textbf{(D)} Perpendicular to \(m = 4 \implies -1/4\).
  \item \textbf{(A)} Midpoint \((20, 28)\).
  \item \textbf{(A)} \(y = -5x + 15\), \(m = -5\).
  \item \textbf{(B)} Distance \((24, 7)\) to origin: \(\sqrt{24^2+7^2} = 25\).
  \item \textbf{(A)} \(y - 6 = -2(x - 5) \implies y = -2x + 16\).
  \item \textbf{(A)} Midpoint \((30, 35)\).
  \item \textbf{(C)} Gradient \((20-8)/(5-1) = 12/4 = 3\).
  \item \textbf{(C)} Distance \(\sqrt{(6-1)^2 + (8-(-4))^2} = \sqrt{5^2 + 12^2} = 13\).
  \item \textbf{(A)} Parallel to \(y = -7x + 3\), \(m = -7\).
  \item \textbf{(A)} Midpoint \((30, 40)\).
  \item \textbf{(B)} \(4y = 8x + 20 \implies m = 2\).
  \item \textbf{(C)} Vertical distance \(84 - 0 = 84\).
  \item \textbf{(A)} \(y = 6x - 2\).
  \item \textbf{(A)} Midpoint \((40, 50)\).
  \item \textbf{(B)} Gradient \((22-10)/(9-3) = 12/6 = 2\).
  \item \textbf{(B)} Distance \((11, 60)\) to origin: \(\sqrt{11^2+60^2} = 61\).
  \item \textbf{(D)} Perpendicular to \(m = -5/2 \implies 2/5\).
  \item \textbf{(A)} Midpoint \((40, 45)\).
  \item \textbf{(A)} \(7y = -14x + 21 \implies m = -2\).
  \item \textbf{(C)} Midpoint of \((-3, 5)\) and \((5, -3)\) is \((1, 1)\).
  \item \textbf{(B)} Gradient \((1-7)/(5-2) = -6/3 = -2\).
  \item \textbf{(A)} Diameter PQ: \(\sqrt{12^2 + 5^2} = 13\), so radius = 6.5.
  \item \textbf{(C)} Distance between \((4, 3)\) and \((1, -1)\): \(\sqrt{3^2 + 4^2} = 5\).
  \item \textbf{(A)} \(y - 3 = -2(x - 2) \implies y = -2x + 7\).
  \item \textbf{(B)} Area of triangle (0,0), (5,0), (0,8): \(1/2 \times 5 \times 8 = 20\) square units.
\end{enumerate}