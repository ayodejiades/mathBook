\chapter{Geometry}
\section{Circle Geometry}
\subsection{Questions}
\begin{multicols}{2}
\begin{enumerate}[label={\arabic*.}]

\item A chord of length $24\text{ cm}$ is drawn in a circle of radius $13\text{ cm}$. Calculate the distance of the chord from the center.
	\begin{enumerate}[label={\Alph*.}]
	\item $5\text{ cm}$
	\item $7\text{ cm}$
	\item $9\text{ cm}$
	\item $11\text{ cm}$
	\end{enumerate}

\item Two chords $AB$ and $CD$ intersect at point $E$ inside a circle. If $AE = 4\text{ cm}$, $EB = 6\text{ cm}$ and $CE = 3\text{ cm}$, find $ED$.
	\begin{enumerate}[label={\Alph*.}]
	\item $6\text{ cm}$
	\item $8\text{ cm}$
	\item $10\text{ cm}$
	\item $12\text{ cm}$
	\end{enumerate}

\item A tangent from an external point $P$ to a circle with center $O$ and radius $5\text{ cm}$ has length $12\text{ cm}$. Find the distance $OP$.
	\begin{enumerate}[label={\Alph*.}]
	\item $7\text{ cm}$
	\item $11\text{ cm}$
	\item $13\text{ cm}$
	\item $17\text{ cm}$
	\end{enumerate}

\item In a circle, a chord subtends an angle of $60^\circ$ at the center. If the radius is $10\text{ cm}$, find the length of the chord.
	\begin{enumerate}[label={\Alph*.}]
	\item $5\text{ cm}$
	\item $8\text{ cm}$
	\item $10\text{ cm}$
	\item $12\text{ cm}$
	\end{enumerate}

\item An arc of a circle subtends an angle of $120^\circ$ at the center. If the radius is $7\text{ cm}$, what angle does it subtend at the circumference?
	\begin{enumerate}[label={\Alph*.}]
	\item $30^\circ$
	\item $45^\circ$
	\item $60^\circ$
	\item $90^\circ$
	\end{enumerate}

\item Two tangents from an external point to a circle are $8\text{ cm}$ each. If the angle between them is $60^\circ$, find the radius of the circle.
	\begin{enumerate}[label={\Alph*.}]
	\item $4\text{ cm}$
	\item $4\sqrt{3}\text{ cm}$
	\item $8\text{ cm}$
	\item $8\sqrt{3}\text{ cm}$
	\end{enumerate}

\item In a cyclic quadrilateral $ABCD$, $\angle A = 70^\circ$ and $\angle C = 110^\circ$. Find $\angle B$.
	\begin{enumerate}[label={\Alph*.}]
	\item $70^\circ$
	\item $90^\circ$
	\item $110^\circ$
	\item $180^\circ$
	\end{enumerate}

\item A chord of a circle is $16\text{ cm}$ long and is $6\text{ cm}$ from the center. Calculate the radius of the circle.
	\begin{enumerate}[label={\Alph*.}]
	\item $8\text{ cm}$
	\item $10\text{ cm}$
	\item $12\text{ cm}$
	\item $14\text{ cm}$
	\end{enumerate}

\item Two circles with radii $4\text{ cm}$ and $6\text{ cm}$ touch externally. Calculate the distance between their centers.
	\begin{enumerate}[label={\Alph*.}]
	\item $2\text{ cm}$
	\item $5\text{ cm}$
	\item $10\text{ cm}$
	\item $12\text{ cm}$
	\end{enumerate}

\item A chord of length $20\text{ cm}$ subtends an angle of $90^\circ$ at the center of a circle. Find the radius.
	\begin{enumerate}[label={\Alph*.}]
	\item $10\text{ cm}$
	\item $10\sqrt{2}\text{ cm}$
	\item $15\text{ cm}$
	\item $20\text{ cm}$
	\end{enumerate}

\item In a circle with center $O$, a chord $AB$ is $12\text{ cm}$ long. If the perpendicular from $O$ to $AB$ is $8\text{ cm}$, find the radius.
	\begin{enumerate}[label={\Alph*.}]
	\item $9\text{ cm}$
	\item $10\text{ cm}$
	\item $11\text{ cm}$
	\item $12\text{ cm}$
	\end{enumerate}

\item Two tangents are drawn from point $P$ to a circle with center $O$. If the angle between the tangents is $80^\circ$, find $\angle POQ$ where $Q$ is one point of tangency.
	\begin{enumerate}[label={\Alph*.}]
	\item $40^\circ$
	\item $50^\circ$
	\item $80^\circ$
	\item $100^\circ$
	\end{enumerate}

\item A cyclic quadrilateral has angles $x$, $2x$, $3x$, and $4x$. Find the largest angle.
	\begin{enumerate}[label={\Alph*.}]
	\item $72^\circ$
	\item $108^\circ$
	\item $120^\circ$
	\item $144^\circ$
	\end{enumerate}

\item In a circle, an inscribed angle is $40^\circ$. Find the central angle subtending the same arc.
	\begin{enumerate}[label={\Alph*.}]
	\item $20^\circ$
	\item $40^\circ$
	\item $60^\circ$
	\item $80^\circ$
	\end{enumerate}

\item Two chords $PQ$ and $RS$ of a circle intersect at $T$ outside the circle. If $PT = 6\text{ cm}$, $TQ = 4\text{ cm}$ and $RT = 8\text{ cm}$, find $TS$.
	\begin{enumerate}[label={\Alph*.}]
	\item $2\text{ cm}$
	\item $3\text{ cm}$
	\item $4\text{ cm}$
	\item $5\text{ cm}$
	\end{enumerate}

\item A tangent to a circle makes an angle of $30^\circ$ with a chord drawn from the point of contact. Find the angle subtended by the chord at the center.
	\begin{enumerate}[label={\Alph*.}]
	\item $60^\circ$
	\item $90^\circ$
	\item $120^\circ$
	\item $150^\circ$
	\end{enumerate}

\item In a circle of radius $15\text{ cm}$, a chord is $18\text{ cm}$ long. Calculate its distance from the center.
	\begin{enumerate}[label={\Alph*.}]
	\item $6\text{ cm}$
	\item $9\text{ cm}$
	\item $12\text{ cm}$
	\item $15\text{ cm}$
	\end{enumerate}

\item Two circles with radii $8\text{ cm}$ and $3\text{ cm}$ touch internally. Find the distance between their centers.
	\begin{enumerate}[label={\Alph*.}]
	\item $3\text{ cm}$
	\item $5\text{ cm}$
	\item $8\text{ cm}$
	\item $11\text{ cm}$
	\end{enumerate}

\item A chord of a circle subtends an angle of $144^\circ$ at the center. What angle does it subtend at a point on the minor arc?
	\begin{enumerate}[label={\Alph*.}]
	\item $36^\circ$
	\item $72^\circ$
	\item $108^\circ$
	\item $144^\circ$
	\end{enumerate}

\item In a cyclic quadrilateral $PQRS$, $\angle P = 85^\circ$. Find $\angle R$.
	\begin{enumerate}[label={\Alph*.}]
	\item $85^\circ$
	\item $95^\circ$
	\item $105^\circ$
	\item $175^\circ$
	\end{enumerate}

\item A tangent and a chord meet at a point on a circle, making an angle of $50^\circ$. Find the angle in the alternate segment.
	\begin{enumerate}[label={\Alph*.}]
	\item $25^\circ$
	\item $40^\circ$
	\item $50^\circ$
	\item $100^\circ$
	\end{enumerate}

\item Two parallel chords in a circle have lengths $16\text{ cm}$ and $12\text{ cm}$. If the radius is $10\text{ cm}$, find the distance between the chords (both on the same side of center).
	\begin{enumerate}[label={\Alph*.}]
	\item $2\text{ cm}$
	\item $4\text{ cm}$
	\item $6\text{ cm}$
	\item $8\text{ cm}$
	\end{enumerate}

\item A circle passes through the vertices of a square with side $8\text{ cm}$. Find the radius of the circle.
	\begin{enumerate}[label={\Alph*.}]
	\item $4\text{ cm}$
	\item $4\sqrt{2}\text{ cm}$
	\item $8\text{ cm}$
	\item $8\sqrt{2}\text{ cm}$
	\end{enumerate}

\item In a circle, a chord $AB = 14\text{ cm}$ and its perpendicular distance from center $O$ is $7\sqrt{3}\text{ cm}$. Find the radius.
	\begin{enumerate}[label={\Alph*.}]
	\item $12\text{ cm}$
	\item $14\text{ cm}$
	\item $16\text{ cm}$
	\item $18\text{ cm}$
	\end{enumerate}

\item Two secants are drawn from point $P$ to a circle. If the external parts are $4\text{ cm}$ and $5\text{ cm}$ and one whole secant is $12\text{ cm}$, find the other whole secant.
	\begin{enumerate}[label={\Alph*.}]
	\item $9\text{ cm}$
	\item $9.6\text{ cm}$
	\item $10\text{ cm}$
	\item $11\text{ cm}$
	\end{enumerate}

\item A cyclic trapezium has parallel sides of lengths $10\text{ cm}$ and $6\text{ cm}$. If the radius is $5\text{ cm}$, find the length of one of the non-parallel sides.
	\begin{enumerate}[label={\Alph*.}]
	\item $4\text{ cm}$
	\item $5\text{ cm}$
	\item $6\text{ cm}$
	\item $8\text{ cm}$
	\end{enumerate}

\item In a circle with radius $r$, two parallel chords are on opposite sides of the center. If their lengths are $16\text{ cm}$ and $12\text{ cm}$ and $r = 10\text{ cm}$, find the distance between them.
	\begin{enumerate}[label={\Alph*.}]
	\item $10\text{ cm}$
	\item $12\text{ cm}$
	\item $14\text{ cm}$
	\item $16\text{ cm}$
	\end{enumerate}

\item A chord subtends an angle of $120^\circ$ at the center of a circle with radius $6\text{ cm}$. Find the area of the minor segment.
	\begin{enumerate}[label={\Alph*.}]
	\item $(12\pi - 9\sqrt{3})\text{ cm}^2$
	\item $(9\pi - 9\sqrt{3})\text{ cm}^2$
	\item $(6\pi - 9\sqrt{3})\text{ cm}^2$
	\item $(12\pi - 18\sqrt{3})\text{ cm}^2$
	\end{enumerate}

\item From an external point, two tangents are drawn to a circle of radius $5\text{ cm}$. If each tangent is $12\text{ cm}$ long, find the distance between the points of tangency.
	\begin{enumerate}[label={\Alph*.}]
	\item $8.46\text{ cm}$
	\item $9.23\text{ cm}$
	\item $10\text{ cm}$
	\item $12\text{ cm}$
	\end{enumerate}

\item In a circle, an angle in a semicircle is:
	\begin{enumerate}[label={\Alph*.}]
	\item $45^\circ$
	\item $60^\circ$
	\item $90^\circ$
	\item $180^\circ$
	\end{enumerate}

\item A chord of length $8\text{ cm}$ is at distance $3\text{ cm}$ from center. Another chord is at distance $4\text{ cm}$. Find its length.
	\begin{enumerate}[label={\Alph*.}]
	\item $4\text{ cm}$
	\item $6\text{ cm}$
	\item $2\sqrt{9}\text{ cm}$
	\item $2\sqrt{9}\text{ cm}$
	\end{enumerate}

\item Two circles of equal radii $5\text{ cm}$ each intersect such that each passes through the center of the other. Find the length of the common chord.
	\begin{enumerate}[label={\Alph*.}]
	\item $5\text{ cm}$
	\item $5\sqrt{2}\text{ cm}$
	\item $5\sqrt{3}\text{ cm}$
	\item $10\text{ cm}$
	\end{enumerate}

\item In a cyclic quadrilateral, two opposite angles are $(3x + 10)^\circ$ and $(2x + 20)^\circ$. Find $x$.
	\begin{enumerate}[label={\Alph*.}]
	\item $30^\circ$
	\item $35^\circ$
	\item $40^\circ$
	\item $45^\circ$
	\end{enumerate}

\item A tangent to a circle of radius $7\text{ cm}$ from point $P$ is $24\text{ cm}$ long. Find $OP$.
	\begin{enumerate}[label={\Alph*.}]
	\item $17\text{ cm}$
	\item $20\text{ cm}$
	\item $25\text{ cm}$
	\item $31\text{ cm}$
	\end{enumerate}

\item In a circle, a chord $AB = 10\text{ cm}$ subtends an angle of $60^\circ$ at the center. Find the radius.
	\begin{enumerate}[label={\Alph*.}]
	\item $5\text{ cm}$
	\item $8\text{ cm}$
	\item $10\text{ cm}$
	\item $12\text{ cm}$
	\end{enumerate}

\item Two chords $AB$ and $CD$ of a circle intersect at right angles at point $E$ inside the circle. If $AE = 3\text{ cm}$, $EB = 4\text{ cm}$ and $CE = 2\text{ cm}$, find $ED$.
	\begin{enumerate}[label={\Alph*.}]
	\item $5\text{ cm}$
	\item $6\text{ cm}$
	\item $7\text{ cm}$
	\item $8\text{ cm}$
	\end{enumerate}

\item A circle has diameter $26\text{ cm}$. A chord parallel to the diameter is $24\text{ cm}$ long. Find its distance from the diameter.
	\begin{enumerate}[label={\Alph*.}]
	\item $3\text{ cm}$
	\item $4\text{ cm}$
	\item $5\text{ cm}$
	\item $6\text{ cm}$
	\end{enumerate}

\item In a cyclic quadrilateral $ABCD$, $\angle DAB = 65^\circ$ and $\angle ABC = 75^\circ$. Find $\angle BCD$.
	\begin{enumerate}[label={\Alph*.}]
	\item $105^\circ$
	\item $115^\circ$
	\item $125^\circ$
	\item $140^\circ$
	\end{enumerate}

\item The angle between a tangent and a chord drawn from the point of contact is $35^\circ$. Find the angle subtended by the chord at the center.
	\begin{enumerate}[label={\Alph*.}]
	\item $35^\circ$
	\item $55^\circ$
	\item $70^\circ$
	\item $110^\circ$
	\end{enumerate}

\item A chord of a circle is equal to its radius. Find the angle subtended by the chord at the center.
	\begin{enumerate}[label={\Alph*.}]
	\item $30^\circ$
	\item $45^\circ$
	\item $60^\circ$
	\item $90^\circ$
	\end{enumerate}

\item Two tangents from external point $T$ touch a circle at $A$ and $B$. If $\angle ATB = 50^\circ$ and radius is $6\text{ cm}$, find $TA$.
	\begin{enumerate}[label={\Alph*.}]
	\item $6\tan 25^\circ\text{ cm}$
	\item $6/\tan 25^\circ\text{ cm}$
	\item $6\sin 25^\circ\text{ cm}$
	\item $6/\sin 25^\circ\text{ cm}$
	\end{enumerate}

\item In a circle with center $O$ and radius $10\text{ cm}$, two chords $AB$ and $CD$ are each $12\text{ cm}$ long. Find the distance between them if they are on the same side.
	\begin{enumerate}[label={\Alph*.}]
	\item $0\text{ cm}$
	\item $8\text{ cm}$
	\item $12\text{ cm}$
	\item $16\text{ cm}$
	\end{enumerate}

\item A cyclic quadrilateral has three angles $80^\circ$, $95^\circ$, and $105^\circ$. Find the fourth angle.
	\begin{enumerate}[label={\Alph*.}]
	\item $70^\circ$
	\item $75^\circ$
	\item $80^\circ$
	\item $85^\circ$
	\end{enumerate}

\item Two circles with radii $9\text{ cm}$ and $4\text{ cm}$ intersect. The distance between their centers is $10\text{ cm}$. Find the length of the common chord.
	\begin{enumerate}[label={\Alph*.}]
	\item $6\text{ cm}$
	\item $7.2\text{ cm}$
	\item $8\text{ cm}$
	\item $9.6\text{ cm}$
	\end{enumerate}

\item A secant and a tangent are drawn from external point $P$ to a circle. If the tangent is $8\text{ cm}$ and the external part of the secant is $4\text{ cm}$, find the whole secant.
	\begin{enumerate}[label={\Alph*.}]
	\item $12\text{ cm}$
	\item $16\text{ cm}$
	\item $18\text{ cm}$
	\item $20\text{ cm}$
	\end{enumerate}

\item In a circle, a chord $PQ$ is $16\text{ cm}$ long and subtends angle $90^\circ$ at center $O$. Find the area of triangle $POQ$.
	\begin{enumerate}[label={\Alph*.}]
	\item $32\text{ cm}^2$
	\item $48\text{ cm}^2$
	\item $64\text{ cm}^2$
	\item $72\text{ cm}^2$
	\end{enumerate}

\item Two parallel chords of lengths $6\text{ cm}$ and $8\text{ cm}$ are on opposite sides of center in a circle of radius $5\text{ cm}$. Find distance between them.
	\begin{enumerate}[label={\Alph*.}]
	\item $6\text{ cm}$
	\item $7\text{ cm}$
	\item $8\text{ cm}$
	\item $9\text{ cm}$
	\end{enumerate}

\item The angle in the alternate segment equals the angle between:
	\begin{enumerate}[label={\Alph*.}]
	\item Two chords
	\item Two radii
	\item A tangent and a chord
	\item Two tangents
	\end{enumerate}

\item In a circle of radius $13\text{ cm}$, a chord $AB = 24\text{ cm}$. Find the angle subtended by the chord at center.
	\begin{enumerate}[label={\Alph*.}]
	\item $67.4^\circ$
	\item $112.6^\circ$
	\item $134.8^\circ$
	\item $157.4^\circ$
	\end{enumerate}

\item A circle is inscribed in an equilateral triangle of side $12\text{ cm}$. Find the radius of the circle.
	\begin{enumerate}[label={\Alph*.}]
	\item $2\sqrt{3}\text{ cm}$
	\item $3\sqrt{3}\text{ cm}$
	\item $4\sqrt{3}\text{ cm}$
	\item $6\sqrt{3}\text{ cm}$
	\end{enumerate}

\item Two chords intersect inside a circle at point $M$. If the segments of one chord are $5\text{ cm}$ and $7\text{ cm}$, and one segment of other is $10\text{ cm}$, find the other segment.
	\begin{enumerate}[label={\Alph*.}]
	\item $3.5\text{ cm}$
	\item $4\text{ cm}$
	\item $4.5\text{ cm}$
	\item $5\text{ cm}$
	\end{enumerate}

\item In a cyclic quadrilateral, if one angle is three times another and their sum is $180^\circ$, find the smaller angle.
	\begin{enumerate}[label={\Alph*.}]
	\item $30^\circ$
	\item $45^\circ$
	\item $60^\circ$
	\item $90^\circ$
	\end{enumerate}

\item A tangent of length $15\text{ cm}$ is drawn from point $P$ to circle with radius $8\text{ cm}$. Find the angle between the tangent and the line joining $P$ to the center.
	\begin{enumerate}[label={\Alph*.}]
	\item $28.1^\circ$
	\item $32.0^\circ$
	\item $58.0^\circ$
	\item $62.0^\circ$
	\end{enumerate}

\item In a circle, a diameter $AB = 20\text{ cm}$. A chord $CD$ perpendicular to $AB$ at point $E$ where $AE = 4\text{ cm}$. Find $CD$.
	\begin{enumerate}[label={\Alph*.}]
	\item $12\text{ cm}$
	\item $14\text{ cm}$
	\item $16\text{ cm}$
	\item $18\text{ cm}$
	\end{enumerate}

\item Two circles touch externally. The sum of their radii is $15\text{ cm}$ and distance between centers is $15\text{ cm}$. If one radius is $9\text{ cm}$, find the other.
	\begin{enumerate}[label={\Alph*.}]
	\item $4\text{ cm}$
	\item $5\text{ cm}$
	\item $6\text{ cm}$
	\item $7\text{ cm}$
	\end{enumerate}

\item A chord subtends angle $\theta$ at center and angle $\alpha$ at circumference. The relationship is:
	\begin{enumerate}[label={\Alph*.}]
	\item $\theta = \alpha$
	\item $\theta = 2\alpha$
	\item $\alpha = 2\theta$
	\item $\theta + \alpha = 180^\circ$
	\end{enumerate}

\item In a circle with radius $r$, a chord of length $r\sqrt{3}$ is drawn. Find the angle it subtends at center.
	\begin{enumerate}[label={\Alph*.}]
	\item $60^\circ$
	\item $90^\circ$
	\item $120^\circ$
	\item $150^\circ$
	\end{enumerate}

\item Two tangents from point $P$ make angle $60^\circ$ with each other. If radius is $5\text{ cm}$, find $OP$.
	\begin{enumerate}[label={\Alph*.}]
	\item $5\text{ cm}$
	\item $10\text{ cm}$
	\item $5\sqrt{3}\text{ cm}$
	\item $10\sqrt{3}\text{ cm}$
	\end{enumerate}

\item A cyclic quadrilateral has consecutive angles in ratio $2:3:4:x$. Find $x$.
	\begin{enumerate}[label={\Alph*.}]
	\item $3$
	\item $4$
	\item $5$
	\item $6$
	\end{enumerate}

\item In circle with center $O$, chord $AB = 8\text{ cm}$ and $\angle AOB = 90^\circ$. Find radius.
	\begin{enumerate}[label={\Alph*.}]
	\item $4\text{ cm}$
	\item $4\sqrt{2}\text{ cm}$
	\item $6\text{ cm}$
	\item $8\text{ cm}$
	\end{enumerate}

\item Two circles of radii $7\text{ cm}$ and $5\text{ cm}$ touch internally. Their common tangent divides the line joining centers in ratio:
	\begin{enumerate}[label={\Alph*.}]
	\item $5:7$
	\item $7:5$
	\item $2:5$
	\item $5:2$
	\end{enumerate}

\item A chord is $12\text{ cm}$ from center of circle with radius $13\text{ cm}$. Find its length.
	\begin{enumerate}[label={\Alph*.}]
	\item $5\text{ cm}$
	\item $8\text{ cm}$
	\item $10\text{ cm}$
	\item $12\text{ cm}$
	\end{enumerate}

\item In a cyclic trapezium $ABCD$ with $AB \parallel CD$, if $\angle A = 70^\circ$, find $\angle C$.
	\begin{enumerate}[label={\Alph*.}]
	\item $70^\circ$
	\item $90^\circ$
	\item $110^\circ$
	\item $140^\circ$
	\end{enumerate}

\item From external point, secant and tangent are drawn. If tangent is $12\text{ cm}$ and internal part of secant is $9\text{ cm}$, find external part.
	\begin{enumerate}[label={\Alph*.}]
	\item $6\text{ cm}$
	\item $7\text{ cm}$
	\item $8\text{ cm}$
	\item $9\text{ cm}$
	\end{enumerate}

\item A regular hexagon is inscribed in circle of radius $8\text{ cm}$. Find side of hexagon.
	\begin{enumerate}[label={\Alph*.}]
	\item $4\text{ cm}$
	\item $6\text{ cm}$
	\item $8\text{ cm}$
	\item $10\text{ cm}$
	\end{enumerate}

\item Angle between tangent and radius at point of contact is:
	\begin{enumerate}[label={\Alph*.}]
	\item $0^\circ$
	\item $45^\circ$
	\item $60^\circ$
	\item $90^\circ$
	\end{enumerate}

\item Two chords $AB = 10\text{ cm}$ and $CD = 24\text{ cm}$ are in circle of radius $13\text{ cm}$. Find sum of their distances from center.
	\begin{enumerate}[label={\Alph*.}]
	\item $10\text{ cm}$
	\item $12\text{ cm}$
	\item $17\text{ cm}$
	\item $20\text{ cm}$
	\end{enumerate}

\item In cyclic quadrilateral, diagonal divides it into two triangles. If angles of one triangle are $40^\circ$, $60^\circ$, $80^\circ$, find angle opposite to $80^\circ$ in other triangle.
	\begin{enumerate}[label={\Alph*.}]
	\item $40^\circ$
	\item $60^\circ$
	\item $80^\circ$
	\item $100^\circ$
	\end{enumerate}

\item A circle passes through vertices of rectangle with sides $6\text{ cm}$ and $8\text{ cm}$. Find diameter.
	\begin{enumerate}[label={\Alph*.}]
	\item $7\text{ cm}$
	\item $10\text{ cm}$
	\item $12\text{ cm}$
	\item $14\text{ cm}$
	\end{enumerate}

\item Chord $AB$ of circle with center $O$ and radius $5\text{ cm}$ has length $8\text{ cm}$. Point $M$ is midpoint of $AB$. Find $OM$.
	\begin{enumerate}[label={\Alph*.}]
	\item $2\text{ cm}$
	\item $3\text{ cm}$
	\item $4\text{ cm}$
	\item $5\text{ cm}$
	\end{enumerate}

\item Two tangents from external point are $6\text{ cm}$ and $8\text{ cm}$ long. This is:
	\begin{enumerate}[label={\Alph*.}]
	\item Possible
	\item Impossible
	\item Possible only if radii differ
	\item None of these
	\end{enumerate}

\item A chord subtends $45^\circ$ at center. Angle at major arc is:
	\begin{enumerate}[label={\Alph*.}]
	\item $22.5^\circ$
	\item $45^\circ$
	\item $67.5^\circ$
	\item $90^\circ$
	\end{enumerate}

\item In circle, perpendicular from center to chord bisects:
	\begin{enumerate}[label={\Alph*.}]
	\item The chord
	\item The arc
	\item The angle at center
	\item All of these
	\end{enumerate}

\item Tangents from external point to circle are:
	\begin{enumerate}[label={\Alph*.}]
	\item Unequal
	\item Equal
	\item Parallel
	\item Perpendicular
	\end{enumerate}

\item A cyclic quadrilateral with all sides equal is a:
	\begin{enumerate}[label={\Alph*.}]
	\item Rectangle
	\item Rhombus
	\item Square
	\item Trapezium
	\end{enumerate}

\item In circle of radius $10\text{ cm}$, chord at $6\text{ cm}$ from center has length:
	\begin{enumerate}[label={\Alph*.}]
	\item $8\text{ cm}$
	\item $12\text{ cm}$
	\item $14\text{ cm}$
	\item $16\text{ cm}$
	\end{enumerate}

\item Two circles touch externally. Line through point of contact passes through:
	\begin{enumerate}[label={\Alph*.}]
	\item One center only
	\item Both centers
	\item Neither center
	\item Midpoint of line joining centers
	\end{enumerate}

\item Angle subtended by diameter at circumference is:
	\begin{enumerate}[label={\Alph*.}]
	\item $45^\circ$
	\item $60^\circ$
	\item $90^\circ$
	\item $180^\circ$
	\end{enumerate}

\item Common chord of two intersecting circles is:
	\begin{enumerate}[label={\Alph*.}]
	\item Parallel to line joining centers
	\item Perpendicular to line joining centers
	\item Equal to sum of radii
	\item Equal to difference of radii
	\end{enumerate}

\item In cyclic quadrilateral, product of diagonals equals:
	\begin{enumerate}[label={\Alph*.}]
	\item Sum of products of opposite sides
	\item Difference of products of opposite sides
	\item Product of adjacent sides
	\item Twice the area
	\end{enumerate}

\item A chord of length $2r$ in circle of radius $r$ is:
	\begin{enumerate}[label={\Alph*.}]
	\item Impossible
	\item The diameter
	\item Any chord
	\item The radius
	\end{enumerate}

\item Maximum number of common tangents to two circles touching externally is:
	\begin{enumerate}[label={\Alph*.}]
	\item $1$
	\item $2$
	\item $3$
	\item $4$
	\end{enumerate}

\item In circle, equal chords are:
	\begin{enumerate}[label={\Alph*.}]
	\item Equidistant from center
	\item At different distances from center
	\item Parallel
	\item Perpendicular
	\end{enumerate}

\item Angle in semicircle is always:
	\begin{enumerate}[label={\Alph*.}]
	\item Acute
	\item Right
	\item Obtuse
	\item Reflex
	\end{enumerate}

\item Two chords $AB$ and $CD$ intersect at $E$. $AE \times EB$ equals:
	\begin{enumerate}[label={\Alph*.}]
	\item $CE + ED$
	\item $CE - ED$
	\item $CE \times ED$
	\item $CE / ED$
	\end{enumerate}

\item Radius of circle inscribed in right triangle with legs $3\text{ cm}$ and $4\text{ cm}$ is:
	\begin{enumerate}[label={\Alph*.}]
	\item $1\text{ cm}$
	\item $1.5\text{ cm}$
	\item $2\text{ cm}$
	\item $2.5\text{ cm}$
	\end{enumerate}

\item Circle circumscribing square of side $a$ has radius:
	\begin{enumerate}[label={\Alph*.}]
	\item $a/2$
	\item $a/\sqrt{2}$
	\item $a\sqrt{2}/2$
	\item $a\sqrt{2}$
	\end{enumerate}

\item Tangent to circle is perpendicular to:
	\begin{enumerate}[label={\Alph*.}]
	\item Chord
	\item Diameter
	\item Radius at point of contact
	\item Another tangent
	\end{enumerate}

\item In circle of radius $13\text{ cm}$, two parallel chords of lengths $10\text{ cm}$ and $24\text{ cm}$ are on same side. Distance between them is:
	\begin{enumerate}[label={\Alph*.}]
	\item $7\text{ cm}$
	\item $10\text{ cm}$
	\item $12\text{ cm}$
	\item $17\text{ cm}$
	\end{enumerate}

\item Angle between two tangents from external point is $60^\circ$. Angle between radii to points of tangency is:
	\begin{enumerate}[label={\Alph*.}]
	\item $60^\circ$
	\item $90^\circ$
	\item $120^\circ$
	\item $150^\circ$
	\end{enumerate}

\item A chord divides circle into two parts. Angle subtended by smaller arc at center is $100^\circ$. Angle at circumference on larger arc is:
	\begin{enumerate}[label={\Alph*.}]
	\item $50^\circ$
	\item $80^\circ$
	\item $100^\circ$
	\item $130^\circ$
	\end{enumerate}

\item Locus of centers of circles passing through two fixed points is:
	\begin{enumerate}[label={\Alph*.}]
	\item Circle
	\item Perpendicular bisector of line joining points
	\item Line joining points
	\item Parallel line
	\end{enumerate}

\item Circle inscribed in equilateral triangle of side $a$ has radius:
	\begin{enumerate}[label={\Alph*.}]
	\item $a/2\sqrt{3}$
	\item $a/\sqrt{3}$
	\item $a\sqrt{3}/6$
	\item $a\sqrt{3}/3$
	\end{enumerate}

\item Two chords intersect inside circle. One is divided into segments $3\text{ cm}$ and $8\text{ cm}$. Other is $12\text{ cm}$. One segment of second chord is:
	\begin{enumerate}[label={\Alph*.}]
	\item $2\text{ cm}$
	\item $4\text{ cm}$
	\item $6\text{ cm}$
	\item $8\text{ cm}$
	\end{enumerate}

\item Cyclic quadrilateral with two adjacent right angles is a:
	\begin{enumerate}[label={\Alph*.}]
	\item Square
	\item Rectangle
	\item Trapezium
	\item Impossible
	\end{enumerate}

\item Maximum distance between two points on circle of radius $r$ is:
	\begin{enumerate}[label={\Alph*.}]
	\item $r$
	\item $r\sqrt{2}$
	\item $2r$
	\item $\pi r$
	\end{enumerate}

\item In circle, angle between tangent and chord equals angle in:
	\begin{enumerate}[label={\Alph*.}]
	\item Semicircle
	\item Alternate segment
	\item Same segment
	\item Minor segment
	\end{enumerate}

\item Number of tangents from external point to circle:
	\begin{enumerate}[label={\Alph*.}]
	\item $0$
	\item $1$
	\item $2$
	\item Infinite
	\end{enumerate}

\item Circle touching all sides of triangle is called:
	\begin{enumerate}[label={\Alph*.}]
	\item Circumcircle
	\item Incircle
	\item Excircle
	\item Concentric circle
	\end{enumerate}

\item The angle formed by a diameter at any point on the circumference is:
	\begin{enumerate}[label={\Alph*.}]
	\item $45^\circ$
	\item $90^\circ$
	\item $180^\circ$
	\item $360^\circ$
	\end{enumerate}

\end{enumerate}
\end{multicols}