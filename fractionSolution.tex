\subsection{Solution}
\begin{multicols}{2}
\begin{enumerate}[label={\arabic*.}]
    \item We begin by converting all the mixed fractions to improper fractions,\\
    \(3\dfrac{7}{8} = \dfrac{31}{8} \text { and } 1\dfrac{1}{3} = \dfrac{4}{3}\) and add them up. \\ 
    Taking LCM of \(8\) and \(3\) we get \(24\),
    \[\frac{31}{8} + \frac{4}{3} = \frac{(31 \times 3) + (8 \times 4) }{24} = \frac{93 + 32}{24} = \dfrac{125}{24} \]
    Similarly, 
    \[1\frac{2}{3} - \frac{3}{8} \Rightarrow \frac{5}{3} - \frac{3}{8}  \frac{(5 \times 8) - (3 \times 3)}{24} = \frac{31}{24}\]
    Now substract \(\dfrac{31}{24} \text { from } \dfrac{125}{24}\)
    \[\frac{125}{24} - \frac{31}{24} = \frac{125 - 31}{24} = \frac{94}{24} = 3\frac{11}{12} \text{(D)} \]
    \item \textbf{Amount he earned before the increase} i.e his Initial Salary \\
    Let initial Salary = \(x\) \hspace {10px}  new salary = N345\\
    So, listen He had 15\% of his initial salary (x) added to his initial earn (x)
    \[\text{initial salary + 15\% of initial = new salary}\]
    \[x + \dfrac{15}{100}x = 345 \Rightarrow \dfrac{100x + 15x}{100} = 345\]
    \[115x = 345 \times 100 \Rightarrow x = \left(\frac{345}{115}\right) \times 100\]
    \(\therefore 3 \times 100 = 300 \text{ (B)}\)\\
    \textbf{Note:} when give this: \( \dfrac{x \times 100}{y}\)its advisable to group in the form
    \(\left(\frac{x}{y}\right) \times 100\) , incase the fraction part becomes decimal the 100 only shift the
    decimal point. 

    \item
    \item The \textbf{time taken (t)} to complete a job is inversely proportional to \textbf{numbers ot men (n)} 
    i.e as the number of men increase the job is done faster. 
    \begin{equation}
        \text{time taken} \propto \dfrac{1}{\text{number of men}} \Rightarrow t \propto \dfrac{1}{n}
    \end {equation}
    Removing proportionality sign
    \[t = \dfrac{k}{n} \Rightarrow tn = k\]
    Generally: \(t_{1}n_{1} = t_{2}n_2 = \cdots = t_{k}n_{k}\)\\
    given \(t_{1} = 9\ , n_{1} = 12 , t_{2} = 6 \text{ and } n_{2} = ?\) \\
    Using: \(t_{1}n_{1} = t_{2}n_{2} \Rightarrow 9 \times 12 = 6 \times n_{2}\) \\
    taking the advantage that 6 $\mid$ 12 
    \[n_{2} = \left(\frac{12}{6}\right) \times 9 = 18 \text { men (E)}\]

    \item Remember convert to improper fraction first
    \[2\frac{5}{12} - 1\frac{7}{8} \times \frac{6}{5} \Rightarrow \frac{29}{12} - \frac{15}{8} \times \frac{6}{5}\]
    Applying BODMAS multilplication comes before substraction\\
    \[\therefore \hspace{10px} \dfrac{29}{12} - \left(\dfrac{\cancelto{3}{15} \times \cancelto{3}{6}}{\cancelto{4}{8} \times \cancelto{1}{5}}\right) \Rightarrow \dfrac{29}{12} - \dfrac{9}{4} = \dfrac{29 - (9 \times 3)}{12} = \frac{2}{12} = \frac{1}{6} \text{(C)}\]
    \item Selling Price = N45.00, profit\% = 8\%, let cost price = \(x\)
       \[ \%profit = \frac{Selling Price - Cost Price }{Cost Price} \times 100 \]
       \[8 = \frac{45 - x }{x} \times 100 \Rightarrow 8x = 4500 -100x\]
       \[108x = 4500 \Rightarrow x = \dfrac{4500}{108}  \]
       \(\therefore \hspace{20px} \text{ cost price = }\dfrac{4500}{108}\) \\
       Now the question says He had made a \%profit of 32\%, how much did he sell it? \\
       Using Formular, We have: 
       \[32 = \frac{SP - \dfrac{4500}{108}}{\dfrac{4500}{108}} \times 100 =  \frac{108\left(SP - \dfrac{4500}{108}\right)}{4500} \times 100\]
       \(\Rightarrow \hspace {10px} 32 =  \dfrac{108SP - 4500}{4500} \times 100 \) \\
       \(\Rightarrow \hspace{10px} \dfrac{32 \times 45\cancel{00}}{\cancel{100}} = 108SP - 4500\) \\
       \(\Rightarrow  \hspace{10px} SP = \dfrac{\left(32 \times 45\right) + 4500}{108} = \dfrac{5940}{108} = 55\) \vspace{5px}\\
       \(\therefore \hspace{10px} N55.00 \text{ Option } (B)\)
    \item 
    \item \textbf{In 1980:} The intial price \(a : x\) rose by 25\% and 10\% respectively, hence:
    \[a:x \Rightarrow \hspace{10px} a + \dfrac{25}{100}a : x + \dfrac{10}{100}x\]
    Which reduces to \(\dfrac{125}{\cancel{100}}a : \dfrac{110}{\cancel{100}}x \Rightarrow {125}a : 110x  \) \vspace{5px} \\
    dividing by 5 \( \dfrac{\cancelto{25}{125}}{\cancel{5}} a : \dfrac{\cancelto{22}{110}}{\cancel{5}} x \hspace {10px}\Rightarrow 25a : 22x \) \vspace{5px}\\
    Multiplying by 2 \hspace{10px} \(50a : 44x\) \hspace{5px}Option (A)
    \[\]
    \item For Questions like this its best to start from the bottom \\
    \textbf{Pre-requesite:} \\
     \textbf{Step 1:} \hspace{10px}\( 5 + \dfrac{6}{7} \Rightarrow \hspace{10px} \dfrac{41}{7}\) \vspace{5px} \\
     \textbf{Step 2:} \hspace{10px}\(4 \mathbin{/} \dfrac{41}{7} \Rightarrow \dfrac{28}{41}  \) \vspace{5px} \\
     \textbf{Step 3:} \hspace{10px}\(3 + \dfrac{28}{41} \Rightarrow \dfrac{151}{41}\) \vspace{5px}\\
     \textbf{Step 4:} \hspace{10px}\(2 ÷ \dfrac{151}{41} \Rightarrow \dfrac{82}{151}\) \vspace{5px} \\
     \textbf{Step 5:} \hspace{10px} \( 1 + \dfrac{82}{151} \Rightarrow \dfrac{233}{151}\) \\
    Option (C) 
    
    \item 
    \item 
    \item 
    \item 
    \item
    \item
    \item
    \item 
    \item
    \item
    \item 
    \item 
    \item 
    \item 
    \item
    \item
    \item
    \item 
    \item
    \item
    \item 
    \item 
    \item 
    \item 
    \item
    \item
    \item
    \item 
    \item
    \item
    \item 
    \item 
    \item 
    \item 
    \item
    \item
    \item
    \item 
    \item
    \item
    \item 
    \item 
    \item 
    \item 
    \item
    \item
    \item
    \item 
    \item
    \item
    \item 
    \item 
    \item 
    \item 
    \item
    \item
    \item
    \item 
    \item
    \item
    \item 
    \item 
    \item 
    \item 
    \item
    \item
    \item
    \item 
    \item
    \item
    \item 
    \item 
    \item 
    \item 
    \item
    \item
    \item
    \item 
    \item
    \item
    \item 
    \item 
    \item 
    \item 
    \item
    \item
    \item
    \item 
    \item
    \item
    \item 
    \item 
\end{enumerate}
\end {multicols}