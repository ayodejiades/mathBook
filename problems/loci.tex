\section{Loci}
\subsection{Questions}
\begin{multicols}{2}
\begin{enumerate}[label={\arabic*.}]

\item The locus of a point $P$ such that it is always $5\text{ cm}$ from a fixed point $O$ is a:
\begin{enumerate}[label={\Alph*.}, nosep]
\item Straight line
\item Circle of radius $5\text{ cm}$
\item Pair of parallel lines
\item Perpendicular bisector
\end{enumerate}

\item Points $A$ and $B$ are $8\text{ cm}$ apart. The locus of points equidistant from $A$ and $B$ is:
\begin{enumerate}[label={\Alph*.}, nosep]
\item A circle passing through $A$ and $B$
\item The perpendicular bisector of $AB$
\item A line parallel to $AB$
\item An angle bisector
\end{enumerate}

\item The locus of points at a constant distance of $3\text{ cm}$ from a straight line $l$ consists of:
\begin{enumerate}[label={\Alph*.}, nosep]
\item Two parallel lines, each $3\text{ cm}$ from $l$
\item A circle of radius $3\text{ cm}$
\item A single line perpendicular to $l$
\item Four lines forming a square
\end{enumerate}

\item Two lines $AB$ and $CD$ intersect at point $O$. The locus of points equidistant from both lines is:
\begin{enumerate}[label={\Alph*.}, nosep]
\item A circle centered at $O$
\item The perpendicular bisector of $AB$
\item The pair of angle bisectors of the angles formed at $O$
\item A line parallel to $AB$
\end{enumerate}

\item A point $P$ moves such that its distance from a fixed point $Q$ is always $7\text{ cm}$. What is the locus of $P$?
\begin{enumerate}[label={\Alph*.}, nosep]
\item A sphere of radius $7\text{ cm}$
\item A circle of radius $7\text{ cm}$ centered at $Q$
\item A straight line through $Q$
\item Two parallel lines $7\text{ cm}$ apart
\end{enumerate}

\item Points $X$ and $Y$ are $10\text{ cm}$ apart. How many points are exactly $6\text{ cm}$ from $X$ and equidistant from $X$ and $Y$?
\begin{enumerate}[label={\Alph*.}, nosep]
\item $0$
\item $1$
\item $2$
\item $4$
\end{enumerate}

\item The locus of the center of a circle that rolls without slipping along a straight line is:
\begin{enumerate}[label={\Alph*.}, nosep]
\item A circle
\item A straight line parallel to the given line
\item A curve called a cycloid
\item A perpendicular line
\end{enumerate}

\item A point $M$ is equidistant from two fixed points $P$ and $Q$ which are $12\text{ cm}$ apart. If $M$ is also $8\text{ cm}$ from $P$, what is the distance from $M$ to $Q$?
\begin{enumerate}[label={\Alph*.}, nosep]
\item $4\text{ cm}$
\item $8\text{ cm}$
\item $12\text{ cm}$
\item $20\text{ cm}$
\end{enumerate}

\item The locus of points in a plane that are $4\text{ cm}$ from a fixed point $O$ and also $4\text{ cm}$ from another fixed point $P$ where $OP = 6\text{ cm}$ consists of:
\begin{enumerate}[label={\Alph*.}, nosep]
\item $0$ points
\item $1$ point
\item $2$ points
\item Infinitely many points
\end{enumerate}

\item A ladder of length $5\text{ m}$ leans against a wall with its foot on the ground. As the ladder slides down, the locus of its midpoint is:
\begin{enumerate}[label={\Alph*.}, nosep]
\item A straight line
\item A circle
\item A quarter circle
\item A parabola
\end{enumerate}

\item Two points $A$ and $B$ are $14\text{ cm}$ apart. The locus of point $P$ such that $\angle APB = 90^\circ$ is:
\begin{enumerate}[label={\Alph*.}, nosep]
\item A straight line
\item A circle with $AB$ as diameter
\item The perpendicular bisector of $AB$
\item Two circles
\end{enumerate}

\item The locus of points at a distance of $5\text{ cm}$ from a circle of radius $3\text{ cm}$ centered at $O$ consists of:
\begin{enumerate}[label={\Alph*.}, nosep]
\item A single circle of radius $8\text{ cm}$
\item Two concentric circles of radii $2\text{ cm}$ and $8\text{ cm}$
\item A circle of radius $5\text{ cm}$
\item An ellipse
\end{enumerate}

\item A point $P$ moves such that it is always equidistant from two parallel lines $l_1$ and $l_2$ that are $10\text{ cm}$ apart. The locus of $P$ is:
\begin{enumerate}[label={\Alph*.}, nosep]
\item A line parallel to both $l_1$ and $l_2$, midway between them
\item A circle
\item The perpendicular bisector of the distance between $l_1$ and $l_2$
\item Two lines perpendicular to $l_1$ and $l_2$
\end{enumerate}

\item Points $R$ and $S$ are $6\text{ cm}$ apart. How many points are $4\text{ cm}$ from $R$ and $5\text{ cm}$ from $S$?
\begin{enumerate}[label={\Alph*.}, nosep]
\item $0$
\item $1$
\item $2$
\item $3$
\end{enumerate}

\item The locus of the vertex of a right angle that moves such that its arms always pass through two fixed points $A$ and $B$ is:
\begin{enumerate}[label={\Alph*.}, nosep]
\item A straight line
\item A circle with $AB$ as diameter
\item The perpendicular bisector of $AB$
\item An ellipse
\end{enumerate}

\item A point $Q$ is $8\text{ cm}$ from a fixed point $O$. How many points are $5\text{ cm}$ from $O$ and equidistant from $Q$ and $O$?
\begin{enumerate}[label={\Alph*.}, nosep]
\item $0$
\item $1$
\item $2$
\item $4$
\end{enumerate}

\item The locus of points that are $3\text{ cm}$ from a line segment $AB$ of length $8\text{ cm}$ forms:
\begin{enumerate}[label={\Alph*.}, nosep]
\item Two parallel line segments
\item A rectangle with semicircular ends (a stadium shape)
\item A circle
\item An ellipse
\end{enumerate}

\item Two intersecting lines form four angles at their intersection point $O$. If a point $P$ is equidistant from both lines, then $P$ lies on:
\begin{enumerate}[label={\Alph*.}, nosep]
\item A circle centered at $O$
\item One of the two angle bisectors
\item A line perpendicular to one of the given lines
\item The perpendicular bisector of $O$
\end{enumerate}

\item Points $M$ and $N$ are $20\text{ cm}$ apart. The number of points that are $12\text{ cm}$ from $M$ and also $12\text{ cm}$ from $N$ is:
\begin{enumerate}[label={\Alph*.}, nosep]
\item $0$
\item $1$
\item $2$
\item $4$
\end{enumerate}

\item A point moves such that the sum of its distances from two fixed points $F_1$ and $F_2$ is constant. The locus is:
\begin{enumerate}[label={\Alph*.}, nosep]
\item A circle
\item An ellipse with foci at $F_1$ and $F_2$
\item A hyperbola
\item The perpendicular bisector of $F_1F_2$
\end{enumerate}

\item The locus of centers of all circles passing through two given points $A$ and $B$ is:
\begin{enumerate}[label={\Alph*.}, nosep]
\item A circle with center at the midpoint of $AB$
\item The perpendicular bisector of $AB$
\item A line parallel to $AB$
\item A circle with $AB$ as diameter
\end{enumerate}

\item A point $P$ is equidistant from two points $A$ and $B$ where $AB = 16\text{ cm}$. If $PA = 10\text{ cm}$, then $PB$ equals:
\begin{enumerate}[label={\Alph*.}, nosep]
\item $6\text{ cm}$
\item $10\text{ cm}$
\item $16\text{ cm}$
\item $26\text{ cm}$
\end{enumerate}

\item The locus of points at a perpendicular distance of $6\text{ cm}$ from a straight line is:
\begin{enumerate}[label={\Alph*.}, nosep]
\item A single line parallel to the given line
\item Two lines parallel to the given line, one on each side
\item A circle of radius $6\text{ cm}$
\item A perpendicular line
\end{enumerate}

\item Points $C$ and $D$ are $9\text{ cm}$ apart. How many points are $5\text{ cm}$ from $C$ and $6\text{ cm}$ from $D$?
\begin{enumerate}[label={\Alph*.}, nosep]
\item $0$
\item $1$
\item $2$
\item $3$
\end{enumerate}

\item A point $P$ moves such that it is always twice as far from point $A$ as it is from point $B$, where $AB = 12\text{ cm}$. The locus of $P$ is:
\begin{enumerate}[label={\Alph*.}, nosep]
\item A straight line
\item A circle
\item An ellipse
\item A hyperbola
\end{enumerate}

\item The locus of points equidistant from three non-collinear points $A$, $B$, and $C$ is:
\begin{enumerate}[label={\Alph*.}, nosep]
\item The circumcenter of triangle $ABC$
\item The centroid of triangle $ABC$
\item The incenter of triangle $ABC$
\item The orthocenter of triangle $ABC$
\end{enumerate}

\item A circle of radius $4\text{ cm}$ is centered at point $O$. How many points are $7\text{ cm}$ from $O$ and also $3\text{ cm}$ from the circle?
\begin{enumerate}[label={\Alph*.}, nosep]
\item $0$
\item $1$
\item $2$
\item $4$
\end{enumerate}

\item Two points $E$ and $F$ are $5\text{ cm}$ apart. The locus of points $P$ such that $PE = 2 \cdot PF$ forms:
\begin{enumerate}[label={\Alph*.}, nosep]
\item A straight line
\item A circle (Circle of Apollonius)
\item The perpendicular bisector of $EF$
\item Two parallel lines
\end{enumerate}

\item The locus of the midpoint of all chords of a circle that are parallel to a given chord is:
\begin{enumerate}[label={\Alph*.}, nosep]
\item A diameter perpendicular to the given chord
\item A chord parallel to the given chord
\item A circle concentric with the given circle
\item The center of the circle
\end{enumerate}

\item A point $P$ is $10\text{ cm}$ from a line $l$ and also $10\text{ cm}$ from a point $Q$ that is $8\text{ cm}$ from line $l$. How many such points $P$ are possible?
\begin{enumerate}[label={\Alph*.}, nosep]
\item $1$
\item $2$
\item $3$
\item $4$
\end{enumerate}

\item The locus of points that are $4\text{ cm}$ from a given point and also $4\text{ cm}$ from a given line (not passing through the point) can have:
\begin{enumerate}[label={\Alph*.}, nosep]
\item At most $1$ point
\item At most $2$ points
\item At most $3$ points
\item At most $4$ points
\end{enumerate}

\item Points $G$ and $H$ are $15\text{ cm}$ apart. The number of points that are $10\text{ cm}$ from $G$ and equidistant from $G$ and $H$ is:
\begin{enumerate}[label={\Alph*.}, nosep]
\item $0$
\item $1$
\item $2$
\item $3$
\end{enumerate}

\item A point moves such that the difference of its distances from two fixed points is constant. The locus is:
\begin{enumerate}[label={\Alph*.}, nosep]
\item A circle
\item An ellipse
\item A hyperbola
\item A parabola
\end{enumerate}

\item The locus of points in a plane at a distance of $5\text{ cm}$ from a circle of radius $3\text{ cm}$ and lying outside the circle is:
\begin{enumerate}[label={\Alph*.}, nosep]
\item A circle of radius $2\text{ cm}$
\item A circle of radius $8\text{ cm}$
\item An annulus between radii $3\text{ cm}$ and $8\text{ cm}$
\item Two circles of radii $2\text{ cm}$ and $8\text{ cm}$
\end{enumerate}

\item Two perpendicular lines intersect at point $O$. The locus of points equidistant from these two lines forms:
\begin{enumerate}[label={\Alph*.}, nosep]
\item A circle centered at $O$
\item Two perpendicular lines bisecting the angles at $O$
\item Four lines
\item A square
\end{enumerate}

\item A point $P$ is equidistant from the sides of an angle. The locus of $P$ is:
\begin{enumerate}[label={\Alph*.}, nosep]
\item The angle bisector
\item A circle
\item A parallel line
\item The perpendicular bisector
\end{enumerate}

\item Points $J$ and $K$ are $18\text{ cm}$ apart. How many points are $10\text{ cm}$ from $J$ and $10\text{ cm}$ from $K$?
\begin{enumerate}[label={\Alph*.}, nosep]
\item $0$
\item $1$
\item $2$
\item $4$
\end{enumerate}

\item The locus of the center of a circle of radius $r$ that rolls on the outside of a fixed circle of radius $R$ is:
\begin{enumerate}[label={\Alph*.}, nosep]
\item A circle of radius $R$
\item A circle of radius $R + r$
\item A circle of radius $R - r$
\item A straight line
\end{enumerate}

\item A point is $12\text{ cm}$ from point $A$ and $5\text{ cm}$ from point $B$ where $AB = 13\text{ cm}$. How many such points exist?
\begin{enumerate}[label={\Alph*.}, nosep]
\item $0$
\item $1$
\item $2$
\item Infinitely many
\end{enumerate}

\item The locus of points from which tangents of equal length can be drawn to a circle is:
\begin{enumerate}[label={\Alph*.}, nosep]
\item A line perpendicular to the radius
\item Any point outside the circle
\item A concentric circle
\item The center of the circle
\end{enumerate}

\item A point $P$ moves such that its distance from a fixed line $l$ is always equal to its distance from a fixed point $F$ not on $l$. The locus of $P$ is:
\begin{enumerate}[label={\Alph*.}, nosep]
\item A circle
\item An ellipse
\item A parabola
\item A hyperbola
\end{enumerate}

\item Points $L$ and $M$ are $7\text{ cm}$ apart. How many points are $4\text{ cm}$ from $L$ and also $4\text{ cm}$ from $M$?
\begin{enumerate}[label={\Alph*.}, nosep]
\item $0$
\item $1$
\item $2$
\item $3$
\end{enumerate}

\item The locus of points at a distance of $2\text{ cm}$ from a square of side $6\text{ cm}$ forms:
\begin{enumerate}[label={\Alph*.}, nosep]
\item A larger square
\item A square with rounded corners
\item A circle
\item An octagon
\end{enumerate}

\item A point $P$ is such that $PA^2 + PB^2 = k$ where $A$ and $B$ are fixed points and $k$ is constant. The locus of $P$ is:
\begin{enumerate}[label={\Alph*.}, nosep]
\item A straight line
\item A circle
\item An ellipse
\item A parabola
\end{enumerate}

\item Two parallel lines $m$ and $n$ are $12\text{ cm}$ apart. The locus of points that are $5\text{ cm}$ from $m$ and $7\text{ cm}$ from $n$ consists of:
\begin{enumerate}[label={\Alph*.}, nosep]
\item $0$ points
\item $1$ line
\item $2$ lines
\item $4$ lines
\end{enumerate}

\item A point moves such that it is always $3\text{ cm}$ closer to point $A$ than to point $B$ where $AB = 10\text{ cm}$. The locus is:
\begin{enumerate}[label={\Alph*.}, nosep]
\item A circle
\item The perpendicular bisector of $AB$
\item A line parallel to $AB$
\item A hyperbola
\end{enumerate}

\item The locus of the center of a circle that touches two parallel lines is:
\begin{enumerate}[label={\Alph*.}, nosep]
\item A line perpendicular to both given lines
\item A line parallel to both given lines, midway between them
\item A circle
\item Two parallel lines
\end{enumerate}

\item Points $N$ and $O$ are $20\text{ cm}$ apart. How many points are $15\text{ cm}$ from $N$ and $15\text{ cm}$ from $O$?
\begin{enumerate}[label={\Alph*.}, nosep]
\item $0$
\item $1$
\item $2$
\item $4$
\end{enumerate}

\item A point $P$ moves such that the ratio $\frac{PA}{PB} = 2$ where $A$ and $B$ are fixed points $9\text{ cm}$ apart. The locus of $P$ is:
\begin{enumerate}[label={\Alph*.}, nosep]
\item A straight line
\item A circle
\item An ellipse
\item The perpendicular bisector of $AB$
\end{enumerate}

\item The locus of points that are $6\text{ cm}$ from point $O$ and also lie on a circle of radius $10\text{ cm}$ centered at $O$ consists of:
\begin{enumerate}[label={\Alph*.}, nosep]
\item $0$ points
\item $1$ point
\item Infinitely many points
\item The entire circle of radius $10\text{ cm}$
\end{enumerate}

\item A line segment $PQ$ of length $8\text{ cm}$ moves such that $P$ always lies on line $l_1$ and $Q$ always lies on line $l_2$ where $l_1 \perp l_2$. The locus of the midpoint of $PQ$ is:
\begin{enumerate}[label={\Alph*.}, nosep]
\item A straight line
\item A quarter circle of radius $4\text{ cm}$
\item A circle of radius $4\text{ cm}$
\item An ellipse
\end{enumerate}

\item Points $R$ and $S$ are $11\text{ cm}$ apart. The number of points that are $7\text{ cm}$ from $R$ and $8\text{ cm}$ from $S$ is:
\begin{enumerate}[label={\Alph*.}, nosep]
\item $0$
\item $1$
\item $2$
\item $3$
\end{enumerate}

\item The locus of points equidistant from two concentric circles of radii $4\text{ cm}$ and $8\text{ cm}$ is:
\begin{enumerate}[label={\Alph*.}, nosep]
\item A circle of radius $6\text{ cm}$
\item A circle of radius $12\text{ cm}$
\item The perpendicular bisector
\item No such locus exists
\end{enumerate}

\item A point $P$ is $5\text{ cm}$ from a line $l$. How many points are $3\text{ cm}$ from $P$ and $3\text{ cm}$ from line $l$?
\begin{enumerate}[label={\Alph*.}, nosep]
\item $1$
\item $2$
\item $3$
\item $4$
\end{enumerate}

\item The locus of the vertices of all right-angled triangles on the same base is:
\begin{enumerate}[label={\Alph*.}, nosep]
\item A straight line
\item A circle with the base as diameter
\item The perpendicular bisector of the base
\item Two circles
\end{enumerate}

\item Points $T$ and $U$ are $16\text{ cm}$ apart. How many points are $10\text{ cm}$ from $T$ and equidistant from $T$ and $U$?
\begin{enumerate}[label={\Alph*.}, nosep]
\item $0$
\item $1$
\item $2$
\item $3$
\end{enumerate}

\item A point moves such that the product of its distances from two fixed points is constant. The locus is:
\begin{enumerate}[label={\Alph*.}, nosep]
\item A circle
\item An ellipse
\item A cassinian oval
\item A hyperbola
\end{enumerate}

\item The locus of points that are $3\text{ cm}$ from a fixed point $O$ and also $3\text{ cm}$ from a fixed line $l$ that is $5\text{ cm}$ from $O$ consists of:
\begin{enumerate}[label={\Alph*.}, nosep]
\item $0$ points
\item $1$ point
\item $2$ points
\item $4$ points
\end{enumerate}

\item Two lines intersect at $60^\circ$. The locus of points $4\text{ cm}$ from the point of intersection and equidistant from both lines consists of:
\begin{enumerate}[label={\Alph*.}, nosep]
\item $2$ points
\item $4$ points
\item $6$ points
\item $8$ points
\end{enumerate}

\item A circle of radius $5\text{ cm}$ rolls along a straight line without slipping. The locus of a point on the circumference of the circle is:
\begin{enumerate}[label={\Alph*.}, nosep]
\item A straight line parallel to the given line
\item A circle
\item A cycloid
\item A sine wave
\end{enumerate}

\item Points $V$ and $W$ are $13\text{ cm}$ apart. The locus of points $P$ such that $\angle VPW = 90^\circ$ is:
\begin{enumerate}[label={\Alph*.}, nosep]
\item A straight line perpendicular to $VW$
\item A circle with $VW$ as diameter
\item The perpendicular bisector of $VW$
\item Two circles
\end{enumerate}

\item The locus of points in a plane such that the sum of squares of distances from two fixed points is constant forms:
\begin{enumerate}[label={\Alph*.}, nosep]
\item A straight line
\item A circle
\item An ellipse
\item A hyperbola
\end{enumerate}

\item A point $P$ is equidistant from two intersecting lines that meet at $45^\circ$. If $P$ is $6\text{ cm}$ from the point of intersection, how many such points exist?
\begin{enumerate}[label={\Alph*.}, nosep]
\item $2$
\item $4$
\item $6$
\item $8$
\end{enumerate}

\item The locus of the midpoints of all chords of length $8\text{ cm}$ in a circle of radius $5\text{ cm}$ is:
\begin{enumerate}[label={\Alph*.}, nosep]
\item The center of the circle only
\item A circle of radius $3\text{ cm}$
\item A circle of radius $4\text{ cm}$
\item A straight line
\end{enumerate}

\item Points $X$ and $Y$ are $24\text{ cm}$ apart. How many points are $15\text{ cm}$ from $X$ and $20\text{ cm}$ from $Y$?
\begin{enumerate}[label={\Alph*.}, nosep]
\item $0$
\item $1$
\item $2$
\item $3$
\end{enumerate}

\item A point moves such that its distance from point $A$ is always half its distance from point $B$ where $AB = 12\text{ cm}$. The locus is:
\begin{enumerate}[label={\Alph*.}, nosep]
\item A straight line
\item A circle
\item The perpendicular bisector of $AB$
\item A parabola
\end{enumerate}

\item The locus of centers of all circles of radius $3\text{ cm}$ that touch a given line $l$ is:
\begin{enumerate}[label={\Alph*.}, nosep]
\item A single line parallel to $l$
\item Two lines parallel to $l$, each $3\text{ cm}$ away
\item A circle
\item The line $l$ itself
\end{enumerate}

\item A point $P$ is $8\text{ cm}$ from point $A$ and $6\text{ cm}$ from point $B$ where $AB = 10\text{ cm}$. How many such points $P$ exist?
\begin{enumerate}[label={\Alph*.}, nosep]
\item $0$
\item $1$
\item $2$
\item Infinitely many
\end{enumerate}

\item The locus of points at a distance of $4\text{ cm}$ from a line segment $AB$ of length $6\text{ cm}$ and lying in the same plane consists of:
\begin{enumerate}[label={\Alph*.}, nosep]
\item Two parallel line segments only
\item A rectangle
\item A stadium shape (rectangle with semicircular ends)
\item A circle
\end{enumerate}

\item Points $Z$ and $A_1$ are $30\text{ cm}$ apart. The number of points that are $18\text{ cm}$ from $Z$ and $18\text{ cm}$ from $A_1$ is:
\begin{enumerate}[label={\Alph*.}, nosep]
\item $0$
\item $1$
\item $2$
\item $4$
\end{enumerate}

\item A square $ABCD$ has side length $8\text{ cm}$. The locus of points equidistant from all four vertices is:
\begin{enumerate}[label={\Alph*.}, nosep]
\item The center of the square only
\item A circle passing through all vertices
\item The perpendicular bisectors of all sides
\item Four points
\end{enumerate}

\item The locus of points that are $5\text{ cm}$ from a circle of radius $5\text{ cm}$ and lie inside the circle is:
\begin{enumerate}[label={\Alph*.}, nosep]
\item A circle of radius $0\text{ cm}$ (the center only)
\item A circle of radius $10\text{ cm}$
\item No such points exist
\item The entire interior of the circle
\end{enumerate}

\item A point $P$ moves such that $2 \cdot PA = 3 \cdot PB$ where $A$ and $B$ are fixed points. The locus of $P$ is:
\begin{enumerate}[label={\Alph*.}, nosep]
\item A straight line
\item A circle (Apollonius circle)
\item The perpendicular bisector of $AB$
\item An ellipse
\end{enumerate}

\item Two perpendicular lines $l_1$ and $l_2$ intersect at point $O$. The locus of points that are $7\text{ cm}$ from $O$ and equidistant from $l_1$ and $l_2$ consists of:
\begin{enumerate}[label={\Alph*.}, nosep]
\item $2$ points
\item $4$ points
\item $6$ points
\item $8$ points
\end{enumerate}

\item Points $B_1$ and $C_1$ are $14\text{ cm}$ apart. How many points are $9\text{ cm}$ from $B_1$ and $10\text{ cm}$ from $C_1$?
\begin{enumerate}[label={\Alph*.}, nosep]
\item $0$
\item $1$
\item $2$
\item $3$
\end{enumerate}

\item A point moves such that its distance from a fixed point is equal to its distance from a fixed circle (not containing the point). The locus is:
\begin{enumerate}[label={\Alph*.}, nosep]
\item A circle
\item An ellipse
\item A parabola
\item A conic section depending on the configuration
\end{enumerate}

\item The locus of the center of a circle of radius $4\text{ cm}$ that touches a fixed circle of radius $6\text{ cm}$ externally is:
\begin{enumerate}[label={\Alph*.}, nosep]
\item A circle of radius $2\text{ cm}$
\item A circle of radius $6\text{ cm}$
\item A circle of radius $10\text{ cm}$
\item A circle of radius $4\text{ cm}$
\end{enumerate}

\item A point $P$ is $12\text{ cm}$ from a line $l$ and also $13\text{ cm}$ from a point $Q$ that is $5\text{ cm}$ from line $l$. How many such points $P$ are possible?
\begin{enumerate}[label={\Alph*.}, nosep]
\item $0$
\item $1$
\item $2$
\item $4$
\end{enumerate}

\item Points $D_1$ and $E_1$ are $25\text{ cm}$ apart. The locus of points equidistant from $D_1$ and $E_1$ and also $20\text{ cm}$ from $D_1$ consists of:
\begin{enumerate}[label={\Alph*.}, nosep]
\item $0$ points
\item $1$ point
\item $2$ points
\item $4$ points
\end{enumerate}

\item The locus of points from which the angle subtended by a line segment $AB$ is $60^\circ$ forms:
\begin{enumerate}[label={\Alph*.}, nosep]
\item Two circular arcs on opposite sides of $AB$
\item A complete circle
\item The perpendicular bisector of $AB$
\item Two straight lines
\end{enumerate}

\item A triangle $ABC$ has $AB = 10\text{ cm}$, $BC = 8\text{ cm}$, and $AC = 6\text{ cm}$. The locus of points equidistant from all three sides is:
\begin{enumerate}[label={\Alph*.}, nosep]
\item The circumcenter
\item The centroid
\item The incenter
\item The orthocenter
\end{enumerate}

\item Points $F_1$ and $G_1$ are $21\text{ cm}$ apart. How many points are $13\text{ cm}$ from $F_1$ and $13\text{ cm}$ from $G_1$?
\begin{enumerate}[label={\Alph*.}, nosep]
\item $0$
\item $1$
\item $2$
\item $4$
\end{enumerate}

\item The locus of points at which a line segment subtends a constant angle $\alpha$ (where $0^\circ < \alpha < 180^\circ$) forms:
\begin{enumerate}[label={\Alph*.}, nosep]
\item Two straight lines
\item Two circular arcs
\item A complete circle
\item An ellipse
\end{enumerate}

\item A point $P$ moves such that $3 \cdot PA = 4 \cdot PB$ where $A$ and $B$ are $14\text{ cm}$ apart. The locus is:
\begin{enumerate}[label={\Alph*.}, nosep]
\item A straight line
\item A circle
\item A parabola
\item An ellipse
\end{enumerate}

\item The locus of midpoints of all line segments of length $12\text{ cm}$ with endpoints on two parallel lines $8\text{ cm}$ apart is:
\begin{enumerate}[label={\Alph*.}, nosep]
\item A line parallel to the given lines
\item Two lines parallel to the given lines
\item A region between two parallel lines
\item The entire plane
\end{enumerate}

\item Points $H_1$ and $I_1$ are $17\text{ cm}$ apart. The number of points that are $10\text{ cm}$ from $H_1$ and $10\text{ cm}$ from $I_1$ is:
\begin{enumerate}[label={\Alph*.}, nosep]
\item $0$
\item $1$
\item $2$
\item $4$
\end{enumerate}

\item A point moves such that the sum of its distances from two perpendicular lines is constant and equals $10\text{ cm}$. The locus is:
\begin{enumerate}[label={\Alph*.}, nosep]
\item A circle
\item A square
\item A straight line
\item A line segment
\end{enumerate}

\item The locus of the vertex of an isosceles triangle with a fixed base $AB$ and constant vertical angle $\alpha$ is:
\begin{enumerate}[label={\Alph*.}, nosep]
\item The perpendicular bisector of $AB$
\item A circle passing through $A$ and $B$
\item Two circular arcs
\item A straight line through $A$ and $B$
\end{enumerate}

\item Points $J_1$ and $K_1$ are $28\text{ cm}$ apart. How many points are $16\text{ cm}$ from $J_1$ and $20\text{ cm}$ from $K_1$?
\begin{enumerate}[label={\Alph*.}, nosep]
\item $0$
\item $1$
\item $2$
\item $3$
\end{enumerate}

\item A point $P$ is such that the angle $\angle APB = 45^\circ$ where $A$ and $B$ are fixed points $10\text{ cm}$ apart. The complete locus of $P$ consists of:
\begin{enumerate}[label={\Alph*.}, nosep]
\item One circular arc
\item Two circular arcs on opposite sides of $AB$
\item A complete circle
\item Two semicircles
\end{enumerate}

\item The locus of centers of all circles passing through a fixed point $P$ and having radius $r$ is:
\begin{enumerate}[label={\Alph*.}, nosep]
\item A point
\item A circle of radius $r$ centered at $P$
\item A line through $P$
\item The entire plane
\end{enumerate}

\item Points $L_1$ and $M_1$ are $32\text{ cm}$ apart. The locus of points equidistant from $L_1$ and $M_1$ and also $25\text{ cm}$ from $L_1$ consists of:
\begin{enumerate}[label={\Alph*.}, nosep]
\item $0$ points
\item $1$ point
\item $2$ points
\item $4$ points
\end{enumerate}

\item A rectangle has length $12\text{ cm}$ and width $5\text{ cm}$. The locus of points equidistant from all four vertices is:
\begin{enumerate}[label={\Alph*.}, nosep]
\item The center of the rectangle only
\item A circle
\item The diagonals of the rectangle
\item Four points
\end{enumerate}

\item The locus of points whose distance from a fixed point $O$ is twice their distance from a fixed line $l$ (where $O$ is not on $l$) is:
\begin{enumerate}[label={\Alph*.}, nosep]
\item A circle
\item An ellipse
\item A parabola
\item A hyperbola
\end{enumerate}

\item Points $N_1$ and $O_1$ are $19\text{ cm}$ apart. How many points are $12\text{ cm}$ from $N_1$ and $12\text{ cm}$ from $O_1$?
\begin{enumerate}[label={\Alph*.}, nosep]
\item $0$
\item $1$
\item $2$
\item $4$
\end{enumerate}

\item A circle has center $O$ and radius $8\text{ cm}$. The locus of midpoints of all chords of the circle that have length $12\text{ cm}$ is:
\begin{enumerate}[label={\Alph*.}, nosep]
\item The center $O$ only
\item A circle of radius $2\sqrt{7}\text{ cm}$
\item A circle of radius $6\text{ cm}$
\item A circle of radius $4\text{ cm}$
\end{enumerate}

\item A point moves such that it is always $4\text{ cm}$ from a fixed line and also $4\text{ cm}$ from a fixed point on that line. The locus consists of:
\begin{enumerate}[label={\Alph*.}, nosep]
\item $0$ points
\item $1$ point
\item $2$ points
\item $3$ points
\end{enumerate}

\item The locus of the center of a circle that passes through two fixed points $A$ and $B$ and has radius $r$ (where $r > \frac{AB}{2}$) is:
\begin{enumerate}[label={\Alph*.}, nosep]
\item A single point
\item The perpendicular bisector of $AB$
\item A circle centered at the midpoint of $AB$
\item Two points on the perpendicular bisector of $AB$
\end{enumerate}

\item Points $P_1$ and $Q_1$ are $26\text{ cm}$ apart. The number of points that are $17\text{ cm}$ from $P_1$ and equidistant from $P_1$ and $Q_1$ is:
\begin{enumerate}[label={\Alph*.}, nosep]
\item $0$
\item $1$
\item $2$
\item $3$
\end{enumerate}

\item A point $P$ moves in a plane such that the absolute difference of its distances from two fixed points is zero. The locus of $P$ is:
\begin{enumerate}[label={\Alph*.}, nosep]
\item A straight line
\item A circle
\item The perpendicular bisector of the line joining the two points
\item A single point
\end{enumerate}

\end{enumerate}
\end{multicols}