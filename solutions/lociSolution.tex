\subsection{Solutions}
\begin{enumerate}[label={\arabic*.}]
  \item \textbf{(B)} Circle of radius 5 cm.
  \item \textbf{(B)} The perpendicular bisector of $AB$.
  \item \textbf{(A)} Two parallel lines, each 3 cm from $l$.
  \item \textbf{(C)} The pair of angle bisectors.
  \item \textbf{(B)} A circle of radius 7 cm centered at $Q$.
  \item \textbf{(C)} 2 points (Intersection of circle rad 6 and perp bisector 5 cm from X).
  \item \textbf{(B)} A straight line parallel to the given line.
  \item \textbf{(B)} 8 cm.
  \item \textbf{(C)} 2 points (Intersect of circles radii 4 cm, OP=6 cm).
  \item \textbf{(C)} A quarter circle.
  \item \textbf{(B)} A circle with $AB$ as diameter.
  \item \textbf{(B)} Two concentric circles of radii 2 cm and 8 cm.
  \item \textbf{(A)} A line parallel to both $l_1$ and $l_2$, midway.
  \item \textbf{(C)} 2 points (Intersection of circles radii 4 and 5, RS=6).
  \item \textbf{(B)} A circle with $AB$ as diameter.
  \item \textbf{(C)} 2 points (Intersect of circle rad 5 and perp bis 4 cm from O).
  \item \textbf{(B)} A stadium shape.
  \item \textbf{(B)} One of the two angle bisectors.
  \item \textbf{(C)} 2 points (Intersect of circles rad 12, MN=20).
  \item \textbf{(B)} An ellipse.
  \item \textbf{(B)} The perpendicular bisector of $AB$.
  \item \textbf{(B)} 10 cm.
  \item \textbf{(B)} Two lines parallel to the given line.
  \item \textbf{(C)} 2 points (Intersect of circles radii 5 and 6, CD=9).
  \item \textbf{(B)} A circle.
  \item \textbf{(A)} The circumcenter of triangle $ABC$.
  \item \textbf{(C)} 2 points.
  \item \textbf{(B)} A circle (Circle of Apollonius).
  \item \textbf{(A)} A diameter perpendicular to the given chord.
  \item \textbf{(B)} 2 points.
  \item \textbf{(D)} At most 4 points.
  \item \textbf{(C)} 2 points (Intersect of circle rad 10 and perp bis 7.5 cm from G).
  \item \textbf{(C)} A hyperbola.
  \item \textbf{(B)} A circle of radius 8 cm.
  \item \textbf{(B)} Two perpendicular lines bisecting the angles.
  \item \textbf{(A)} The angle bisector.
  \item \textbf{(C)} 2 points (Intersect of circles rad 10, JK=18).
  \item \textbf{(B)} A circle of radius $R + r$.
  \item \textbf{(C)} 2 points (Intersect of circles 12 and 5, AB=13).
  \item \textbf{(C)} A concentric circle.
  \item \textbf{(C)} A parabola.
  \item \textbf{(C)} 2 points (Intersect of circles rad 4, LM=7).
  \item \textbf{(B)} A square with rounded corners.
  \item \textbf{(B)} A circle.
  \item \textbf{(B)} 1 line.
  \item \textbf{(D)} A hyperbola.
  \item \textbf{(B)} A line parallel to both, midway.
  \item \textbf{(C)} 2 points (Intersect of circles rad 15, NO=20).
  \item \textbf{(B)} A circle.
  \item \textbf{(A)} 0 points (Radius 6 != 10).
  \item \textbf{(C)} A circle of radius 4 cm.
  \item \textbf{(C)} 2 points.
  \item \textbf{(A)} A circle of radius 6 cm.
  \item \textbf{(B)} 2 points.
  \item \textbf{(B)} A circle with the base as diameter.
  \item \textbf{(C)} 2 points.
  \item \textbf{(C)} A cassinian oval.
  \item \textbf{(C)} 2 points.
  \item \textbf{(B)} 4 points.
  \item \textbf{(C)} A cycloid.
  \item \textbf{(B)} A circle with $VW$ as diameter.
  \item \textbf{(B)} A circle.
  \item \textbf{(B)} 4 points.
  \item \textbf{(B)} A circle of radius 3 cm.
  \item \textbf{(C)} 2 points.
  \item \textbf{(B)} A circle.
  \item \textbf{(B)} Two lines parallel to $l$.
  \item \textbf{(C)} 2 points.
  \item \textbf{(C)} A stadium shape.
  \item \textbf{(C)} 2 points.
  \item \textbf{(A)} The center of the square only.
  \item \textbf{(A)} The center only.
  \item \textbf{(B)} A circle (Apollonius circle).
  \item \textbf{(B)} 4 points.
  \item \textbf{(C)} 2 points.
  \item \textbf{(D)} A conic section.
  \item \textbf{(C)} A circle of radius 10 cm.
  \item \textbf{(C)} 2 points.
  \item \textbf{(C)} 2 points.
  \item \textbf{(A)} Two circular arcs.
  \item \textbf{(C)} The incenter.
  \item \textbf{(C)} 2 points.
  \item \textbf{(B)} Two circular arcs.
  \item \textbf{(B)} A circle.
  \item \textbf{(A)} A line parallel to the given lines.
  \item \textbf{(C)} 2 points.
  \item \textbf{(B)} A square.
  \item \textbf{(C)} Two circular arcs.
  \item \textbf{(C)} 2 points.
  \item \textbf{(B)} Two circular arcs.
  \item \textbf{(B)} A circle of radius $r$ centered at $P$.
  \item \textbf{(C)} 2 points.
  \item \textbf{(A)} The center of the rectangle only.
  \item \textbf{(D)} A hyperbola.
  \item \textbf{(C)} 2 points.
  \item \textbf{(B)} A circle of radius $2\sqrt{7}$ cm.
  \item \textbf{(C)} 2 points.
  \item \textbf{(D)} Two points on the perpendicular bisector.
  \item \textbf{(C)} 2 points.
  \item \textbf{(C)} The perpendicular bisector.
\end{enumerate}
