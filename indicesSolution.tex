\subsection{Solutions}
\begin{multicols}{2}
\begin{enumerate}[label={\arabic*.}]

    \item \textbf{(B)} First we separate the powers, and notice $2^6 = 64$, so it cancels out \\
    $25^{x-1} = \cancel{64} \left(\cfrac{5^6}{\cancel{2^6}}\right) \rightarrow 25^{x-1} = 5^6$, rewrite 25 as $5^2$ and resolve $(5^2)^{x-1} = 5^{2(x-1)}$
    \begin{align*} 
        \cancel{5}^{2(x-1)} &= \cancel{5}^6 \\
        2x - 2 &= 6 \rightarrow x = 4
    \end{align*}

    \item \textbf{(E)} The least number for which all the numbers can be written is 5:
    \[25^{x-1} = (5^2)^{x-1} = 5^{2(x-1)} = 5^{2x - 2}\]
    \[125^{x+1} = (5^3)^{x+1} = 5^{3(x+1)} = 5^{3x + 3}\]
    \[\cfrac{5^x \times 5^{2x-2}}{5^{3x+3}} = \cfrac{5^{x + 2x - 2}}{5^{3x + 3}} = 5^{3x - 2 - (3x + 3)} = 5^{-5}\]

    \item \textbf{(D)} Express them individually in standard form 
    $$(3.705 \times 10^1) \times (4.2 \times 10^{-3})$$ 
    rearrange and combine powers
    \begin{align*} 
        3.705 \times 4.2  \times 10^{-2} &= 15.561 \times 10^{-2} \\
        & = 1.5561 \times 10^{-1}
    \end{align*}

    \item \textbf{(B)} notice $64 = 2^6$ making it possible to group $64$ and $r^{-6}$, $64r^{-6} = 2^6\cdot r^{-6} = 2^6 \cdot (r^{-1})^6 $ \\
    Using the property: $$a^k \cdot b^k = (ab)^k \Rightarrow 2^6 \cdot (r^{-1})^6 = (2r^{-1})^6$$ \\
    hence, $\sqrt[3]{(64r^{-6})^{\frac{1}{2}}} = \sqrt[3]{((2r^{-1})^6)^{\frac{1}{2}}} = \sqrt[3]{(2r^{-1})^{\frac{6}{2}}}  = \sqrt[3]{(2r^{-1})^3}$ \\
    $$\sqrt[m]{a^{n}} = a^{\frac{m}{n}} \Rightarrow (2r^{-1})^{\frac{3}{3}} = 2r^{-1} = \frac{2}{r} $$

    \item \textbf{(C)} Rewrite $9^y$ as $(3^2)^y = (3^y)^2$ let $3^y = p$
    \begin{align*}
        9^y - 4(3^y) + 3 &\Rightarrow p^2 - 4p + 3 = 0 \\
        & = p^2 -3p - p + 3 \\ 
        &= p(p-3)-1(p-3) \\ 
        &= (p-1)(p-3) \\ 
        & = (3^y -1)(3^y -3) = 0
    \end{align*}
    $$\begin{tabular}{c|c}
        for $3^y - 1 = 0$ &  for $3^y - 3 = 0$ \\
        $\cancel{3}^y = 1 = \cancel{3}^0$ &  $\cancel{3}^y = \cancel{3}^1$ \\
        $\therefore \hspace{10pt} y = 0 $ & $\therefore \hspace{10pt} y = 1$
    \end{tabular} $$

    \item Working on the numerator:
    \begin{align*}
        9^{\frac{1}{3}} \times 27^{\frac{1}{2}} &= 3^{\frac{2}{3}} \times 3^{\frac{3}{2}} = 3^{\frac{2}{3} + \frac{3}{2}}
    \end{align*}
    \item \textbf{(B)} $ 3^{\frac{x}{2}} = 9^{\frac{1}{3}} \Rightarrow \cancel{3}^{\frac{x}{2}} = \cancel{3}^{\frac{2}{3}} \therefore x = \dfrac{4}{3} $
    \item $4^{\frac{1}{2}} = \sqrt{4} = 2 \hspace{15pt} \therefore 2^6 = 64$
    \item \textbf{(D)} If we let $2^{n-1} = k$ then the numerator $$3k - 4k  = -k = -2^{n-1} = -2^n\cdot 2^{-1}  $$We can also resolve the denominator if rewrite $2^{n+1}$ as $2\cdot 2^n$ and let $2^n = p$ 
    $$2\cdot2^n - 2^n = 2p - p = p = 2^n$$
    $\therefore \hspace{15pt} \dfrac{-\cancel{2^n} \cdot 2^{-1}}{\cancel{2^n}} = -2^{-1}$

    \item \textbf{(A)} $27^{\frac{1}{3}} = 3, 8^{\frac{2}{3}} = (2^3)^{\frac{2}{3}} = 4, 16^{\frac{2}{4}} = 4$ 
    $$\frac{27^{\frac{1}{3}} - 8^{\frac{2}{3}}}{16^{\frac{2}{4}} \times 2} = \dfrac{3 - 4}{4 \times 2} = -\dfrac{1}{8}$$
    \item \textbf{(E)} $16^{2x+3} = 4^{2(2x-3)}$ \\
    $\dfrac{4^{x+3}}{4^{2(2x - 3)}} = 1 \Rightarrow \cancel{4}^{x+3 - (4x - 6)} = \cancel{4}^0$ 
    \begin{align*}
    \Rightarrow \hspace{10pt} x + 3 - 4x + 6 &= 0\\
        -3x + 9 &= 0 & \therefore x  =-3 
    \end{align*}

    \item 
    \begin{align*} 
        (0.008)^{-\frac{1}{3}} = (8 \times 10^{-3})^{-\frac{1}{3}} &= 8^{-\frac{1}{3}} \times (10^{-3})^{-\frac{1}{3}} \\
        & = (8^{\frac{1}{3}})^{-1} \times 10 \\
        & = 2^{-1} \times 10  = \dfrac{10}{2}\\
        (0.16)^{-\frac{3}{2}} = (16 \times 10^{-2})^{-\frac{3}{2}} &= 16^{-\frac{3}{2}} \times (10^{-2})^{-\frac{3}{2}} \\
        &= (16^{-\frac{1}{2}})^3 \times 10^3 \\
        & = 4^{-3} \times 10^3 =\cfrac{1000}{4}
    \end{align*}

    \item \textbf{(E)} On the numerator: $3^{n-1} = \cfrac{3^n}{3}$,  can be simplified to
    $$3^n - 3^{n-1}= 3^n - \cfrac{3^n}{3} = 3^n\left(1- \cfrac{1}{3}\right) = 3^n \cdot \cfrac{2}{3}$$
    On the denominator: \(27 = 3^3\)
    \[3^3 \times 3^n - 3^3 \times 3^n \cdot \cfrac{2}{3} = 3^n \times 3^3\left(1-\cfrac{2}{3}\right) = 3^n \times 3^3 \cfrac{1}{3}\] 
    \(= \cfrac{\cancel{3^n} \cdot \frac{2}{3}}{\cancel{3^n} \times 3^3 \times \frac{1}{3}}  = \cfrac{1}{3^{3}} \times \cfrac{\frac{2}{3}}{\frac{1}{3}} = \cfrac{2}{27}\)

    \item \textbf{(B)} Express all in index form \\
    $0.0048 = 48 \times 10^{-4} = 2^4 \times 3 \times 10^{-4}$, $0.81 = 81 \times 10^{-2} = 3^4 \times 10^{-2}$, $0.027 = 27 \times 10^{-3} = 3^3 \times 10^{-3}$ and $0.04 = 4 \times 10^{-2} = 2^2 \times 10^{-2}$ \\
    The expression as it appears can be rewritten as: 
    \begin{align*}
        &= \sqrt{\cfrac{(2^4 \times 3 \times 10^{-4}) \times (3^4 \times 10^{-2}) \times 10^{-7}}{(3^3 \times 10^{-3}) \times (2^2 \times 10^{-2}) \times 10^6 }} \\
        & = \sqrt{\cfrac{2^4 }{2^2}\left( \cfrac{3^4 \times 3}{3^3} \right) \left(\cfrac{10^{-4} \times \cancel{10^{-2}} \times 10^{-7}}{10^{-3} \times \cancel{10^{-2}} \times 10^6}\right) } \\
        & = \sqrt{2^2 \times 3^2 \times 10^{-14} } = 2 \times 3 \times 10^{-7} = 6 \times 10^{-7}
    \end{align*}

    \item \textbf{(C)} \begin{align*} 
        3^{2y} -6(3^y) & = (3^y)^2 - 6(3^y) = 27 
    \end{align*}
    Let $3^y = k$ , $k^2 - 6k = 27 $
    \begin{align*} 
            & \Rightarrow k^2 - 6k -27 = 0 \\
            & = k^2 - 9k + 3k - 27 = 0 \\
            & = k(k-9)+3(k-9) = 0 \\
        &= (k+3)(k-9) = (3^y + 3)(3^y - 9) = 0
    \end{align*}
    \begin{tabular}{c|c}
        for $3^y + 3 = 0$ & for $3^y - 9 = 0$ \\
        $3^y = -3 $& $3^y = 9 = 3^2$ \\
            & $y = 2$
    \end{tabular} \\
<<<<<<< HEAD
   
    \item \textbf{(A)} Rewrite $5^{x+1} = 5^x \cdot 5$ and factor the expression 
   $ \Rightarrow \hspace{5pt} 5^{x} \cdot 5 + 5^x = 5^x (5 + 1) = \cfrac{5^x \times \cancel{6}}{\cancel{6}} = \cfrac{150}{6} \\
    5^x = 25 = 5^2  , \hspace{5pt} \therefore x = 2$ 
    \item \textbf{(E)} Just count the number of zeros you see \\
    $0.0000001 = 10^{-7}$ $ \hspace{5pt}\therefore  2n + 1 = -7, n = -4$
    \item
    \item \textbf{(A)} Rewrite $81$ using $3$, \\ $\sqrt[3]{81} = 81^{\frac{1}{3}} = (3^4)^{\frac{1}{3}} = 3^{\frac{4}{3}} = 3^x$,  $x = \dfrac{4}{3}$
=======
>>>>>>> f386337d19ddb25d828937a689d8af7981fe4234

    \item \textbf{(A)} Rewrite \( 5^{x+1} = 5^x \cdot 5 \) and factor the expression: 
    \[\Rightarrow \hspace{5pt} 5^x \cdot 5 + 5^x = 5^x (5 + 1) = \cfrac{5^x \times \cancel{6}}{\cancel{6}} = \cfrac{150}{6}\]
    \[5^x = 25 = 5^2 \, \hspace{5pt} \therefore x = 2 \]

    \item \textbf{(E)} Just count the number of zeros you see: \\
    \( 0.0000001 = 10^{-7} \) \(\hspace{5pt}\therefore 2n + 1 = -7, n = -4 \)

    \item 
    \item \textbf{(A)} Rewrite \( 81 \) using \( 3 \): \\ 
    \(\sqrt[3]{81} = 81^{\frac{1}{3}} = (3^4)^{\frac{1}{3}} = 3^{\frac{4}{3}} = 3^x, \, x = \dfrac{4}{3}\)
    \item \textbf{(B)} Simplifying:    
    \[x(x+1)^{-\frac{1}{2}} = \cfrac{x}{(x+1)^{\frac{1}{2}}} = \cfrac{x}{\sqrt{x+1}}\]
    \begin{align*}
    \text{Rationalizing:} \cfrac{x}{\sqrt{x+1}} &= \cfrac{x}{\sqrt{x+1}} \times \cfrac{\sqrt{x+1}}{\sqrt{x+1}} \\
    &= \cfrac{x\sqrt{x+1}}{x+1} = \cfrac{x(x+1)^{\frac{1}{2}}}{x+1}
    \end{align*} 
    The question can be rewritten as: 
    \begin{align*} 
        &= \cfrac{\cfrac{x(x+1)^{\frac{1}{2}}}{x+1} - (x+1)^{\frac{1}{2}}}{(x+1)^{\frac{1}{2}}} \\
        &= \cfrac{\cancel{(x+1)^{\frac{1}{2}}}\left(\cfrac{x}{x+1} - 1\right)}{\cancel{(x+1)^{\frac{1}{2}}}} \\
        &= \cfrac{x - (x+1)}{x+1} = -\cfrac{1}{x+1}
    \end{align*}

    \item \textbf{(C)} Approaching this problem directly is really slow and time-consuming. Instead, you'll want to let \( 0.8 = a \) and \( 0.5 = b \):
    \[\Rightarrow \cfrac{a \times a \times a - b \times b \times b}{a \times a + a \times b + b \times b} = \cfrac{a^3 - b^3}{a^2 + ab + b^2} = a - b\]
    \[\therefore \hspace{5pt} 0.8 - 0.5 = 0.3 = 3 \times 10^{-1}\]
        \textbf{Workings:} \vspace{-10pt}
        \begin{align*} 
            & a - b\\
            a^2 + ab + b^2 \,\, &\overline{ \hspace{3pt} a^3 - b^3 \hspace{50pt}} \\
            -& \hspace{5pt} \underline{a^3 + a^2b + ab^2} \\ 
            & \hspace{20pt} -a^2b - ab^2 - b^3 \\
            & \hspace{22pt} \underline{-a^2b - ab^2 - b^3} \\
            & \hspace{22pt} \underline{-------}
        \end{align*}
        On the day of the exam, you would avoid the long division if you remember it just like \( x^2 - y^2 = (x+y)(x-y) \). 

    \item \textbf{(E)} Apply same technique as before, let $a = 69842$ and $b = 30158$ \\
    $$\cfrac{a \times a - b \times b }{a - b} = \cfrac{a^2 - b^2}{a-b} = \cfrac{(a-b)(a+b)}{(a-b)} = a+b $$
    $ a + b = 69842 + 30158  = 100000 = 10^5$

    \item \textbf{(D)} \, \,$\cfrac{(3^2)^2 \times (2 \times 3^2)^4}{3^{16}} = 2^4 \cdot \cfrac{3^4 \times 3^8}{3^{16}} = \cfrac{2^4}{3^4} = \cfrac{16}{81}$

    \item \textbf{(C)} $121$ can be expressed in only two forms 
    \begin{tabular}{c|c}
        $m^n = 121 = 121^1$ & $m^n = 121 = 11^2$ \\
        $m = 121, n = 1$& $m = 11, n = 2$ \\
        $(m-1)^{n+1}$ & \\
            $= (121-1)^{1+1} = 120^2$ & $=(11 - 1)^{2+1} = 10^3$ \\    
    \end{tabular}

    \item Rewrite \( a^{-\frac{1}{2}} = \cfrac{1}{a^{\frac{1}{2}}} = \cfrac{1}{\sqrt{a}} \), then rationalize:
    \[\cfrac{1}{\sqrt{a}} \cdot \cfrac{\sqrt{a}}{\sqrt{a}} = \cfrac{\sqrt{a}}{a}\]
    Rationalize:
    \[\cfrac{1 - \sqrt{a}}{1 + \sqrt{a}} \cdot \cfrac{1 - \sqrt{a}}{1 - \sqrt{a}} = \cfrac{(1 - \sqrt{a})^2}{1 - a} = \cfrac{1 - 2\sqrt{a} + a}{1 - a}\]
    Since both expressions share the same base, we can rewrite the expression as:
    \[\cfrac{\left(\sqrt{a} + \cfrac{\sqrt{a}}{a}\right) + (1 - 2\sqrt{a} + a)}{1 - a}\]    

    \item \textbf{(A)}
    \begin{itemize} 
    \item $\left(\cfrac{1}{64}\right)^0 = 1$\\
    \item $64^{-\frac{1}{2}} = (64^{\frac{1}{2}})^{-1} = (\sqrt{64})^{-1}= 8^{-1} = \cfrac{1}{8}$ \\
    \end{itemize}
<<<<<<< HEAD
    The resulting solution: \\
    $$1 + \cfrac{1}{8} + 16 = 17 + \cfrac{1}{8} = 17\frac{1}{8}$$ \\
    \textbf{NOTE:} \textit{the reason} $(-32)^4 = 32^4$ \textit{ is because 4 is even, which is} $-32 \times -32 \times -32 \times -32, \therefore - \times - \times - \times - = +$  \\
    if the index was an odd number like 3, $(-32)^3 = -32 \times -32 \times -32, \therefore - \times - \times - = -$ 
=======
>>>>>>> f386337d19ddb25d828937a689d8af7981fe4234

\item \textbf{(A)} If \( a^{-1} = \cfrac{1}{a} \) and \( \left(\cfrac{1}{b}\right)^{-1} = b \),
    \[\left(\cfrac{a}{b}\right)^{-1} = \left(a \times \cfrac{1}{b}\right)^{-1} = a^{-1} \times \left(\cfrac{1}{b}\right)^{-1} = \cfrac{1}{a} \times b = \cfrac{b}{a}\]
    \[\boxed{\therefore \left(\cfrac{a}{b}\right)^{-1} = \cfrac{b}{a}}\]
    \[\left(\cfrac{x}{y}\right)^{5a - 3} = \left(\cfrac{x}{y}\right)^{-(17 - 3a)}, \, 5a - 3 = 3a - 17 \implies a = -7\]

    \item When this kind of expression is being posed, I've said your best bet isn't to solve it directly. Rather, you should think of a way to cancel out some numbers using a possible identity, with the most basic one being:
    \[a^2 - b^2 = (a + b)(a - b)\]
    With nothing to think of, my eyes took me to \( (0.4 + 0.2)^3 \). Because of the index 3, I decided to look at \( 0.064 - 0.008 \), where \( 64 = 4^3 \) and \( 8 = 2^3 \):
    \begin{align*}
    (0.064 - 0.008) &= (64 \times 10^{-3}) - (8 \times 10^{-3}) \\
    &= (4^3 \times (10^{-1})^3) - (2^3 \times (10^{-1})^3) \\
    &= (4 \times 10^{-1})^3 - (2 \times 10^{-1})^3 \\
    &= 0.4^3 - 0.2^3
    \end{align*}
    So, I just believed every other number contains at least \( 0.4 \) or \( 0.2 \):
    \[0.16 = (0.4)^2, \, 0.08 = 0.4 \times 0.2, \, 0.04 = (0.2)^2\]
    Rewrite the question as:
    \[\cfrac{(0.4^3 - 0.2^3)(0.4^2 - 0.2^2)}{(0.4^2 + 0.4 \times 0.2 + 0.2^2)(0.4 + 0.2)^3}\]
    Let \( 0.4 = a \) and \( 0.2 = b \), then:
    \[\cfrac{(a^3 - b^3)(a^2 - b^2)}{(a^2 + ab + b^2)(a + b)^3}\]
    \[= \cfrac{\cancel{(a^2 + ab + b^2)}(a - b)}{\cancel{a^2 + ab + b^2}} \cdot \cfrac{\cancel{(a + b)}(a - b)}{\cancel{(a + b)^3}} \]
    \[ = \cfrac{(a - b)(a - b)}{(a + b)^2} = \cfrac{(a - b)^2}{(a + b)^2} = \cfrac{a^2 - 2ab + b^2}{a^2 + 2ab + b^2} = \cfrac{0.16 - 2(0.08) + 0.04}{0.16 + 2(0.08) + 0.04} = \cfrac{0.04}{0.36} = \cfrac{1}{9}\]  

    \item \textbf{(C)}
    \begin{align*} 
    \text{ from inside}  \sqrt[n]{x^2} &= x^{\frac{2}{n}} \\
    (\sqrt[n]{x^2})^{\frac{n}{2}} & = (x^{\frac{2}{n}})^{\frac{n}{2}} = x^{\frac{2}{n} \cdot \frac{n}{2}} = x 
    \end{align*}
    then just raise what's inside to the power of $2 \rightarrow x^2$
    \item \textbf{(B)} \begin{align*} 
        3^x - 3^{x-1} &= 3^x -\cfrac{3^x}{3} = 3^x\left(1- \cfrac{1}{3}\right) =3^x \cdot \cfrac{2}{3} =  486 \\ \rightarrow \hspace{5pt}  3^x &=  \cfrac{\cancel{486}\times 3}{2} = 243 \times 3 = 3^5 \times 3 = 3^6, x = 6
    \end{align*}
    \item
    \begin{itemize} 
    \item $5\sqrt{5} = 5^1 \times 5^{\frac{1}{2}} = 5^{\frac{3}{2}}$
    \item $5^3 \divisionsymbol 5^{-\frac{3}{2}} = \cfrac{5^3}{5^{-\frac{3}{2}}} = 5^3 \times \cfrac{1}{5^{-\frac{3}{2}}} = 5$
    \end{itemize}

    \item \textbf{(D)} $(\sqrt{3})^5 = (3^{\frac{1}{2}})^5 = 3^{\frac{5}{2}}, \hspace{5pt} 9^2 =(3^2)^2 = 3^4  $ and $3\sqrt{3} = 3^1 \times 3^{\frac{1}{2}} = 3^{\frac{3}{2}}$ \\
        $$3^{\frac{5}{2}} \times 3^4 = 3^n \times 3^{\frac{3}{2}}\hspace{5pt} \Rightarrow 3^{\frac{5}{2} + 4} = 3^{n + \frac{3}{2}} $$ 
        $ \cfrac{5}{2} + 4 = n + \cfrac{3}{2} \Rightarrow \hspace{5pt} \therefore \hspace{5pt} n = \cfrac{5}{2} - \cfrac{3}{2} + 4 = 1 + 4 = 5  $
    \item \textbf{(C)} $243^{\frac{n}{5}} = (3^5)^{\frac{n}{5}} = 3^n$ and $9^n = (3^2)^n = 3^{2n} = \cfrac{3^n \times 3^{2n + 1}}{3^{2n} \times 3^{n-1}} = \cfrac{3^{n + 2n + 1}}{3^{2n + n -1}} = \cfrac{3^{3n + 1}}{3^{3n - 1}} = 3^{3n + 1 - (3n - 1)} = 3^2 = 9 $

    \item \( k^a k^b k^c = \cancel{k}^{a + b + c} = 1 = \cancel{k}^0, \,\, \therefore \, a + b + c = 0 \)
    \begin{align*} 
        (a + b + c)^3 &= a^3 + 3a^2(b + c) + 3a(b + c)^2 + (b + c)^3 \\
        &= a^3 + 3a^2(b + c) + 3a(b^2 + 2bc + c^2) + b^3 + 3b^2c + 3bc^2 + c^3 \\
        &= a^3 + 3a^2b + 3a^2c + 3ab^2 + 6abc + 3ac^2 + b^3 + 3b^2c + 3bc^2 + c^3 \\
        &= a^3 + b^3 + c^3 + (3a^2c + 3abc + 3ac^2) + (3a^2b + 3ab^2 + 3abc) + (3b^2c + 3bc^2) \\
        &= a^3 + b^3 + c^3 + 3ac(a + b + c) + 3ab(a + b + c) + 3bc(a + b + c) - 3abc
    \end{align*}
    \vspace{-20pt}
    \begin{align*}
    (a + b + c)^3 - 3ac(a + b + c) &= a^3 + b^3 + c^3 \\
    - 3ab(a + b + c) - 3bc(a + b + c) + 3abc
    \end{align*}
    So, if \( a + b + c = 0 \):
    \[0^3 - 3ac(0) - 3ab(0) - 3bc(0) + 3abc = 3abc\]

    \item \textbf{(C)}
        \begin{itemize} 
        \item \( 81^{3.6} = (3^4)^{3.6} = 3^{4 \times 3.6} = 3^{14.4} \)
        \item \( 9^{2.7} = (3^2)^{2.7} = 3^{2 \times 2.7} = 3^{5.4} \)
        \end{itemize}
        \[\therefore \hspace{10pt} 3^{14.4} \times 3^{5.4} = 3^{14.4 + 5.4} = 3^{19.8}\]
        \begin{itemize} 
        \item \( 81^{4.2} = (3^4)^{4.2} = 3^{16.8} \)
        \end{itemize}
        \[\therefore \hspace{10pt} 3^{16.8} \times 3^{1} = 3^{16.8 + 1} = 3^{17.8}\]
        \[\cfrac{3^{19.8}}{3^{17.8}} = 3^{19.8 - 17.8} = 3^2 = 9\]

    \item
        \begin{align*}
        6^{2n + 1} &= (2 \times 3)^{2n + 1} = 2^{2n + 1} \times 3^{2n + 1} \\
            &= 2^{2n} \cdot 2 \times 3^{2n} \cdot 3
        \end{align*}
        \( 18^n = (9 \times 2)^n = (3^2 \times 2)^n = 3^{2n} \times 2^n \) \\
        \( 12^{2n} = (2 \times 3)^{2n} = 2^{2n} \times 3^{2n} \)
        \[\therefore \cfrac{(\cancel{2^{2n}} \times 2 \times \cancel{3^{2n}} \times 3) \times \cancel{3^{2n}} \times 2^{4n}}{(\cancel{3^{2n}} \times 2^n) \times 2^n \times (\cancel{2^{2n}} \times \cancel{3^{2n}})} = \cfrac{2^{4n + 1} \times 3}{2^{2n}}\]
        \[= 2^{4n - 2n + 1} \times 3 = 2^{2n + 1} \times 3\]

<<<<<<< HEAD
    $\therefore \cfrac{(\cancel{2^{2n}} \times 2 \times \cancel{3^{2n}} \times 3) \times \cancel{3^{2n}} \times 2^{4n}}{(\cancel{3^{2n}} \times 2^{n}) \times 2^{n} \times (\cancel{2^{2n}} \times \cancel{3^{2n}})}  = \cfrac{2^{4n + 1} \times 3}{2^{2n}}$\\
     $= 2^{4n - 2n + 1} \times 3  = 2^{2n + 1} \times 3$
    \item \textbf{(C)} \begin{tabular}{c|c}
        $\cancel{2}^{x+y} = \cancel{2}^5$, $x+y = 5$ & $\cancel{3}^{3y - x} = \cancel{3}^3$, $3y -x  = 3$
=======
    \item \textbf{(C)}
    \begin{tabular}{c|c}
        $\cancel{2}^{x+y} = \cancel{2}^5$, $x + y = 5$ & $\cancel{3}^{3y - x} = \cancel{3}^3$, $3y - x = 3$
>>>>>>> f386337d19ddb25d828937a689d8af7981fe4234
    \end{tabular}
    \begin{align*}
        x + y &= 5 \hspace{.2in} ...(i) \\
        3y - x &= 3 \hspace{.2in} ...(ii) \\
        x + y &= 5 \\
        3y - x &= 3 \\
        4y &= 8 \implies y = 2
    \end{align*}
    If \( x + y = 5 \), and \( y = 2 \), then:    
    \[x + 2 = 5 \implies x = 3 \]
    \item 
    \item
    \item
    \item 
    \item 
    \item 
    \item 
    \item
    \item
    \item
    \item 
    \item
    \item
    \item 
    \item 
    \item 
    \item 
    \item
    \item
    \item
    \item 
    \item
    \item
    \item 
    \item 
    \item 
    \item 
    \item
    \item
    \item
    \item 
    \item
    \item
    \item 
    \item 
    \item 
    \item 
    \item
    \item
    \item
    \item 
    \item
    \item
    \item 
    \item 
    \item 
    \item 
    \item
    \item
    \item
    \item 
    \item
    \item
    \item 
    \item 
    \item 
    \item 
    \item
    \item
    \item
    \item 
    \item
    \item
    \item 
\end{enumerate}
\end{multicols}