\subsection{Solutions}
\begin{multicols}{2}
\begin{enumerate}[label={\textbf{\arabic*.}}]
\item \textbf{(A)} For the expression $f(x + 2)$ to equal $-1$, we must have $x = -3$.
    Let's verify by substitution:
    \begin{align*}
    f(-1) &= 2(-3)^2 + 7(-3) - 5 \\
    &= 2(9) - 21 - 5 \\
    &= 18 - 21 - 5 = -8
    \end{align*}

\item \textbf{(B)} Let's factor the expression \(a^2x + a^2y - b^2y - b^2x\):
    \begin{align*}
    &= a^2(x + y) - b^2(x + y) \\
    &= (a^2 - b^2)(x + y) \\
    &= (a + b)(a - b)(x + y)
    \end{align*}

\item \textbf{(A)} Given that $x - 1 = 0$ and $x + 1 = 0$ are factors:
    Let $x = 1$ in $x^3 + px^2 + qx + 6 = 0$:
    \begin{align*}
    1 + p + q + 6 &= 0 \\
    p + q &= -7 \quad ...(i)
    \end{align*}
    For $x = -1$:
    \begin{align*}
    -1 + p - q + 6 &= 0 \\
    p - q &= -5 \quad ...(ii)
    \end{align*}
    Solving by elimination:
    \begin{align*}
    2p &= -12 \implies p = -6 \\
    -6 + q &= -7 \implies q = -1
    \end{align*}
    Therefore, $p = -6$ and $q = -1$

\item \textbf{(B)} Using the difference of squares formula:
    \begin{align*}
    \frac{x - 1}{x^2 - 1} &= \frac{x - 1}{(x+1)(x-1)} = \frac{1}{x + 1}
    \end{align*}
    Therefore:
    \begin{align*}
    \frac{1}{x - 1} + \frac{x - 1}{x^2 - 1} &= \frac{1}{x - 1} + \frac{1}{x + 1}
    \end{align*}
    Replacing $x$ with $1-x$:
    \begin{align*}
    \frac{1}{(1-x) - 1} + \frac{1}{(1-x) + 1} &= -\frac{1}{x} + \frac{1}{2-x} \\
    &= -\frac{1}{x} - \frac{1}{x - 2}
    \end{align*}

\item Expand $(x + 3y + 5)(2x^2 + 5y + 2)$:
    \begin{align*}
    &= 2x^3 + 5xy + 2x + 6x^2y + 15y^2 + 6y + 10x^2 + 25y + 10 \\
    &= 2x^3 + 6x^2y + 5xy + 15y^2 + 31y + 10x^2 + 10
    \end{align*}
    \item If $x-1$ divides $kx^3 + 10x^2 + lx - 3$, then $x = 1$ must give zero:
    \begin{align*}
    k(1)^3 + 10(1)^2 + l(1) - 3 &= 0 \\
    k + 10 + l - 3 &= 0 \\
    k + l &= -7 \quad ...(i)
    \end{align*}
    When $x = -2$:
    \begin{align*}
    k(-2)^3 + 10(-2)^2 + l(-2) - 3 &= 27 \
    -8k + 40 - 2l - 3 &= 27 \
    -8k - 2l &= -10 \
    4k + l &= 5 \quad ...(ii)
    \end{align*}
    Solving these equations:
    \begin{align*}
    3k &= 12 \implies k = 4 \
    4 + l &= -7 \implies l = -11
    \end{align*}

    \item \textbf{(E)} Polynomial long division:
        \[
        \begin{array}{r|l}
        & 3x^2 + 7x - 6 \\
        \hline
        x-1 & 3x^3 + 4x^2 - 13x + 6 \\
        \cline{2-2}
        & 3x^3 - 3x^2 \\
        & \underline{\phantom{3x^3} 7x^2 - 13x} \\
        & 7x^2 - 7x \\
        & \underline{\phantom{3x^3 + 7x^2} -6x + 6} \\
        & -6x + 6 \\
        & \underline{\phantom{3x^3 + 7x^2 - 6x +} 0}
        \end{array}
        \]

        Factor \(3x^2 + 7x - 6\):
        \begin{align*}
        3x^2 + 7x - 6 &= 3x^2 + 9x - 2x - 6 \\
        &= 3x(x + 3) - 2(x + 3) \\
        &= (3x - 2)(x + 3)
        \end{align*}
<<<<<<< HEAD
    Solving by Elimination
    \begin{align*}
        p + q &= -7 \hspace{20pt} ...(i)\\
       \hspace{6pt} p - q &= -5  \hspace{20pt} ...(ii)\\
        2p &= -12 \hspace{10pt} \Rightarrow  p = -6
    \end{align*}
    substitute $p = -6$ in any equation \\
        $$p + q = -7 \Rightarrow -6 + q = -7 \Rightarrow q =  -1$$ \\
    $\therefore p \text{ and } q \text{ are } -6 \text{ and } -1$ respectively
    

\item \textbf{(B)} First resolve $x^2 -1$ to $(x - 1)(x + 1)$ difference of two square \\
    \[\therefore \hspace{10px} \dfrac{x - 1}{x^2 - 1} = \dfrac{\cancel{x -1}}{\cancel{(x-1)}(x+1)} = \dfrac{1}{x + 1}\]
    Now, \hspace{10px} $\dfrac{1}{x -1} + \dfrac{x -1}{x^2 - 1} = \dfrac{1}{x - 1} + \dfrac{1}{x + 1}$ \\\\
    Replace all $x$ with $1-x$ 
    \begin{align*}
    \dfrac{1}{(1-x) -1} + \dfrac{1}{(1-x) + 1} &= -\dfrac{1}{x} + \dfrac{1}{2 - x} \\
     &= -\dfrac{1}{x} - \dfrac{1}{x - 2} 
    \end{align*}
    \begin{align*}
    \textbf{Explanation:} \hspace{10px} \dfrac{1}{2 - x} = \dfrac{1}{-(-2 + x)} &= \dfrac{1}{-(x - 2)} \\ &= -\dfrac{1}{x - 2}
    \end{align*}
\item $(x + 3y + 5)(2x^2 + 5y + 2)$ \\
    $= 2x^3 + 5xy + 2x + 6x^2y + 15y^2 +  6y + 10x^2 + 25y + 10$ \\
    $= 2x^3 + 6x^2y + 5xy + 15y^2 + 31y + $

    \item If $x-1$ divides $kx^3+10x^2+lx-3$, means substituting $x=1$ in the equation the result is $0$ 
    \begin{align*}
        k(1)^3+10(1)^2+l(1)-3 &= 0 \\
        k + 10 + l - 3 &= 0 \\
        k + l + 7 &= 0 
    \end{align*} 
    \begin{equation}\tag{i}
        k + l = -7
    \end{equation} \vspace{-5pt}
    but if you subsittute for $x = 2$ you get $27$ \\ \vspace{-10pt}
    \begin{align*}
        k(-2)^3 + 10(-2)^2 + l(-2) - 3 &= 27 \\
        -8k + 40 -2l -3 &= 27 \\
        -8k - 2l + 37 &= 27 \\
        -8k -2l &= 27 - 37 \\
        -2(4k + l) = -10 \\
    \end{align*}
    \begin{equation}\tag{ii} 
        4k + l = 5
    \end{equation}
    Substracting equ(i) from equ(ii)
    \begin{align*} 
        k + l &= -7 \\
       -\hspace{6pt} 4k + l  &= 5 \\
       (k - 4k ) + (l - l) & = -7 - 5  \\
       -3k & = -12 \\
       \cfrac{\cancel{-3}k}{\cancel{-3}} & = \cfrac{\cancelto{4}{-12}}{\cancel{-3}}
    \end{align*} 
    $\therefore k = 4, \text{ substitute } k = 4 \text{ in any equation }$ 
       $$ k+l = -7 \,\, \Rightarrow 4 + l = -7 \,\, \Rightarrow l = -11$$
    $\therefore k = 4 \text{ and } l = -11$

    \item \textbf{(E)}
    \begin{align*}
        &\hspace{3pt} 3x^2 + 7x - 6 \vspace{-10pt}\\ 
        x-1 \hspace{3pt}&\overline{)3x^3+4x^2-13x +6} \\
        -\hspace{3pt}& \hspace{3pt} \underline{3x^3 - 3x^2} \\
        \hspace{-40pt} & \hspace{35pt}  7x^2 - 13x \\ 
       -\hspace{-28pt} & \hspace{35pt} \underline{7x^2 - 7x \hspace{10pt}} \\
        & \hspace{55pt} {-6x + 6} \\
        -\hspace{-50pt}& \hspace{55pt} \underline{-6x + 6} \\
        & \hspace{55pt} \underline{----}
    \end{align*}
    Find the roots of $3x^2 + 7x -6 = 0$ using factorization 
    \begin{align*}
        3x^2 + 7x -6 &= 3x^2 + 9x - 2x -6  \\
        & = 3x(x+ 3) -2(x + 3) \\
        & = (3x - 2)(x + 3)
    \end{align*}
    $\therefore \text{ the roots are } (x-1)(x+3)(3x - 2)$

    \item $(x^2 + x + 1)(x^2 - x + 1)$ \\
        $ =  x^4 - x^3 + x ^2 + x^3 - x^2 + x + x^2 -x + 1$ \\
        $ =  x^4 - x^3 + x^3 + x ^2 - x^2 + x^2 + x  -x + 1$ \\
        $= x^4 + x^2 + 1$
    \item \begin{align*}
         & \hspace{ 4 pt} x ^2 - x -6 \\
        x -1 & \overline{) x^3 - 2x^2 - 5x + 6} \\
        -\hspace{-1pt} & \underline{ \hspace{4pt} x^ 3 - x^2} \\
        -\hspace{-23pt} & \hspace{25pt} x^2 - 5x  \\
       - \hspace{-23pt} &  \hspace{ 25pt} \underline{x^2 + x} \\
       - \hspace{-43pt} & \hspace{ 45pt} 6x + 6 \\
       - \hspace{-43pt} & \hspace{ 45pt} \underline{6x + 6}  \\
        & \hspace{ 45pt} \underline{---} 
    \end{align*}
   Now solve the Quadratic equation 
   \begin{align*} 
    x^2 - x - 6 &= x^2 - 3x + 2x - 6 = 0 \\
    &  = x(x - 3) + 2(x - 3) = 0 \\
    & =  (x + 2)(x - 3) = 0
   \end{align*}
   $\therefore x = -2 \text{ or } 3$, so its other roots are $3 \text{ and} -2$
    \item for $x-1 = 0$ substitute $x = 1$ 
    \begin{align*}
       lx^3 + 2kx^2 + 24 &= l(1)^3 + 2k(1)^2 + 24 = 0 \\
       & = 3l + 2k + 24 = 0 \\ 
       &= 3l + 2k = -24 \hspace{15pt}...(i)
    \end{align*}
    for $x+2 = 0$ substitute $x = -2$ 
    \begin{align*}
       lx^3 + 2kx^2 + 24 &= l(-2)^3 + 2k(-2)^2 + 24 = 0 \\
       & = -8l + 8k + 24 = 0 \\ 
       & = -l + k + 3 = 0  \\
       &= -l + k = -3 \hspace{15pt} ...(ii)
    \end{align*}
    We would be solving by Elimation to take out $k$, so we would 
    have to modify $-l + k = -3 $ into $-2l + 2k = -6$ by multiplying by $2$ \\
    So, because the sign of the $k$ on both equation are same, we would use substraction \vspace{-8pt}
    \begin{align*} 
         3l + 2k &= -24 \hspace{15pt}...(i) \\
        - \hspace{7pt} (-2l + 2k &= -6) \hspace{15pt} ...(ii) \\
        5l & = 18 \Rightarrow l = \frac{18}{5} 
    \end{align*} 
    substitute $l = \frac{18}{5}$ into any equation \vspace{-10pt}
    \begin{align*}
      -l + k = -3 &\Rightarrow -\cfrac{18}{5} + k = -3 \\
       k &= -3 + \cfrac{18}{5} =\frac{3}{5} \
    \end{align*}
=======
        Therefore, the roots are \((x-1)(x+3)(3x - 2)\).
>>>>>>> f386337d19ddb25d828937a689d8af7981fe4234


    \item Factor $(x^2 + x + 1)(x^2 - x + 1)$:
        \begin{align*}
        &= x^4 - x^3 + x^2 + x^3 - x^2 + x + x^2 - x + 1 \
        &= x^4 + x^2 + 1
        \end{align*}
    
    \item 
        \[
        \begin{array}{r|l}
        & x^2 - x - 6 \\
        \hline
        x - 1 & x^3 - 2x^2 - 5x + 6 \\
        \cline{2-2}
        & x^3 - x^2 \\
        & \underline{\phantom{x^3} - x^2 - 5x} \\
        & - x^2 + x \\
        & \underline{\phantom{x^3 - x^2} - 6x + 6} \\
        & - 6x + 6 \\
        & \underline{\phantom{x^3 - x^2 - 6x} 0}
        \end{array}
        \]
        Now solve the quadratic equation:
        \begin{align*}
            x^2 - x - 6 &= x^2 - 3x + 2x - 6 = 0 \\
            &= x(x - 3) + 2(x - 3) = 0 \\
            &= (x + 2)(x - 3) = 0
        \end{align*}
        \[\therefore x = -2 \text{ or } 3\]
        So, its other roots are \(3\) and \(-2\).

    
   \item For \( x - 1 = 0 \), substitute \( x = 1 \):
        \begin{align*}
            lx^3 + 2kx^2 + 24 &= l(1)^3 + 2k(1)^2 + 24 = 0 \\
            &= l + 2k + 24 = 0 \\ 
            &= l + 2k = -24 \hspace{15pt}...(i)
        \end{align*}
        For \( x + 2 = 0 \), substitute \( x = -2 \):
        \begin{align*}
            lx^3 + 2kx^2 + 24 &= l(-2)^3 + 2k(-2)^2 + 24 = 0 \\
            &= -8l + 8k + 24 = 0 \\ 
            &= -8l + 8k = -24 \hspace{15pt}...(ii)
        \end{align*}
        We will solve by elimination to eliminate \( k \). Therefore, we must modify equation (ii) by multiplying by 1:
        \begin{align*}
            -8l + 8k &= -24 \hspace{15pt}...(ii)
        \end{align*}
        So, because the sign of \( k \) in both equations is the same, we will use subtraction:
        \begin{align*} 
<<<<<<< HEAD
            (p + q)^3 &= p^3 + 3p^2q + 3pq^2 + q^3 \\
            (p + q)^3 &= p^3 + q^3 + 3pq(p + q)  \\
            (p + q)^3 - 3pq(p + q) & = p^3 + q^3 \\
            (p+q)((p+q)^2 - 3pq) &= p^3 + q^3
        \end{align*}
        So, let replace them back, \\
        $ \Rightarrow (2a)^3 + (3b)^3 =(\underbrace{2a + 3b})\left((2a + 3b) - 3(2a)(3b)\right) $ \\
        $\therefore (2a + 3b)$ is common to all 3 expressions
    \item kaldk label
    ajdkj lfkasld 

    \item \textbf{Advice:} \textit{you shouldn't be tempted to expand every expression you see, as for this question $9$ is a perfect square, so this allow the use of }difference of two square \\
    $$\boxed{p^2 - q^2 = (p+q)(p-q)}$$ 
        $9 - (x^2 - 3x - 1)^2 = \underbrace{3^2}_{p} - (\underbrace{x^2 - 3x -1}_{q})^2$  \\
        $= (3 - (x^2 - 3x  - 1) )(3 + (x^2 - 3x  - 1) )$ \\
        $= (3 - x^2 + 3x  + 1 )(3 + x^2 - 3x  - 1 )$ \\
        $= (- x^2 + 3x  + 4 )(x^2 - 3x  + 2 )$ \\
    lets resolve each part, \vspace{-10pt}
=======
                l + 2k &= -24 \hspace{15pt}...(i) \\
            - &\hspace{7pt} -8l + 8k = -24 \hspace{15pt}...(ii) \\
            9l + 10k &= 24
        \end{align*}
        Substitute \( l = -\frac{1}{2} \) into any equation:
        \begin{align*}
            l + 2k = -24 &\Rightarrow -\cfrac{1}{2} + 2k = -24 \\
            k &= -12
        \end{align*}

    \item \textbf{(E)} \( 81a^4 - 16b^4 = \left(9a^2\right)^2 - \left(4b^2\right)^2 \)
        \[\left(9a^2\right)^2 - \left(4b^2\right)^2 = (9a^2 - 4b^2)(9a^2 + 4b^2)\]
        \[ = \left((3a)^2 - (2b)^2\right)(9a^2 + 4b^2)\]
        \[= (3a - 2b)(3a + 2b)(9a^2 + 4b^2)\] 

    \item 
    \[4a^2 - 9b^2 = (2a)^2 - (3b)^2 = (2a - 3b)(2a + 3b)\]
    \[(4a + 6b)^2 = \left(2(2a + 3b)\right)^2 = 2^2 (2a + 3b)^2\]
    \[ 8a^3 + 27b^3 = (2a)^3 + (3b)^3 \]
    To simplify \( (2a)^3 + (3b)^3 \), we should call \( 2a = p \) and \( 3b = q \), and follow along.
>>>>>>> f386337d19ddb25d828937a689d8af7981fe4234
    \begin{align*}
        (p + q)^3 &= p^3 + 3p^2q + 3pq^2 + q^3 \\
        &= p^3 + q^3 + 3pq(p + q) \\
        (p + q)^3 - 3pq(p + q) &= p^3 + q^3 \\
        (p + q)((p + q)^2 - 3pq) &= p^3 + q^3
    \end{align*}
    So, let's replace them back:
    \[\Rightarrow (2a)^3 + (3b)^3 = (2a + 3b)\left((2a + 3b)^2 - 3(2a)(3b)\right)\]
    \(\therefore (2a + 3b)\) is common to all 3 expressions.

    \item
    \item \textbf{Advice:} 
    \[ 9 - (x^2 - 3x - 1)^2 = 3^2 - (x^2 - 3x - 1)^2\]  
    \[ = (3 - (x^2 - 3x - 1))(3 + (x^2 - 3x - 1))\]
    \[ = (3 - x^2 + 3x + 1)(3 + x^2 - 3x - 1) \]
    \[= (-x^2 + 3x + 4)(x^2 - 3x + 2)\]
    Let's resolve each part:
    \begin{align*}
    -x^2 + 3x + 4 &= -x^2 - x + 4x + 4 \\
    &= -x(x + 1) + 4(x + 1) \\
    &= (-x + 4)(x + 1)\\
    x^2 - 3x + 2 &= x^2 - x - 2x + 2 \\
    &= x(x - 1) - 2(x - 1) \\
    &= (x - 2)(x - 1)
    \end{align*}
    The factors are \((-x + 4)(x + 1)(x - 2)(x - 1)\).

    \item \textbf{(D)} $f(x - 2) = 4x^2 + x + 7 $ equate $x - 2 = 1 \Rightarrow x = 3$ 
    \begin{align*} 
        f(x - 2) &= 4x^2 + x + 7 \\
        \Rightarrow  f(3 - 2) = f(1) &= 4(3)^2 + 3 + 7 \\
        & = 36 + 10 = 46
    \end{align*}

    \item Replace all \(y\)'s with \(y + 3\)
    \begin{align*} 
        g(y) &= \cfrac{y - 3}{11} + \cfrac{11}{y^2 - 9} \\
       \Rightarrow  g(y + 3) &= \cfrac{(y + 3) - 3}{11} + \cfrac{11}{(y + 3)^2 - 9}  \\
        & = \cfrac{y}{11} + \cfrac{11}{y^2 + 6y + 9 -9} \\
<<<<<<< HEAD
        & = \cfrac{y}{11} + \cfrac{11}{y^2 + 6y} = \cfrac{y}{11} + \cfrac{11}{y (y + 6 )}
    \end{align*}

    \item 
    \item \textbf{(C)} \begin{align*} 
        x^2 + 2a +ax + 2x &= x^2 + ax + 2a + 2x  \\
        & = x ( x + a ) + 2 (x + a) \\
        & = (x + 2)(x+a)
    \end{align*}
    \item 
    \item Applying difference of two square \\ \vspace{-5pt} $$p^2 - q^2 = (p + q)(p-q)$$ 
    \begin{align*}
        &\Rightarrow [\underbrace{(4a + 3)}_{p} + \underbrace{(3a -2)}_{q} ][\underbrace{(4a + 3)}_{p} - \underbrace{(3a -2)}_{q}] \\
        & = (7a + 1)(a + 5)
    \end{align*}
    \item \textbf{(D)} We should first reduce the equation to its quadratic form 
      \begin{align*}
        & \hspace{5pt} x^2 + 2x - 8 \\
        x - 2& \overline{)x^3 - 12x -16 } \\ 
        -\hspace{2pt} & \hspace{5pt}  \underline{x^3 - 2x^2} \\
        & \hspace{30pt} 2x^2 - 12x \\
    - \hspace{-23pt}& \hspace{30pt} \underline{2x^2 - 4x} \\
    -\hspace{-55pt} & \hspace{55pt} 8x - 16 \\
   -\hspace{-50pt} & \hspace{55pt} \underline{(-8x - 16)} \\
    \end{align*} 
    Factorizing the equation into roots 
    \begin{align*} 
        x^2 + 2x - 8 = 0 &\Rightarrow x^2 + 4x - 2x - 8 = 0 \\
        & = x(x+4) -2(x+4)  \\ &= (x-2)(x+4)
    \end{align*} 
    $x^3 - 12x -16 = (x-2)(x-2)(x+4)$ \\
    $\therefore$ the equation has 2 equal roots and a third different root

    
    \item \textbf{(D)} for $x + 1 = 0,  \Rightarrow x = -1 $ is a root \\
    substituting $x = -1$ in the equation \\
    $x^3 -4x^2 +cx + d$ 
    \begin{align*}
        &= (-1)^3 -4(-1)^2 + c(-1) + d = 0 \\
        & = -1 -4 -c + d = 0 \\
        & = -c + d = 5
    \end{align*}
    Substituting $x = -2$ gives 1 \\
    $x^3 -4x^2 +cx + d$
    \begin {align*}
     &= (-2)^3 -4(-2)^2 + c(-2) + d = 1 \\
    & = -8 -16 -2c + d = 1 \\
    & = -2c + d = 25
    \end{align*}
    Adding the two Equations
    \begin{align*}
         -c + d &= 5 \\
        -\hspace{-5pt} -2c + d &= 25 \\
        c &= -20
    \end{align*} 
    substitute $c= -20$ in any equation 
    \begin{align*}
        -c + d &= -(-20) +d = 5 \\
        & \Rightarrow \hspace{5pt}   d = 5 - 20  = -15
    \end{align*}
    $\therefore c = -20 \text{ and } d = -15$
    
    \item 

    \item Using the difference of two sqauare's \\
    $$p^2 - q^2 = (p + q)(p - q)$$
    $(x^2 +x)^2 - (2x + 2)^2$
    \begin{align*}
         &= \left[(x^2 + x) + (2x + 2)\right]\left[(x^2 +x ) - (2x + 2)\right] \\
         &= \left[x^2 + x + 2x + 2\right]\left[x^2 +x  -2x - 2\right] \\
         &= \left[x(x+ 1) + 2(x+ 1)\right]\left[x(x+1) -2(x+1)\right] \\
         & = (x+1)(x+1)(x+2)(x-2) \\
         & = (x+1)^2(x+2)(x-2)
=======
        & = \cfrac{y}{11} + \cfrac{11}{y^2 + 6y} = \cfrac{y}{11} + \cfrac{11}{y(y + 6)}
>>>>>>> f386337d19ddb25d828937a689d8af7981fe4234
    \end{align*}
    
    \item 
    \item \textbf{(C)}
        \begin{align*} 
            x^2 + 2a + ax + 2x &= x^2 + ax + 2a + 2x \\
            &= x(x + a) + 2(x + a) \\
            &= (x + 2)(x + a)
        \end{align*}

    \item give
    
    \item Applying the difference of two squares: \(p^2 - q^2 = (p + q)(p - q)\)
        \begin{align*}
            &\Rightarrow [\underbrace{(4a + 3)}_{p} + \underbrace{(3a - 2)}_{q}][\underbrace{(4a + 3)}_{p} - \underbrace{(3a - 2)}_{q}] \\
            &= (7a + 1)(a + 5)
        \end{align*}
    
    \item \textbf{(D)} We should first reduce the equation to its quadratic form:
        \begin{align*}
            & \hspace{5pt} x^2 + 2x - 8 \\
            x - 2 & \overline{x^3 - 12x - 16} \\ 
            - & \hspace{5pt} \underline{x^3 - 2x^2} \\
            & \hspace{30pt} 2x^2 - 12x \\
            - & \hspace{-23pt} \underline{2x^2 - 4x} \\
            & \hspace{55pt} 8x - 16 \\
            - & \hspace{-50pt} \underline{(-8x - 16)}
        \end{align*}
        Factorizing the equation into roots:
        \begin{align*} 
            x^2 + 2x - 8 = 0 &\Rightarrow x^2 + 4x - 2x - 8 = 0 \\
            &= x(x + 4) - 2(x + 4) \\
            &= (x - 2)(x + 4)
        \end{align*}
        Therefore, \(x^3 - 12x - 16 = (x - 2)(x - 2)(x + 4) \) \\
        \(\therefore\) The equation has \(2\) equal roots and a third different root. 
            
    \item \textbf{(D)} For \( x + 1 = 0 \), \(\Rightarrow x = -1\) is a root. \\
        Substituting \( x = -1 \) in the equation \( x^3 - 4x^2 + cx + d \):
        \begin{align*}
            (-1)^3 - 4(-1)^2 + c(-1) + d &= 0 \\
            -1 - 4 - c + d &= 0 \\
            -c + d &= 5
        \end{align*}
        Substituting \( x = -2 \) gives 1:
        \begin{align*}
            (-2)^3 - 4(-2)^2 + c(-2) + d &= 1 \\
            -8 - 16 - 2c + d &= 1 \\
            -2c + d &= 25
        \end{align*}
        Adding the two equations:
        \begin{align*}
            -c + d &= 5 \\
            -(-2c + d) &= -25 \\
            c &= -20
        \end{align*} 
        Substitute \( c = -20 \) in any equation:
        \begin{align*}
            -c + d &= -(-20) + d = 5 \\
            & \Rightarrow \hspace{5pt} d = 5 - 20 = -15
        \end{align*}
        Therefore, \( c = -20 \) and \( d = -15 \). 
    \item 
    
    \item Using the difference of squares: \(p^2 - q^2 = (p + q)(p - q)\)
    \[(x^2 + x)^2 - (2x + 2)^2\]
    \begin{align*}
         &= \left[(x^2 + x) + (2x + 2)\right]\left[(x^2 + x) - (2x + 2)\right] \\
         &= \left[x^2 + x + 2x + 2\right]\left[x^2 + x - 2x - 2\right] \\
         &= \left[x(x + 1) + 2(x + 1)\right]\left[x(x + 1) - 2(x + 1)\right] \\
         &= (x + 1)(x + 1)(x + 2)(x - 2) \\
         &= (x + 1)^2(x + 2)(x - 2)
    \end{align*}
    
    \item Given \(f(x) = 2x^2 + 5x + 3\), replace all \(x\) with \(x + 1\) 
     \begin{align*}
        f(x + 1) &= 2(x + 1)^2+ 5(x +1) + 3 \\
        & = 2(x^2 + 2x + 1) + 5(x + 1) + 3 \\
        & = 2x^2 + 4x + 2 + 5x + 1 + 3 \\
        & = 2x^2 + 9x + 6
     \end{align*}

    \item \textbf{(A)} \begin{align*} 
        & \hspace{5pt} x^ 2 + 2^{-1}x -4^{-1} \\ \vspace{-20pt} 
        x - 2^{-1}& \overline{)x^3 - 8^{-1} \hspace{.5in}} \\
        -\hspace{-1pt} & \hspace{5pt} \underline{x^3 - 2^{-1}x^2} \\
        & \hspace{30pt} 2^{-1}x^2 - 8^{-1} \\
        -\hspace{-25pt} & \hspace{30pt} \underline{2^{-1}x^2 - 4^{-1}x} \\
       -\hspace{-65pt} & \hspace{68pt} 4^{-1}x - 8^{-1} \\
       -\hspace{-65pt} & \hspace{68pt} \underline{4^{-1}x - 8^{-1} } \\
       & \hspace{68pt} \underline{-----} 
    \end{align*}
    The other factor is $x^ 2 + 2^{-1}x -4^{-1}$

    \item \textbf{(D)} Upon careful inspection, we notice that the question allows us to use the difference of squares:
        \[9(x + y)^2 - 4(x - y)^2\]
        Let's make it more useful by setting:
        \[\underbrace{3(x + y)}_{p}]^2 - [\underbrace{2(x - y)}_{q}]^2\]
        \begin{align*}
            & = \left[3(x + y) - 2(x - y)\right]\left[3(x + y) + 2(x - y)\right] \\
            & = \left(3x + 3y - 2x + 2y\right)\left(3x + 3y + 2x - 2y\right) \\
            & = (x + 5y)(5x - y)
        \end{align*}
    

    \item \textbf{(C)} The first thing to notice is that \(4\) and \(9\) are perfect squares. Additionally, they share factors with \(12\).
        \[4a^2 - 12ab - c^2 + 9b^2 = (2a)^2 - 2 \cdot 2a \cdot 3b - c^2 + (3b)^2\]
        Let \(p = 2a\) and \(q = 3b\),
        \begin{align*}
            &= (2a)^2 - 2 \cdot 2a \cdot 3b - c^2 + (3b)^2 \\
            &= p^2 - 2 \cdot p \cdot q - c^2 + q^2 \\
            &= p^2 - 2pq - c^2 + q^2 \\
            &= p^2 - 2pq + q^2 - c^2 \\
            &= (p - q)^2 - c^2
        \end{align*}
        Applying the difference of squares:
        \begin{align*}
            (p - q)^2 - c^2 &= (p - q - c)(p - q + c)
        \end{align*}
        Replacing them with their initial values:
        \begin{align*}
            (2a - 3b)^2 - c^2 &= (2a - 3b - c)(2a - 3b + c)
        \end{align*}

        \item \textbf{(B)} 
        \begin{align*}
            \frac{1}{2}(3y - 4x)^2 &= \frac{1}{2}(3y - 4x)(3y - 4x) \\
            &= \frac{1}{2}(9y^2 - 12xy - 12xy + 16x^2) \\
            &= \frac{1}{2}(9y^2 - 24xy + 16x^2) \\
            &= \frac{9}{2}y^2 - 12xy + 8x^2 \\ 
            &= 8x^2 - 12xy + \frac{9}{2}y^2 \\ 
            &= 8x^2 + (-12xy) + \frac{9}{2}y^2
        \end{align*}
        By direct comparison, we have:
        \[8x^2 + (-12xy) + \frac{9}{2}y^2 = 8x^2 + kxy + ly^2\]
        Thus, \( k = -12 \) and \( l = \frac{9}{2} \).
        

    \item \textbf{(C)} For \(x - 4 = 2\), \(x = 6\) 
    \begin{align*} 
        f(6 - 4) = f(2) &= 6^2 + 2(6) + 3 \\
        & = 36 + 12 + 3 \\
        & = 51
    \end{align*}

    \item \textbf{(D)} 
        \begin{align*}
            y^3 - 4xy + xy^3 - 4y &= y^3 + xy^3 - 4xy - 4y \\
            &= {y}^{3}(1 + x) - 4y(x + 1) \\
            &= (y^3 - 4y)(x + 1) \\
            &= y(y^2 - 4)(x + 1) \\
            &= y(y - 2)(y + 2)(x + 1) \\
            &= y(x + 1)(y + 2)(y - 2) \\
            &= (y + xy)(y + 2)(y - 2)
        \end{align*}
    
    \item \textbf{(A)} \begin{align*}
        g(x + 1) &= (x + 1) ^2 + 3(x + 1) + 4  \\
        g(x) & = x^2 + 2x + 4 \\
        g(x + 1) - g(x) & = (x+ 1)^2 - x^2 + 3(x+1)  \\
        & \hspace{10pt} - 3x + 4 - 4 \\
        & = (x + 1 - x)(x + 1 + x) \\
        &\hspace{10pt} + 3x + 3 - 3x \\
        & = 2x + 1 + 3 = 2x + 4 = 2(x + 2)
    \end{align*}
    
    \item \textbf{(D)} \begin{align*} 
        m^3 - m -2m^2 + 2 &= m(m^2 -1) -2 (m^2 -1) \\
        & = (m - 2)(m^2 - 1^2) \\
        & = (m-2)(m-1)(m+1)
    \end{align*}

    \item \textbf{(B)} \begin{align*} 
        & = rs -ps + tr - pt \\
        & = s(r-p) + t(r-p) \\
        & = (s + t)(r-p)
    \end{align*}
    $\therefore \hspace{10pt} (r-p) $ is a factor.

    \item \textbf{(A)} If $x + 1$ is a factor, $\therefore$ the value of the $x^3 + 3x^2 + kx + 4 = 0$ when $x = -1$ \\
    \begin{align*}
        x^3 + 3x^2 + kx + 4 &= (-1)^3 + 3(-1)^2 +k(-1) + 4 \\
        & = -1 + 3 -k + 4 = 0 \\
        k & = -6 
    \end{align*}

    \item Notice, \(-q^2\), \(6qr\) and \(-9r^2\) all have a \(q\), \(r\) or both and \(9p^2\) can be expressed as $(3r)^2$, the best i can think of is a difference of two squares. \\
    $$9p^2 - q^2 + 6qr - 9r^2 = (3p)^2 - \underbrace{(9r^2 - 6qr + q^2)}_{\text{like quadratic} }$$ 
    $9r^2 - 6qr + q^2$ can be expressed without $q$,  $9r^2 - 6r + 1$ 
    \begin{align*}
    9r^2 - 6r + 1 &= 9r^2 - 3r - 3r + 1 \\
    &= 3r(3r -1) -(3r -1) \\ 
    &= (3r -1)^2 \\
    9r^2 - 6qr + q^2  &= (3r -q)^2 \\
    9p^2 - q^2 + 6qr - 9r^2 &= (3p)^2 - (3r -q)^2 \\
    &= (3p + 3r - q)(3p - 3r + q) 
    \end{align*}

    \item \textbf{(B)} $x+1 = 0$ when $x = -1$ 
    \begin{align*} 
        f(0) = f(-1 + 1) &= 3(-1)^2 - (-1) + 4 \\
        & = 3 + 1 + 4 = 8
    \end{align*}

    \item $x+2 = 1$ at $x = -1$ 
    \begin{align*} 
        f(1) = f(-1 + 2) &= 3(-1)^2 + 4(-1) + 1 \\
        & = 3 - 4 + 1 = 0
    \end{align*}

    \item \textbf{(B)} they all share factors with 2 
     \begin{align*} 
        2(3x^2 - 7x - 6) & = 2(3x^2 - 9x + 2x -6) \\
        & = 2\left(3x(x - 3) + 2(x - 3)\right) \\
        & = 2(x-3)(3x +2)
    \end{align*}
    
    \item \textbf{(A)} $abx^2 -4bx -2axy + 8y$  \\
    $= bx(ax - 4) - 2y(ax - 4)$ \\
    $=(ax -4)(bx-2y)$

    \item \textbf{(C)} $1^2 - (a -b)^2 = (1 +a -b)(1 -a + b)$

    \item\textbf{(B)} \begin{align*}
        -2x^2 + 7x + 15 &= -2x^2 + 10x - 3x + 15 \\
        & = -2x(x -5) -3(x-5) \\
        & = (-2x -3)\underbrace{(x-5)}_{factor}
    \end{align*}
    
    \item \textbf{(A)} I'd like to take a different approach for this problem rather than long division \\
    \begin{align*} 
        \cfrac{x^3 - x  + 7x^2 - 7}{x^2 - 1} & = \cfrac{x(x^2 -1) + 7(x^2 -1)}{x^2 - 1}  \\
        & = \cfrac{(x+7)(\cancel{x^2 - 1})}{\cancel{x^2 -1}} = x+7
    \end{align*}
    \item if $x-p$ leaves remainder of 6  \\
    $2(p)^2 -p(p) + p = 6 \hspace{7pt} \Rightarrow  \underbrace{p}_{2} \underbrace{(p+1)}_{3} = 6$ \\
    $\therefore \hspace{10pt} p = 2$
    \item when divided by $(m-1)$ the remainder is $2$ \\
    $\therefore p(1)^2 + q(1) + 1 = 2 \text{ and } p + q = 1$\\
    when divided by $(m+1)$, remainder is $4$ \\
    $p(-1)^2 + q(-1) = p -q = 4$
    \begin{align*} 
        p + q &= 2 \hspace{20pt} ...(i) \\
       +\hspace{5pt}  p - q &= 4 \hspace{20pt} ...(ii) \\
       2p & = 6, \hspace{10pt} p = 3, q = -1
    \end{align*}
\item \textbf{(A)} 
    \(r^2 - 2pr - rq + 2pq = r^2 - rq - 2pr + 2pq \)\\
    \(= r(r - q) - 2p(r - q) = (r - q)(r - 2p)\)    
\item \textbf{(A)} 
        \begin{align*}
            & \hspace{5pt} x^2 + 5x + 6 \\
            2x + 1 \hspace{5pt} & \overline{) \hspace{3pt} 2x^3 + 11x^2 + 17x + 6} \\
            - & \hspace{5pt} \underline{2x^3 + x^2} \\
            & \hspace{30pt} 10x^2 + 17x \\
            - \hspace{-25pt} & \hspace{30pt} \underline{10x^2 + 5x} \\
            & \hspace{63pt} 12x + 6 \\
            - \hspace{-60pt} & \hspace{63pt} \underline{12x + 6} \\
            & \hspace{63pt} \underline{----}
        \end{align*}    
\item \textbf{(A)} \( x^2 + 2xy + y^2 = (x+y)^2 \) \\
    \( (x+y)^2 + 3(x+y) - 18 = 0 \) let \( x+y = k \) 
    \begin{align*}
    k^2 + 3k - 18 &= k^2 + 6k - 3k - 18 \\
    &= k(k+6) - 3(k+6) \\
    &= (k+6)(k-3)
    \end{align*}
    replace \( k \) back \( (x+y+6)(x+y-3) \)
\item 
    \begin{align*}
        & 2x^2 + x - 1  \\
        2x - 1 \hspace{5pt} & \overline{)\hspace{3pt} 4x^3 - 3x + 1} \\
        -\hspace{-2pt} & \hspace{8pt} \underline{4x^3 - 2x^2} \\
        & \hspace{36pt} 2x^2 - 3x \\
        -\hspace{-30pt} & \hspace{36pt} \underline{2x^2 - x} \\
        & \hspace{51pt} {-2x + 1} \\
        -\hspace{-45pt} & \hspace{51pt} \underline{-2x + 1} \\
        & \hspace{51pt} \underline{----}
    \end{align*}
<<<<<<< HEAD
    \item
\item \textbf{(B)} $-9y^2 = -(3y)^2 $ carries the (-ve) sign, so this is a difference of two square problem. We would rearrange
    $$(4x^2 + 20x + 25) - 9y^2$$ 
    producing a square for the left hand side 
    \begin{align*}
    4x^2 + 20x + 25 &= 4x^2 + 10x  +10x + 25\\ &= 2x(2x + 5) + 5(2x+5)\\ &= (2x+5)^2
    \end{align*}
    $$(2x+5)^2 - (3y)^2 = (2x+ 5 - 3y)(2x + 5 + 3y)$$
    \item \textbf{(B)} To avoid difficulties let $a^x = p$ \\
    now divide\,\, $p^3 - 26p^2 + 156p - 216$ by $p^2 - 24p + 108$ 
=======
\item
\item \textbf{(B)} \( -9y^2 = -(3y)^2 \) carries the negative sign, so this is a difference of two squares problem. We would rearrange
        \[(4x^2 + 20x + 25) - 9y^2 \]
        Producing a square for the left-hand side
        \begin{align*}
        &4x^2 + 20x + 25 
        &= 4x^2 + 10x + 10x + 25 \\
        &= 2x(2x + 5) + 5(2x + 5) \\
        &= (2x + 5)^2
        \end{align*}
    \[(2x + 5)^2 - (3y)^2 = (2x + 5 - 3y)(2x + 5 + 3y) \] 
\item \textbf{(B)} To avoid difficulties let \( a^x = p \) \\
    now divide \( p^3 - 26p^2 + 156p - 216 \) by \( p^2 - 24p + 108 \):
>>>>>>> f386337d19ddb25d828937a689d8af7981fe4234
    \begin{align*} 
        & \hspace{8pt} p - 2 \\
        p^2 - 24p + 108 \hspace{5pt} & \overline{)\hspace{3pt} p^3 - 26p^2 + 156p - 216} \\
       -\hspace{-1pt} & \hspace{7pt} \underline{p^3 - 24p^2 + 108p} \\
       & \hspace{25pt} {-2p^2 + 48p - 216} \\
      -\hspace{-20pt} & \hspace{25pt} \underline{-2p^2 + 48p - 216} \\
      & \hspace{25pt} \underline{--------}
    \end{align*}
    The result is \( p - 2 = a^x - 2 \)
\item
\item \textbf{(A)} \( x = \cfrac{4}{3} \) and \( x = -\cfrac{3}{5} \), \hspace{5pt} \( \therefore  x-\cfrac{4}{3} = 0 \) and \( x+\cfrac{3}{5} = 0 \) 
    \begin{align*}
    \left(x-\cfrac{4}{3}\right)\left(x+\cfrac{3}{5}\right) &= x^2 + \cfrac{3}{5}x - \cfrac{4}{3}x + \left(-\cfrac{4}{3} \times \cfrac{3}{5}\right) \\
    &= x^2  +  \cfrac{9 - 20}{15}\,x -\cfrac{4}{5} \\
    &= x^2 - \cfrac{11}{15}\,x -\cfrac{4}{5} \\
    &= 15x^2 - 11x - 12 = 0
    \end{align*}
\item 
    \begin{align*}
    &\hspace{6pt} 3x - 2y \\
        9x^2 + 6xy + 4y^2 \hspace{5pt}  & \overline{\hspace{2pt} 27x^3 - 8y^3 \hspace{1in}} \\
      -\hspace{-3pt} & \hspace{6pt} \underline{27x^3 + 18x^2y + 12xy^2} \\
      & \hspace{32pt} {-18x^2y - 12xy^2 - 8y^3} \\
     -\hspace{-26pt} & \hspace{32pt} \underline{-18x^2y - 12xy^2 -8y^3}\\
     & \hspace{32pt} \underline{----------}
    \end{align*}
    The other factor is \(3x - 2y\)
\item 
    \begin{align*}
    &\hspace{6pt} 3x - 2y \\
        9x^2 + 6xy + 4y^2 \hspace{5pt}  & \overline{\hspace{2pt} 27x^3 - 8y^3 \hspace{1in}} \\
      -\hspace{-3pt} & \hspace{6pt} \underline{27x^3 + 18x^2y + 12xy^2} \\
      & \hspace{32pt} {-18x^2y - 12xy^2 - 8y^3} \\
     -\hspace{-26pt} & \hspace{32pt} \underline{-18x^2y - 12xy^2 -8y^3}\\
     & \hspace{32pt} \underline{----------}
    \end{align*}
    The other factor is $3x - 2y$
<<<<<<< HEAD
    \item \textbf{(C)} \\
    $\cfrac{x(x^2 + 3x - 10)}{2(x^2 - 4)} = \cfrac{x\cancel{(x-2)}(x+5)}{2\cancel{(x-2)}(x+2)} = \cfrac{x(x+5)}{2(x+2)}$ \\
    \begin{align*}
    x^2 + 3x - 10 &= x^2 + 5x - 2x - 10 \\ 
    &= x(x+5) -2(x+5) \\ 
    &= (x-2)(x+5) \\
    x^2 - 4 = x^2 - 2^2 &= (x-2)(x+2) 
    \end{align*}
    \item 
    \item for $x+3 = 0 , x = -3$, subsitute ${x=-3}$ \\
    into the polynomial \\
=======
\item \textbf{(C)} \\
    \[\cfrac{x(x^2 + 3x - 10)}{2(x^2 - 4)} = \cfrac{x\cancel{(x-2)}(x+5)}{2\cancel{(x-2)}(x+2)} = \cfrac{x(x+5)}{2(x+2)}\]
        \begin{align*}
        x^2 + 3x - 10 &= x^2 + 5x - 2x - 10 \\ 
        &= x(x+5) - 2(x+5) \\ 
        &= (x-2)(x+5) \\
        x^2 - 4 &= x^2 - 2^2 \\ 
        &= (x-2)(x+2)
        \end{align*}    
\item 
\item for $x+3 = 0 , x = -3$, subsitute $x=-3$ into the polynomial \\
>>>>>>> f386337d19ddb25d828937a689d8af7981fe4234
    $$3(-3)^3 + 5(-3)^2 + 11(-3) - 4 = -81 + 45 -33 - 4$$ 
    \item \textbf{(B)} solve it without one of the variable, i'd work \\ without $x$ 
    \begin{align*}
    2y^2 - 15y + 18 &= 2y^2 -12y - 3y + 18  \\ 
    &= 2y(y-6) -3(y-6)  \\ 
    &= (2y-3)(y-6)
    \end{align*}
    Replace the $x$ on those without any variable 
    $$(2y-3x)(y-6x)$$
\item if $y-1$ is a factor then $1$ is a root 
    \begin{align*}
    1^3 + 4(1)^2 + k(1) - 6 &= 0 \\
    3+ 4 + k - 6 \hspace{5pt} &\Rightarrow \hspace{5pt} k = -1
    \end{align*}
    \item \textbf{(B)} 
    \begin{align*} 6x^2 + 3x + 10x + 5 &= 3x(2x + 1) + 5(2x+1) \\ &= (3x+5)(2x+1) \end{align*}
    \item \textbf{(C)} if $-2, -1 \text{ and } 3$ are roots, then $(x+2), (x+1) \text{ and } (x-3)$ are factors 
        \begin{align*} 
            (x+2)(x+1)(x-3) &= 0 \\
            (x^2 + 2x + x + 2)(x-3) &= 0 \\
            (x^2 + 3x + 2)(x-3) &= 0 \\
            x^3 - 3x^2 +3x^2 - 9x + 2x -6 & = 0 \\
            x^3 -7x -6 & = 0 
        \end{align*}
    \item for $2x+1 = 0, x = -\cfrac{1}{2}$ \\
    \begin{align*} 
        k\left(-\cfrac{1}{2}\right)^3 + \left(-\cfrac{1}{2}\right)^2 - 5\left(-\cfrac{1}{2}\right) - 2 &= 2 \\
         -\cfrac{k}{8} + \cfrac{1}{4}+ \cfrac{5}{2}  = 4\hspace{10pt} \rightarrow -k + 2 + 20 &= 16 \hspace{10pt} \rightarrow k = 6
    \end{align*}
\item start with trial and error, for $x = 1$ 
    $$(1)^3 - 2(1)^2-5(1) + 6 = 0 $$ Since $1$ is a root and $(x-1)$ is a factor \\
    i'd take a different approach, rather than the long division using synthethic division \\ Divide $x^3 - 2x^2 - 5x + 6 = 0$ by $x-1$
     \begin{itemize}
        \item  hello man
     \end{itemize}

\item  
    $$2t^2 -5t + 6t - 15 = 2t(t + 3)-5(t-3) = (2t-5)(t+3)$$
\item \textbf{(B)} $2(4)^2 -k(4) -12 = 32-4k -12=0  \rightarrow k = 5$
\item 
\item \textbf{(A)} Long Division works for this, so use it. let's work it out manually 
    \begin{align*}
     x^3 -2x^2 + 3x - 3 &= x^3 + x -2x^2 + 2x -3 \\
     & = x(x^2 + 1) -2x^2 -2 + 2x -1 \\
     & = x(x^2 + 1) -2(x^2 + 1) + (2x -1)
    \end{align*}
So we see all the parts are divisible by $x^2 + 1$ except $2x-1$ because we can't factor $x^2 + 1$ from it, $\therefore $ the remainder is $2x -1$

\item \textbf{(D)} Substitute $x=-3$ \\
$2(-3)^3 - 11(-3)^2 + 8(-3) -1$
\begin{align*}
 &= 2(-27) -11(9) -24 -1 \\
& = -54 - 99 -24 -1 = -178
\end{align*}
\item \textbf{(A)} let $2x+3y =k$ and $2x -3y =p$, then we can rewrite it as 
\begin{align*}
k^2 + 2kp + p^2  = (k+p)^2 &= (2x + 3y + 2x - 3y)^2 \\
& = (4x)^2 = 16x^2
\end{align*}
\item \textbf{(D)}
$\begin{tabular}{c|c|c|c}
    $5$ &$45a^3b$ & $5ab^3$ & $-30a^2b^2$ \\ \hline 
    $a$ & $9a^3b$ & $ab^3$ & $-6a^2b^2$ \\ \hline 
    $b$ & $9a^2b$ & $b^3$ & $-6a^2b^2$ \\ \hline
    & $9a^2$ & $b^2$ & $-6a^2b$ \\
\end{tabular}$ \\\\
$\therefore \,\, 5ab(9a^2 + b^2 -6a^2b)$ \\

\item \textbf{(B)} let $(a-b) = p, (b-c) = q \text{ and } c-a = r$ \\
\begin{align*}
p^3 + q^3 + r^3 & = (p + q + r)^3 - 3pq(p + q + r) \\
& - 3pr(p+q+r) - 3qr(p + q + r) + 3pqr
\end{align*}
Working on each of them \\
$(p + q + r)^3 = (a - b + b-c + c-a)^3 = 0$ \\
so every part that has $(p + q + r) = 0 $ , except \\
$$3pqr = 3(a-b)(b-c)(c-a)$$
\textit{if this was an exam like jamb, my first choice would have been B, because after replacement the values they would most likely retain themselves and 3 and 6 are usually the coefficient of any cubic expansion }
\item \textbf{(B)} $-2a - \cfrac{2}{a} = -2\left(a+ \cfrac{1}{a}\right)$ \\
$ \hspace{5pt}\therefore \, a^2 + \cfrac{1}{a^2}  \,-2\left(a+ \cfrac{1}{a}\right) + 3$ \\
$\left(a + \cfrac{1}{a}\right)^2 = a^2 + \cfrac{1}{a^2}+ 2a\cdot\cfrac{1}{a} = a^2 + \cfrac{1}{a^2} + 2$ \\
$\therefore a^2 + \cfrac{1}{a^2} = \left(a + \cfrac{1}{a} \right)^2 - 2$ \\
let's compose the question as: 
$$\left(a + \cfrac{1}{a}\right)^2 -2\left(a + \cfrac{1}{a}\right) - 1 = k^2 -2k +1 = (k+1)^2$$
replacing the value of $k$ back \\
$(k+1)^2 = \left(a+ \cfrac{1}{a} + 1\right)\left(a + \cfrac{1}{a} + 1\right)$

\item let $3x + 5y = k$ and $2x + 3y = p$ 
\begin{align*}
\therefore \hspace{5pt} 9k^2 -12kp + 4p^2 &= 9k^2 -6kp -6kp + 4p^2  \\
&= 3k(k-p)-4p(3k-p)
\end{align*}
\item \textbf{(B)} let $a - b + c = k $ and $b-c +a = p$ \\
$\therefore k^2 + 2kp + p^2 = k^2 + kp + kp +p^2 = (k + p)^2$ \\
$\Rightarrow \, (a - b + c  + b -c +a )^2 = (2a)^2 = 4a^2$
\item $81(xy)^2 - 108(xy)z + 36z^2$ 
\begin{align*}
&= 9\left(9(xy)^2 - 12(xy)z + 4z^2\right) \\
&= 9\left(9(xy)^2 - 6(xy)z - 6(xy)z + 4z^2\right) \\
&= 9\left(3xy(3xy - 2z) - 2z(3xy - 2z)\right) \\
&= 9\left(3xy - 2z\right)\left(3xy - 2z\right) = 9(3xy -2z)^2 \\
& = 3^2(3xy -2z)^2 = (3 \cdot (3xy -2z))^2 = (9xy -6z)^2
\end{align*}
\item \textbf{(A)} if $(x+1)$ is a factor, then  $-1$ is a root, $\therefore  f(-1) = 0$ 
\begin{align*}
    f(-1) &= 2(-1)^3 -a(-1)^2 -(2a-3)(-1) + 2 = 0 \\
    & = -2 -a + 2a -3 + 2 = a -3 =0 , \hspace{5pt} \therefore a = 3
\end{align*}
\item \textbf{(B)} If $x-2$ is a factor, then $2$ is a root\\
$(2)^3 -2a(2)^2 + a(2) - 1$ \vspace{-10pt}
\begin{align*} 
     &= 8 - 8a + 2a - 1 = -6a + 7 = 0 \\
     & \Rightarrow -6a = -7, \hspace{10pt} \therefore \,\, a = \cfrac{7}{6}
\end{align*}
\item sustitute $p = 3$ into the polynomial 
    \begin{align*} 
        6(3)^3 -(3)^2 -47(3) + 30 = 
    \end{align*}
\item \textbf{(A)} $ \, k^2 -kp -kp + p^2 = k(k-p) -p(k-p) = (k-p)^2$
\item \textbf{(C)} substitute $x=-1$ into the polynomial, and then equate it to zero because $x+1$ is a factor \\
\begin{align*} 
    (-1)^3 + 3(-1)^2 + m(-1) + 4 & = -1 + 3 -m + 4 = 0 \\
    & = 6- m  = 0, \, \therefore m = 6
\end{align*}
\item \textbf{(D)}
\begin{align*} 
    & \hspace{7pt} x + 7\\
   x^2 - 1\hspace{3pt}  & \overline{) \hspace{3pt} x^3 + 7x^2 - x - 7} \\
   -& \hspace{7pt} \underline{x^3 - x} \\
   & \hspace{28pt} 7x^2 - 7 \\
   -& \hspace{28pt} \underline{7x^2 - 7} \\
   & \hspace{28pt} \underline{----} \\
\end{align*}
\item then $x = -a$ is a root
\begin{align*} 
     \Rightarrow \hspace{5pt} &(-a)^3 + a(-a)^2 - 2(-a) + a + 4  = 0\\
      = &-a^3 + a^3 + 2a + a + 4 = 3a + 4 = 0 \hspace{10pt} \therefore a = -\cfrac{4}{3}
\end{align*}
\item \textbf{(D)} we can rewrite the entire expression like so: \vspace{-5pt}
$$(4x)^2 - 2(4x)(9y) + (9y)^2 - 3(4x) + 3(9y)$$
let $4x$ and $9y$ be $a$ and $b$ respectively \\
\begin{align*}
\underbrace{a^2 -2ab +b^2}_{(a-b)^2} -3(a - b) &= (a-b)^2 -3(a-b) \\
& = (a-b)(a-b -3) \\
& = (4x - 9y)(4x - 9y - 3)
\end{align*}
\item \textbf{(C)}  $4^2 (x-y)^2 - 3^2(x+y)$
\begin{align*} 
     & = [\underbrace{4(x-y)}_{a}]^2 -[\underbrace {3(x+y)}_{b}]^2 \\
    & = \left[4(x-y) -3(x+y)\right]\left[4(x-y) + 3(x+y)\right] \\
    & = \left[4x - 4y -3x - 3y\right]\left[4x - 4y + 3x +  3y\right] \\
    & = (x - 7y)(7x -y)
\end{align*}
\item \textbf{(B)} \begin{align*} 
    a^2 + \cfrac{1}{4} + a &=\cfrac{1}{4}(4a^2 + 4a + 1) = \cfrac{1}{4}(4a^2 + 2a + 2a + 1) \\
    & = \cfrac{1}{4} (2a(2a + 1) + (2a+1)) \\
    &= \left( \cfrac{1}{2} \right)^2 (2a+1)^2 = \left(a + \cfrac{1}{2}\right)^2
\end{align*}
\item \textbf{(B)} Because $a^2 - b^2 = (a-b)(a+b)$, the question becomes
$$(a+b)^2 -2(a+b)(a-b) + (a-b)^2$$ 
let $(a+b) = k$ and $a-b = r$ 
\begin{align*} 
     &= k^2 - 2kr + r^2 = k^2 -kr - kr + r^2 \\
     &= k(k-r)-r(k-r) = (k-r)^2 \\
     & = (a+b-a + b)^2 = (2b)^2 = 4b^2
\end{align*}
\item \textbf{(C)} let $a+b=k$ the new expression
\begin{align*}
    &=  k^2 - 14ck + 49c^2 = k^2 - 7ck -7ck + 49c^2 \\
    & = k(k-7c) -7c(k -7c) = (k-7c)^2 \\
    & = (a + b -7c)^2
\end{align*}
\item 
\item 
\item 
\item 
\item \textbf{(B)} let $p+q = k$ and $p-q = r$ 
\begin{align*} 
    &= k^2 - r^2 = (k-r)(k+r) \\
    &= [(p+q) -(p-q)][(p+q)+(p-q)] \\
   2apq & = (2q)(2p) = 4pq = 2\cdot 2 pq
\end{align*}
$\therefore \,\, a = 2$
\item \textbf{(B)} On the numerator $(a-b)^3=(a-b)^2(a-b)$ \vspace{5pt}\\
$(a^2 + b^2)(a-b) - (a-b)^2(a-b)$ \vspace{-8pt}
    \begin{align*}
        & = (a-b)[a^2 +b^2 -(a-b)^2] \\
        & = (a-b)(a^2 + b^2 - (a^2 + 2ab + b^2)) \\
        & = (a-b)(a^2 + b^2 - a^2 -2ab -b^2) \\
        &= -2ab(a-b)
     \end{align*}
On the denominator $a^2b -ab^2 = ab(a-b)$ \\
$$\therefore \,\, \cfrac{-2ab(a-b)}{ab(a-b)}= -2$$
\item 
\item 
\item \textbf{(E)} $$(x+2)(x+b) = x^2 + 2x + bx + 2b = x^2 + (2 +b)x + 2b$$
$\therefore \,\, 2b = 6,b = 3 \text{ and } 2 + b = c, 2+ 3 = 5 = c$
\item \textbf{(D)} It equals $x^3 + \cfrac{1}{x^3} + 3\left(x+ \cfrac{1}{x}\right)$ \\
$x^3 + \cfrac{1}{x^3}= \left(x + \cfrac{1}{x}\right)^3 - 3\left(x + \cfrac{1}{x}\right)$ \\
\begin{align*} 
   x^3 + 3x + \cfrac{3}{x} + \cfrac{1}{x^3} &= \left(x + \cfrac{1}{x}\right)^3 - 3\left(x + \cfrac{1}{x}\right) +3\left(x+ \cfrac{1}{x}\right) \\
   & = \left(x + \cfrac{1}{x}\right)^3
\end{align*}
\item \textbf{(D)} $x^{\frac{3}{2}} = (x^{\frac{1}{2}})^3$, $y^{\frac{3}{2}} = (y^{\frac{1}{2}})^3$ and $x = x^{\frac{2}{2}} = (x^{\frac{1}{2}})^{2}$ \\
let $x^{\frac{1}{2}} = k$ and $y^{\frac{1}{2}} = p$, make all replacement and we have
\begin{align*}
    & \hspace{7pt} k^2 + p^2 \\
    k - p \hspace{3pt} & \overline{)\hspace{3pt} k^3  -k^2p  + kp^2 -p^3} \\
    -& \hspace{7pt} \underline{k^3 -k^2p} \\
     & \hspace{55pt} kp^2 - p^3 \\
     -\hspace{-50pt} & \hspace{55pt} \underline{kp^2 - p^3} \\
     & \hspace{55pt} \underline{----} 
\end{align*}
The result is $k^2 + p^2 = (x^{\frac{1}{2}})^2 + (y^{\frac{1}{2}})^2 = x + y$
\item \textbf{(A)} If it is divided by the $ax-b = 0$, then the value of $x$ to be substituted into the function is $x = \cfrac{b}{a}$, $\therefore$  the remainder is  $f\left(\cfrac{b}{a}\right)$
\end{enumerate}
\end{multicols}